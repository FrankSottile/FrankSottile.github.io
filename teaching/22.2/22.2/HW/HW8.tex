%HW8.tex
%Eighth Homework -- Math 300H 
%
%  The percent sign is a comment character
%
%%%%%%%%%%%%%%%%%%%%%%%%%%%%%%%%%%%%%%%%%%%%%%%%%%%%%%%%%%%%%%%%%%%%%%%%%%%%%%%%%%
%
%   Look these up on line.  The first sets the type of document, and the next are for mathematics symbols, graphics and color
%
\documentclass[12pt]{article}
\usepackage{amssymb,amsmath,mathtools}
\usepackage{graphicx}
\usepackage[usenames,dvipsnames,svgnames,table]{xcolor}
\usepackage{multirow}   % This is for more control over tables
%%%%%%%%%%%%%%%%%%%%%%%%%%%%%%%%  Layout     %%%%%%%%%%%%%%%%%%%%%%%%%%%%%%%%%%%%%%
\usepackage{vmargin}
\setpapersize{USletter}
\setmargrb{1cm}{1cm}{2cm}{1cm} % --- sets all four margins LTRB


\newcommand{\RR}{{\mathbb R}}  % This is the backboard symbol for the real numbers.  Note how it is used below
\newcommand{\NN}{{\mathbb N}}  % 
\newcommand{\QQ}{{\mathbb Q}}  % 
\newcommand{\ZZ}{{\mathbb Z}}  %

\newcommand{\calP}{{\mathcal P}}  %Caligraphic P for power set

\newcommand{\sep}{{\ :\ }}     % This is for the : in our notation for building sets.
\newcommand{\lsim}{\mathord{\sim}}  %  This is to remove the extra spacing around \sim (it is a binary relation) for the logical not

\newcommand{\defcolor}[1]{{\color{blue}{#1}}}
\newcommand{\demph}[1]{{\color{blue}\sl{#1}}}
%%%%%%%%%%%%%%%%%%%%%%%%%%%%%%%%%%%%%%%%%%%%%%%%%%%%%%%%%%%%%%%%%%%%%%%%%%%%%%%%%
%
%   Please edit this as appropriate
%
\begin{document}
\LARGE 
\noindent
{\color{Maroon}Foundations of Mathematics \hfill Math 300H Section 970}\vspace{2pt}\\
\Large \vspace{2pt}\\
\large
Eighth Homework: \hfill Due \  1 November 2022
\normalsize\medskip

%%%%%%%%%%%%%%%%%%%%%%%%%%%%%%%%%%%%%%%%%%%%%%%%%%%%%%%%%%%%%%%%%%%%%%%%%%%%%%%%%%%%%%%%%%%%%%%%%%%%
\noindent{\color{blue}\rule{528.3675pt}{2pt}}

\noindent {\color{Maroon}\bf Definition:}
The \demph{Fibonacci sequence} $\{f_n\mid n\geq 1\}$ is defined by $f_1=f_2=1$ and for $n\geq 2$, $f_{n+1}=f_{n}+f_{n-1}$.

\noindent{\color{blue}\rule{528.3675pt}{2pt}}
%%%%%%%%%%%%%%%%%%%%%%%%%%%%%%%%%%%%%%%%%%%%%%%%%%%%%%%%%%%%%%%%%%%%%%%%%%%%%%%%%%%%%%%%%%%%%%%%%%%%

%%%%%%%%%%%%%%%%%%%%%%%%%%%%%%%%%%%%%%%%%%%%%%%%%%%%%%%%%%%%%%%%%%%%%%%%%%%%%%%%%
\begin{enumerate}  %  the \begin{..}  \end{..} stars and ends an environment.



%%%%%%%%%%%%%%%%%%%%%%%%%%%%%%%%%%%%%%%%%%%%%%%%%%%%%%%%%%%%%%%%%%%%%%%%%%%%%%%%%
\item Let $a,r\in{\mathbb R}$ with $r\neq 1$.
  Prove that for every number $n\in{\mathbb N}$, $a+ar+ar^2+\dotsb+ar^{n-1}= \frac{a(1-r^n)}{1-r}$.
%%%%%%%%%%%%%%%%%%%%%%%%%%%%%%%%%%%%%%%%%%%%%%%%%%%%%%%%%%%%%%%%%%%%%%%%%%%%%%%%%


%%%%%%%%%%%%%%%%%%%%%%%%%%%%%%%%%%%%%%%%%%%%%%%%%%%%%%%%%%%%%%%%%%%%%%%%%%%%%%%%%
\item Explore the divisibility by 3 of positive powers of 4.  (E.g. $4^n \mod 3$, for $n\in\NN$.)
        Make a conjecture and prove it.
%%%%%%%%%%%%%%%%%%%%%%%%%%%%%%%%%%%%%%%%%%%%%%%%%%%%%%%%%%%%%%%%%%%%%%%%%%%%%%%%%

%%%%%%%%%%%%%%%%%%%%%%%%%%%%%%%%%%%%%%%%%%%%%%%%%%%%%%%%%%%%%%%%%%%%%%%%%%%%%%%%%
\item Write a proof in paragraph form of the inequality
$3^n > 1+2^n$  for $n\geq 2$ using mathematical induction.

%%%%%%%%%%%%%%%%%%%%%%%%%%%%%%%%%%%%%%%%%%%%%%%%%%%%%%%%%%%%%%%%%%%%%%%%%%%%%%%%%


%%%%%%%%%%%%%%%%%%%%%%%%%%%%%%%%%%%%%%%%%%%%%%%%%%%%%%%%%%%%%%%%%%%%%%%%%%%%%%%%%
 \item{} Consider the sequence $\{a_n\mid n\in\NN\}$  defined by $a_1=1$, $a_2=3$ and for each $n\in \NN$,
           $a_{n+2}=3 a_{n+1}-2a_n$.
   Calculate the first eight terms of this sequence. 

   Conjecture a formula for $a_n$ and prove it using induction.
%%%%%%%%%%%%%%%%%%%%%%%%%%%%%%%%%%%%%%%%%%%%%%%%%%%%%%%%%%%%%%%%%%%%%%%%%%%%%%%%%

%%%%%%%%%%%%%%%%%%%%%%%%%%%%%%%%%%%%%%%%%%%%%%%%%%%%%%%%%%%%%%%%%%%%%%%%%%%%%%%%%
 \item{} Consider the sequence $\{a_n\mid n\in\NN\}$  defined by $a_1=a_2=1$ and for each $n\in \NN$,
           $a_{n+2}=\frac{1}{2}\left(a_{n+1}+\frac{2}{a_n}\right)$.
   Calculate the first six terms of this sequence.

   Prove, for all $n\in\NN$, that $1\leq a_n\leq 2$.
%%%%%%%%%%%%%%%%%%%%%%%%%%%%%%%%%%%%%%%%%%%%%%%%%%%%%%%%%%%%%%%%%%%%%%%%%%%%%%%%%  

%%%%%%%%%%%%%%%%%%%%%%%%%%%%%%%%%%%%%%%%%%%%%%%%%%%%%%%%%%%%%%%%%%%%%%%%%%%%%%%%%
\item  Compute the first 15 terms of the Fibonacci sequence (this will help for later problems).
           Note that the recursion $f_{n+1}=f_{n}+f_{n-1}$ may be rewritten $f_{n-1}=f_{n+1}-f_n$.
           Use this to extend the Fibonacci sequence to {\sl negative} integers and compute the values of $f_n$ for
           $-10\leq n \leq 0$.
           Conjecture a formula for $f_{-n}$ for $n\in\NN$ and prove it by induction.
%%%%%%%%%%%%%%%%%%%%%%%%%%%%%%%%%%%%%%%%%%%%%%%%%%%%%%%%%%%%%%%%%%%%%%%%%%%%%%%%%


%%%%%%%%%%%%%%%%%%%%%%%%%%%%%%%%%%%%%%%%%%%%%%%%%%%%%%%%%%%%%%%%%%%%%%%%%%%%%%%%%
\item Explore sums of squares of the Fibonacci numbers and conjecture a formula for 
  \[
       f_1^2 + f_2^2 + f_3^2 + \dotsb + f_n^2\,.
  \]
  Prove your formula.
%%%%%%%%%%%%%%%%%%%%%%%%%%%%%%%%%%%%%%%%%%%%%%%%%%%%%%%%%%%%%%%%%%%%%%%%%%%%%%%%%

%%%%%%%%%%%%%%%%%%%%%%%%%%%%%%%%%%%%%%%%%%%%%%%%%%%%%%%%%%%%%%%%%%%%%%%%%%%%%%%%%
\item Evaluate the proposed proof of the following statement.\newline
  {\bf Theorem.}  {\sl For every positive integer $n$, we have \ 
    $1+3+5+\dotsb+(2n-1)=n^2$.}\newline
  {\bf Proof.}  We proceed by induction.  Note that the formula holds for  $n=1$.
  Assume that  $1+3+5+\dotsb+(2k-1)=k^2$ for a positive integer $k$.
  We prove that $1+3+5+\dotsb+(2k+1)=(k+1)^2$.
  Observe that\vspace{-5pt}
  \begin{eqnarray*}
   1+3+5+\dotsb+(2k+1)&=&(k+1)^2\\
   1+3+5+\dotsb+(2k-1)+(2k+1)&=&(k+1)^2\\
   k^2+(2k+1)&=&(k+1)^2\\
   (k+1)^2&=&(k+1)^2\,.
   \begin{picture}(1,0)\put(110,0){$\Box$}\end{picture}
  \end{eqnarray*}
%%%%%%%%%%%%%%%%%%%%%%%%%%%%%%%%%%%%%%%%%%%%%%%%%%%%%%%%%%%%%%%%%%%%%%%%%%%%%%%%%


  
\end{enumerate}
%%%%%%%%%%%%%%%%%%%%%%%%%%%%%%%%%%%%%%%%%%%%%%%%%%%%%%%%%%%%%%%%%%%%%%%%%%%%%%%%%%%%%%%%%%%%%%%%%%%%

\end{document}


%%%%%%%%%%%%%%%%%%%%%%%%%%%%%%%%%%%%%%%%%%%%%%%%%%%%%%%%%%%%%%%%%%%%%%%%%%%%%%%%%
\item \ 
%%%%%%%%%%%%%%%%%%%%%%%%%%%%%%%%%%%%%%%%%%%%%%%%%%%%%%%%%%%%%%%%%%%%%%%%%%%%%%%%%
