%Homework1.tex
%First Homework -- Math 300 
%
%  The percent sign is a comment character
%
%%%%%%%%%%%%%%%%%%%%%%%%%%%%%%%%%%%%%%%%%%%%%%%%%%%%%%%%%%%%%%%%%%%%%%%%%%%%%%%%%%
%
%   Look these up on line.  The first sets the type of document, and the next are for mathematics symbols, graphics and color
%
\documentclass[12pt]{article}
\usepackage{amssymb,amsmath}
\usepackage{graphicx}
\usepackage[usenames,dvipsnames,svgnames,table]{xcolor}
\usepackage{multirow}   % This is for more control over tables
%%%%%%%%%%%%%%%%%%%%%%%%%%%%%%%%  Layout     %%%%%%%%%%%%%%%%%%%%%%%%%%%%%%%%%%%%%%
\usepackage{vmargin}
\setpapersize{USletter}
\setmargrb{2cm}{1cm}{2cm}{1cm} % --- sets all four margins LTRB


%%%%%%%%%%%%%%%%%%%%%%%%%%%%%%%%%%%%%%%%%%%%%%%%%%%%%%%%%%%%%%%%%%%%%%%%%%%%%%%%%
\begin{document}
\LARGE 
\noindent
{\color{Maroon}Foundations of Mathematics \hfill Math 300 Sections 902, 905}\vspace{2pt}\\
\Large YOUR NAME\vspace{2pt}\\
\large
First Homework: \hfill 31 August 2020
\normalsize\vspace{10pt}


%%%%%%%%%%%%%%%%%%%%%%%%%%%%%%%%%%%%%%%%%%%%%%%%%%%%%%%%%%%%%%%%%%%%%%%%%%%%%%%%%
\begin{enumerate}


\item Which of the following sentences are statements?
  \begin{enumerate}
   \item $3^2+4^2=5^2$.
   \item $a^2+b^2=c^2$.
   \item There exist integers $a$, $b$, and $c$ such that  $ a^2+b^2=c^2$.
   \item If $x^2= 4$, then $x=2$.
   \item For each real number $x$,  if $x^2= 4$, then $x=2$.
   \item For each real number $t$ , $\sin^2 t + \cos^2 t =1$.
   \item If $n$ is a prime number, then $n^2$ has three positive factors.
   \item $\sin x < \sin(\pi/4)$.
   \item Every rectangle is a parallelogram.
  \end{enumerate}

 %%%%%%%%%%%%%%%%%%%%%%%%%%%%%%%%%%%%%%%%%%%%%%%%%%%%%%%%%%%%%%%%%%%%%%%%%%%%%%%%%
 \item  Identify the hypothesis and the conclusion for each of the following conditional statements.
  \begin{enumerate}
   \item If $a$ is an irrational number and $b$ is an irrational number, then $a\cdot b$ is an irrational number.
   \item If $p \neq 2$ and $p$ is an even number, then $p$ is not prime.
  \end{enumerate}

 %%%%%%%%%%%%%%%%%%%%%%%%%%%%%%%%%%%%%%%%%%%%%%%%%%%%%%%%%%%%%%%%%%%%%%%%%%%%%%%%%
 \item   Determine whether each of the following conditional statements is true or false.
  \begin{enumerate}
   \item  If $10 < 7$, then $3 = 4$.
   \item If $7 < 10$, then $3 = 4$.
   \item If $10 < 7$, then $3 + 5 = 8$.
   \item If $7 < 10$, then $3 + 5 = 8$.

  \end{enumerate}

%%%%%%%%%%%%%%%%%%%%%%%%%%%%%%%%%%%%%%%%%%%%%%%%%%%%%%%%%%%%%%%%%%%%%%%%%%%%%%%%%
%
%  Above is the worksheet from Day 1
%  Next comes the worksheet from Day 2
%%%%%%%%%%%%%%%%%%%%%%%%%%%%%%%%%%%%%%%%%%%%%%%%%%%%%%%%%%%%%%%%%%%%%%%%%%%%%%%%%

\item Give a valid definition of an odd integer.
  
%%%%%%%%%%%%%%%%%%%%%%%%%%%%%%%%%%%%%%%%%%%%%%%%%%%%%%%%%%%%%%%%%%%%%%%%%%%%%%%%%
\item  Consider the following statement:
%   ``If $m$ is an even integer, then $m{+}1$ is an odd integer.''    %This is for the worksheet
 ``If $m$ is an odd integer, then $m{+}1$ is an even integer.''   %This is for the homework

  \begin{enumerate}
  \item   Construct a know-show table for a proof of this statement.
    %
    %  Take a picture of your know-show table, perhas calling it KnowShow5a.jpg and placing it in the folder with this file.
    %    (Any name will do, but it is better to not put spaces in the filenames, as some compilers complain. Many use underscore
    %      in place of a space.  Note that this is case-sensitive)
    %  Uncomment the following line:
    %  \includegraphics[height=3cm]{KnowShow5a}
    %
    %  Note: any extension (except .eps for postscript) will be OK, the compiler will find the unique file that begins 'KnowShow5a'
    %  The command height can be changed if the image is the wrong size
    %                           (you might try to alter the image file (rotate it), if it is not horizontal

   \item     Write a proof of this statement in paragraph form.
  \end{enumerate}

%%%%%%%%%%%%%%%%%%%%%%%%%%%%%%%%%%%%%%%%%%%%%%%%%%%%%%%%%%%%%%%%%%%%%%%%%%%%%%%%%  
\item  Consider the following   ``proof'' that if $m$ and $n$ are even, then $m{+}n$ is even:

  We know that $n=2t$ and $m=2t$, so $m{+}n=2t+2t= 4t$.  Therefore $m{+}n$ is even.


  \begin{enumerate}
   \item   Criticize (discuss its shortcomings).
  
   \item   Construct a know-show table for a correct  proof of this statement.
    %
    %  Take a picture of your know-show table, perhas calling it KnowShow6b.jpg and placing it in the folder with this file.
    %    (Any name will do, but it is better to not put spaces in the filenames, as some compilers complain.)
    %  Uncomment the following line:
    %  \includegraphics[height=3cm]{KnowShow6b}
    %
    %  Note: any extension (except .eps for postscript) will be OK, the compiler will find the unique file that begins 'KnowShow6b'
    %  The command height can be changed if the image is the wrong size
    %                           (you might try to alter the image file (rotate it), if it is not horizontal

   \item     Write a  correct proof of this statement in paragraph form.
  \end{enumerate}

%%%%%%%%%%%%%%%%%%%%%%%%%%%%%%%%%%%%%%%%%%%%%%%%%%%%%%%%%%%%%%%%%%%%%%%%%%%%%%%%%
\item  Consider the following statement:
  
  ``If $m$ is an even integer and $n$ is an integer, then $mn$ is an even integer.''    

  \begin{enumerate}
   \item   Construct a know-show table for a proof of this statement.
    %
    %  Take a picture of your know-show table, perhas calling it Know_Show_7a.jpg and placing it in the folder with this file.
    %    (Any name will do, but it is better to not put spaces in the filenames, as some compilers complain. Many use underscore
    %      in place of a space)
    %  Uncomment the following line:
    %  \includegraphics[height=3cm]{Know_Show_7a}
    %
    %  Note: any extension (except .eps for postscript) will be OK, the compiler will find the unique file that begins 'Know_Show_7a'
    %  The command height can be changed if the image is the wrong size
    %                           (you might try to alter the image file (rotate it), if it is not horizontal

   \item     Write a proof of this statement in paragraph form.
  \end{enumerate}


%%%%%%%%%%%%%%%%%%%%%%%%%%%%%%%%%%%%%%%%%%%%%%%%%%%%%%%%%%%%%%%%%%%%%%%%%%%%%%%%%
\item
  Which of the following statement are true and which are false?  Justify your conclusions.
  
 \begin{enumerate}
   \item
     If $a$, $b$, and $c$ are integers, then $ab + ac$ is an even integer.
   \item
     If  $a$, $b$, and $c$ are integers with both  $b$ and $c$ odd integers, then $ab + ac$ is an even integer.
 \end{enumerate}


%%%%%%%%%%%%%%%%%%%%%%%%%%%%%%%%%%%%%%%%%%%%%%%%%%%%%%%%%%%%%%%%%%%%%%%%%%%%%%%%%
\item  An integer $n$ is a {\bf type 0 integer} if there is an integer $a$ such that $n=3a$.
       An integer $n$ is a {\bf type 1 integer} if there is an integer $a$ such that $n=3a+1$.
       An integer $n$ is a {\bf type 2 integer} if there is an integer $a$ such that $n=3a+2$.
  
 \begin{enumerate}
   \item Give examples of at least four different integers that are type 1.
   \item Give examples of at least four different integers that are type 2.
   \item Multiplying pairs of integers from the first part, what do you believe about the truth value of the following statement:
     \[  \mbox{If $m$ and $n$ are both type 1 integers, then $m\cdot n$ is a type 1 integer.} \]
 \end{enumerate}


%%%%%%%%%%%%%%%%%%%%%%%%%%%%%%%%%%%%%%%%%%%%%%%%%%%%%%%%%%%%%%%%%%%%%%%%%%%%%%%%%
\item  Using the definitions from the previous exercise to help write a proof (in paragraph form) of the following statements.
  
 \begin{enumerate}
   \item  If $m$ and $n$ are both type 1 integers, then $m+n$ is a type 2 integer.
   \item  If $m$ and $n$ are both type 2 integers, then $m+n$ is a type 1 integer.
 \end{enumerate}


\end{enumerate}
%%%%%%%%%%%%%%%%%%%%%%%%%%%%%%%%%%%%%%%%%%%%%%%%%%%%%%%%%%%%%%%%%%%%%%%%%%%%%%%%%

\noindent\hrule\smallskip

For numbers 5, 6, and 7, you have three options for the  know-show tables: (1) Create them in LaTeX (see the example below, and note the
cell with two rows),  or make the table by hand, take a picture, and (2) follow the instructions for including an image in the LaTeX
file given in comments above, or else (3) convert the images to .pdf  and merge them with the .pdf for this file.


Here is the know-show table on page 20 of the book, for the assertion:
``If $x$ and $y$ are odd integers, then $x\cdot y$ is an odd integer.''\medskip

%%%%%%%%%%%%%%%%%%%%%%%%%%%%%%%%%%%%%%%%%%%%%%%%%%%%%%%%%%%%%%%%%%%%%%%%%%%%%%%%%
\begin{tabular}{|l|l|l|}\hline   % This gets us ready for a table with three columns and vertical lines, and a line at the top
  %
  %  The table is typeset one row at a time.  The columns are separated by '&' and at the end '\\ \hline' 
  %
  {\bf Step } & {\bf Statement} & {\bf Reason} \\ \hline    %   This is the header row
  %
  $P$  & $x$ and $y$ are odd integers. & Hypothesis \\ \hline
  $P1$ & \multirow{2}{250pt}{There exist integers $m$ and $n$ such that $x=2m+1$ and $y=2n+1$} & Definition of an odd integer. \\
  & & \\ \hline     %   This is needed because of \multirow command in the previous row. 
  $P2$& $xy=(2m+1)(2n+1)$& Substitution \\ \hline
  $P3$& $xy=4mn + 2m + 2n +1$ & Algebra \\ \hline
  $P4$& $xy=2(2mn +m+n)+1$ & Algebra \\ \hline
  $Q1$& There exists an integer $q$ such that $xy=2q+1$ & Use $q=2mn +m+n$ \\ \hline
  $Q$&  $x\cdot y$ is an odd integer & Definition of an odd integer \\ \hline
\end{tabular}   %  This ends the table

\end{document}
%%%%%%%%%%%%%%%%%%%%%%%%%%%%%%%%%%%%%%%%%%%%%%%%%%%%%%%%%%%%%%%%%%%%%%%%%%%%%%%%%


\end{document}
%%%%%%%%%%%%%%%%%%%%%%%%%%%%%%%%%%%%%%%%%%%%%%%%%%%%%%%%%%%%%%%%%%%%%%%%%%%%%%%%%
