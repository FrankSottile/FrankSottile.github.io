%HWC.tex
%Twelfth Homework -- Math 300H 
%
%  The percent sign is a comment character
%
%%%%%%%%%%%%%%%%%%%%%%%%%%%%%%%%%%%%%%%%%%%%%%%%%%%%%%%%%%%%%%%%%%%%%%%%%%%%%%%%%%
%
%   Look these up on line.  The first sets the type of document, and the next are for mathematics symbols, graphics and color
%
\documentclass[12pt]{article}
\usepackage{amssymb,amsmath,mathtools}
\usepackage{graphicx}
\usepackage[usenames,dvipsnames,svgnames,table]{xcolor}
\usepackage{multirow}   % This is for more control over tables
%%%%%%%%%%%%%%%%%%%%%%%%%%%%%%%%  Layout     %%%%%%%%%%%%%%%%%%%%%%%%%%%%%%%%%%%%%%
\usepackage{vmargin}
\setpapersize{USletter}
\setmargrb{1cm}{1cm}{2cm}{1cm} % --- sets all four margins LTRB


\newcommand{\RR}{{\mathbb R}}  % This is the backboard symbol for the real numbers.  Note how it is used below
\newcommand{\NN}{{\mathbb N}}  % 
\newcommand{\QQ}{{\mathbb Q}}  % 
\newcommand{\ZZ}{{\mathbb Z}}  %

\newcommand{\calP}{{\mathcal P}}  %Caligraphic P for power set

\newcommand{\sep}{{\ :\ }}     % This is for the : in our notation for building sets.
\newcommand{\lsim}{\mathord{\sim}}  %  This is to remove the extra spacing around \sim (it is a binary relation) for the logical not

\newcommand{\defcolor}[1]{{\color{blue}{#1}}}
\newcommand{\demph}[1]{{\color{blue}\sl{#1}}}
%%%%%%%%%%%%%%%%%%%%%%%%%%%%%%%%%%%%%%%%%%%%%%%%%%%%%%%%%%%%%%%%%%%%%%%%%%%%%%%%%
%
%   Please edit this as appropriate
%
\begin{document}
\LARGE 
\noindent
{\color{Maroon}Foundations of Mathematics \hfill Math 300H Section 970}\vspace{2pt}\\
\Large \vspace{2pt}\\
\large
Twelfth Homework: \hfill Due \  6 December 2021
\normalsize\medskip


\noindent{\color{blue}\rule{528.3675pt}{2pt}}

For this homework, we use that $\NN=\{1,2,3,4,\dotsc\}$.

\noindent{\color{blue}\rule{528.3675pt}{2pt}}


%%%%%%%%%%%%%%%%%%%%%%%%%%%%%%%%%%%%%%%%%%%%%%%%%%%%%%%%%%%%%%%%%%%%%%%%%%%%%%%%%
\begin{enumerate}  %  the \begin{..}  \end{..} stars and ends an environment.



%%%%%%%%%%%%%%%%%%%%%%%%%%%%%%%%%%%%%%%%%%%%%%%%%%%%%%%%%%%%%%%%%%%%%%%%%%%%%%%%%
\item {\bf Hilbert's Hotel.}
      Prove that the sets $\NN$ and $\NN\smallsetminus\{1\}$ are equinumerous.
%%%%%%%%%%%%%%%%%%%%%%%%%%%%%%%%%%%%%%%%%%%%%%%%%%%%%%%%%%%%%%%%%%%%%%%%%%%%%%%%%



%%%%%%%%%%%%%%%%%%%%%%%%%%%%%%%%%%%%%%%%%%%%%%%%%%%%%%%%%%%%%%%%%%%%%%%%%%%%%%%%%
\item Show that the function $f\colon\NN\to\ZZ$ defined by $f(n)=\frac{1}{4}(1+(-1)^n(2n-1))$ is a bijection.
       What does this demonstrate (prove)?
%%%%%%%%%%%%%%%%%%%%%%%%%%%%%%%%%%%%%%%%%%%%%%%%%%%%%%%%%%%%%%%%%%%%%%%%%%%%%%%%%


%%%%%%%%%%%%%%%%%%%%%%%%%%%%%%%%%%%%%%%%%%%%%%%%%%%%%%%%%%%%%%%%%%%%%%%%%%%%%%%%%
\item A function $f\colon\NN\times\NN\to\NN$ is defined by $f(a,b)=2^{a-1}(2b-1)$.
       Prove that $f$ is one-to-one and onto.
       Show that $\NN\times\NN$ is denumerable.
%%%%%%%%%%%%%%%%%%%%%%%%%%%%%%%%%%%%%%%%%%%%%%%%%%%%%%%%%%%%%%%%%%%%%%%%%%%%%%%%%



%%%%%%%%%%%%%%%%%%%%%%%%%%%%%%%%%%%%%%%%%%%%%%%%%%%%%%%%%%%%%%%%%%%%%%%%%%%%%%%%%
\item Prove that the set  $\RR\smallsetminus\QQ$ of irrational numbers is uncountable.
%%%%%%%%%%%%%%%%%%%%%%%%%%%%%%%%%%%%%%%%%%%%%%%%%%%%%%%%%%%%%%%%%%%%%%%%%%%%%%%%%



%%%%%%%%%%%%%%%%%%%%%%%%%%%%%%%%%%%%%%%%%%%%%%%%%%%%%%%%%%%%%%%%%%%%%%%%%%%%%%%%%
\item Prove or disprove the following:   (All have short demonstrations, at most a couple of lines, if you use results from
  the class and text.)
 \begin{enumerate}

  \item If $X$ is an uncountable set, then $|X|=|\RR|$.

  \item There exists a bijection $f\colon \QQ\to\RR$.

  \item If $F,U,N$ are sets such that $F\subseteq U\subseteq N$ and $F$ and $N$ are denumerable, then $U$ is denumerable.

  \item The set $P:=\{ n\pi\mid n\in \NN\}$ is denumerable.  (Here $\pi$ is Archimedes' constant, the ratio of the diameter
    to the circumference of any Euclidean circle.)

  \item There exists a denumerable subset of the set of irrational numbers.

  \item Every infinite set is a subset of some denumerable set.

  \item If $P$ and $Q$ are sets with the property that there exists an injection $f\colon P\to Q$, then $|P|=|Q|$.

 \end{enumerate} 
%%%%%%%%%%%%%%%%%%%%%%%%%%%%%%%%%%%%%%%%%%%%%%%%%%%%%%%%%%%%%%%%%%%%%%%%%%%%%%%%%
 

%%%%%%%%%%%%%%%%%%%%%%%%%%%%%%%%%%%%%%%%%%%%%%%%%%%%%%%%%%%%%%%%%%%%%%%%%%%%%%%%%
\item {\bf Thought-provoking.}
  Prove or disprove: there is no set $X$ whose power set ${\mathcal P}(X)$ is denumerable.
%%%%%%%%%%%%%%%%%%%%%%%%%%%%%%%%%%%%%%%%%%%%%%%%%%%%%%%%%%%%%%%%%%%%%%%%%%%%%%%%%


\end{enumerate}
%%%%%%%%%%%%%%%%%%%%%%%%%%%%%%%%%%%%%%%%%%%%%%%%%%%%%%%%%%%%%%%%%%%%%%%%%%%%%%%%%%%%%%%%%%%%%%%%%%%%

\end{document}

  

%%%%%%%%%%%%%%%%%%%%%%%%%%%%%%%%%%%%%%%%%%%%%%%%%%%%%%%%%%%%%%%%%%%%%%%%%%%%%%%%%
\item \qquad
%%%%%%%%%%%%%%%%%%%%%%%%%%%%%%%%%%%%%%%%%%%%%%%%%%%%%%%%%%%%%%%%%%%%%%%%%%%%%%%%%
 

%%%%%%%%%%%%%%%%%%%%%%%%%%%%%%%%%%%%%%%%%%%%%%%%%%%%%%%%%%%%%%%%%%%%%%%%%%%%%%%%%
\item \ 
%%%%%%%%%%%%%%%%%%%%%%%%%%%%%%%%%%%%%%%%%%%%%%%%%%%%%%%%%%%%%%%%%%%%%%%%%%%%%%%%%
