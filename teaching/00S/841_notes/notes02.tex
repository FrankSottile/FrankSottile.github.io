%notes02.tex
%
% Frank Sottile
% 1 February 2000
% Madison, WI
%
%
\documentclass[12pt]{amsart}
\usepackage{amssymb}
\usepackage{epsf}

\def\silentfootnote#1{{\let\thefootnote\relax\footnotetext{#1}}}


%\def\baselinestretch{1.2}

\headheight=8pt     \topmargin=0pt
\textheight=624pt   \textwidth=442pt
\oddsidemargin=13pt \evensidemargin=13pt


\begin{document}
\begin{center}
\Large
The Geometry of Algebraic Groups\\
\large
Math 841\\
Frank Sottile\\
{\tt notes02}
\end{center}\bigskip

\silentfootnote{\sl Version of 7 February 2000.} 

\section*{Miscellany}

The (first) chapter of {\it Studies in Algebraic Geometry}, A.~Seidenberg,
ed., MAA Studies in Mathematics, Volume 20, 1980, is by M.~Rosenlicht, and
has a brief and elegant introduction to linear algebraic groups.
For orthogonal groups in characteristic 2, the first chapter of R.~Carter, 
{\it Simple groups of Lie Type}, Wiley, 1989, is a good source.
\bigskip


When discussing the Zariski topology, I neglected to mention that points of
an algebraic variety are closed.
The point $a=(a_1,\ldots,a_n)\in{\mathbb A}^n_k$ is the zero set of the
(maximal) ideal generated by the polynomials $x_i-a_i$ for $i=1,\ldots,n$.
We may sometimes write $X(k)$ or $X_k$ for the zeroes of the polynomials
defining $X$ in $k^n$.
(This is useful when changing fields, but keeping the
same equations.)

This is also useful as the set of maximal ideals of $k[X]$ with residue
field $k$ is the set of points of $X$.
This gives the reverse direction of the algebraic-geometric dictionary.
Thus a finite group is seen to be the set of maximal ideals in the dual Hopf
algebra to its group algebra.
The maximal ideals of $k[X]$ whose residue field is an (algebraic) extension
$L$ of $k$ give rise to points in $X(L)$, as described above.
\bigskip

Another point that was skipped in the first handout was that the Hilbert 
Basis Theorem  (finite generation of ideals) and Hilbert's Nullstellensatz
(when $k$ is algebraically closed) both descend to the coordinate ring
$k[X]$ of an affine algebraic variety $X$, giving a bijective correspondence
between radical 
ideals of $k[X]$ and algebraic subvarieties of $X$.
\bigskip


\noindent{\bf Hopf Algebras. }
({\sl Courtesy of Sarah Witherspoon})
A $k$-{\em algebra} $A$ (with unit) is a $k$-vector space with
($k$-linear) maps $m:A\otimes A\rightarrow A$, 
$u:k\rightarrow A$ such that the following diagrams
commute:
$$
\begin{picture}(135,58)
\put(0,45){$A\otimes A\otimes A$}   \put(105,45){$A\otimes A$}
\put(13,0){$A\otimes A$}             \put(116,0){$A$}
\put(70,51){\scriptsize $m\otimes 1$}       
\put( 2,25){\scriptsize $1\otimes m$}
\put(75, 6){\scriptsize $m$}      \put(125,25){\scriptsize $m$}
\put(60,48){\vector(1,0){43}}   \put(50, 3){\vector(1,0){60}}
\put(28,40){\vector(0,-1){27}}  \put(121,40){\vector(0,-1){27}}
\end{picture}
\qquad\qquad
\begin{picture}(103,58)
\put(18,45){$A$}                 \put(70,45){$A\otimes A$}
\put(6,0){$A\otimes A$}        \put(80,0){$A$}
\put(40,51){\scriptsize $u\otimes 1$}       
\put( 0,25){\scriptsize $1\otimes u$}
\put(50, 6){\scriptsize$m$}     \put(89,25){\scriptsize$m$}
\put(30,48){\vector(1,0){35}}   \put(40, 3){\vector(1,0){38}}
\put(23,40){\vector(0,-1){27}}  \put(85,40){\vector(0,-1){27}}
\put(29,42){\vector(3,-2){50}} \put(51,31){\scriptsize $1$}
\end{picture}
$$
In the second diagram, we use the canonical isomorphisms 
$k\otimes A \simeq A\simeq A\otimes k$.

A {\em coalgebra} $C$ is a vector space with maps $\Delta:C\rightarrow 
C\otimes C$ ({\em comultiplication}), $\epsilon:C\rightarrow k$ ({\em counit}
or {\em augmentation}) such that the following diagrams commute:
$$
\begin{picture}(140,58)
\put(13,45){$C$}              \put(100,45){$C\otimes C$}
\put(0,0){$C\otimes C$}      \put(85, 0){$C\otimes C\otimes C$}
\put(53,51){\scriptsize $\Delta$}       \put( 8,25){\scriptsize $\Delta$}
\put(47.5, 6){\scriptsize $\Delta\otimes 1$}      
\put(120,25){\scriptsize $1\otimes\Delta$}
\put(30,48){\vector(1,0){62}}   \put(40, 3){\vector(1,0){40}}
\put(18,40){\vector(0,-1){27}}  \put(116,40){\vector(0,-1){27}}
\end{picture}
\qquad\qquad
\begin{picture}(107,58)
\put(18,45){$C$}                 \put(75,45){$C\otimes C$}
\put(6,0){$C\otimes C$}        \put(85,0){$C$}
\put(45,51){\scriptsize $\epsilon\otimes 1$}       
\put( 0,25){\scriptsize $1\otimes\epsilon$}
\put(60, 6){\scriptsize$\Delta$}     \put(94,25){\scriptsize$\Delta$}
\put(70,48){\vector(-1,0){40}}   \put(77, 3){\vector(-1,0){32}}
\put(23,13){\vector(0,1){27}}  \put(90,13){\vector(0,1){27}}
\put(80,10){\vector(-3,2){50}} \put(51,31){\scriptsize $1$}
\end{picture}
$$
The first of these diagrams is the {\em coassociativity} property.


A {\em bialgebra} $B$ is an algebra $(B,m,u)$
and a coalgebra $(B,\Delta,\epsilon)$ such that $\Delta$,
$\epsilon$ are algebra maps, or equivalently, such that $m$, $u$
are coalgebra maps.  Here we use the fact that if $C$, $C'$ are coalgebras,
then $C\otimes C'$ is as well by the following ($\tau$ denotes the twist
map $C\otimes C'\rightarrow C'\otimes C$):
$$
\begin{picture}(280,47)
\put(0,37){$C\otimes C'$}
\put(170,37){$(C\otimes C')\otimes(C\otimes C')$}
\put(70,0){$(C\otimes C)\otimes (C'\otimes C')$}
\put(42,40){\vector(1,0){120}}
\put(38,34){\vector(3,-1){63}}
\put(140,13){\vector(3,1){60}}
\put(32,18){\scriptsize $\Delta\otimes\Delta'$} 
\put(180,18){\scriptsize $1\otimes\tau\otimes 1$} 
\end{picture}
$$

A {\em Hopf algebra} $H$ is a bialgebra $(H,m,u,\Delta,
\epsilon)$ with a map $s:H\rightarrow
H$ ({\em coinverse} or {\em antipode}) such that the following diagram 
commutes:
$$
\begin{picture}(260,81)
\put(0,32){$H$}    \put(120,32){$k$}     \put(237,32){$H$}
\put( 50,67){$H\otimes H$}   \put( 50, 0){$H\otimes H$}
\put(165,67){$H\otimes H$}   \put(165, 0){$H\otimes H$}
\put(15,37){\vector(1,0){100}}  \put(135,37){\vector(1,0){100}}
\put(90,5){\vector(1,0){70}}  \put(90,72){\vector(1,0){70}}
\put(15,32){\vector(2,-1){40}} \put(195,62){\vector(2,-1){40}}
\put(15,42){\vector(2, 1){40}} \put(195,12){\vector(2, 1){40}}
\put(28,14){\scriptsize$\Delta$}\put(28,54){\scriptsize$\Delta$}
\put(218,14){\scriptsize$m$}   \put(218,54){\scriptsize$m$}
\put(115,8){\scriptsize$1\otimes s$}\put(115,75){\scriptsize$s\otimes 1$}
\put(70,40){\scriptsize$\epsilon$} \put(180,40){\scriptsize$u$} 
\end{picture}
$$
This property implies that $s$ is an algebra anti-homomorphism, i.e.
$s(hh')=s(h')s(h)$ for all $h,h'\in H$.

\noindent{\bf References:}
\begin{itemize}
\item[] S. Montgomery, {\em Hopf Algebras and Their Actions on
Rings,} Regional Conference Series in Mathematics, Number 82, American
Mathematical Society, 1993.
 
\item[]
M. E. Sweedler, {\em Hopf Algebras,} W. A. Benjamin, Inc., 1969.
\end{itemize}




\section{Elementary structure of (affine) algebraic groups}

\noindent{\bf Theorem.} 
{\it The closure of a subgroup of an algebraic group is a subgroup.
If the subgroup is normal, then its closure is, too.}\medskip

An example of such a non-closed subgroup is any compact torus 
$(S^1)^m$ inside the complex diagonal $n\times n$-matrices 
$({\mathbb C}^\times)^n$.

\noindent{\sl Proof.}
Let $H\subset G$ be a subgroup.
Consider the map $\nu\colon G\times G\to G$ defined by 
$\nu\colon (g,h)\mapsto gh^{-1}$.
Then the restriction of $\nu$ maps $H\times H$ to $H$,
and hence to the closure $\overline{H}$ of $H$.
The inverse image of $\overline{H}$ under $\nu$ is closed and it contains
$H\times H$, and hence it contains 
$\overline{H\times H}=\overline{H}\times \overline{H}$.
This shows that $\overline{H}$
is a subgroup of $G$.

If $H$ is also normal, then for every $g\in G$, we can consider the map 
$\varphi_g\colon h\mapsto ghg^{-1}$.
As before $\varphi_g\colon\overline{H}\to\overline{H}$, and so
$\overline{H}$ is normal.
\qed

We now prove a more substantial structure theorem about affine algebraic
groups.

\noindent{\bf Definition}
A topological space is {\sl Noetherian} if it has the descending chain
condition on closed sets.
The Zariski topology on an affine algebraic variety is Noetherian.
If we have $X=X_1\cup X_2$, where $X_1$ and $X_2$ are proper closed subsets,
then $X$ is {\sl reducible}.
Otherwise $X$ is {\it irreducible}.
\medskip


\noindent{\bf Theorem}
{\it Any closed subset of a Noetherian topological space is uniquely
expressible as a union of finitely many closed irreducible subsets, none of
which contains another.}\medskip

\noindent{\sl Proof. }
Suppose that a Noetherian topological space $X$ has a closed subset $Y$
which is not the union of finitely many irreducible closed subsets.
Then we may write $Y=Y_1\cup Y_2$, with each $Y_i$ a proper closed subset
of $Y$.
Because $Y$ is not the union of finitely many proper closed subsets, than at
least one of $Y_1$ or $Y_2$ is not either.
Continuing in this fashion, we construct an infinite descending chain of
closed subsets of $X$, and so $X$ is not Noetherian.

Suppose $Y$ is a closed subset of $X$, and we have 
$Y=Y_1\cup \cdots \cup Y_n$ with each $Y_i$ closed and irreducible.
We may if necessary remove any $Y_i$ with $Y_i\subset Y_j$ for some 
$i\neq j$, and assume that we do not have $Y_i\subset Y_j$ with $i\neq j$.
Suppose we have another such representation 
$Y=Z_1\cup\cdots\cup Z_m$.
Then $Y_i=\bigcup_j (Y_i\cap Z_j)$.
Since $Y_i$ is irreducible, there is some index $j$ with $Y_i=Y_i\cap Z_j$,
so we have $Y_i\subset Z_j$.
Similarly, there is an index $l$ with $Z_j\subset Y_l$, which implies 
$Y_i=Z_j$, and hence these two representations are equal.
\qed\bigskip


When $Y=Y_1\cup\cdots\cup Y_n$ is the unique expression of $Y$ as a finite
union of irreducible subsets, we call each $Y_i$ a {\sl component} of $Y$.
If $Z\subset Y$ is irreducible, then the arguments in the last paragraph
above show that $Z$ is contained wholly in a single
component of $Y$.
Similarly, the image of an irreducible set under a continuous map is
irreducible.


It is an exercise in algebra that an algebraic variety $X$ is irreducible is
and only if its coordinate ring $k[X]$ has no zero divisors, which is
equivalent to the ideal ${\mathcal I}(X)$ of polynomials vanishing on $X$
being a prime ideal. 
When $X$ is reducible, then the components of $X$ correspond to minimal
prime ideals of $k[X]$.
If $f\colon X \to Y$ is a map with $X$ irreducible, then the image of $X$
lies in a single component of $Y$:
The map $f^*:k[Y]\to k[X]$ must factor through the quotient of $k[Y]$ by one
of these minimal primes.

Observe that irreducibility may not be preserved under field extensions;
some irreducible components of $X(k)$ may become reducible in the algebraic
closure ${\mathbb K}$ of $k$.
When this occurs, these components are permuted by the Galois group of 
${\mathbb K}$ over $k$.
A variety $X(k)$ is {\sl absolutely irreducible} if $X({\mathbb K})$ is
irreducible.
\medskip

\noindent{\bf Theorem. }
{\it A product of irreducible closed sets is irreducible.}\medskip

\noindent{\sl Proof. }
Suppose that $X\times Y=Z_1\cup Z_2$,
with each $Z_i$ a closed subset of $X\times Y$.
For each $x\in X$, the closed set $\{x\}\times Y$ is isomorphic to $Y$, and
is therefore irreducible.
Since 
$$
  \{x\}\times Y\ =\ ((\{x\}\times Y)\cap Z_1)\   \cup\  
                   ((\{x\}\times Y)\cap Z_2)\,,
$$
either $\{x\}\times Y\subset Z_1$ or else 
$\{x\}\times Y\subset Z_2$.
The subset $X_1\subset X$ consisting of those $x\in X$ with 
$\{x\}\times Y\subset Z_1$ is a closed subset:
We have $X_1=\bigcap_{y\in Y} X_y$, where $X_y$ is the collection of points
$x\times y\in Z_1$.
Since $X_y\times \{y\}=(X\times \{y\})\cap Z_1$, $X_y$ and hence $X_1$ is
closed.
Similarly define the closed subset $X_2$.
Since $X=X_1\cup X_2$ and $X$ is irreducible, we either have $X=X_1$ or
$X=X_2$.
But $X=X_i$ implies $X\times Y=Z_i$, which proves $X\times Y$ is
irreducible.
\qed
\bigskip

\noindent{\bf Theorem. }
{\it
An affine algebraic group $G$ has a unique component $G_0$ containing the
identity element $e$.
This identity component $G_0$ is a closed normal subgroup of $G$ of finite
index.
The various components of $G$ are the (left or right) cosets of $G_0$ in
$G$, and these are disjoint.}\medskip

\noindent{\sl Proof.} 
Each component of $G$ is not contained in the union of the other components,
and so contains a point not lying in any other component.
Since $G$ is homogeneous
($G$ acts transitively on itself via isomorphisms),
each point of $G$ is contained in a unique component of $G$.
Let $G_0$ be the component of $G$ containing the identity element.
Then the various components of $G$ are translates of $G_0$.

Consider the map $G\times G\to G$ defined by $(g,h)\mapsto gh^{-1}$.
Since the restriction of this map $G_0\times G_0\to G$ sends 
$(e,e)\mapsto e$, and $G_0\times G_0$ is irreducible, we must have
$G_0\times G_0 \subset G_0$. 
Thus $G_0$ is a subgroup of $G$.
To see that $G_0$ is normal, let $H\in G$ and consider the map
$\varphi\colon G\to G$ defined by $g\mapsto hgh^{-1}$.
Again, $\varphi(e)=e$ and so $\varphi_h(G_0)\subset G_0$, which completes
the proof.
\qed\medskip


We say that $G_0$ is {\sl connected} if $G=G_0$.
Previously, we defined the special orthogonal groups to be the subgroup of
the orthogonal group consisting of matrices with determinant 1.
We could also have defined the special orthogonal groups to be the identity
component of the orthogonal groups.
%{\bf Something about a connected $k$-group being absolutely irreducible.}

\end{document}
