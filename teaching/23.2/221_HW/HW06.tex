%HW06.tex
%Sixth Homework -- Math 221H
% 
%
% Frank Sottile
% 24 September 2023 
%
%%%%%%%%%%%%%%%%%%%%%%%%%%%%%%%%%%%%%%%%%%%%%%%%%%%%%%%%
\documentclass[12pt]{article}
\usepackage{amssymb,amsmath}
\usepackage{graphicx}
\usepackage[usenames,dvipsnames,svgnames,table]{xcolor}
\usepackage{multirow}   % This is for more control over tables
%%%%%%%%%%%%%%%%%%%%%%%%%%%%%%%%  Layout     %%%%%%%%%%%%%%%%%%%%%%%%%%%%%%%%%%%%%%
\usepackage{vmargin}
\setpapersize{USletter}
\setmargrb{1cm}{0.25cm}{1cm}{0.5cm} % --- sets all four margins LTRB

%%%%%%%%%%%%%%%%%%%%%%%%%%%%%%%%%%%%%%%%  Macros  %%%%%%%%%%%%%%%%%%%%%%%%%%%%%%%%%%%%%%%%
\newcommand{\RR}{{\mathbb R}}  % This is the backboard bold symbol for the real numbers.  Note how it is used below
\newcommand{\NN}{{\mathbb N}}  % 
\newcommand{\ZZ}{{\mathbb Z}}  %

\newcommand{\calP}{{\mathcal P}}  %Caligraphic P for power set

\newcommand{\bfa}{{\bf a}}    %Vectors
\newcommand{\bfb}{{\bf b}}    %Vectors
\newcommand{\bfi}{{\bf i}}    %Unit Vectors
\newcommand{\bfj}{{\bf j}}    %Unit Vectors
\newcommand{\bfk}{{\bf k}}    %Unit Vectors

\newcommand{\sep}{{\ :\ }}     % This is for the : in our notation for building sets.
                               % an acceptable (and common) alternative is \mid  (try it!)
%%%%%%%%%%%%%%%%%%%%%%%%%%%%%%%%%%%%%%%%%%%%%%%%%%%%%%%%%%%%%%%%%%%%%%%%
\begin{document}
\LARGE 
\noindent
{\color{Maroon}Honors Multivariate Calculus \hfill Math 221H Section 201}\vspace{2pt}\\
\large
Sixth Homework:\hfill 
Due in recitation: Thursday 28 September 2023\vspace{2pt}

\normalsize
    {\bf {\color{Maroon}Homework about the chain rule and directional derivatives}}\vspace{2pt}

%%%%%%%%%%%%%%%%%%%%%%%%%%%%%%%%%%%%%%%%%%%%%%%%%%%%%%%%%%%%%%%%%%%%%%%%%%%%%%%%%%%%%%%%%%%%%%%%%%%%
\begin{enumerate}


%%%%%%%%%%%%%%%%%%%%%%%%%%%%%%%%%%%%%%%%%%%%%%%%%%%%%%%%%%%%%%%%%%%%%%%%%%%%%%%%%%%%%%%%%%%%%%%%%%%%
  
\item The length $\ell$, width $w$, and height $h$ of a box change with time.
      At some time, we have $\ell=1$ and $w=h=2$ (all in metres), and $\ell$ and $w$ are decreasing at a rate of 1 m/s, while $h$ is
      increasing at a rate of 3 m/s.
      At this point in time, find the rates at which the following quantities are changing.\vspace{-2pt}
      \[
       (a)\ \mbox{The volume.}\qquad
       (b)\ \mbox{The surface area.}\qquad
       (c)\ \mbox{The main diagonal.}\vspace{-2pt}
      \]      
  

%%%%%%%%%%%%%%%%%%%%%%%%%%%%%%%%%%%%%%%%%%%%%%%%%%%%%%%%%%%%%%%%%%%%%%%%%%%%%%%%%%%%%%%%%%%%%%%%%%%%

%%%%%%%%%%%%%%%%%%%%%%%%%%%%%%%%%%%%%%%%%%%%%%%%%%%%%%%%%%%%%%%%%%%%%%%%%%%%%%%%%%%%%%%%%%%%%%%%%%%%
  
\item  Write out the chain rule for the partial derivatives of $u$ with respect to $x,y,z$, and $w$, if\vspace{-2pt}
      \[
      u\ =\ f(s,t)\,,\qquad
      s\ =\ s(w,x,y,z)\,,\quad\mbox{and}\quad
      t\ =\ t(w,x,y,z)\,.\vspace{-2pt}
      \]
%%%%%%%%%%%%%%%%%%%%%%%%%%%%%%%%%%%%%%%%%%%%%%%%%%%%%%%%%%%%%%%%%%%%%%%%%%%%%%%%%%%%%%%%%%%%%%%%%%%%

%%%%%%%%%%%%%%%%%%%%%%%%%%%%%%%%%%%%%%%%%%%%%%%%%%%%%%%%%%%%%%%%%%%%%%%%%%%%%%%%%%%%%%%%%%%%%%%%%%%%
  
\item  If $z=f(x,y)$, where $x=s+t$ and $y=s-t$, show that
  $   \Bigl(\frac{\partial z}{\partial x}\Bigr)^2  - 
   \Bigl(\frac{\partial z}{\partial y}\Bigr)^2\ =\
   \frac{\partial z}{\partial s}\ 
   \frac{\partial z}{\partial t}$.\vspace{-2pt}
%%%%%%%%%%%%%%%%%%%%%%%%%%%%%%%%%%%%%%%%%%%%%%%%%%%%%%%%%%%%%%%%%%%%%%%%%%%%%%%%%%%%%%%%%%%%%%%%%%%%

%%%%%%%%%%%%%%%%%%%%%%%%%%%%%%%%%%%%%%%%%%%%%%%%%%%%%%%%%%%%%%%%%%%%%%%%%%%%%%%%%%%%%%%%%%%%%%%%%%%%
\item Suppose that $f(x,y)=\sin(x+2y)$.
  Find the directional derivative of $f$ at the point $(x,y)=(4,-2)$ in the direction of the angle $\theta=-2\pi/3$.\vspace{-2pt}
%%%%%%%%%%%%%%%%%%%%%%%%%%%%%%%%%%%%%%%%%%%%%%%%%%%%%%%%%%%%%%%%%%%%%%%%%%%%%%%%%%%%%%%%%%%%%%%%%%%%

 
%%%%%%%%%%%%%%%%%%%%%%%%%%%%%%%%%%%%%%%%%%%%%%%%%%%%%%%%%%%%%%%%%%%%%%%%%%%%%%%%%%%%%%%%%%%%%%%%%%%%
\item Suppose that $f(x,y,z)=x^3 y^2 z$.
  Find the gradient of $f$, evaluate it at the point $(1,-2,1)$, and find the rate of change of $f$ at this point
  in the direction of $\langle \frac{1}{\sqrt{3}},- \frac{1}{\sqrt{3}}, \frac{1}{\sqrt{3}}\rangle$.\vspace{-2pt}
%%%%%%%%%%%%%%%%%%%%%%%%%%%%%%%%%%%%%%%%%%%%%%%%%%%%%%%%%%%%%%%%%%%%%%%%%%%%%%%%%%%%%%%%%%%%%%%%%%%%

%%%%%%%%%%%%%%%%%%%%%%%%%%%%%%%%%%%%%%%%%%%%%%%%%%%%%%%%%%%%%%%%%%%%%%%%%%%%%%%%%%%%%%%%%%%%%%%%%%%%
\item Suppose that $f(x,y,z)=x^3-y^2z$.
  Find the directional derivative $D_uf(2,1,6)$ where $u$ has the same direction as $\langle 12,3,4\rangle$.\vspace{-2pt}
%%%%%%%%%%%%%%%%%%%%%%%%%%%%%%%%%%%%%%%%%%%%%%%%%%%%%%%%%%%%%%%%%%%%%%%%%%%%%%%%%%%%%%%%%%%%%%%%%%%%
  
%%%%%%%%%%%%%%%%%%%%%%%%%%%%%%%%%%%%%%%%%%%%%%%%%%%%%%%%%%%%%%%%%%%%%%%%%%%%%%%%%%%%%%%%%%%%%%%%%%%%
\item  Find the maximum rate of change of the function at the given point, and the direction at which it occurs.\vspace{-2pt}
  \[
  f(x,y)=\ln(x^2+2y^2)\ \mbox{at }(x,y)=(3,4) \qquad
  f(x,y,z)=\tfrac{x}{y} +\tfrac{y}{z}\ \mbox{at }(x,y,z)=(9,3,1)\,.\vspace{-2pt}
 \]
%%%%%%%%%%%%%%%%%%%%%%%%%%%%%%%%%%%%%%%%%%%%%%%%%%%%%%%%%%%%%%%%%%%%%%%%%%%%%%%%%%%%%%%%%%%%%%%%%%%%

     
%%%%%%%%%%%%%%%%%%%%%%%%%%%%%%%%%%%%%%%%%%%%%%%%%%%%%%%%%%%%%%%%%%%%%%%%%%%%%%%%%%%%%%%%%%%%%%%%%%%%
\item Suppose that the electric potential $V$ in a region of space has the formula $xyz + 5y^2 - 3xy$.
  What is the rate of change of the potential at the point $(x,y,z)=(4,3,5)$ in the direction of $\bfi+\bfj-\bfk$?\vspace{-2pt}

  In what direction does $V$ change most rapidly at this point?\vspace{-2pt}

  What is the maximum rate of change at this point?\vspace{-2pt}
%%%%%%%%%%%%%%%%%%%%%%%%%%%%%%%%%%%%%%%%%%%%%%%%%%%%%%%%%%%%%%%%%%%%%%%%%%%%%%%%%%%%%%%%%%%%%%%%%%%%

    
%%%%%%%%%%%%%%%%%%%%%%%%%%%%%%%%%%%%%%%%%%%%%%%%%%%%%%%%%%%%%%%%%%%%%%%%%%%%%%%%%%%%%%%%%%%%%%%%%%%%
\item Find equations of the tangent plane and the normal line to the implicit surface(s) at the  given point(s).\vspace{-2pt}
  \[
  (a)\  x^2-2y^2+z^2=3\ \mbox{at }(x,y,z)= (-1,1,2) \qquad\qquad
  (b)\  x e^{yz}=1\ \mbox{at }(x,y,z)= (1,0,2)\,. \vspace{-2pt}
 \]
%%%%%%%%%%%%%%%%%%%%%%%%%%%%%%%%%%%%%%%%%%%%%%%%%%%%%%%%%%%%%%%%%%%%%%%%%%%%%%%%%%%%%%%%%%%%%%%%%%%%

   
%%%%%%%%%%%%%%%%%%%%%%%%%%%%%%%%%%%%%%%%%%%%%%%%%%%%%%%%%%%%%%%%%%%%%%%%%%%%%%%%%%%%%%%%%%%%%%%%%%%%
\item If $f(x,y)=2x^3-3xy+y^2$ find the gradient vector $\nabla f(1,3)$ and use it to give an equation for the tangent line
  to the curve $f(x,y)=2$ at the point $(x,y)=(1,3)$.\vspace{-2pt}
%%%%%%%%%%%%%%%%%%%%%%%%%%%%%%%%%%%%%%%%%%%%%%%%%%%%%%%%%%%%%%%%%%%%%%%%%%%%%%%%%%%%%%%%%%%%%%%%%%%%

   
%%%%%%%%%%%%%%%%%%%%%%%%%%%%%%%%%%%%%%%%%%%%%%%%%%%%%%%%%%%%%%%%%%%%%%%%%%%%%%%%%%%%%%%%%%%%%%%%%%%%
\item Show that the tangent plane to the ellipsoid $x^2/a^2 + y^2/b^2 + z^2/c^2 =1$ at the point $(x_0,y_0,z_0)$
  is given by the equation  $  \frac{x x_0}{a^2} + \frac{y y_0}{b^2} + \frac{z z_0}{c^2} = 1$. \vspace{-2pt}
%%%%%%%%%%%%%%%%%%%%%%%%%%%%%%%%%%%%%%%%%%%%%%%%%%%%%%%%%%%%%%%%%%%%%%%%%%%%%%%%%%%%%%%%%%%%%%%%%%%%

   
%%%%%%%%%%%%%%%%%%%%%%%%%%%%%%%%%%%%%%%%%%%%%%%%%%%%%%%%%%%%%%%%%%%%%%%%%%%%%%%%%%%%%%%%%%%%%%%%%%%%
\item Show that the sum of the $x$-, $y$-, and $z$-intercepts of any tangent plane to the surface
  $\sqrt{x} + \sqrt{y} + \sqrt{z} = a$ ($a$ is a fixed positive real number) is a constant.\vspace{-2pt}
%%%%%%%%%%%%%%%%%%%%%%%%%%%%%%%%%%%%%%%%%%%%%%%%%%%%%%%%%%%%%%%%%%%%%%%%%%%%%%%%%%%%%%%%%%%%%%%%%%%%

   
%%%%%%%%%%%%%%%%%%%%%%%%%%%%%%%%%%%%%%%%%%%%%%%%%%%%%%%%%%%%%%%%%%%%%%%%%%%%%%%%%%%%%%%%%%%%%%%%%%%%
\item Find the points on the ellipsoid $3x^2+2y^2+z^2=1$ where the tangent plane is parallel to the plane
  $3x+y-3z=9$.\vspace{-2pt}
%%%%%%%%%%%%%%%%%%%%%%%%%%%%%%%%%%%%%%%%%%%%%%%%%%%%%%%%%%%%%%%%%%%%%%%%%%%%%%%%%%%%%%%%%%%%%%%%%%%%

  
   
%%%%%%%%%%%%%%%%%%%%%%%%%%%%%%%%%%%%%%%%%%%%%%%%%%%%%%%%%%%%%%%%%%%%%%%%%%%%%%%%%%%%%%%%%%%%%%%%%%%%
\item Find the points on the graph of the function $z=f(x,y)=\frac{(x+y+1)^2}{x^2+y^2+1}$ where the tangent plane is
  horizontal.\vspace{-2pt} 
%%%%%%%%%%%%%%%%%%%%%%%%%%%%%%%%%%%%%%%%%%%%%%%%%%%%%%%%%%%%%%%%%%%%%%%%%%%%%%%%%%%%%%%%%%%%%%%%%%%%

\end{enumerate}



\end{document}
%%%%%%%%%%%%%%%%%%%%%%%%%%%%%%%%%%%%%%%%%%%%%%%%%%%%%%%%%%%%%%%%%%%
