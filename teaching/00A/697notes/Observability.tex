%Observability.tex
%%%%%%%%%%%%%%%%%%%%%%%%%%%%%%%%%%%%%%%%%%%%%
%
% Frank Sottile
% 20 September 2000
% Amherst
%
%%%%%%%%%%%%%%%%%%%%%%%%%%%%%%%%%%%%%%%%%%%%%
%
%  These are notes from my class lectures on 
%  observability, from 19 September 2000
%
%%%%%%%%%%%%%%%%%%%%%%%%%%%%%%%%%%%%%%%%%
\documentclass[12pt]{amsart}
\usepackage{amssymb}
\textwidth 6.4in\oddsidemargin 0.1in
\textheight 9.5in\topmargin -0.8in   % For regular US paper
\begin{document}

\begin{center}
\Large\bf Supplemental Notes\\
Math 697R\\
21 September, 2000
\end{center}
\noindent{\large\bf Observability}

In an actual system, it may not be possible to 
observe the full state $x_t\in{\mathbb R}^n$ at a given time,
but rather only some projection of that state, called the 
{\it output vector}
$$
  y_t\ =\ Cx_t+Du_t\,,
$$
where $C\in\mbox{\rm Mat}_{p\times n}({\mathbb R})$ and 
$D\in\mbox{\rm Mat}_{p\times m}({\mathbb R})$.

Our discrete-time linear system becomes
%
 \begin{equation}\label{system}
  \begin{array}{rcl}
    x_{t+1}&\ =\ & A x_t + B u_t\\
    y_t & = & C x_t + D u_t
  \end{array}
 \end{equation}
%

\noindent{\bf Definition. }
A system is called {\it observable} for a sequence of inputs
$(u_0,u_1,\ldots)$ if we can deduce the initial state 
$x_0$ by knowing the sequence of outputs
$(y_t\mid t\in{\mathbb N})$.

We first compute the sequence of outputs
%
\begin{eqnarray*}
  y_d &=& A x_d + D u_d\,,\\
      &=& CA^d x_0 + [CA^{d-1}B\ \ldots\ CAB\ CB\ D]\ 
           \left[\begin{array}{c} u_0\\\vdots\\ 
                  u_{d-1}\\ u_d\end{array}\right]\ .
\end{eqnarray*}
%
We can discern the difference between two initial states
$x_0$ and $x_0'$ (and hence identify states) only if they 
give different sequences of outputs $y_t$ and $y_t'$, 
that is only if the sequence of differences
$$
  y_t-y'_t\ =\ CA^t(x_t-x'_t)
$$ 
is not identically zero.
Note the happy fact that this does not depend upon the chosen
sequence of inputs nor on the matrices $B$ and $D$.


We may deduce the following Theorem in much the same way as we deduced
the similar theorem about controllability.
\medskip


\noindent{\bf Theorem.}
{\it 
The following statements about the system~{\rm (\ref{system})} are equivalent.
\begin{enumerate}
 \item[{\rm(i)}] 
      The system~{\rm (\ref{system})}  is observable for one sequence 
      $(u_t\mid t\in{\mathbb N})$ of inputs.
 \item[{\rm(ii)}]  
      The system~{\rm (\ref{system})} is observable for all sequences 
      $(u_t\mid t\in{\mathbb N})$ of inputs.
 \item[{\rm(iii)}]  
      ${\displaystyle \bigcap_{t\geq 0} C A^t\ =\ \{0\}}$.
 \item[{\rm(iv)}]  
      The row space of ${\displaystyle 
       \left[\begin{array}{c}CA^{n-1}\\\vdots\\CA\\C\end{array}\right]}$
      is equal to ${\mathbb R}^n$.
 \item[{\rm(v)}]  
      The matrix ${\displaystyle 
           \left[\begin{array}{c}CA^{n-1}\\\vdots\\CA\\C\end{array}\right]}$
      has (full) rank $n$.
 \item[{\rm(vi)}]  
      ${\mathbb R}^n$ is the minimal subspace containing 
      the row space of $C$ which is stable under multiplication 
      by $A$ from the right.
\end{enumerate}
}\medskip



\noindent{\bf Exercise.}
Prove the previous Theorem.

\end{document}