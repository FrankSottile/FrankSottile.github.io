%notes01.tex
%
% Frank Sottile
% 25 January 2000
% Madison, WI
%
%
\documentclass[12pt]{amsart}
\usepackage{amssymb}
\usepackage{epsf}

\def\silentfootnote#1{{\let\thefootnote\relax\footnotetext{#1}}}


\headheight=8pt     \topmargin=0pt
\textheight=624pt   \textwidth=452pt
\oddsidemargin=8pt \evensidemargin=8pt


\begin{document}
\begin{center}
\Large
The Geometry of Algebraic Groups\\
\large
Math 841\\
Frank Sottile\\
{\tt notes01}
\end{center}\bigskip

\silentfootnote{\sl Version of 7 February 2000.} 

\section*{Prelude}

Let $k$ be a field.
Consider the following important, but completely impossible problem:
Let $f_1,\ldots,f_N\in k[x_1,\ldots,x_n]$ be polynomials.
Describe the (isolated) solutions in $k^n$ to the system of polynomial
equations 
$$
   f_1(x_1,\ldots,x_n)\ =\ \cdots\ =\ f_N(x_1,\ldots,x_n)\ =\ 0\,.
$$
In this generality, we cannot even hope to count the number of solutions.

However, in many important cases, these polynomials come from a very nice
geometric situation, and it is possible to say quite a bit about this
problem.
In these geometric situations, an important role is played by a
certain group which acts transitively through polynomial equations, and much
of what we can say about this problem comes from analyzing and understanding
the structure of this group.
The purpose of this class is to begin a study of these 
{\it affine algebraic groups} and the special {\it flag varieties} on
which they act transitively, then formulate these geometric problems and
give some answers to the question above about their solutions.

These objects (algebraic groups, flag varieties, and the geometric
problems) all have importance and life in mathematics beyond the picture I
will present.
Our quest will also touch on many topics that are interesting in their own
right, including algebraic geometry, cohomology, combinatorics, and maybe
even symbolic computation, and will give a hint of some interactions
between these subjects.

Before beginning the course, I leave you with one such geometric situation:
Suppose $k$ is algebraically closed and we have $2m$ general
2-planes $L_1,\ldots, L_{2m}$ in $k^{m+2}$.
The question is: how many $m$-planes $H\subset k^{m+2}$ intersect each of
these 2-planes $L_i$ non-trivially?
The answer is $\frac{1}{m+1}\binom{2m}{m}$, and later in the course, we will
prove this when $k$ has characteristic zero.


\section{Affine Algebraic Groups}

\subsection{Affine Algebraic Varieties}
We would like to work over an arbitrary field $k$, but the strongest results
in algebraic geometry are true only for algebraically closed fields
${\mathbb K}$. 
Even worse, algebraic geometry over an algebraically closed field  is
somewhat elementary and easily motivated, while the foundations necessary to
treat arbitrary fields are not.
On the other hand, the objects we will study have very similar structures
{\sl independent of the field over which we work}, so there is an advantage
to be as general as possible.

Affine $n$-space over $k$ (${\mathbb A}^n$ or ${\mathbb A}^n_k$) is the set 
$k^n$.
An {\sl affine algebraic variety} $X\subset {\mathbb A}^n$ is the set of
common zeroes of some finite collection of polynomials
$$
   X\ :=\ \{(x_1,\ldots,x_n)\in{\mathbb A}^n \mid 
   f_1(x)\ =\ \cdots\ =\ f_N(x)\ =\ 0\}\,,
$$
with each $f_i\in k[x_1,\ldots,x_n]=k[x]$.
Since any polynomial $g$ in the ideal $I$ generated by these
polynomials $f_1,\ldots,f_N$ also vanishes on $X$, we might better proclaim
an affine variety to be the common zeroes of an {\sl ideal} of 
$k[x_1,\ldots,x_n]$.
However, the Hilbert Basis Theorem tells us that ideals of 
$k[x_1,\ldots,x_n]$ are finitely generated, so every ideal 
$I$ corresponds to some affine variety, which we denote by 
${\mathcal V}(I)$.

If we consider the ideal of polynomials vanishing on some subset $X$ of
${\mathbb A}^n$
$$
  {\mathcal I}(X)\ :=\ 
         \{ f\in k[x_1,\ldots,x_n]\mid f|_X\equiv 0\}\,,
$$
then we have the inclusions
$$
  X\ \subset\ {\mathcal V}({\mathcal I}(X))
  \qquad\mbox{and}\qquad
  I\ \subset\ {\mathcal I}({\mathcal V}(I))
$$
but neither need be an equality.

To refine the second inclusion, recall that the radical $\sqrt{I}$ of an
ideal $I$ is $\{f\in k[x]\mid \exists n f^n\in I\}$.
An ideal is {\it radical} if it equals its own radical.
One of the foundational results in algebraic geometry is Hilbert's
Nullstellensatz.\medskip

\noindent{\bf Theorem (Hilbert's Nullstellensatz).}
{\it Suppose ${\mathbb K}$ is algebraically closed.
If $I$ is any ideal of ${\mathbb K}[x_1,\ldots,x_n]$, then 
$\sqrt{I}={\mathcal I}({\mathcal V}(I))$.}
\medskip

This leads to a one-to-one correspondence between affine algebraic
subvarieties of ${\mathbb A}^n_{\mathbb K}$ and radical ideals in 
${\mathbb K}[x_1,\ldots,x_n]$.
When $k$ is not algebraically closed, the situation is more complicated.
In this case (and this is at the heart of the revolution in the foundations
in algebraic geometry in the last century),
we use algebra (or the case when $k$ is algebraically closed) to guide our
geometric definitions. 
This is appropriate, as it is important in algebraic geometry and
particularly in algebraic groups to consider not only the zeroes $X(k)$ of an
ideal $I\subset k[x_1,\ldots,x_n]$ in ${\mathbb A}^n_k$ but also its
superset $X({\mathbb K})\subset {\mathbb A}^n_{\mathbb K}$, which is the set
of zeroes on $I$ in ${\mathbb A}^n_{\mathbb K}$, for any field extension
${\mathbb K}$ of $k$.
\medskip

We define the {\it Zariski topology} on ${\mathbb A}^n$ by declaring the
affine varieties to be closed.
The axioms for a topology are satisfied:
$\emptyset$ and ${\mathbb A}^n$ are both closed. Finite unions of closed
sets are closed, as they are the zero sets of the intersection of the
corresponding ideals.
Finally, arbitrary intersections of varieties correspond to the ideal
generated by an arbitrary collection of ideals, and hence are closed.

The Zariski topology is quite strange:
The closed subsets of ${\mathbb A}^1$ are $\emptyset$, ${\mathbb A}^1$, and
all finite subsets of ${\mathbb A}^1$.
Thus the usual separation property (any 2 points can be covered by disjoint
open sets) of Hausdorff spaces fails 
spectacularly.
On the other hand, the Hilbert Basis Theorem gives the ascending chain
condition for ideals, and thus the descending chain
condition on closed sets, which implies ACC for open sets, that is, 
the Zariski topology is  compact.
When $k={\mathbb C}$, this is certainly much weaker than the traditional
topology. 
(Weirder, too: $S^1$ and ${\mathbb R}$ are Zariski-dense in ${\mathbb C}$!)

On the other hand, is it a useful notion, and it is very handy to have some
topology when $k$ is infinite with positive characteristic.\smallskip

The Zariski topology is also interesting for products of algebraic
varieties. 
Suppose $X\subset{\mathbb A}^n$ and $Y\subset{\mathbb A}^m$ are algebraic
varieties with ideals $I\subset k[x_1,\ldots,x_n]$ and 
$J\subset k[y_1,\ldots,y_m]$.
Then $X\times Y\subset {\mathbb A}^{m+n}$ is defined by the ideal of
$k[x_1,\ldots,x_n,y_1,\ldots,y_m]$ generated by $I+J$.
Observe that the Zariski topology on $X\times Y$ is {\it not} in general the
product topology.
In the product topology on ${\mathbb A}^2$, the closed sets are finite
collections of points, sets of the form $\{x\}\times{\mathbb A}$ and 
${\mathbb A}\times\{x\}$, and ${\mathbb A}^2$ itself.
On the other hand, ${\mathbb A}^2$ has a rich collection of 1-dimensional
subvarieties (called {\sl curves}), including the diagonal 
$\{(x,x)\mid x\in{\mathbb A}\}={\mathcal V}(x_1-x_2)$, which is not in the
previous list.
\medskip

Any reasonable mathematical structure has a notion of maps, and in algebraic
geometry the reasonable maps are those given by polynomials.
Let $I\subset k[x_1,\ldots,x_n]$ be a radical ideal, and 
$X:={\mathcal V}(I)$ be its associated variety.
Elements of $k[x_1,\ldots,x_n]/I$ are called {\it regular functions} on $X$
and we write $k[X]$ for the ring of regular functions, called the 
{\it coordinate ring} of $X$.
When $k$ is algebraically closed, a regular function is determined by its
values on $X$, which justifies our terminology.
On the other hand, this is not true when $k$ is finite.
We see that $k[X]\otimes_k k[Y]$ is the coordinate ring of 
$X\times Y$.
Observe that a $k$-algebra $R$ is the coordinate ring of an algebraic
variety if and only if it is finitely generated and {\sl reduced} 
(has no nilpotent elements).

An $m$-tuple $(f_1,\ldots,f_m)$ of regular functions in $X$ defines a 
{\it regular map} from $X$ to ${\mathbb A}^m$.
$$
  \varphi\colon  x\in X\ \longmapsto\ (f_1(x),\ldots,f_m(x))\,.
$$
The map $\varphi$ induces a map 
$\varphi^*\colon k[y_1,\ldots, y_m]\to k[X]$
(in fact it is equivalent to such a map) defined by
$g\mapsto g\circ\varphi=g(f_1,\ldots,f_m)$.
Let $Y\subset {\mathbb A}^n$ be an algebraic subvariety with radical ideal 
$J$.
The image of $\varphi$ lies in $Y$ when $\varphi^*(J)=0$ in $k[X]$.
Thus we obtain a map $\varphi^*\colon k[Y]\to k[X]$.

This algebraic formulation of regular maps shows they are continuous in the
Zariski topology.
We say that two varieties $X$ and $Y$ are {\it isomorphic} if there is a
regular map $\varphi\colon X\to Y$ for which $\varphi^*$ is an isomorphism.
A {\sl principal open subset} $Z\subset X$ is an open subset of the form 
$X-{\mathcal V}(f)$ for some regular function $f\in k[X]$.
Principal open subsets are themselves affine algebraic varieties.
The map $Z\ni z\mapsto (z,f(z)^{-1})\in{\mathbb A}^{n+1}$
is a homeomorphism of $Z$ with its image, and we see that the coordinate
ring of $Z$ is $k[X][y]/(yf-1)$.
\medskip

\noindent{\bf Algebraic-Geometric Dictionary.}
{\it 
Let ${\mathbb K}$ be algebraically closed.
The association of an algebraic variety to its coordinate ring, 
and a regular map of algebraic varieties to the corresponding map of
coordinate rings is a contravariant equivalence of categories
$$
  \left\{\begin{minipage}[c]{1.5in}
   Algebraic ${\mathbb K}$-varieties with regular ${\mathbb K}$-maps
  \end{minipage}\right\}
   \quad \longrightarrow \quad
  \left\{\begin{minipage}[c]{2.35in}
   Finitely generated reduced ${\mathbb K}$-algebras with 
   ${\mathbb K}$-homomorphisms
  \end{minipage}\right\}\,.
$$
}

$$
\left[\ \begin{minipage}[c]{5.65in}
    Modern algebraic geometry extends this to all fields by adding
    structures to affine varieties so that the category of affine
    $k$-varieties is contravariantly equivalent to the category of finitely
    generated reduced $k$-algebras. 
    We even go one step further and define {\sl schemes}, and the category
    of affine $k$-schemes is contravariantly equivalent to the category of
    finitely generated $k$-algebras. 
 \end{minipage}\ \right]
$$



We are now ready to make the most important definition in this course.
\medskip

\noindent{\bf Definition.}
An {\it affine algebraic group} $G$ defined over $k$ is an affine
$k$-variety whose multiplication maps $\mu\colon G\times G\to G$ and inverse 
map $\iota\colon G\to G$ are regular $k$-maps, and whose identity element 
$e\in G(k)$ is a (closed) point of $G$.\medskip

\noindent{\bf Examples.}
The {\sl additive group} ${\mathbb G}_a$ is ${\mathbb A}^1$ with group
operation addition ($\mu(x,y)=x+y$, $\iota(x)=-x$, and $e=0$).
In a similar fashion, any affine space is an algebraic group under addition.
The {\sl multiplicative group} ${\mathbb G}_a$ is the principal open subset  
${\mathbb A}-\{0\}$ with group operation multiplication ($\mu(x,y)=xy$,
$\iota(x)=x^{-1}$, and $e=1$). 

The set $\mbox{\rm Mat}_{p\times m}(k)$ of $p\times m$ matrices under
addition is identified with ${\mathbb G}_a^{pm}$.
More interesting is the {\it general linear group} $GL(n,k)$, which is the
set of invertible $n\times n$ matrices with entries in $k$.
We shall see that all affine algebraic groups are closed subgroups of some
general linear group.
(This explains why they are sometimes called linear algebraic groups.)
For example, the {\sl special linear group} $SL(n,k)$ is the subgroup of
matrices with determinant 1.
Some other important subgroups of $GL(n,k)$ are the subgroup $B$ of upper
triangular matrices, and its subgroups $U$ of upper triangular matrices with
1's on the diagonal, and $T$ of diagonal matrices.
We have $T\simeq {\mathbb G}_m^n$.
Another affine algebraic subgroup is 
${\mathcal S}_n$, the collection of $n\times n$
permutation matrices. 
($x=(x_{ij})$ with $x_{ij}^2=x_{ij}$ and $^tx x = I_n$.)

The special linear group is the first of the 4 infinite families of
classical groups.
The others are:
The {\sl symplectic group} $Sp(2n,k)$ is the subgroup of 
$GL(2n,k)$ consisting of those matrices $x$ such that 
$$
  ^tx\left(\begin{array}{cc}0&J_n\\-J_n&0\end{array}\right)x\ =\ 
  \left(\begin{array}{cc}0&J_n\\-J_n&0\end{array}\right)\,,
$$
where $J_n$ is the $n\times n$-matrix whose only entries are 1's on its
anti-diagonal and $^tx$ is the transpose of the matrix $x$.
The (split) orthogonal group $O(2n+1,k)$ is the subgroup of
$GL(2n+1,k)$ consisting of those matrices $x$ which satisfy 
$^txJ_{2n+1}x=J_{2n+1}$ and the other {\sl orthogonal group}
$O(2n,k)$ is the subgroup of $GL(2n,k)$ consisting of those matrices $x$
satisfying $^txJ_{2n}x=J_{2n}$.
These definitions of the orthogonal groups are valid only when the
characteristic of $k$ is not 2, and even though the definitions are very
similar, the resulting odd orthogonal and even orthogonal groups are quite
different. 
The special orthogonal group $SO(m,k)$ is the subgroup of $O(m,k)$
consisting of matrices of determinant 1.
\bigskip

Before we commence a finer study of affine algebraic groups, we remark on an 
interesting connection to algebra through the algebraic-geometric
dictionary.
Let $G$ be an affine algebraic group and consider its coordinate ring
$k[G]$.
The multiplication map $\mu:G\times G\to G$ gives rise to ring homomorphism
$\mu^*\colon k[G]\to k[G]\otimes k[G]$, the inverse to a ring homomorphism
$\iota^*\colon k[G]\to k[G]$, and the identity $e\in G$ to map 
$p^*\colon k[G]\to k[G]/{\mathcal I}(e)\simeq k$.
These maps satisfy certain properties coming from the group axioms for
$(G,\mu,\iota,e)$.
Since they are also algebra homomorphisms, we see that $k[G]$ becomes a 
{\sl Hopf algebra}.

Conversely, if a reduced $k$-algebra has the structure of a Hopf algebra,
then its associated affine algebraic variety is an algebraic group, with the  
group structures induced from the Hopf-algebra structures.
Thus there is an equivalence of categories between reduced commutative Hopf
algebras over a field $k$ and affine algebraic groups.
When the Hopf algebra is not necessarily reduced, then the associated object
is called a {\sl group scheme}.
Finite groups are algebraic groups; the dual of their group algebra is their
coordinate ring.

\end{document}



