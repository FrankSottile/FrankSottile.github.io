%notes08.tex
%
% Frank Sottile
% 25 March 2000
% Madison, WI
%
\documentclass[12pt]{amsart}
\usepackage{amssymb}
\usepackage{epsf}

\headheight=8pt     \topmargin=-10pt
\textheight=644pt   \textwidth=432pt
\oddsidemargin=18pt \evensidemargin=18pt

\def\silentfootnote#1{{\let\thefootnote\relax\footnotetext{#1}}}

\newcommand{\G}{{\bf G}}
\newcommand{\lra}{\relbar\joinrel\relbar\joinrel\longrightarrow}
%\def\baselinestretch{1.2}

\begin{document}
\begin{center}
\Large
The Geometry of Algebraic Groups\\
\large
Math 841\\
Frank Sottile\\
{\tt notes08}
\end{center}\bigskip

\silentfootnote{\sl Version of 9 April 2000} 

\section{Lie Algebras}


The set Mat$_{n\times n}(k)$ of $n\times n$ matrices with entries in $k$ 
has a bilinear anti-symmetric {\sl Bracket} operation $[x,y]:=xy -yx$
defined for $x,y\in\mbox{\rm Mat}_{n\times n}(k)$.
We call the set of $n\times n$ matrices with this operation 
${\mathfrak{g}\mathfrak{l}}_nk$.
In general, a {\sl Lie algebra} is a linear subspace of an associative
algebra closed under the bracket.
\medskip


\noindent{\bf The Lie algebra of an affine algebraic group.}
Let $G$ be an affine algebraic group.
Recall that $G$ acts on $k[G]$ by left translation:
for $x,y\in G$ and $f\in k[G]$, we have $\lambda_x.f(y)=f(x^{-1}y)$.
More generally, this map is the composite:
$$
   \lambda_x\ \colon\ 
    k[G] \stackrel{\Delta}{\longrightarrow} k[G]\otimes k[G]
    \stackrel{1\otimes ev_x}{\lra}
    k[G]\otimes (k[G]/{\mathfrak m}_x)\ \simeq\ k[G]\,,
$$
where $\Delta$ is the coproduct map $\Delta$ and $ev_x$ is the canonical map
$k[G]\twoheadrightarrow k[G]/{\mathfrak m}_k\simeq k$.


A {\sl derivation} $\delta$ on $k[G]$ is a $k$-linear map
$\delta\colon k[K]\to k[G]$ such that 
$$
  \delta(fg)\ =\ \delta(f)g - f \delta(g)\,,
$$
for all $f,g\in k[G]$.
The set $\mbox{Der}(k[G])$ of derivations on $k[G]$ form a Lie algebra:
\begin{eqnarray*}
[\delta,\gamma](fg)&=& \delta\gamma(fg)-\gamma\delta(fg)\\
&=& \delta(\gamma(f)g+f\gamma(g))-\gamma(\delta(f)g+f\delta(g))\\
&=& \delta\gamma(f)g + \gamma(f)\delta(g) + \delta(f)\gamma(g)
    + f\delta\gamma(g)\\
&& - \gamma\delta(f)g - \delta(f)\gamma(g) - \gamma(f)\delta(g) 
    - f\gamma\delta(g)\\
&=& (\delta\gamma(f)-\gamma\delta(f))g +
    f(\delta\gamma(g)-\gamma\delta(g))\\
&=& [\delta,\gamma](f) g - f [\delta,\gamma](g)\,.
\end{eqnarray*}

The Lie algebra ${\mathcal L}(G)$ of $G$  is the set 
$$
   {\mathcal L}(G)\ :=\ 
        \{\delta\in \mbox{Der}(k[G])\mid \delta \lambda_x=\lambda_x\delta
        \  \forall x\in G\}\,
$$
of left-invariant derivations on $k[G]$.
The behavior of this Lie algebra under maps between algebraic groups is not
evident from this definition.
The set of derivations of $k[G]$ is not obviously functorial under
homomorphisms.
We shall see below that the Lie algebra of $G$ is functorial; the basic idea
is that a left-invariant derivation is determined by its action at the
identity, thus we may `push this action at the identity forward' along a
homomorphism, and then `spread it out' using the left-invariance.

To this end, we define a map $\theta: {\mathcal L}(G)\to (k[G])^*$, the
space of linear maps on $k[G]$, to be the composition
$$
  \theta v\ \colon\ k[G]\stackrel{v}{\longrightarrow} k[G]
             \stackrel{ev_e}{\longrightarrow} k\,,
$$
for $v\in{\mathcal L}(G)$.
That is, for $f\in k[G]$, we have $\theta v (f)=(vf)(e)\in k$.
Since $v$ is a derivation, for $f,g\in k[G]$, we have
%
\begin{eqnarray*}
  \theta v(fg)&=& (vf\cdot g)(e) + (f\cdot vg)(e)\\
              &=& (vf)(e)\cdot g(e) + f(e)\cdot (vg)(e)\\
              &=& \theta v(f)\cdot g(e) + f(e)\cdot (\theta v)(g)
                  \makebox[.01in][l]{\hspace{1in}$(*)$}
\end{eqnarray*}
%
Since $v$ is a derivation, we have $v(k)=0$ and so 
$v(f)=v(f-f(e))$.
Thus by $(*)$, if $h\in ({\mathfrak m}_e)^2$, then $\theta v(h)=0$.
It follows that $\theta v$ is completely determined by its action as
a linear map on ${\mathfrak m}_e/{\mathfrak m}_e^2$,
and so we have that $\theta v$ is an element of the tangent space to $G$ 
at the identity.

To that end, we define ${\mathfrak g}:=\theta_{G,e}$, the tangent space to
$G$ at the identity.
For $v\in {\mathfrak g}$, we may consider $v$ to be a linear map on $k[G]$ by
$v(f)=v(f-f(e))$. 
Then, for $f,g\in k[G]$, we have
$$
  0\ =\ v((f-f(e))(g-g(e))\ =\ 
         v(fg)-v(f)\cdot g(e) - f(e)\cdot v(g)\,,
$$
and so the map $v$ satisfies $(*)$.
(It is common to call a map satisfying $(*)$ a {\sl point derivation} at the
maximal ideal ${\mathfrak m}_e$.)
\medskip


\noindent{\bf Theorem. }
{\it 
The $k$-linear map $\theta\colon{\mathcal L}(G)\to {\mathfrak g}$
is an isomorphism and the bracket on ${\mathfrak g}$ is preserved under
group homomorphisms.}\medskip

\begin{proof}
We assume that $k$ is algebraically closed, so that we may show that regular
functions agree by showing that they define the same function on the points
of $G$.
Define a $k$-linear map 
$\eta\colon{\mathfrak g}\to \mbox{End}(k[G])$ as follows.
For $v\in {\mathfrak g}$  and $f\in k[G]$, let $\eta(v)(f)$ be the function
whose value at $x\in G$ is $v(\lambda_{x^{-1}}f)$.
That is $(\eta(v)(f))(x)= v(\lambda_{x^{-1}}f)$.

We observe that $\eta(v)$ is a derivation.
Let $f,g\in k[G]$ and $x\in G$.
Then
\begin{eqnarray*}
\eta(v)(fg)(x)&=& v(\lambda_{x^{-1}}f\cdot g)\ =\ 
         v(\lambda_{x^{-1}}f \cdot\lambda_{x^{-1}}g)\\
   &=& v(\lambda_{x^{-1}}f)\cdot \lambda_{x^{-1}}g(e) +
       \lambda_{x^{-1}}f(e)\cdot v(\lambda_{x^{-1}}g)\\
    &=& (\eta(v)f\cdot g + f\cdot \eta(v)g) (x)
\end{eqnarray*}
Now we show that $\eta(v)$ is left-invariant.
Let $f\in k[G]$ and $x,y\in G$.
\begin{eqnarray*}
[\lambda_y(\eta(v)f)](x)&=&
           (\eta(v)f)(y^{-1}x) \ =\ v(\lambda_{x^{-1}y}f)\\
    &=& v(\lambda_{x^{-1}}(\lambda_yf))\ =\ 
       [\eta(v)(\lambda_y f)](x)
\end{eqnarray*}

These maps $\theta$ and $\eta$ are inverses.
Let $\delta\in{\mathcal L}(G)$, $v\in{\mathfrak g}$, $f\in k[G]$, and 
$x\in G$.
\begin{eqnarray*}
 [\eta(\theta \delta).f](x) 
   &=& \theta\delta(\lambda_{x^{-1}}f)\\
   &=& [\delta(\lambda_{x^{-1}}f)](e)\\
   &=& [\lambda_{x^{-1}}(\delta f)](e)\ =\ \delta f(x)\,,
\end{eqnarray*}
so $\eta(\theta\delta).f=\delta.f$.
And
\begin{eqnarray*}
  [\theta(\eta v)]f  &=& [\eta(v).f](e)\\
   &=& v(\lambda_{e^{-1}}f)\ =\ v.f\,.
\end{eqnarray*}


We now show that the barcket is preserved under group homomorphisms.
Let $\varphi\colon G\to H$ be a homomorphism of agebraic groups.
Then the map $d\varphi$ on tangent spaces
$d\varphi\colon{\mathfrak g}\to{\mathfrak h}$ is $k$-linear.
We show that $d\varphi[w,v]=[d\varphi w,d\varphi v]$, for
$v,w\in{\mathfrak g}$.
Let $x=d\varphi w$ and $y=d\varphi v$, and $f\in k[H]$.
Then
\begin{eqnarray*}
[x,y].f&=& 
  (\eta(x).\eta(y).f)e - (\eta(y).\eta(x).f)e\\
  &=& x(\eta(y).f) - y(\eta(x).f)\\
  &=& w(\varphi^*(\eta(y).f)-v(\varphi^*(\eta(x).f))\,.
\end{eqnarray*}
We also compute
\begin{eqnarray*}
(d\varphi[w,v]).f&=& [w,v].\varphi^*f\\
  &=& (\eta(w).\eta(v)(\varphi^*f))e-(\eta(v).\eta(w)(\varphi^*f))e\\
  &=& w(\eta(v)\varphi^*f)-v(\eta(w)\varphi^*f)\,.
\end{eqnarray*}
To show these are equal, we show that 
$\varphi^*(\eta(d\varphi v)f)=\eta(v)\varphi^*f$ for $v\in{\mathfrak g}$ and
$f\in k[H]$. 
We show both sides give the same function on $G$, as both are in 
$k[G]$.
We first see that
$$
  (\eta(v)\varphi^*f)g\ =\ v(\lambda_{g^{-1}}\varphi^*f)\,.
$$
And also 
$$
  [\varphi^*(\eta(d\varphi v)f)](g)\ =\ (\eta(d\varphi v)f)(\varphi(g)\ =\ 
  d\varphi v(\lambda_{\varphi(g)^{-1}}f)\ =\ 
  v(\varphi^*(\lambda_{\varphi(g)^{-1}}f))\,.
$$

We show that 
$\lambda_{g^{-1}}\varphi^*f=\varphi^*(\lambda_{\varphi(g)^{-1}}f)$, as
regular functions in $k[G]$, which will complete the proof.
Let $h\in G$.
Then
$$
  (\lambda_{g^{-1}}\varphi^*f)h\ =\ \varphi^*f(gh)\ =\ f(\varphi(gh)\ =\ 
  f(\varphi(g)\varphi(h))\,,
$$
and
$$
  (\varphi^*(\lambda_{\varphi(g)^{-1}}f))h\ =\ 
  \lambda_{\varphi(g)^{-1}}f(\varphi(h))\ =\ 
  f(\varphi(g)\varphi(h))\,.
$$
\end{proof}


\section{Chevalley's Theorem}
A first step towards proving Chevalley's Theorem that if $H$ is a closed
subgroup of a linear algebraic group, then $G/H$ is naturally a
quasi-projective variety












\end{document}
