%HW11.tex
%Eleventh Homework -- Math 629 
%
%  The percent sign is a comment character
%
%%%%%%%%%%%%%%%%%%%%%%%%%%%%%%%%%%%%%%%%%%%%%%%%%%%%%%%%%%%%%%%%%%%%%%%%%%%%%%%%%%
%
%   Look these up on line.  The first sets the type of document, and the next are for mathematics symbols, graphics and color
%
\documentclass[12pt]{article}
\usepackage{amssymb,amsmath}
\usepackage{graphicx}
\usepackage[usenames,dvipsnames,svgnames,table]{xcolor}
\usepackage{multirow}   % This is for more control over tables
%%%%%%%%%%%%%%%%%%%%%%%%%%%%%%%%  Layout     %%%%%%%%%%%%%%%%%%%%%%%%%%%%%%%%%%%%%%
\usepackage{vmargin}
\setpapersize{USletter}
\setmargrb{2cm}{1cm}{2cm}{1cm} % --- sets all four margins LTRB


%%%%%%%%%%%%%%%%%%%%%%%%%%%%%%%%%%%%%%%%%%%%%%%%%%%%%%%%%%%%%%%%%%%%%%%%%%%%%%%%%
\begin{document}
\LARGE 
\noindent
{\color{Maroon}History of Mathematics \hfill Math 629}\vspace{2pt}\\
\large
Tenth Homework: \hfill 2 April 2022\\
Due Monday 11 April 2022.
\normalsize\vspace{10pt}

To hand in: We are using Gradescope for homework submission.


\begin{enumerate}

\item  {[10]}
     Exercise 17.7.1 from Stillwell (Section 17.6).
\item  {[15]}
  Exercise 17.7.2 from Stillwell (Section 17.6).
  Include a proof that the sum of the interior angles of any simple polygon with $n$ sides is $\pi(n-2)$ (or $(n-2)\cdot 180^\circ$,
  if you prefer degrees).

\item  {[10]}
  Exercise 18.2.1 from Stillwell.
  
\item{[10]} A paragraph on Gauss.

\item  {[25]}
     Describe what you did with lines on the hyperolic soccerball.  (About 2 paragraphs.)
\item  {[25]}
       What did you learn from this activity?  (About 2 paragraphs.)
\item {[5]}
  (Not for Gradescope. Ignore Question 6 there.).  Send triangle data and photos to Frank in a serial post in Piazza.
  He will credit/grade your Question 6 in Gradescope, as he gets your triangle data.


  \end{enumerate}
\end{document}
%%%%%%%%%%%%%%%%%%%%%%%%%%%%%%%%%%%%%%%%%%%%%%%%%%%%%%%%%%%%%%%%%%%%%%%%%%%%%%%%%
