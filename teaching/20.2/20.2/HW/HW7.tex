%HW7.tex
%Sixth Homework -- Math 300 
%
%  The percent sign is a comment character
%
%%%%%%%%%%%%%%%%%%%%%%%%%%%%%%%%%%%%%%%%%%%%%%%%%%%%%%%%%%%%%%%%%%%%%%%%%%%%%%%%%%
%
%   Look these up on line.  The first sets the type of document, and the next are for mathematics symbols, graphics and color
%
\documentclass[12pt]{article}
\usepackage{amssymb,amsmath}
\usepackage{graphicx}
\usepackage[usenames,dvipsnames,svgnames,table]{xcolor}
\usepackage{multirow}   % This is for more control over tables
%%%%%%%%%%%%%%%%%%%%%%%%%%%%%%%%  Layout     %%%%%%%%%%%%%%%%%%%%%%%%%%%%%%%%%%%%%%
\usepackage{vmargin}
\setpapersize{USletter}
\setmargrb{1cm}{.1cm}{2cm}{1cm} % --- sets all four margins LTRB


\newcommand{\defn}[1]{{\color{blue}\sl #1}}
\newcommand{\deco}[1]{{\color{blue} #1}}

\newcommand{\calP}{{\mathcal P}}

\newcommand{\NN}{{\mathbb N}}
\newcommand{\QQ}{{\mathbb Q}}
\newcommand{\RR}{{\mathbb R}}
\newcommand{\ZZ}{{\mathbb Z}}

%%%%%%%%%%%%%%%%%%%%%%%%%%%%%%%%%%%%%%%%%%%%%%%%%%%%%%%%%%%%%%%%%%%%%%%%%%%%%%%%%
\begin{document}
\LARGE 
\noindent
{\color{Maroon}Foundations of Mathematics \hfill Math 300 Sections 902, 905}\vspace{2pt}\\
\Large YOUR NAME\vspace{2pt}\\
\large
Seventh Homework: \hfill Due 19 October 2020
\normalsize    %%%% This returns the fontsize to 12 point
%\vspace{10pt}  %%%% This makes a vertical space of 10 points (approx 70 points/inch) after the line break

\noindent{\color{blue}\rule{529pt}{2pt}}


\noindent {\color{Maroon}\bf Definition:}  Let $A$ be a set.  The \defn{power set of $A$}, \deco{$\calP(A)$} is the set whose elements are
exactly all of the subsets of $A$.

\noindent {\color{Maroon}\bf Recall:} The product of two real numbers is positive if and only if either both numbers are positive or both
numbers are negative.

Also,  the product of two real numbers is negative if and only if one number is positive and one number is negative.

  
%%%%%%%%%%%%%%%%%%%%%%%%%%%%%%%%%%%%%%%%%%%%%%%%%%%%%%%%%%%%%%%%%%%%%%%%%%%%%%%%%
\begin{enumerate}


%\newpage
%%%%%%%%%%%%%%%%%%%%%%%%%%%%%%%%%%%%%%%%%%%%%%%%%%%%%%%%%%%%%%%%%%%%%%%%%%%%%%%%%
\item{[15]} Do all parts of Problem 17 in the Exercises for Section 5.2 in the Sundstrom book.
%%%%%%%%%%%%%%%%%%%%%%%%%%%%%%%%%%%%%%%%%%%%%%%%%%%%%%%%%%%%%%%%%%%%%%%%%%%%%%%%%

%\newpage
%%%%%%%%%%%%%%%%%%%%%%%%%%%%%%%%%%%%%%%%%%%%%%%%%%%%%%%%%%%%%%%%%%%%%%%%%%%%%%%%%
\item{[5]} Write the defining property of the power set of a set $A$ as a logical statment, using quantifiers and logical operators.
%%%%%%%%%%%%%%%%%%%%%%%%%%%%%%%%%%%%%%%%%%%%%%%%%%%%%%%%%%%%%%%%%%%%%%%%%%%%%%%%%

%\newpage
%%%%%%%%%%%%%%%%%%%%%%%%%%%%%%%%%%%%%%%%%%%%%%%%%%%%%%%%%%%%%%%%%%%%%%%%%%%%%%%%%
\item{[12]} Let $A=\{\emptyset, \spadesuit, \Psi\}$.  Determine which of the following are true or false. (no proof needed)
  
  \begin{itemize}
  \item[(a)] \makebox[100pt][l]{$\spadesuit\subseteq \calP(A)$}
        (e)  \makebox[100pt][l]{$\emptyset\subseteq \calP(A)$}
        (i)  \makebox[100pt][l]{$\{\emptyset,\{\spadesuit\}\}\subseteq \calP(A)$}
  \item[(b)] \makebox[100pt][l]{$\Psi\in \calP(A)$}
        (f)  \makebox[100pt][l]{$\emptyset\in \calP(A)$}
        (j)  \makebox[100pt][l]{$\{\emptyset,\{\spadesuit\}\}\in \calP(A)$}
  \item[(c)] \makebox[100pt][l]{$\{\Psi\}\subseteq \calP(A)$}
        (g)  \makebox[100pt][l]{$\{\emptyset\}\subseteq \calP(A)$}
        (k)  \makebox[100pt][l]{$A\subseteq \calP(A)$}
  \item[(d)] \makebox[100pt][l]{$\{\spadesuit\}\in \calP(A)$}
        (h)  \makebox[100pt][l]{$\{\emptyset\} \in \calP(A)$}
        (l)  \makebox[100pt][l]{$A\in \calP(A)$}
 \end{itemize}

%%%%%%%%%%%%%%%%%%%%%%%%%%%%%%%%%%%%%%%%%%%%%%%%%%%%%%%%%%%%%%%%%%%%%%%%%%%%%%%%%
 
%\newpage
%%%%%%%%%%%%%%%%%%%%%%%%%%%%%%%%%%%%%%%%%%%%%%%%%%%%%%%%%%%%%%%%%%%%%%%%%%%%%%%%%
\item{[5]}  Write a very clean proof of the following statement:

  ``For all sets $A$, $B$, and $C$, if $A\subseteq B$ and $B\subseteq C$, then $A\subseteq C$.''

%\newpage
%%%%%%%%%%%%%%%%%%%%%%%%%%%%%%%%%%%%%%%%%%%%%%%%%%%%%%%%%%%%%%%%%%%%%%%%%%%%%%%%%
\item{[8]} Suppose that $S:=\{n\in\ZZ\mid n\equiv 9 \mod 6\}$ and $T:=\{n\in\ZZ\mid n\equiv 3 \mod 12\}$.
          Prove whichever of $S\subseteq T$, $T\subseteq S$ is true, or give counterexamples.
%%%%%%%%%%%%%%%%%%%%%%%%%%%%%%%%%%%%%%%%%%%%%%%%%%%%%%%%%%%%%%%%%%%%%%%%%%%%%%%%%

%\newpage
%%%%%%%%%%%%%%%%%%%%%%%%%%%%%%%%%%%%%%%%%%%%%%%%%%%%%%%%%%%%%%%%%%%%%%%%%%%%%%%%%
\item{[8]}  Prove the following set equality.
  \[
     \{ x\in\RR\mid x^2-3x-10<0\}\ =\ \{x\in\RR\mid -2<x<5\}\,.
  \]
%%%%%%%%%%%%%%%%%%%%%%%%%%%%%%%%%%%%%%%%%%%%%%%%%%%%%%%%%%%%%%%%%%%%%%%%%%%%%%%%%

%\newpage
%%%%%%%%%%%%%%%%%%%%%%%%%%%%%%%%%%%%%%%%%%%%%%%%%%%%%%%%%%%%%%%%%%%%%%%%%%%%%%%%%
\item{[8]}  Prove the following set equality.
  \[
     \{ x\in\RR\mid x^2\geq 4\}\ =\ \{x\in\RR\mid x\leq -2 \} \bigcup \{x\in\RR\mid x\geq 2\}\,.
  \]
%%%%%%%%%%%%%%%%%%%%%%%%%%%%%%%%%%%%%%%%%%%%%%%%%%%%%%%%%%%%%%%%%%%%%%%%%%%%%%%%%
 
%\newpage
%%%%%%%%%%%%%%%%%%%%%%%%%%%%%%%%%%%%%%%%%%%%%%%%%%%%%%%%%%%%%%%%%%%%%%%%%%%%%%%%%
\item{[8]} Let  $U$ be some universal set.
  Investigate the two sets $A-(B-C)$ and $(A-B)-C$.  Are they the same? different? Is one a subset of the other?

  Make a conjecture about their relation, and prove it.
%%%%%%%%%%%%%%%%%%%%%%%%%%%%%%%%%%%%%%%%%%%%%%%%%%%%%%%%%%%%%%%%%%%%%%%%%%%%%%%%%

%\newpage
%%%%%%%%%%%%%%%%%%%%%%%%%%%%%%%%%%%%%%%%%%%%%%%%%%%%%%%%%%%%%%%%%%%%%%%%%%%%%%%%%
\item{[8]}  Let $A$ and $B$ be subsets of some univesal set $U$.
           Prove De Morgan's Law: $(A\cap B)^c= A^c\cup B^c$.

%%%%%%%%%%%%%%%%%%%%%%%%%%%%%%%%%%%%%%%%%%%%%%%%%%%%%%%%%%%%%%%%%%%%%%%%%%%%%%%%%
 
%\newpage
%%%%%%%%%%%%%%%%%%%%%%%%%%%%%%%%%%%%%%%%%%%%%%%%%%%%%%%%%%%%%%%%%%%%%%%%%%%%%%%%%
\item{[12]}  Let $A$ and $B$ be subsets of some univesal set $U$.
  Give two, independent proofs of the set identity
  \[
  (A\cup B) - (A\cap B)\ =\ (A-B) \cup (B-A)\,.
  \]

%%%%%%%%%%%%%%%%%%%%%%%%%%%%%%%%%%%%%%%%%%%%%%%%%%%%%%%%%%%%%%%%%%%%%%%%%%%%%%%%%

%\newpage
%%%%%%%%%%%%%%%%%%%%%%%%%%%%%%%%%%%%%%%%%%%%%%%%%%%%%%%%%%%%%%%%%%%%%%%%%%%%%%%%%
\item{[11]}   Let $A$ and $B$ be sets.
  For $a\in A$ and $b\in B$, consider the set: $\{\{a\},\{a,b\}\}$.
  What is this set if $a=b$ ?
  
  Prove:  For all $a,c\in A$ and $b,d\in B$, we have 
  $\{\{a\},\{a,b\}\} = \{\{c\},\{c,d\}\}$ if and only if
  $a=c$ and $b=d$.
%%%%%%%%%%%%%%%%%%%%%%%%%%%%%%%%%%%%%%%%%%%%%%%%%%%%%%%%%%%%%%%%%%%%%%%%%%%%%%%%%
 

\end{enumerate}
%%%%%%%%%%%%%%%%%%%%%%%%%%%%%%%%%%%%%%%%%%%%%%%%%%%%%%%%%%%%%%%%%%%%%%%%%%%%%%%%%


\end{document}
%%%%%%%%%%%%%%%%%%%%%%%%%%%%%%%%%%%%%%%%%%%%%%%%%%%%%%%%%%%%%%%%%%%%%%%%%%%%%%%%%


