%HW06.tex
%Sixth Homework -- Math 629 
%
%  The percent sign is a comment character
%
%%%%%%%%%%%%%%%%%%%%%%%%%%%%%%%%%%%%%%%%%%%%%%%%%%%%%%%%%%%%%%%%%%%%%%%%%%%%%%%%%%
%
%   Look these up on line.  The first sets the type of document, and the next are for mathematics symbols, graphics and color
%
\documentclass[12pt]{article}
\usepackage{amssymb,amsmath}
\usepackage{graphicx}
\usepackage[usenames,dvipsnames,svgnames,table]{xcolor}
\usepackage{multirow}   % This is for more control over tables
%%%%%%%%%%%%%%%%%%%%%%%%%%%%%%%%  Layout     %%%%%%%%%%%%%%%%%%%%%%%%%%%%%%%%%%%%%%
\usepackage{vmargin}
\setpapersize{USletter}
\setmargrb{2cm}{1cm}{2cm}{1cm} % --- sets all four margins LTRB


%%%%%%%%%%%%%%%%%%%%%%%%%%%%%%%%%%%%%%%%%%%%%%%%%%%%%%%%%%%%%%%%%%%%%%%%%%%%%%%%%
\begin{document}
\LARGE 
\noindent
{\color{Maroon}History of Mathematics \hfill Math 629}\vspace{2pt}\\
\large
Sixth Homework: \hfill 22 February 2022\\
Due Monday 28 February 2022.
\normalsize\vspace{10pt}

To hand in: We are using Gradescope for homework submission.


\begin{enumerate}

\item  {[10]}
     Exercise 7.4.1 from Stillwell.

\item  {[10]}
     Exercise 7.4.2 from Stillwell.

\item  {[10]}
  Use the same method to find a parametrization for $y^2=x^2(x-1)$.
  What is different about the third "curve" ? (Hint: draw the curve and consider the origin) 
  

\item  {[10]}
     Exercise 7.5.1 from Stillwell.  I think that both conics can be parabolas, but they will need to open in different directions.

\item  {[10]}
     Exercise 7.5.2 from Stillwell.   (Same comment as the previous problem.)
 
\item  {[10]}
     Exercise 9.5.3 from Stillwell.

\item  {[10]}
     Exercise 9.5.4 from Stillwell.

\item  {[10]}
     Exercise 9.6.1 from Stillwell.

\item  {[10]}
     Exercise 9.6.2 from Stillwell.

\item  {[10]}
     Exercise 9.6.3 from Stillwell.  Doing this trio, you will one-up Leibniz!


     


\end{enumerate}

\end{document}
%%%%%%%%%%%%%%%%%%%%%%%%%%%%%%%%%%%%%%%%%%%%%%%%%%%%%%%%%%%%%%%%%%%%%%%%%%%%%%%%%
