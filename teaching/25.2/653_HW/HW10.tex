%HW10.tex
%
% Tenth Homework for Graduate Algebra
% Frank Sottile
%%%%%%%%%%%%%%%%%%%%%%%%%%%%%%%%%%%%%%%%%%%%%%%%%%%%%%%%%%%%%%%%%%%%%%%
\documentclass[12pt]{article}
\usepackage{multicol,amsfonts, amssymb,  mathtools,amsmath}
\usepackage{colordvi,graphicx}
\headheight=8pt
%
\topmargin=-75pt
\textheight=720pt   \textwidth=560pt
\oddsidemargin=-60pt \evensidemargin=-60pt

\pagestyle{empty}

%%%%%%%%%%%%%%%%%%%%%%%%%%%%%%%%%%%%%%%%%%%%
\newcommand{\CC}{{\mathbb C}}
\newcommand{\KK}{{\mathbb K}}
\newcommand{\NN}{{\mathbb N}}
\newcommand{\QQ}{{\mathbb Q}}
\newcommand{\RR}{{\mathbb R}}
\newcommand{\TT}{{\mathbb T}}
\newcommand{\ZZ}{{\mathbb Z}}

\newcommand{\calA}{{\mathcal A}}
\newcommand{\be}{{\bf e}}
\newcommand{\bfi}{{\bf i}}
\newcommand{\bfj}{{\bf j}}

\newcommand{\End}{\mbox{End}}
\newcommand{\Hom}{\mbox{Hom}}
\newcommand{\spec}{\mbox{spec}}
\newcommand{\cone}{\mbox{cone}}

\newcommand{\vect}[2]{(\begin{smallmatrix}#1\\#2\end{smallmatrix})}
\newcommand{\msp}{\hspace{8pt}}
%\newcommand{\Square}{\raisebox{-2pt}{\includegraphics{images/Square.eps}}}
\newcommand{\Square}{\raisebox{-2pt}{\Large$\square$}}

\def\Color#1#2{\special{color push cmyk #1}#2\special{color pop}}
%\def\Indigo#1{\Color{.42 1. 0. .49}{#1}}
\def\Indigo#1{\Color{1. .95 .05 .4}{#1}}
\def\MyViolet#1{\Color{.6 1. 0. .15}{#1}}
\def\TAMU#1{\Color{.15 1. .39 .69}{#1}}

\newcommand{\barsl}{\noindent\begin{minipage}[t]{590pt}
\Indigo{\rule{590pt}{1.2pt}}\vspace{-5.7mm}\\
\MyViolet{\rule{590pt}{1.2pt}}\vspace{-5.7mm}\\
\Blue{\rule{590pt}{1.2pt}}\vspace{-5.7mm}\\
\Green{\rule{590pt}{1.2pt}}\vspace{-5.7mm}\\
\Yellow{\rule{590pt}{1.2pt}}\vspace{-5.7mm}\\
\Orange{\rule{590pt}{1.2pt}}\vspace{-5.7mm}\\
\Red{\rule{590pt}{1.2pt}}\bigskip
\end{minipage}}


\newcommand{\barsn}{\noindent\begin{minipage}[t]{590pt}
\Indigo{\rule{590pt}{1.1pt}}\vspace{-4.5mm}\\
\MyViolet{\rule{590pt}{1.1pt}}\vspace{-4.5mm}\\
\Blue{\rule{590pt}{1.1pt}}\vspace{-4.5mm}\\
\Green{\rule{590pt}{1.1pt}}\vspace{-4.5mm}\\
\Yellow{\rule{590pt}{1.1pt}}\vspace{-4.5mm}\\
\Orange{\rule{590pt}{1.1pt}}\vspace{-4.5mm}\\
\Red{\rule{590pt}{1.1pt}}\bigskip
\end{minipage}}

\def\demph#1{\TAMU{{\sl #1}}}
\def\defcolor#1{\TAMU{#1}}

\begin{document}
\LARGE 
\noindent
Algebra \ \ Autumn 2025\vspace{1pt}\\
Frank Sottile\vspace{2pt}\\
\Large 6 November 2025 \hfill
\sf
 Tenth Homework\makebox[20pt][l]{\ }
\normalsize\vspace{10pt}

\noindent
Write your answers neatly, in complete sentences.  
I highly recommend recopying your work before handing it in.
Correct and crisp proofs are greatly appreciated; oftentimes your work can be shortened and made clearer.

\barsn

\noindent\Maroon{{\sf Hand in at the start of class, Thursday 13 November:}} 

\begin{enumerate}
\setcounter{enumi}{58}

%%%%%%%%%%%%%%%%%%%%%%%%%%%%%%%%%%%%%%%%%%%%%%%%%%%%%%%%%%%%%%%%%%%%%%%%%%%%%%%%%%%%%%%%%%%%%%%%%%%%
\item  For each posiitve integer $n$, let $R_n$ be a ring and suppose that  for $0<m<n$ we have ring homomorphsms
  $\varphi_{n,m}\colon R_n \twoheadrightarrow R_m$ that form an inverse system of groups (if $0<\ell<m<n$, then
  $\varphi_{n,\ell}=\varphi_{m,\ell}\circ\varphi_{n,m}$).
  
  Show that the inverse limit ${\displaystyle \lim_{\leftarrow} R_n}$ of this inverse system of groups
  has the structure of a ring.
%%%%%%%%%%%%%%%%%%%%%%%%%%%%%%%%%%%%%%%%%%%%%%%%%%%%%%%%%%%%%%%%%%%%%%%%%%%%%%%%%%%%%%%%%%%%%%%%%%%%


%%%%%%%%%%%%%%%%%%%%%%%%%%%%%%%%%%%%%%%%%%%%%%%%%%%%%%%%%%%%%%%%%%%%%%%%%%%%%%%%%%%%%%%%%%%%%%%%%%%%
\item  Prove that products exist in the category of rings.
%%%%%%%%%%%%%%%%%%%%%%%%%%%%%%%%%%%%%%%%%%%%%%%%%%%%%%%%%%%%%%%%%%%%%%%%%%%%%%%%%%%%%%%%%%%%%%%%%%%%

%%%%%%%%%%%%%%%%%%%%%%%%%%%%%%%%%%%%%%%%%%%%%%%%%%%%%%%%%%%%%%%%%%%%%%%%%%%%%%%%%%%%%%%%%%%%%%%%%%%%
%
\item
      An element $x$ in a ring $R$ is \demph{nilpotent} if there is a positive integer $n$
      with $x^n=0$.
      Prove that the set $\eta(R)$ of nilpotent elements of a \Magenta{{\sl commutative}}
      ring $R$ forms an ideal, called the \demph{nilradical} of $R$.
      Show that $\eta( R/\eta(R))=\{0\}$.
%%%%%%%%%%%%%%%%%%%%%%%%%%%%%%%%%%%%%%%%%%%%%%%%%%%%%%%%%%%%%%%%%%%%%%%%%%%%%%%%%%%%%%%%%%%%%%%%%%%%  

%%%%%%%%%%%%%%%%%%%%%%%%%%%%%%%%%%%%%%%%%%%%%%%%%%%%%%%%%%%%%%%%%%%%%%%%%%%%%%%%%%%%%%%%%%%%%%%%%%%%  
\item 
  Consider the ring $\End(\ZZ\oplus\ZZ)$ of endomorphisms of the free abelian group $\ZZ\oplus\ZZ$.
  Prove that $\End(\ZZ\oplus\ZZ)$ is noncommutative.
%%%%%%%%%%%%%%%%%%%%%%%%%%%%%%%%%%%%%%%%%%%%%%%%%%%%%%%%%%%%%%%%%%%%%%%%%%%%%%%%%%%%%%%%%%%%%%%%%%%%  

%%%%%%%%%%%%%%%%%%%%%%%%%%%%%%%%%%%%%%%%%%%%%%%%%%%%%%%%%%%%%%%%%%%%%%%%%%%%%%%%%%%%%%%%%%%%%%%%%%%%
%
\item
      Let $R$ be a ring, $I$ an ideal of $R$, and $n\geq 1$ an integer.
      Define $M_n(I)$ to be the set of $n\times n$ matrices with entries in  $I$.

     (1) Prove that for any ideal $I$ of $R$, $M_n(I)$ is an ideal of $M_n(R)$.

     (2) Prove that any ideal of $M_n(R)$ has the form $M_n(I)$ for $I$ an ideal of $R$.
%%%%%%%%%%%%%%%%%%%%%%%%%%%%%%%%%%%%%%%%%%%%%%%%%%%%%%%%%%%%%%%%%%%%%%%%%%%%%%%%%%%%%%%%%%%%%%%%%%%%  


%%%%%%%%%%%%%%%%%%%%%%%%%%%%%%%%%%%%%%%%%%%%%%%%%%%%%%%%%%%%%%%%%%%%%%%%%%%%%%%%%%%%%%%%%%%%%%%%%%%%
%
\item 
       Prove that a ring $R$ is a division ring if and only if it has no proper left ideals.
%%%%%%%%%%%%%%%%%%%%%%%%%%%%%%%%%%%%%%%%%%%%%%%%%%%%%%%%%%%%%%%%%%%%%%%%%%%%%%%%%%%%%%%%%%%%%%%%%%%%  


%%%%%%%%%%%%%%%%%%%%%%%%%%%%%%%%%%%%%%%%%%%%%%%%%%%%%%%%%%%%%%%%%%%%%%%%%%%%%%%%%%%%%%%%%%%%%%%%%%%%
%
\item 
      Determine all prime and maximal ideals in the ring $\ZZ_m$ of integers modulo a positive  integer $m$.
%%%%%%%%%%%%%%%%%%%%%%%%%%%%%%%%%%%%%%%%%%%%%%%%%%%%%%%%%%%%%%%%%%%%%%%%%%%%%%%%%%%%%%%%%%%%%%%%%%%%  



%%%%%%%%%%%%%%%%%%%%%%%%%%%%%%%%%%%%%%%%%%%%%%%%%%%%%%%%%%%%%%%%%%%%%%%%%%%%%%%%%%%%%%%%%%%%%%%%%%%%
%
\item 
      Let $S$ be a \demph{subset} of a ring $R$.
       Show that the intersection of all ideals of $R$ that contain $S$ is the set
\[
   \Bigl\{ \sum_{i=1}^n r_i s_i t_i \mid  r_1,t_1,\dotsc,r_n,t_n\in R
         \quad s_1,\dotsc,s_n\in S\quad n\in\NN\Bigr\}\,.\vspace{-3pt}
\]
%%%%%%%%%%%%%%%%%%%%%%%%%%%%%%%%%%%%%%%%%%%%%%%%%%%%%%%%%%%%%%%%%%%%%%%%%%%%%%%%%%%%%%%%%%%%%%%%%%%%  



      
\end{enumerate}
%%%%%%%%%%%%%%%%%%%%%%%%%%%%%%%%%%%%%%%%%%%%%%%%%%%%%%%%%%%%%%%%%%%%%%%%%%%%%%%%%%%%%%%%%%%%%%%%%%%%

\end{document}
