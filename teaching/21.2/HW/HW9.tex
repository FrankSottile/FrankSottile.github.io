%HW9.tex
%Ninth Homework -- Math 300H 
%
%  The percent sign is a comment character
%
%%%%%%%%%%%%%%%%%%%%%%%%%%%%%%%%%%%%%%%%%%%%%%%%%%%%%%%%%%%%%%%%%%%%%%%%%%%%%%%%%%
%
%   Look these up on line.  The first sets the type of document, and the next are for mathematics symbols, graphics and color
%
\documentclass[12pt]{article}
\usepackage{amssymb,amsmath,mathtools}
\usepackage{graphicx}
\usepackage[usenames,dvipsnames,svgnames,table]{xcolor}
\usepackage{multirow}   % This is for more control over tables
%%%%%%%%%%%%%%%%%%%%%%%%%%%%%%%%  Layout     %%%%%%%%%%%%%%%%%%%%%%%%%%%%%%%%%%%%%%
\usepackage{vmargin}
\setpapersize{USletter}
\setmargrb{1cm}{1cm}{2cm}{1cm} % --- sets all four margins LTRB


\newcommand{\RR}{{\mathbb R}}  % This is the backboard symbol for the real numbers.  Note how it is used below
\newcommand{\NN}{{\mathbb N}}  % 
\newcommand{\QQ}{{\mathbb Q}}  % 
\newcommand{\ZZ}{{\mathbb Z}}  %

\newcommand{\calP}{{\mathcal P}}  %Caligraphic P for power set

\newcommand{\sep}{{\ :\ }}     % This is for the : in our notation for building sets.
\newcommand{\lsim}{\mathord{\sim}}  %  This is to remove the extra spacing around \sim (it is a binary relation) for the logical not

\newcommand{\defcolor}[1]{{\color{blue}{#1}}}
\newcommand{\demph}[1]{{\color{blue}\sl{#1}}}
%%%%%%%%%%%%%%%%%%%%%%%%%%%%%%%%%%%%%%%%%%%%%%%%%%%%%%%%%%%%%%%%%%%%%%%%%%%%%%%%%
%
%   Please edit this as appropriate
%
\begin{document}
\LARGE 
\noindent
{\color{Maroon}Foundations of Mathematics \hfill Math 300H Section 970}\vspace{2pt}\\
\Large \vspace{2pt}\\
\large
Ninth Homework: \hfill Due \  8 November 2021
\normalsize\medskip


\noindent{\color{blue}\rule{528.3675pt}{2pt}}


%%%%%%%%%%%%%%%%%%%%%%%%%%%%%%%%%%%%%%%%%%%%%%%%%%%%%%%%%%%%%%%%%%%%%%%%%%%%%%%%%
\begin{enumerate}  %  the \begin{..}  \end{..} stars and ends an environment.

   
%%%%%%%%%%%%%%%%%%%%%%%%%%%%%%%%%%%%%%%%%%%%%%%%%%%%%%%%%%%%%%%%%%%%%%%%%%%%%%%%%
\item Let $R$ be an equivalence relation on $A = \{a, b, c, d, e, f , g\}$ such that $a R c$, $b R f$,  $c R d$, and $d R g$.
    If there are three distinct equivalence classes resulting from $R$, then determine these equivalence classes and
    determine all elements of $R$.
%%%%%%%%%%%%%%%%%%%%%%%%%%%%%%%%%%%%%%%%%%%%%%%%%%%%%%%%%%%%%%%%%%%%%%%%%%%%%%%%%

%%%%%%%%%%%%%%%%%%%%%%%%%%%%%%%%%%%%%%%%%%%%%%%%%%%%%%%%%%%%%%%%%%%%%%%%%%%%%%%%%
\item  Let $H$ be a nonempty subset of $\ZZ$.
  Suppose that the relation $R$ defined on $\ZZ$ by $a R b$ if $a - b\in  H$ is an equivalence relation.
  Verify the following
  \begin{enumerate}
    \item $0\in H$.
    \item If $a\in H$, then $-a\in H$.
    \item If $a,b\in H$, then $a+b\in H$.
  \end{enumerate}
%%%%%%%%%%%%%%%%%%%%%%%%%%%%%%%%%%%%%%%%%%%%%%%%%%%%%%%%%%%%%%%%%%%%%%%%%%%%%%%%%

%%%%%%%%%%%%%%%%%%%%%%%%%%%%%%%%%%%%%%%%%%%%%%%%%%%%%%%%%%%%%%%%%%%%%%%%%%%%%%%%%
\item
  (a) Let $R$ be the relation defined on $\ZZ$ by $a R b$ if $a+b$ is even.
       Show that $R$ is an equivalence relation and determine the distinct equivalence classes.

 (b) Suppose that ``even'' is replaced by ``odd'' in (a).
       Which of the properties reflexive, symmetric, and/or transitive, does $R$ possess?

%%%%%%%%%%%%%%%%%%%%%%%%%%%%%%%%%%%%%%%%%%%%%%%%%%%%%%%%%%%%%%%%%%%%%%%%%%%%%%%%%

%%%%%%%%%%%%%%%%%%%%%%%%%%%%%%%%%%%%%%%%%%%%%%%%%%%%%%%%%%%%%%%%%%%%%%%%%%%%%%%%%
\item Let $R$ be a relation defined on the set $\NN$ by $a R b$ if either $a \mid 2b$ or $b \mid 2a$.
       Prove or disprove: $R$ is an equivalence relation.
%%%%%%%%%%%%%%%%%%%%%%%%%%%%%%%%%%%%%%%%%%%%%%%%%%%%%%%%%%%%%%%%%%%%%%%%%%%%%%%%%


 
%\newpage
%%%%%%%%%%%%%%%%%%%%%%%%%%%%%%%%%%%%%%%%%%%%%%%%%%%%%%%%%%%%%%%%%%%%%%%%%%%%%%%%%%
%  \item  Determine all the congruence classes (equivalence classes) for the relation on the integers $\ZZ$ of congruence modulo 5.
%%%%%%%%%%%%%%%%%%%%%%%%%%%%%%%%%%%%%%%%%%%%%%%%%%%%%%%%%%%%%%%%%%%%%%%%%%%%%%%%%
  
%%%%%%%%%%%%%%%%%%%%%%%%%%%%%%%%%%%%%%%%%%%%%%%%%%%%%%%%%%%%%%%%%%%%%%%%%%%%%%%%%
\item The relation $R$ on $\ZZ$ defined by $a R b$ if $a^4 \equiv b^4 \mod 8$ is known to be an equivalence relation.
  Determine the distinct equivalence classes.
%%%%%%%%%%%%%%%%%%%%%%%%%%%%%%%%%%%%%%%%%%%%%%%%%%%%%%%%%%%%%%%%%%%%%%%%%%%%%%%%%

 %%%%%%%%%%%%%%%%%%%%%%%%%%%%%%%%%%%%%%%%%%%%%%%%%%%%%%%%%%%%%%%%%%%%%%%%%%%%%%%%%
\item In $\ZZ_{11}$, express the following sums and products as $[r]$, where $0\leq r < 11$.
  
  (a) $[7] + [5]$ \quad
  (b) $[7]\cdot [5]$ \quad
  (c) $[−82] + [207]$ \quad
  (d) $[−82] \cdot [207]$\,.
%%%%%%%%%%%%%%%%%%%%%%%%%%%%%%%%%%%%%%%%%%%%%%%%%%%%%%%%%%%%%%%%%%%%%%%%%%%%%%%%%
  
%\newpage
%%%%%%%%%%%%%%%%%%%%%%%%%%%%%%%%%%%%%%%%%%%%%%%%%%%%%%%%%%%%%%%%%%%%%%%%%%%%%%%%%
\item Compute the addition and multiplication tables for $\ZZ_5$.
%%%%%%%%%%%%%%%%%%%%%%%%%%%%%%%%%%%%%%%%%%%%%%%%%%%%%%%%%%%%%%%%%%%%%%%%%%%%%%%%%
  

%%%%%%%%%%%%%%%%%%%%%%%%%%%%%%%%%%%%%%%%%%%%%%%%%%%%%%%%%%%%%%%%%%%%%%%%%%%%%%%%%
\item Prove that the multiplication in $\ZZ_m$, for $m\geq 2$, defined by $[a]_m\cdot[b]_m=[a\cdot b]_m$ is well-defined.
          (See Result 4.11 in our book.)
%%%%%%%%%%%%%%%%%%%%%%%%%%%%%%%%%%%%%%%%%%%%%%%%%%%%%%%%%%%%%%%%%%%%%%%%%%%%%%%%%

    

  
\end{enumerate}
%%%%%%%%%%%%%%%%%%%%%%%%%%%%%%%%%%%%%%%%%%%%%%%%%%%%%%%%%%%%%%%%%%%%%%%%%%%%%%%%%%%%%%%%%%%%%%%%%%%%

\end{document}

  

%%%%%%%%%%%%%%%%%%%%%%%%%%%%%%%%%%%%%%%%%%%%%%%%%%%%%%%%%%%%%%%%%%%%%%%%%%%%%%%%%
\item \ 
%%%%%%%%%%%%%%%%%%%%%%%%%%%%%%%%%%%%%%%%%%%%%%%%%%%%%%%%%%%%%%%%%%%%%%%%%%%%%%%%%
