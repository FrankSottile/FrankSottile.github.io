%HW3.tex
%Second Homework -- Math 629 
%
%  The percent sign is a comment character
%
%%%%%%%%%%%%%%%%%%%%%%%%%%%%%%%%%%%%%%%%%%%%%%%%%%%%%%%%%%%%%%%%%%%%%%%%%%%%%%%%%%
%
%   Look these up on line.  The first sets the type of document, and the next are for mathematics symbols, graphics and color
%
\documentclass[12pt]{article}
\usepackage{amssymb,amsmath}
\usepackage{graphicx}
\usepackage[usenames,dvipsnames,svgnames,table]{xcolor}
\usepackage{multirow}   % This is for more control over tables
%%%%%%%%%%%%%%%%%%%%%%%%%%%%%%%%  Layout     %%%%%%%%%%%%%%%%%%%%%%%%%%%%%%%%%%%%%%
\usepackage{vmargin}
\setpapersize{USletter}
\setmargrb{2cm}{1cm}{2cm}{1cm} % --- sets all four margins LTRB


%%%%%%%%%%%%%%%%%%%%%%%%%%%%%%%%%%%%%%%%%%%%%%%%%%%%%%%%%%%%%%%%%%%%%%%%%%%%%%%%%
\begin{document}
\LARGE 
\noindent
{\color{Maroon}History of Mathematics \hfill Math 629}\vspace{2pt}\\
\large
Third Homework: \hfill 1 February 2022\\
Due Monday 7 February 2022.
\normalsize\vspace{10pt}

      Some of these problems will require you to find other material than just what is in the readings.
To hand in: We are using Gradescope for homework submission.


\begin{enumerate}

\item    Give the defining property of a perfect number. Prove that if $2^n-1$ is a prime number, $p$, then $2^{n-1}p$ is a perfect number.
  Write down
  four perfect numbers.
  How many perfect numbers are there? 


\item   Continued fractions.  Do the exercises in Stillwell: 3.4.1, 3.4.2, 3.4.3, and 3.4.4.


\item    Consider Archimedes' quote from The Method: "It is of course easier to supply the proof when we have previously acquired some
  knowledge of the questions by the method, than it is to find it without any previous knowledge." What was "the method" he is referring to?
  What does his quote say about the role of experimentation or studying examples in Mathematics?  


\item    Exhaustion.
  Do the exercises in Stillwell's Section 4.4 (4.4.1, 4.4.2, and 4.4.3) to prove the formula for the logarithm of product of rational
  numbers. Make sure to use exhaustion and not calculus tricks.
  
  Hint: From the definition of exhaustion, to show two quantities are equal, say $\log(a)$ and $\log(ab)-\log(b)$,
  is to show how any lower approximation to one can be transformed into a lower approximation for the other, with the same area, and the
  same for upper approximations; I find the correct method here to be quite elegant.
  (Note that, as a mathematician, I use $\log$ for the logarithm with respect to the natural base, Euler's
  number $e$.)  
       
\end{enumerate}

\end{document}
%%%%%%%%%%%%%%%%%%%%%%%%%%%%%%%%%%%%%%%%%%%%%%%%%%%%%%%%%%%%%%%%%%%%%%%%%%%%%%%%%
