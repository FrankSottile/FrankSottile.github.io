%notes04.tex
%
% Frank Sottile
% 8 February 2000
% Madison, WI
%
\documentclass[12pt]{amsart}
\usepackage{amssymb}
\usepackage{epsf}

\headheight=8pt     \topmargin=0pt
\textheight=624pt   \textwidth=432pt
\oddsidemargin=18pt \evensidemargin=18pt

\def\silentfootnote#1{{\let\thefootnote\relax\footnotetext{#1}}}

%\def\baselinestretch{1.2}

\begin{document}
\begin{center}
\Large
The Geometry of Algebraic Groups\\
\large
Math 841\\
Frank Sottile\\
{\tt notes04}
\end{center}\bigskip

\silentfootnote{\sl Version of 18 February 2000} 

\section{Tangent spaces to affine varieties}

Given a polynomial $f\in k[t_1,\ldots,t_n]$ and a point 
$x=(x_1,\ldots,x_n)\in{\mathbb A}^n$, we can write the polynomial in terms
of the differences $\xi_i:=(x_i-t_i)$, and obtain the Taylor expansion 
of $f$ at the point $x$ as follows:
$$
   f\ =\  f(x) + 
   \sum_{i=1}^n \frac{\partial f}{\partial t_i}(t_i-x_i) + \cdots\ \ ,
$$
where the remaining terms are homogeneous with degree greater than 1 in
the differences $\xi_i$.
The linear term above is a linear map from $k^n\to k$ called the
differential of $f$ at the point $x$ and denoted $d_xf$
(we use $\xi$ as coordinates for $k^n$).
This linear map is independent of choice of coordinates.\smallskip

\noindent{\bf Definition. }
Let $X\subset {\mathbb A}^n$ be a subvariety with $I$ its (radical) ideal.
The {\sl Zariski tangent space} $\theta_{X,x}$ to $X$ at a point $x\in X$
is the kernel of the linear maps $\{d_xf\mid f\in I\}$.
Since 
$$
  d_x(f+g)\ =\ d_xf + d_xg 
  \quad\mbox{ and }\quad
  d_x(fg)\ =\ f(x)d_xg + g(x)d_xf\,,
$$ 
we see that only a finite generating set for $I$ is needed to define
$\theta_{X,x}$ at points $x\in X$.
\medskip

\noindent{\bf Example. }
We give a simple example that illustrates many of the notions we have
discussed so far.
Set $X:={\mathcal V}(y^2-x^3)\subset{\mathbb A}^2$.
We display a real picture of $X$ below:
$$
  \epsfxsize 1in\epsfbox{cuspidal.eps}
$$
Consider the map $\varphi:{\mathbb A}^1\to{\mathbb A}^2$ given by 
$\varphi(t)=(t^2,t^3)$.
Then we have
$$
  \varphi^*\ :\ x\longmapsto t^2 \ \mbox{\rm and }\ 
                y\longmapsto t^3\,.
$$
It is easy to see that the kernel of $\varphi^*$ is the principal ideal
generated by $y^2-x^3$.
We know this is a prime ideal (providing a proof the $y^2-x^3$ is
irreducible), as its image is a subalgebra of $k[t]$,
and hence an integral domain.
In fact, ${\rm im}\varphi^*= k[t^2,t^3]$.

We conclude that $X$ is the image of the map $\varphi$, and that the
coordinate ring of $X$ is $k[t^2,t^3]$.
This also shows that $X$ is an irreducible algebraic variety.
Also, as $k[t^2,t^3]\not\simeq k[t]=k[{\mathbb A}^1]$, the map $\varphi$,
despite being bijective on the points of ${\mathbb A}^1$ and $X$, is not an
isomorphism---the obvious reverse map is cannot be given by regular
functions on $X$.

Observe that ${\mathbb G}_m$ acts on $X$ via
$$
   (\alpha,(x,y))\ \longmapsto\ 
   (\alpha^2x,\alpha^3x)\ =:\ \alpha.(x,y)\,.
$$
In fact, using the ${\mathbb G}_m$ action on ${\mathbb A}^1$ by
multiplication, we see that the map $\varphi$ is 
{\sl ${\mathbb G}_m$-equivariant} in that 
$\varphi(\alpha.t)=\alpha.\varphi(t)$ for $\alpha\in{\mathbb G}_m$ and 
$t\in{\mathbb A}^1$.
This action of ${\mathbb G}_m$ on $X$ extends to all of ${\mathbb A}^2$, and
so $X$ is linearized.
If we identify the points of ${\mathbb A}^2$ with $k^2$, then this is a
linear action of ${\mathbb G}_m$ via diagonal matrices:
$$
   {\mathbb G}_m\ni\alpha\ \longmapsto\ 
   \left(\begin{array}{cc}\alpha^2&0\\0&\alpha^3\end{array}\right)
$$
In fact, this realizes ${\mathbb G}_m$ as the subgroup of diagonal matrices
$diag(x,y)$ such that $x^3=y^2$.

The group ${\mathbb G}_m$ acts on $X$ with only 2 orbits:
$$
  {\mathbb G}_m.(1,1)\ \mbox{ and }\ (0,0)\,.
$$
Consider the differential $df$ of the defining equation $f=y^2-x^3$ of $X$.
In vector notation, we have
$df=\langle -3x^2, 2y \rangle$.
This is a non-zero linear function if $(x,y)\neq (0,0)$, and this fact is
independent of the characteristic of $k$.
In particular, if $z\in X-\{(0,0)\}$, then $\theta_{X,z}\simeq k$ and 
we have $\theta_{X,(0,0)}=k^2$.
Thus the above orbit decomposition of $X$ is also its decomposition into
smooth and singular points.
\medskip


We return to our discussion of tangent spaces to varieties and 
make the definition of tangent spaces intrinsic so
that it becomes functorial under maps of algebraic varieties.
To that end, let $x\in X$ and consider a regular function $f\in I$, where
$I$ is the radical ideal of $X$.
If we define $d_xf=d_xF$, where $F\in k[{\mathbb A}^n]$ is any polynomial
lift of $f\in k[X]$, then $d_xf$ is only defined modulo the subspace
$\{d_xF\mid F\in I\}$ of differentials of polynomials vanishing on $X$.
Thus $d_xf$ is a well-defined linear map on the common kernel of the
differentials $\{d_xF\mid F\in I\}$, in other words, 
$d_xf$ is a linear map on the Zariski tangent space $\theta_{X,x}$ 
to $X$ at $x$.

Thus we have a  linear map $d_x\colon k[X]\to \theta_{X,x}^*$, the space of
linear maps on $\theta_{X,x}$.
Since $d_x \alpha=0$ for $\alpha\in k$ a constant,
$d_xf = d_x(f-f(x))$, and so it suffices to consider the restriction of
$d_x$ to the maximal ideal of $x$, 
${\mathfrak m}_x:=\{f\in k[X]\mid f(x)=0\}$.
\medskip

\noindent{\bf Theorem. }
{\it 
The map $d_x$ defines an isomorphism of the vector spaces 
${\mathfrak m}_x/{\mathfrak m}^2_x$ and $\theta_{X,x}^*$.
}\medskip

\begin{proof}
To see that $d_x$ is surjective, observe that 
the differential of any linear map is that linear map.
Since $\theta_{X,x}$ is a linear subspace of $k^n$, the restriction of
linear maps on $k^n$ gives all linear maps on $\theta_{X,x}$.
But this restriction factors through the composition of the
quotient map $k[{\mathbb A}^n]\to k[X]$ and the differential $d_x$.


Suppose now that $g\in{\mathfrak m}_x$ satisfies $d_x=0$.
Let $G$ be any polynomial lift of $g$.
Since the linear map $d_xG$ vanishes on $\theta_{X,x}$ it equals $d_xF$ for
some $F\in I$. 
If we set $H=G-F$, then $d_xH=0$ as a linear map on ${\mathbb A}^n$, and so 
its Taylor expansion has no constant or linear terms.
But this says that $H\in{\mathfrak m}_x^2$ in $k[{\mathbb A}^n]$.
Since $H$ is another polynomial representative of $G$, we conclude that
$g\in{\mathfrak m}_x^2$ in $k[X]$. 
(This is because the maximal ideal ${\mathfrak m}_x$ of $x\in X$ is
generated by polynomials in the maximal ideal ${\mathfrak m}_x$ of 
$x\in {\mathbb A}^n$.)
\end{proof}


Henceforth, we will  define the tangent space $\theta_{X,x}$ at a point $x$
of an algebraic variety $X$ to be $({\mathfrak m}_x/{\mathfrak m}_x^2)^*$,
where ${\mathfrak m}_x$ is the ideal in the coordinate ring of $X$ of
functions vanishing at $x$.
This intrinsic definition is functorial:
A map $\varphi\colon X\to Y$ induces maps 
$d_x\varphi\colon \theta_{X,x}\to \theta_{Y,\varphi(x)}$
and if we have $\psi\colon Y\to Z$ as well, then 
the composite map 
$\theta_{X,x}\to \theta_{Y,\varphi(x)}\to \theta_{Z,\psi(\varphi(x))}$ is
the composition of $d_x\varphi$ and $d_{f(x)}\psi$.

To see this, let $\varphi\colon X\to Y\subset{\mathbb A}^m$ be a map of
affine varieties given by the functions $f_1,\ldots,f_m\in k[X]$.
Then the map $\varphi^*\colon k[Y]\to k[X]$ is given by
$g\mapsto g(f_1,\ldots,f_m)$.
Let $x\in X$ and $g\in{\mathfrak m}_{\varphi(x)}$.
Then $\varphi^*g(x)=g(\varphi(x))=0$, and so we see that 
$\varphi^*\colon {\mathfrak m}_{\varphi(x)}\to{\mathfrak m}_x$, and also
$\varphi^*\colon {\mathfrak m}^2_{\varphi(x)}\to{\mathfrak m}^2_x$.
Thus the map
$$
  {\mathfrak m}_{\varphi(x)}\ \longrightarrow\ 
  {\mathfrak m}_x\ \relbar\joinrel\twoheadrightarrow\ 
  {\mathfrak m}_x/{\mathfrak m}^2_x
$$
factors through the quotient 
${\mathfrak m}_{\varphi(x)}/{\mathfrak m}^2_{\varphi(x)}$.
Dualizing, we obtain the map
$d_x\varphi\colon \theta_{X,x}\to\theta_{Y,\varphi(x)}$.
The functoriality of $d_x\varphi$ follows from the functoriality of the maps
$\varphi^*$. 

This functoriality has an immediate consequence:
If $\varphi\colon X\simeq Y$ is an isomorphism, then 
$d_x\varphi$ is an isomorphism for every $x\in X$.
In particular, we have the following.\medskip

\noindent{\bf Theorem. }(Equisingularity of orbits)
{\it
Let $X$ be an affine variety equipped with the action of a algebraic group
$G$.
Then for any $x\in X$, the tangent spaces at any two points of the orbit
$G.x$ are isomorphic.
In particular, all tangent spaces of $G$ are isomorphic.
}\medskip

\noindent{\bf Example. }
The functoriality of the differential map has an important application.
An algebraic group $G$ acts on itself by conjugation, and this action fixes
the identity element $e$ of $G$.
Thus $G$ acts through linear automorphisms on $\theta_{G,e}$, and this
representation is called the {\sl adjoint representation} of $G$.
\medskip

Unlike algebraic groups, general algebraic varieties will have
singularities.\medskip

\noindent{\bf Example. }
Consider the variety $X={\mathcal V}(xz-yw)\subset{\mathbb A}^4$.
The dimension of the Zariski tangent space at the origin is 4, but at every
other point of $X$ it is 3.
To see this, consider the differential of the defining equation of $X$.
In vector notation, this differential is
$\langle z, -w, x, -y\rangle$,
and this vanishes only at the origin.\medskip

The situation for general varieties is similar.
We first generalize the method of the example.
For $f\in k[{\mathbb A}^n]$, the differential $df$ of $f$
(the vector of its partial derivatives) is the element of
$\mbox{\rm Hom}(k^n,k[{\mathbb A}^n])$
$$
   df\ :\ k^n\ni \xi\ \longmapsto\ 
    \xi_1\frac{\partial f}{\partial t_1}+
    \xi_2\frac{\partial f}{\partial t_2}+ \cdots +
    \xi_n\frac{\partial f}{\partial t_n}\ \in\ 
    k[{\mathbb A}^n]\,.
$$


\noindent{\bf Theorem. }
{\it
Let $X$ be an affine variety.
Then there exists a (non-empty) open subset $X_{sm}$ of $X$ consisting of
the points of $X$ whose Zariski tangent space has minimal
dimension.}\medskip

\noindent{\sl Proof. }
Let $f_1,\ldots,f_s$ be generators of the radical ideal $I$ of $X$.
Let $M\in\mbox{Mat}_{s\times n}(k[{\mathbb A}^n])$ be the map from 
$k^n\to (k[{\mathbb A}^n])^s$ given by the $s$-tuple of their
differentials. 
For $x\in X$, we have $\theta_{X,x}\simeq \mbox{\rm kernel}(M(x))$.

For each number $l=1,2,\ldots,\min\{s,n\}$, define the degeneracy locus 
$\Delta_l\subset{\mathbb A}^n$ to be the zero locus of the collection of all
$l\times l$ minors of the matrix $M$, and set $\Delta_l={\mathbb A}^n$ if 
$l$ is greater than $\min\{s,n\}$.
Then we have
$$
  \Delta_1\ \subset\ \Delta_2\ \subset\ \cdots\ \subset 
  \Delta_{\min\{s,n\}}\ \subset\ {\mathbb A}^n\ 
   =\ \Delta_{1+\min\{s,n\}}\,.
$$
On the set $\Delta_{i+1}-\Delta_i$, the matrix $M(x)$ has constant rank $i$.
Thus if $x\in \Delta_{i+1}-\Delta_i$, the kernel of $M(x)$ has dimension
$n-i$. 
Let $i$ be minimal such that $X\subset \Delta_{i+1}$.
Then $X_{sm}:=X-\Delta_i=\{x\in X\mid \dim\theta_{X,x}=n-i\}$ is open in
$X$, and  $n-i$ is the minimum dimension of a tangent space to $X$.
\qed

When $X$ is irreducible, we call $X_{sm}$ the subset of smooth or
non-singular points of $X$, and the points of $X-X_{sm}$ are called singular
points. 
A variety is {\sl smooth} if $X=X_{sm}$.
By the previous Theorem, algebraic groups are smooth.
The dimension of the tangent space at smooth points is the dimension of the
variety $X$.\medskip

\noindent{\bf Theorem. }
{\it 
If $X$ is irreducible, then 
$\min_{x\in X}\dim_k(\theta_{X,x})=\dim X$.
}\medskip

This is proven in Shafarevich II.1.4.
\medskip

\noindent{\bf Theorem. }
{\it
Suppose $G$ is an affine algebraic group acting on a variety $X$.
Then each orbit of $G$ in $X$ is a smooth, locally closed subset of $X$
whose boundary is a union of orbits of strictly smaller dimension.}\medskip 

\noindent{\sl Proof. }
Let $y\in X$.
Set $Y=G.y$, the orbit of $y$ in $X$.
Since $Y$ is dense in $\overline{Y}$, it meets the open subset 
$(\overline{Y})_{sm}$ of smooth points.
By our Theorem on equisingularity of orbits, all tangent spaces
$\theta_{Y,y}$ have the same dimension, from which it follows
that all points of $Y$ are smooth points in $\overline{Y}$.

We have that $Y$ is the image of the map $G\to \overline{Y}$ given by 
$g\mapsto g.y$, and so $Y$ is constructible.
Let $U$ be an open subset of $\overline{Y}$ contained in $Y$.
Since $G.U=Y$, we see that $Y$ is a union of open subsets, and so is open in
$\overline{Y}$ and thus locally closed.

Finally, $Y$, and hence $\overline{Y}$ are $G$-stable, and so 
$\overline{Y}-Y$ is closed and $G$-stable, and has lower dimension than $Y$,
as $Y$ is dense in $\overline{Y}$.
Thus it is a union of $G$-orbits.
\qed


\end{document}
