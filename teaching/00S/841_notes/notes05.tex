%notes05.tex
%
% Frank Sottile
% 17 February 2000
% Madison, WI
%
\documentclass[12pt]{amsart}
\usepackage{amssymb}
\usepackage{epsf}

\headheight=8pt     \topmargin=-10pt
\textheight=644pt   \textwidth=432pt
\oddsidemargin=18pt \evensidemargin=18pt

\def\silentfootnote#1{{\let\thefootnote\relax\footnotetext{#1}}}

\newcommand{\G}{{\bf G}}

%\def\baselinestretch{1.2}

\begin{document}
\begin{center}
\Large
The Geometry of Algebraic Groups\\
\large
Math 841\\
Frank Sottile\\
{\tt notes05}
\end{center}\bigskip

\silentfootnote{\sl Version of 24 February 2000} 

\section{Projective Algebraic Varieties}
So far, we have been concerned with affine algebraic geometry.
In a sense, this is the local theory of algebraic geometry.
We will now begin studying (quasi-) projective varieties, which are glued
together from affine varieties in much the same way as manifolds are
obtained by gluing together balls in ${\mathbb R}^n$.

Many of the results we had about affine varieties carry over to projective
varieties.
Rather than enumerate them all, we will just use them, making a comment
about their extension at the point of use.
\smallskip

\noindent{\bf Definition. }
Projective $n$-space, ${\mathbb P}^n$, is the collection of all non-zero
vectors $(z_0,z_1,\ldots,z_n)\in k^{n+1}-\{0\}$, where we identify
a non-zero vector $(z_0,z_1,\ldots,z_n)$ with any non-zero scalar multiple 
$(\lambda z_0,\lambda z_1,\ldots,\lambda z_n)$ for 
$\lambda\in k^\times={\mathbb G}_m$.
Equivalently, ${\mathbb P}^n$ is the set of ${\mathbb G}_m$-orbits on
$k^{n+1}-\{0\}$.
Since any such orbit is the set of non-zero vectors in a one-dimensional
subspace of $k^{n+1}$, we also see that ${\mathbb P}^n$ is the collection of
one-dimensional subspaces of $k^{n+1}$.
In fact, we can define ${\mathbb P}(V)$ for any finite-dimensional
$k$-vector space.
\smallskip

Write $[z_0,z_1,\ldots,z_n]$ for the equivalence class of the non-zero
vector $(z_0,z_1,\ldots,z_n)$.
The components $z_i$ are called the {\it homogeneous coordinates} of
projective space.
Given a point $z=[z_0,z_1,\ldots,z_n]$ of projective $n$-space, let $p(z)$
be the smallest index $i$ such that $z_i\neq 0$.
Then $z=[0,\ldots,0,1,x_{i+1},\ldots,x_n]$, where we have $x_j=z_j/z_i$.
This gives a bijection of $\{z\in{\mathbb P}^n\mid p(z)=i\}$ with
${\mathbb A}^{n-i}$, and so we obtain the {\sl affine (or cellular)
decomposition} of projective space:
$$
  {\mathbb P}^n\ =\ {\mathbb A}^n\ {\textstyle \coprod}\ 
  {\mathbb A}^{n-1}\ {\textstyle \coprod}\ \cdots\ {\textstyle \coprod}\ 
  {\mathbb A}^1\ {\textstyle \coprod}\   {\mathbb A}^0\,.
$$

Projective $n$-space has many subsets in bijection with ${\mathbb A}^n$.
One economical list is provided by the {\sl principal affine pieces}
$U_i$ for $i=0,1,\ldots,n$.
Define
$$
  U_i\ :=\ \{[z_0,z_1,\ldots,z_n]\in{\mathbb P}^n\mid z_i\neq 0\}\,.
$$
Points $z=[z_0,z_1,\ldots,z_n]$ in $U_i$ have a canonical representative
$$
  \left[\frac{z_0}{z_i},\,\frac{z_1}{z_i},\,\ldots,\,
   \frac{z_{i-1}}{z_i},\,1,\,\frac{z_{i+1}}{z_i}
   ,\,\ldots,\,\frac{z_n}{z_i}\right]
$$
Thus we may identify $U_i$ with ${\mathbb A}^n$, in fact by
$$
  (x_1,x_2,\ldots,x_n)\ \longmapsto\ 
   [x_1,\ldots,x_i,1,x_{i+1},\ldots,x_n]\,.
$$
More generally, the set of points in ${\mathbb P}^n$ where a linear
form  does not vanish may be identified with affine $n$-space 
${\mathbb A}^n$.
\smallskip

\noindent{\bf Definition. }
The Zariski topology on ${\mathbb P}^n$ is the topology induced by requiring
that these bijections $U_i\simeq{\mathbb A}^n$ are homeomorphisms.
Thus $X\subset{\mathbb P}^n$ is closed if and only if $X\cap U_i$ is
closed for each $i=0,1,\ldots,n$.
A Zariski closed subset $X$ of ${\mathbb P}^n$ is called a 
{\sl projective (sub)variety}.
\smallskip

We consider the implications of this definition.
Identify $U_0$ with ${\mathbb A}^n$ as above, where the coordinate $x_i$ of
${\mathbb A}^n$ is the quotient $z_i/z_0$.
Let $f\in k[t_1,\ldots,t_n]$ be a polynomial function on ${\mathbb A}^n$.
For $z\in U_0$, $f(z)=f(z_1/z_0,\ldots,z_n/z_0)$ is not a polynomial
function of the homogeneous coordinates of $z$.
Multiplying by $z_0^{\deg f}$ clear denominators and 
then $g(z)=z_0^{\deg f}f(z_1/z_0,\ldots,z_n/z_0)$ is a polynomial
function of the homogeneous coordinates of $z$.
In fact $g$ is {\sl homogeneous} in that each of its terms have the same
total degree $\deg f$.

Suppose $X\subset {\mathbb P}^n$ is a subvariety.
Then, given a polynomial $f\in k[t_1,\ldots,t_n]$ vanishing on $X\cap U_0$,
the homogeneous polynomial $z_0^{1+\deg f}f(z_1/z_0,\ldots,z_n/z_0)$ 
vanishes on $X$, as every point in $X-(X\cap U_0)$ has vanishing 0th
homogeneous coordinate.
Repeating this for each polynomial $f$ vanishing on $X\cap U_0$, and then
for each of the principal affine pieces, we obtain a collection of
homogeneous polynomials $g\in k[s_0,\ldots,s_n]$ which vanish on $X$.
Since the principal affine pieces cover ${\mathbb P}^n$, the projective
variety $X$ is the common zeroes of a collection of homogeneous polynomials
in $k[s_0,s_1,\ldots,s_n]$.

As in affine space, the set of zeroes of a polynomial will be closed, but
there is a slight difference for projective space.
Let $z\in{\mathbb P}^n$ and $g\in k[s_0,s_1,\ldots,s_n]$ be a polynomial.
For each $j=0,\ldots,d=\deg g$, let $g_j$ be the terms in $g$ of total
degree $j$.
Then, for $\lambda\in k-\{0\}$, we have
$$
  g(\lambda z)\ =\ g_0(z) + \lambda g_1(z) 
               + \lambda^2 g_2(g) + \cdots + \lambda^d g_d(z)\,,
$$
a polynomial in $\lambda$.
Since $z=\lambda z$ as points in projective space, we see that $g$ has a
well-defined value at $z$ if and only if each homogeneous component $g_j$ of
$g$ for $j>0$ vanishes at the point $z$.


From this, we see that it suffices to only consider homogeneous polynomials,
and ideals generated by homogeneous polynomials
(called {\sl homogeneous ideals}) when defining closed subsets of projective
space.
Given a homogeneous ideal $I\subset k[s_0,s_1,\ldots,s_n]$, let 
${\mathcal V}(I)\subset{\mathbb P}^n$ be the set of common zeroes of
polynomials in $I$.
Recall that in affine algebraic geometry, we have Hilbert's Nullstellensatz,
which asserts that an ideal $I\subset k[t_1,\ldots,t_n]$ defines the empty
set in affine $n$-space ${\mathbb A}^n_{\mathbb K}$ over the algebraic
closure ${\mathbb K}$ of $k$ if  and only if $I=k[t_1,\ldots,t_n]$, that is 
$1\in I$.
The situation in projective space is more complicated.

Let ${\mathfrak m}_0$ be the homogeneous ideal of all non-constant
homogeneous polynomials, so that 
${\mathfrak m}_0=\langle s_0,s_1,\ldots,s_n\rangle$.
Then ${\mathfrak m}_0$ defines the empty set in ${\mathbb P}^n$, and so we
call it the {\sl irrelevant ideal}.
We see also that ${\mathfrak m}_0^l$ also defines the empty set for any
$l>0$.
\medskip

\noindent{\bf Theorem. }
{\it 
A homogeneous ideal $I\subset k[s_0,s_1,\ldots,s_n]$ defines the empty set
in ${\mathbb P}^n_{\mathbb K}$ if and only if 
there is some $l>0$ such that ${\mathfrak m}_0^l\subset I$.
}\medskip

\begin{proof}
For any homogeneous ideal $I\subset k[s_0,\ldots,s_n]$, the intersection
${\mathcal V}(I)\cap U_i$ is defined by dehomogenizing $I$:
$$
  I_i\ :=\ \{g(t_1,\ldots,t_{i-1},1,t_i,\ldots,t_n)\mid g\in I\}\,.
$$
If we have extend scalars to ${\mathbb K}$ and 
${\mathcal V}(I)\cap U_i=\emptyset$, then by Hilbert's Nullstellensatz
$1\in I_i$.
But this implies that some pure power of $z_i$, say $z_i^{a_i}$, lies in
$I$.
Let $a:=\max\{a_0,a_1,\ldots,a_n\}$ and set $l:=(a-1)(n+1)+1$.
Then it is easy to see that every monomial of degree $l$ is divisible by
some $z_i^{a_i}$, and so ${\mathfrak m}_0^l\subset I$.
The reverse implication was already shown, and it holds even over non-closed
fields.
\end{proof}


\noindent{\bf The Grassmannian. }
We give a very important (for us) example of an algebraic variety,
the Grassmannian of $l$-dimensional subspaces of a vector space $V$.
The Grassmannians are a generalization of projective spaces
(which are the case when $l=1$) and are in turn generalized by the flag
varieties we will  study later in the term.


Let $V$ be a $k$-vector space.
The {\sl Grassmannian of $l$-planes in $V$}, $\G$ or $\G(l,V)$, is the set
of all $l$-dimensional subspaces $H$ of $V$ (also called $l$-planes of $V$).
We will endow this set with the structure of a projective
algebraic variety.
When $V=k^n$, then we write $\G(l,n)$ for the Grassmannian of $l$-planes 
in $k^n$.
(The identification $V=k^n$ signifies that we have chosen an ordered basis
$e_1,\ldots,e_n$ for $V$.)


The first observation about $\G$ is that it admits a transitive action of
the algebraic group $GL(V)$ of linear automorphisms of $V$ (acting on the
right).
Indeed, given any two $l$-planes in $V$, there is some linear automorphism
of $V$ sending one to the other.
Moreover, given an $l$-plane $H$, its isotropy subgroup 
$G_H:=\{g\in GL(V)\mid H.g=H\}$ is a closed subgroup of $GL(V)$:
If $f_1,\ldots,f_l$ are a basis for $H$ and $\Lambda_{l+1},\ldots,\Lambda_n$
are independent linear forms vanishing on $H$, then $G_H$ is defined by the
vanishing of the linear polynomials $\Lambda_j(f_i.g)$.
In this way, we identify $\G(l,V)$ as the set of orbits 
$GL(V)/G_H$.
In fact, as we will see from Chevalley's Theorem (proven later) that as
$G_H$ is closed, $GL(V)/G_H$ is an algebraic variety, and thus we have that
the set $\G(l,V)$ is an algebraic variety.

We realize $\G(l,V)$ as a subset of projective space using multilinear
algebra.
Recall that $\wedge^aV$ is the set of all symbols 
$v_1\wedge\cdots\wedge v_a$ for $v_1,\ldots,v_a\in V$ subject to the
following relations:
%
\begin{eqnarray*}
   v_1\wedge\cdots\wedge v_i\wedge v_{i+1}\wedge\cdots\wedge v_a &=&
  -v_1\wedge\cdots\wedge v_{i+1}\wedge v_i\wedge\cdots\wedge v_a\\
   v_1\wedge\cdots\wedge (\alpha v_i+ \beta w_i)\wedge\cdots\wedge v_a &=&
   \alpha v_1\wedge\cdots\wedge v_i\wedge\cdots\wedge v_a\  +\\
    &&\quad
   \beta v_1\wedge\cdots\wedge  w_i\wedge\cdots\wedge v_a\,,
\end{eqnarray*}
%
This construction is functorial in that if we have a linear map 
$\varphi\colon V\to W$, then we get a map 
$\wedge^a\varphi\colon\wedge^aV\to\wedge^aW$ defined by
$$  
  \wedge^a\varphi\ :\ 
   v_1\wedge\cdots\wedge v_a\ \longmapsto\ 
   \varphi(v_1)\wedge\cdots\wedge \varphi(v_a)\,.
$$

Let $H\in \G(l,V)$.
Then we have $H\hookrightarrow V$.
Applying the $l$th exterior product to this inclusion, we get an inclusion
$$
  \wedge^l H\ \hookrightarrow\ \wedge^l V\,,
$$
as taking exterior powers is a functor in the category of finite-dimensional
$k$-vector spaces.
Since $\wedge^l H$ is 1-dimensional, it gives a line in the linear space 
$\wedge^l V$, and hence a point in ${\mathbb P}(\wedge^l V)$.
This defines the {\sl Pl\"ucker embedding} of 
$\G(l,V)\hookrightarrow{\mathbb P}(\wedge^l V)$.

We give a more concrete realization of this Pl\"ucker embedding.
Let $\varphi\in\mbox{Mat}_{n\times l}$ be a matrix, which we identify with
the linear map $M:k^l\to k^n$ it induces.
Let $e_1,\ldots,e_n$ and $f_1,\ldots,f_l$ be bases for $k^n$ and $k^l$,
respectively. 
From the above definition of exterior products, we see that $\wedge^lk^n$
has a basis consisting of vectors
$e_\alpha:=e_{\alpha_1}\wedge e_{\alpha_2}\wedge\cdots\wedge e_{\alpha_l}$,
for $\alpha\colon 1\leq\alpha_1<\alpha_2\cdots<\alpha_l\leq n$.
Write $\binom{[n]}{l}$ for this set of sequences, which are the $l$-subsets
of $[n]:=\{1,\ldots,n\}$, each written in increasing order.
In particular $\wedge^lk^l$ has a basis $f_1\wedge\cdots\wedge f_l$, and so
we see that it is 1-dimensional.

Suppose that the entries of the matrix $\varphi$ are $x_{i,j}$.
Then 
%
\begin{eqnarray*}
  \varphi(f_1\wedge\cdots\wedge f_l)&=&
  \left(\sum_{j_1=1}^n x_{j_1,1}e_{j_1}\right)\wedge\cdots\wedge
  \left(\sum_{j_l=1}^n x_{j_l,l}e_{j_l}\right)\\
   &=&
   \sum_{\alpha\in\binom{[n]}{l}} \det(x_{\alpha_i,j}\mid i,j=1,\ldots,l) 
    \,e_\alpha
\end{eqnarray*}
%
Let $p_\alpha(\varphi)$ be the determinant of the $l\times l$ submatrix of
$\varphi$ given by the rows indexed by $\alpha$.
This maximal minor of $\varphi$ is also called the 
{\sl $\alpha$th Pl\"ucker coordinate} of the matrix $\varphi$.
Suppose that we have another map $\psi\colon k^l\to k^n$ with the same image
as $\varphi$.
Then there exists a $g\in GL(l,k)$ such that $\psi=\varphi g$, and so we see
that $p_\alpha(\psi)=p_\alpha(\varphi)\cdot \det(g)$.
In this way, we see that the collection of Pl\"ucker coordinates
$p_\alpha(\varphi)$ are the homogeneous coordinates of the image of
$\wedge^l\varphi$ in Pl\"ucker space ${\mathbb P}^{\binom{n}{l}-1}$.

(It is not too hard to see that the Pl\"ucker map is an embedding, but we
will postpone a proof of this fact until later.)


From this description, we can describe the homogeneous ideal of polynomials
vanishing on the image of $\G(l,n)$ in Pl\"ucker space 
${\mathbb P}^{\binom{n}{l}-1}$.
We claim this ideal is the ideal of algebraic relations satisfied by the
maximal minors of a generic $n\times l$ matrix in 
$\mbox{Mat}_{n\times l}$.
The map on coordinate rings induced by the map 
$\mbox{Mat}_{n\times l}\to \wedge^lk^n$
given by sending a matrix $\varphi$ to its Pl\"ucker coordinates 
is the map $k[p_\alpha]\to k[x_{i,j}]$ given by sending a symbol $p_\alpha$ 
to the $l\times l$ maximal minor of a generic matrix given by the rows in
$\alpha$, $p_\alpha(x_{i,j})$.

If $I$ is this ideal, then $k[p_\alpha]/I$ is isomorphic to the subring
of $k[x_{i,j}]$ generated by these generic minors.
Since $k[x_{i,j}]$ is an integral domain, $k[p_\alpha]/I$ is as well, and so
we see that $I$ is prime.
It follows that $\G(l,n)$ is an irreducible subset of Pl\"ucker space.
\medskip

\noindent{\bf Example. }
Consider now the example of the Grassmannian of 2-planes in $k^4$, 
$\G(2,4)$.
Here, the Pl\"ucker coordinates are
$p_{12},p_{13},p_{14},p_{23},p_{24},p_{34}$, and if our generic matrix
has coordinates $x_{i,j}$ for $i=1,2,3,4$ and $j=1,2$.
Then
%
\begin{eqnarray*}
  p_{12}&=& x_{11}x_{22}-x_{12}x_{21}\\
  p_{13}&=& x_{11}x_{32}-x_{12}x_{31}\\
  p_{14}&=& x_{11}x_{42}-x_{12}x_{41}\\
  p_{23}&=& x_{21}x_{32}-x_{22}x_{31}\\
  p_{24}&=& x_{21}x_{42}-x_{22}x_{41}\\
  p_{34}&=& x_{31}x_{42}-x_{32}x_{41}
\end{eqnarray*}
%
We search for an algebraic relation by looking among the leading terms of
the $p_{ij}$.
Observe that the leading terms of the products $p_{14}p_{23}$ and
$p_{13}p_{24}$ coincide, so we subtract these two quadratic monomials:
$$
\begin{array}{rcl}
p_{14}p_{23} &=&\ x_{11}x_{21}x_{32}x_{42} - x_{11}x_{22}x_{31}x_{42}
               -x_{12}x_{21}x_{32}x_{41} + x_{12}x_{22}x_{31}x_{41}\\
-p_{13}p_{24}&=&-(x_{11}x_{21}x_{32}x_{42} - x_{11}x_{22}x_{32}x_{41}
                 -x_{12}x_{21}x_{31}x_{42} + x_{12}x_{22}x_{31}x_{41})
            \\\hline
             &=&-x_{11}x_{22}x_{31}x_{42} + x_{11}x_{22}x_{32}x_{41}
                +x_{12}x_{21}x_{31}x_{42} - x_{12}x_{21}x_{32}x_{41}\\
             &=& -p_{12}p_{34}\,,
\end{array}
$$
and thus we obtain the quadratic {\sl Pl\"ucker relation}:
$p_{14}p_{23}-p_{13}p_{24}+p_{12}p_{34}=0$.
Observe that when we first cancel the `leading terms' of $p_{14}p_{23}$ and
$p_{13}p_{24}$, we also cancel their trailing terms.
This miracle gets even better on the next cancellation, as the leading term
of the difference $p_{14}p_{23}-p_{13}p_{24}$ equals the leading term of
$-p_{12}p_{34}$, and when we cancel these, the other 3 terms cancel as well.

I claim that this one quadratic relation generates all algebraic relations
among the maximal minors of a generic $4\times 2$ matrix.
To see this, consider the intersection of the image of the Grassmannian
$\G(2,4)$ with an affine piece $U_\alpha$ of Pl\"ucker space.
Since we may rename coordinates, it suffices to assume that $\alpha=12$.
This intersection is parameterized by $4\times 2$-matrices $\varphi$ whose
$\alpha$th Pl\"ucker coordinate does not vanish.
This means that the first two rows of such a matrix are invertible, and
multiplying a matrix on the right by this inverse gives a matrix whose
transpose has the form:
$$
  \left[\begin{array}{cccc}
    1 & 0 & -p_{23} & -p_{24}\\
    0 & 1 &  p_{13} &  p_{14}
  \end{array}\right]
$$
The entries of this matrix are some of its Pl\"ucker
coordinates and the intersection of the Grassmannian with 
$U_{12}$ is the set of 5-tuples $(p_{13},p_{14},p_{23},p_{24},p_{34})$
where $p_{34}=-p_{23}p_{14}+p_{13}p_{24}$.

Thus $\G(2,4)\cap U_{12}$ is a closed subset of affine 5-space
defined by the single polynomial equation 
$p_{34}=-p_{23}p_{14}+p_{13}p_{24}$.
If we homogenize this, we obtain the the Pl\"ucker relation we had
previously, and we can see that we obtain this same relation if we begin
with another Pl\"ucker coordinate.
\medskip

We generalize this example to prove that the Grassmannian is a smooth,
closed, and irreducible subvariety of projective space, and give necessary
and sufficient equations.\medskip

\noindent{\bf Theorem. }
{\it 
The association of an $l$-plane $H$ to its Pl\"ucker coordinates
defines an injective map $\G(l,n)\to {\mathbb P}^{\binom{n}{l}-1}$
whose image is a smooth irreducible subvariety of dimension
$l(n-l)$.}\medskip 

Our treatment of reducible and irreducible sets was in the context of
Noetherian topological spaces, and so carries over to projective varieties.
The characterization of irreducible varieties as having prime ideals also
carries over to projective varieties as well.
For smoothness, we say a point $x\in X$ of a projective variety $X$ is
smooth if $x$ is a smooth point of the affine variety $X\in U_i$ for some
(and hence any) affine piece $U_i$ containing $x$.
The dimension of an irreducible variety $X$ is the dimension of any of the
non-empty intersections $X\cap U_i$.\medskip



\noindent{\sl Proof. }
We will show that the intersection of the image with each affine piece
$U_\alpha:=
\{[z_\beta]\in{\mathbb P}^{\binom{n}{l}-1}\mid z_\alpha\neq 0\}$ for 
$\alpha\in\binom{[n]}{l}$ is a smooth
irreducible subvariety isomorphic to $\mbox{Mat}_{n\times l}$, 
and this intersection is in bijection with
those $H\in \G(l,n)$ whose $\alpha$th 
Pl\"ucker coordinate is non-zero.
It suffices to do this for the index $\alpha=(1,2,\ldots,l)$, as the general
case follows from this case by permuting the basis $e_1,\ldots,e_n$ of 
$k^n$.

Let $\G_\alpha\subset \G(l,n)$ be the subset consisting of those $l$-planes
whose $\alpha$th Pl\"ucker coordinate is non-zero.
Then $\G_\alpha$ is the inverse image of the affine piece $U_\alpha$ under
the Pl\"ucker map.
If we represent an $l$-plane $H\in\G_\alpha$ as the 
column space of a $n\times l$ matrix $\varphi$, then the first $l$ rows of
$\varphi$ form an invertible matrix $T\in GL(l,k)$, as its
determinant is the $\alpha$th Pl\"ucker coordinate of $H$. 
Replacing $\varphi$ by $\varphi T^{-1}$, we see that we may assume that a
$l$-plane  $H\in \G_\alpha$ is the column space of a matrix of the form
%
\begin{equation}\label{eq:loc-chart}
  \left[\begin{array}{c}I_l\\ X\end{array}\right]\,,
\end{equation}
%
where $I_l$ is the $l\times l$ identity matrix
and $X\in\mbox{Mat}_{(n-l)\times l}$.
Conversely, given a matrix $X\in\mbox{Mat}_{(n-l)\times l}$, 
the column space $H$ of the matrix~(\ref{eq:loc-chart}) is an 
$l$-plane in $\G_\alpha$.
Thus the association
%
\begin{eqnarray*}
  \mbox{Mat}_{(n-l)\times l}&\longrightarrow&\G_\alpha\\
          X       &  \longmapsto & \mbox{column space }
     \left[\begin{array}{c}I_l\\ X\end{array}\right]
\end{eqnarray*}
%
defines a bijection of $\mbox{Mat}_{(n-l)\times l}$ with $\G_\alpha$.

For each $i=l+1,\ldots,n$ and $j=1,\ldots,l$, let 
$\beta_{ij}\in\binom{[n]}{l}$ be the index
$$
  \beta_{ij}\ =\ 1,2,\ldots,j{-}1,j{+}1,\ldots,l,i\,.
$$
Then the $\beta_{ij}$th maximal minor of the matrix~(\ref{eq:loc-chart}) is 
$(-1)^{j-i}x_{ij}$, where $x_{ij}$ is the $i,j$th entry in the matrix $X$
(whose rows are numbered $l+1,\ldots,n$).
In this way, we see that the composition
%
\begin{equation}\label{eq:g-comp}
  \mbox{Mat}_{(n-l)\times l}\ \longrightarrow\ \G_\alpha\ 
  \longrightarrow\ U_\alpha
\end{equation}
%
is one-to-one, and hence $G_\alpha$ is in bijection\index{bijection} 
with its image.
Since the maximal minors of the
matrix~(\ref{eq:loc-chart}) are polynomials in the 
entries of $X\in\mbox{Mat}_{(n-l)\times l}$, the
composition~(\ref{eq:g-comp}) is a regular map.
We claim that its image is an affine subvariety of $U_\alpha$, which proves
that the image of $\G(l,n)$ in ${\mathbb P}^{\binom{n}{l}-1}$ is a closed
subvariety. 

For this, we identify $\mbox{Mat}_{(n-l)\times l}$ with the coordinate
subspace of $U_\alpha$ spanned by $p_{\beta_{ij}}$ where $x_{ij}$
corresponds to  $(-1)^{j-i}p_{\beta_{ij}}$ and set 
${\mathbb A}^N$ to be the complementary coordinate subspace.
Then the image of $\G_\alpha$ in 
$U_\alpha\simeq\mbox{Mat}_{(n-l)\times l}\times{\mathbb A}^N$ is the graph
of the map obtained by following the composition~(\ref{eq:g-comp}) with the 
projection $U_\alpha\twoheadrightarrow {\mathbb A}^N$.
This shows that the image of $\G_\alpha$ is a closed subvariety of
$U_\alpha$, 
as the graph of a regular map is Zariski 
closed.

Lastly, since the image of $\G_\alpha$ is the graph of a regular map
$\mbox{Mat}_{(n-l)\times l}\to{\mathbb A}^N\!$, it is isomorphic
to $\mbox{Mat}_{(n-l)\times l}$, and hence smooth.
\qed\medskip

\noindent{\bf Remark. }
In the proof of this Theorem, we identified 
$\mbox{Mat}_{(n-l)\times l}$
with an affine open subset of $\G(l,n)$ via
%
\begin{equation}\label{eq:loc-coords}
   \mbox{Mat}_{(n-l)\times l} \ni X\ \longmapsto\ 
    \mbox{column space } \left[\begin{array}{c}I_l\\
           X\end{array}\right] \in \G(l,n)\,.
\end{equation}
%
Permuting rows of such a matrix gives systems of local coordinates for the
Grassmannian, and each system is parameterized by the set 
of $(n-l)\times l$ matrices.


\end{document}
