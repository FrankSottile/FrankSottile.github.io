%HW01.tex
%
% First Homework for Graduate Algebra
% Frank Sottile
%%%%%%%%%%%%%%%%%%%%%%%%%%%%%%%%%%%%%%%%%%%%%%%%%%%%%%%%%%%%%%%%%%%%%%%
\documentclass[12pt]{article}
\usepackage{multicol,amssymb,amsmath}
\usepackage{colordvi,graphicx}
\headheight=8pt
%
\topmargin=-75pt
\textheight=720pt   \textwidth=560pt
\oddsidemargin=-60pt \evensidemargin=-60pt


\pagestyle{empty}

%%%%%%%%%%%%%%%%%%%%%%%%%%%%%%%%%%%%%%%%%%%%
\newcommand{\CC}{{\mathbb C}}
\newcommand{\KK}{{\mathbb K}}
\newcommand{\NN}{{\mathbb N}}
\newcommand{\QQ}{{\mathbb Q}}
\newcommand{\TT}{{\mathbb T}}
\newcommand{\ZZ}{{\mathbb Z}}

\newcommand{\calA}{{\mathcal A}}
\newcommand{\be}{{\bf e}}

\newcommand{\Hom}{\mbox{Hom}}
\newcommand{\spec}{\mbox{spec}}
\newcommand{\cone}{\mbox{cone}}

\newcommand{\vect}[2]{(\begin{smallmatrix}#1\\#2\end{smallmatrix})}

\def\Color#1#2{\special{color push cmyk #1}#2\special{color pop}}
%\def\Indigo#1{\Color{.42 1. 0. .49}{#1}}
\def\Indigo#1{\Color{1. .95 .05 .4}{#1}}
\def\MyViolet#1{\Color{.6 1. 0. .15}{#1}}


\newcommand{\barsl}{\noindent\begin{minipage}[t]{590pt}
\Indigo{\rule{590pt}{1.2pt}}\vspace{-5.7mm}\\
\MyViolet{\rule{590pt}{1.2pt}}\vspace{-5.7mm}\\
\Blue{\rule{590pt}{1.2pt}}\vspace{-5.7mm}\\
\Green{\rule{590pt}{1.2pt}}\vspace{-5.7mm}\\
\Yellow{\rule{590pt}{1.2pt}}\vspace{-5.7mm}\\
\Orange{\rule{590pt}{1.2pt}}\vspace{-5.7mm}\\
\Red{\rule{590pt}{1.2pt}}\bigskip
\end{minipage}}


\newcommand{\barsn}{\noindent\begin{minipage}[t]{590pt}
\Indigo{\rule{590pt}{1.1pt}}\vspace{-4.5mm}\\
\MyViolet{\rule{590pt}{1.1pt}}\vspace{-4.5mm}\\
\Blue{\rule{590pt}{1.1pt}}\vspace{-4.5mm}\\
\Green{\rule{590pt}{1.1pt}}\vspace{-4.5mm}\\
\Yellow{\rule{590pt}{1.1pt}}\vspace{-4.5mm}\\
\Orange{\rule{590pt}{1.1pt}}\vspace{-4.5mm}\\
\Red{\rule{590pt}{1.1pt}}\bigskip
\end{minipage}}

\def\demph#1{\Maroon{{\sl #1}}}
\def\defcolor#1{\Maroon{#1}}

\begin{document}
\LARGE 
\noindent
Algebra \ \ Autumn 2023\vspace{1pt}\\
Frank Sottile\vspace{1pt}\\
\Large 21 August 2023 \hfill
\sf
 First Homework\makebox[40pt][l]{\ }
\large\vspace{10pt}

\noindent
Write your answers neatly, in complete sentences.  
I highly recommend recopying your work before handing it in.
Correct and crisp proofs are greatly appreciated; oftentimes your work can be shortened and made clearer.

\barsl

\noindent\Maroon{{\large\sf Hand in to Frank Monday 28 August:}}  \Blue{(Have this on a separate sheet of paper.)}

\begin{enumerate}

%%%%%%%%%%%%%%%%%%%%%%%%%%%%%%%%%%%%%%%%%%%%%%%%%%%%%%%%%%%%%%%%%%%%%%%%%%%%%%%%%%%%%%%%%%%%%%%%%%%%
\item
  Let $\alpha$ be a rotation about a point $p$ in the plane and $\rho$ be a reflection in a line through $p$.
  What is $\alpha^{-1}$, geometrically?
  Show that $\rho\alpha\rho=\alpha^{-1}$.  
  Let $\sigma$ also be a reflection in a line.  
  What is  $\rho\sigma$? 
 (There are several cases to consider.)

%%%%%%%%%%%%%%%%%%%%%%%%%%%%%%%%%%%%%%%%%%%%%%%%%%%%%%%%%%%%%%%%%%%%%%%%%%%%%%%%%%%%%%%%%%%%%%%%%%%%  
\end{enumerate}

\barsl

\noindent\Maroon{{\large\sf Hand in for the grader Monday 28 August:}}
\Blue{(Have this separate from \#1.)}\bigskip

A \demph{subgroup} of a group $G$ is a subset  $H$ of $G$ which is a group in its own right, under the group operations
of $G$.
For example, the set $2\ZZ$ of even integers is a subgroup of the additive group of integers.

\begin{enumerate}
\setcounter{enumi}{1}

%%%%%%%%%%%%%%%%%%%%%%%%%%%%%%%%%%%%%%%%%%%%%%%%%%%%%%%%%%%%%%%%%%%%%%%%%%%%%%%%%%%%%%%%%%%%%%%%%%%%
\item 
     Show that a finite group $G$ of even order ($G$ has an even number of elements) has a non-identity element $a$ with $a^2=e$.
%%%%%%%%%%%%%%%%%%%%%%%%%%%%%%%%%%%%%%%%%%%%%%%%%%%%%%%%%%%%%%%%%%%%%%%%%%%%%%%%%%%%%%%%%%%%%%%%%%%%
  
%%%%%%%%%%%%%%%%%%%%%%%%%%%%%%%%%%%%%%%%%%%%%%%%%%%%%%%%%%%%%%%%%%%%%%%%%%%%%%%%%%%%%%%%%%%%%%%%%%%%
\item Let $GL(2,\ZZ)$ be the collection of $2\times 2$ matrices with integer entries and determinant $\pm 1$.
      This is a group under multiplication of matrices, with identity $I=\vect{1&0}{0&1}$.
      Let $A:=\vect{0&-1}{1&0}$ and  $B:=\vect{0&1}{-1&1}$.
      What is the order of $A$ ?  Of $B$ ? Of $AB$?
%%%%%%%%%%%%%%%%%%%%%%%%%%%%%%%%%%%%%%%%%%%%%%%%%%%%%%%%%%%%%%%%%%%%%%%%%%%%%%%%%%%%%%%%%%%%%%%%%%%%

%%%%%%%%%%%%%%%%%%%%%%%%%%%%%%%%%%%%%%%%%%%%%%%%%%%%%%%%%%%%%%%%%%%%%%%%%%%%%%%%%%%%%%%%%%%%%%%%%%%%
\item Suppose that $G$ is a group in which the square of every element is the identity.
      Prove that $G$ is abelian.
%%%%%%%%%%%%%%%%%%%%%%%%%%%%%%%%%%%%%%%%%%%%%%%%%%%%%%%%%%%%%%%%%%%%%%%%%%%%%%%%%%%%%%%%%%%%%%%%%%%%

%%%%%%%%%%%%%%%%%%%%%%%%%%%%%%%%%%%%%%%%%%%%%%%%%%%%%%%%%%%%%%%%%%%%%%%%%%%%%%%%%%%%%%%%%%%%%%%%%%%%
\item
  Let $p$ be a prime number.
  Consider the following subsets of the rational numbers, $\QQ$:
%
  \begin{eqnarray*}
    \QQ[\tfrac{1}{p}] &:=& \{ \tfrac{a}{b}\in\QQ \mid
        a,b\in\ZZ\,,\ \mbox{gcd}(a,b)=1\,,\ \mbox{and}\ b=p^i\ \mbox{some } i\geq0\}\\
    \QQ^{(p)} &:=& \{ \tfrac{a}{b}\in\QQ \mid
       a,b\in\ZZ\,,\ \mbox{gcd}(a,b)=1\,,\ \mbox{and}\ p\nmid b\}
  \end{eqnarray*}
%
  Prove that these are subgroups of $\QQ$ under addition.

  (We intend that $\gcd(0,1)=1$.)
%%%%%%%%%%%%%%%%%%%%%%%%%%%%%%%%%%%%%%%%%%%%%%%%%%%%%%%%%%%%%%%%%%%%%%%%%%%%%%%%%%%%%%%%%%%%%%%%%%%%
  
    
      
\end{enumerate}
%%%%%%%%%%%%%%%%%%%%%%%%%%%%%%%%%%%%%%%%%%%%%%%%%%%%%%%%%%%%%%%%%%%%%%%%%%%%%%%%%%%%%%%%%%%%%%%%%%%%

\end{document}
