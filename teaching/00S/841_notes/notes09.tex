%notes09.tex
%
% Frank Sottile
% 25 March 2000
% Madison, WI
%
\documentclass[12pt]{amsart}
\usepackage{amssymb}
\usepackage{epsf}

\headheight=8pt     \topmargin=-10pt
\textheight=644pt   \textwidth=432pt
\oddsidemargin=18pt \evensidemargin=18pt

\def\silentfootnote#1{{\let\thefootnote\relax\footnotetext{#1}}}

\newcommand{\G}{{\bf G}}

%\def\baselinestretch{1.2}

\begin{document}
\begin{center}
\Large
The Geometry of Algebraic Groups\\
\large
Math 841\\
Frank Sottile\\
{\tt notes09}
\end{center}\bigskip

\newcommand{\DOT}{{\mbox{\Large\bf .}}}

\silentfootnote{\sl Version of 25 March 2000} 

\section{Borel Subgroups}


Let $V$ be an $n$-dimensional  $k$-vector space and let ${\mathcal F}(V)$ be
the set of 
(complete) flags in $V$, which are sequence of $k$-subspaces
$$
   F_{\DOT}\ :\ F_1\subset F_2\subset\cdots\subset F_n\ =\ V\,.
$$
We may regard ${\mathcal F}(V)$ as a projective variety as follows:
It is naturally a subset of the product of Grassmannians
$$
  {\bf G}_1V\times {\bf G}_2 V\times \cdots\times {\bf G}_nV\,.
$$
To see that is it closed, and hence a projective variety,
consider the condition for $F_{i-1}\subset F_i$ in
the product of affine pieces $G_{\alpha}\times G_\beta$ of 
$\G_{i-1}\times \G_i$.
It is an easy exercise to write down $n-i$ linear forms whose entries depend
linearly on the coordinates in $G_\beta$ such that a plane $F_i\in G_\beta$
is the kernel of the forms evaluated at $F_i$.
Then the equations for $F_{i-1}\subset F_i$ in $G_{\alpha}\times G_\beta$ 
are simply the multilinear equations that $F_{i-1}\in G_\alpha$
be annihilated by these linear forms.


Recall Borel's Fixed Point Theorem:  A connected solvable linear algebraic
group acting  on a projective variety has a fixed point.
This has an important consequence in the theory of algebraic groups, the 
Lie-Kolchin Theorem:
\medskip

\noindent{\bf Lie-Kolchin Theorem. }
{\it
Let $G\subset GL(V)$ be a connected solvable subgroup.
Then there is an ordered  basis for $V$ for which $G$ is a subset of the
upper triangular matrices with respect to that basis.
}\medskip

\begin{proof}
We may replace $G$ by its closure in $GL(V)$, which is still solvable.
Then the action of $G$ on ${\mathcal F}(V)$ has a fixed point, $F_{\DOT}$,
by the Borel Fixed Point Theorem.
If we choose an ordered basis $e_1,\ldots,e_n$ for $V$ such that 
$F_i=\langle e_1,\ldots,e_i\rangle$, then $G$ will be upper triangular with
respect to that basis.
(Here, we have $G$ acting on the right on $V$.)
\end{proof}


A {\sl Borel subgroup} of an algebraic group $G$ is a maximal, connected
solvable subgroup.  (Hence closed).
Since any Borel subgroup $B$ has $B\subset G^{\circ}$, the identity
component of $G$, we assume that $G$ is connected.
\medskip

\noindent{\bf Theorem. }
{\it
Let $B$ be any Borel subgroup of $G$.
Then $G/B$ is a projective variety and all other Borel subgroups are
conjugate to $B$ in $G$.}\medskip

\begin{proof}
Let $S\subset G$ be a Borel subgroup with maximal dimension.
Then there exists a representation $\rho:G\to GL(V)$ with a 1-dimensional
subspace $F_1$ of $V$ for which $S$ is its stabilizer, by Chevalley's
Theorem. 
By the Lie-Kolchin Theorem, $S$ stabilizes a flag in $V/F_1$.
Let $F_{\DOT}$ be the lift of that flag to $V$.
Then $S$ stabilizes $F_{\DOT}$, and since $S$ is the stabilizer of $F_1$,
$S$ is the stabilizer of $F_{\DOT}$.

Thus we have $G/B\simeq G.F_{\DOT}\subset {\mathcal F}(V)$.
Let $x\in \overline{G.F_{\DOT}}$.
Then the stabilizer of the flag represented by $x$ is a solvable subgroup of
$G$ (as it is a subgroup of the solvable subgroup of upper triangular
matrices in $GL(V)$ with respect to that flag).
This subgroup is closed, and thus its dimension is at most that of $S$, as
$S$ has maximal dimension among all solvable subgroups of $G$.
Thus the dimension of the orbit $G.x$ is at least the dimension of the orbit
$G.F_{\DOT}$.
This is by the Theorem on the dimension of fibres:
\medskip

\noindent{\bf Theorem. }
{\it 
Suppose $f:X\to Y$ is a map of irreducible algebraic varieties with
$f(X)=Y$. 
Then there is an open subset $U$ of $Y$ such that for every $y\in U$, the
fibre $f^{-1}(y)$ of the map over $y$ has  dimension equal to
$\dim X - \dim Y$.
}\medskip

We apply this Theorem to the map $G\to G.x$, whose fibre is the isotropy
subgroup of $x$.)
But by our Theorem on orbits, either the dimension of $g.x$ is less than
that of $G.F_{\DOT}$ or else $G.x=G.F_{\DOT}$.

We conclude that $G.F_{\DOT}$ is closed, which implies that $G/S$ is a
projective variety.
Any Borel subgroup $B$ of $G$ acts on $G/S$ on the left, and thus has a
fixed point $xS$.
Thus we have $BxS=xS$, which implies that $x^{-1}BxS=S$ or that 
$x^{-1}Bx\subset S$.
Since each is a Borel subgroup, we have $x^{-1}Bx=S$, which proves both
parts of the Theorem.
\end{proof}


Suppose that $H$ is a closed subgroup of $G$ and that the quotient $G/H$ is
a projective variety.
Let $B$ be a Borel subgroup of $G$, which acts on the left of $G/H$.
Then by the Borel Fixed Point Theorem, $B$ has a fixed point $x.H$ on $G/H$. 
As in the proof of the proceeding Theorem, this implies that 
$B\subset x^{-1}Hx$.
Similarly, if $H$ is a closed subgroup of $G$ containing a Borel subgroup
$B$ of $G$, then we have a surjection $G/B\to G/H$ of a projective variety
onto a quasi-projective variety.
Since the image of a projective variety under any algebraic map is closed,
we conclude that $G/H$ is a projective variety.
We make a definition.
A subgroup $P$ of an algebraic group $G$ is {\sl parabolic} if $G/P$ is
projective. 
\medskip

\noindent{\bf Corollary. }
{\it
A closed subgroup $P$ of $G$ is parabolic if and only if it contains a Borel
subgroup of $G$.
A connected subgroup $H$ of $G$ is a Borel subgroup if and only if $H$ is
solvable and $G/H$ is projective.}\medskip


\noindent{\bf Definition. }
A {\sl flag variety} (or generalized flag manifold) is a projective
homogeneous space $G/P$ of an algebraic group $G$.
We wish to classify these flag varieties and study their geometry.
\medskip

Consider the group $G=GL(n,k)$.
The group $B$ of upper triangular matrices is a Borel subgroup of $G$, and
the flag variety $G/B$ in this case is the space of complete flags
${\mathcal F}_n:={\mathcal F}(k^n)$ defined previously.
Given any subset $S\subset \{1,2,\ldots,n-1\}$, consider the variety of
partial flags ${\mathcal F}_S(k^n)$ given by flags
$$
   F_{s_1}\subset F_{s_2}\subset \cdots \subset F_{s_m}\subset k^n\,,
$$
where $S=\{s_1<s_2<\cdots<s_m\}$ and $\dim F_{s_i}=s_i$.
We call $S$ the {\sl type} of the flag.

There is an obvious surjective `forgetful' map from 
${\mathcal F}_n\to {\mathcal F}_S$ given by sending a complete flag
$F_{\DOT}$ to the partial flag formed by the subspaces in $F_{\DOT}$ whose
dimension is in $S$.
Thus we see that these partial flag varieties are projective varieties;
either as they are the image of a projective variety, or else directly in
the same way we argued that ${\mathcal F}_n$ was projective.

If $e_1,\ldots, e_n$ is a basis for $k^n$, then we let the standard flag
$E_{\DOT}$ be the flag whose $i$th subspace $E_i$ is spanned by
$e_1,\ldots, e_i$.
Then $B$ is the isotropy subgroup of $E_{\DOT}$, and we define $P_S$ to be
the isotropy subgroup of the subflag of $E_{\DOT}$ of type $S$.
If $T\supset S$, then $P_T\subset P_S$ and so the projection from 
${\mathcal F}_n$ to ${\mathcal F}_S$ factors through ${\mathcal F}_T$. 

In this way we obtain a lattice of parabolic subgroups of $G$ containing $B$
anti-isomorphic to the lattice of subsets of $\{1,\ldots,n-1\}$.

There is even more geometric information in this structure.
Let $A_{n-1}$ be the graph consisting of a path of length $n-1$:
$$
  \epsfbox{figures/An.eps}
$$
This graph corresponds to the flag manifold ${\mathcal F}_n$.
If we have a subset $S\subset \{1,\ldots,n-1\}$, then we may consider the 
induced subgraph that remains when we remove the vertices from $A_{n-1}$
labeled by $S$. 
Below, we show $A_5$ and the induced subgraph when $S=\{2,3\}$:
$$
  \epsfbox{figures/A6.eps}
$$
This subgraph encodes the geometry of the fibres 
${\mathcal F}_n\to{\mathcal F}_S$ in the following way.
Each connected component of the subgraph is isomorphic to some $A_{m-1}$.
Let $Fib_S$ be the product of flag varieties ${\mathcal F}_m$, one for each
component of the subgraph.
Then $Fib_S$ is isomorphic to the fibre of the the projection
${\mathcal F}_n\to{\mathcal F}_S$.

Consider this in the example of 
$\pi\colon{\mathcal F}_6\to {\mathcal F}_{\{2,3\}}$.
This map is given by the forgetful map
$$
  \pi\colon
  F_1\subset F_2\subset F_3\subset F_4\subset F_5\ \longmapsto\ 
  F_2\subset F_3\,.
$$
The inverse image $\pi^{-1}(F_2\subset F_3)$ is given by choosing
subspaces $F_1,F_4$, and $F_5$ which `fill out' the flag 
$F_2\subset F_3$.
We may choose $F_1$ to be any 1-dimensional subspace of the 2-dimensional
vector space $F_2$, and $F_4\subset F_5$ corresponds to any complete flag in 
$k^6/F_3\simeq k^3$.
In this way we see that 
$\pi^{-1}(F_2\subset F_3)\simeq 
   {\mathcal F}(F_2)\times {\mathcal F}(k^6/F_3)$, in agreement with the
description given above.










\end{document}
