%Homework2.tex
%Second Homework -- Math 300H 
%
%  The percent sign is a comment character
%
%%%%%%%%%%%%%%%%%%%%%%%%%%%%%%%%%%%%%%%%%%%%%%%%%%%%%%%%%%%%%%%%%%%%%%%%%%%%%%%%%%
%
%   Look these up on line.  The first sets the type of document, and the next are for mathematics symbols, graphics and color
%
\documentclass[12pt]{article}
\usepackage{amssymb,amsmath,mathtools}
\usepackage{graphicx}
\usepackage[usenames,dvipsnames,svgnames,table]{xcolor}
\usepackage{multirow}   % This is for more control over tables
%%%%%%%%%%%%%%%%%%%%%%%%%%%%%%%%  Layout     %%%%%%%%%%%%%%%%%%%%%%%%%%%%%%%%%%%%%%
\usepackage{vmargin}
\setpapersize{USletter}
\setmargrb{2cm}{1cm}{2cm}{1cm} % --- sets all four margins LTRB


\newcommand{\RR}{{\mathbb R}}  % This is the backboard symbol for the real numbers.  Note how it is used below
\newcommand{\NN}{{\mathbb N}}  % 
\newcommand{\ZZ}{{\mathbb Z}}  %

\newcommand{\calP}{{\mathcal P}}  %Caligraphic P for power set

\newcommand{\sep}{{\ :\ }}     % This is for the : in our notation for building sets.
\newcommand{\lsim}{\mathord{\sim}}  %  This is to remove the extra spacing around \sim (it is a binary relation) for the logical not

%%%%%%%%%%%%%%%%%%%%%%%%%%%%%%%%%%%%%%%%%%%%%%%%%%%%%%%%%%%%%%%%%%%%%%%%%%%%%%%%%
%
%   Please edit this as appropriate
%
\begin{document}
\LARGE 
\noindent
{\color{Maroon}Foundations of Mathematics \hfill Math 300H Section 970}\vspace{2pt}\\
\Large YOUR NAME\vspace{2pt}\\
\large
Second Homework: \hfill 13 September 2021\\
Use English when possible.  Answers should not just be symbols.
\normalsize\vspace{10pt}


%%%%%%%%%%%%%%%%%%%%%%%%%%%%%%%%%%%%%%%%%%%%%%%%%%%%%%%%%%%%%%%%%%%%%%%%%%%%%%%%%
\begin{enumerate}  %  the \begin{..}  \end{..} stars and ends an environment.


%%%%%%%%%%%%%%%%%%%%%%%%%%%%%%%%%%%%%%%%%%%%%%%%%%%%%%%%%%%%%%%%%%%%%%%%%%%%%%%%%%%%%%%%%%%%%%%%%%%%
\item  For a real number $r$, define $A_r\coloneqq\{r^2\}$,  $B_r$ to be the closed interval $[r{-}1, r{+}2]$, and $C_r$ to be the interval
  $(r,\infty)$.
  For $S=\{1,2,4\}$ determine each of the following 

  $\bigcup_{s\in S} A_s $ \ and \ $\bigcap_{s\in S} A_s $\ and\ 
  $\bigcup_{s\in S} B_s $ \ and \ $\bigcap_{s\in S} B_s $\ and\ 
  $\bigcup_{s\in S} C_s $ \ and \ $\bigcap_{s\in S} C_s $\,.
%%%%%%%%%%%%%%%%%%%%%%%%%%%%%%%%%%%%%%%%%%%%%%%%%%%%%%%%%%%%%%%%%%%%%%%%%%%%%%%%%%%%%%%%%%%%%%%%%%%


%%%%%%%%%%%%%%%%%%%%%%%%%%%%%%%%%%%%%%%%%%%%%%%%%%%%%%%%%%%%%%%%%%%%%%%%%%%%%%%%%%%%%%%%%%%%%%%%%%%%
\item    Determine each of the  following:

  ${\displaystyle \bigcup_{n=1}^\infty \left(-\frac{1}{n},\frac{1}{n}\right)}$ \ and \ 
  ${\displaystyle \bigcap_{n=1}^\infty \left(-\frac{1}{n},\frac{1}{n}\right)}$ \ \ and \ \ 
  ${\displaystyle \bigcup_{n=1}^\infty \left(\frac{n{-}1}{n},\frac{n{+}1}{n}\right)}$ \ and \;\ 
  ${\displaystyle \bigcap_{n=1}^\infty \left(\frac{n{-}1}{n},\frac{n{+}1}{n}\right)}$
     
%%%%%%%%%%%%%%%%%%%%%%%%%%%%%%%%%%%%%%%%%%%%%%%%%%%%%%%%%%%%%%%%%%%%%%%%%%%%%%%%%%%%%%%%%%%%%%%%%%%

%%%%%%%%%%%%%%%%%%%%%%%%%%%%%%%%%%%%%%%%%%%%%%%%%%%%%%%%%%%%%%%%%%%%%%%%%%%%%%%%%%%%%%%%%%%%%%%%%%%
\item  Which of the following sets are partitions of $A=\{a,b,c,d,e,f,g\}$?
       For each collection of subsets that is not a partition of $A$, explain your answer.
  \begin{enumerate} 

  \item[(a)] \makebox[200pt][l]{$S_1=\left\{ \{a,c,e,g\},\{b,f\},\{d,d\} \right\}$}
        (b)                     $S_2=\left\{ \{a,b,c,d\},\{e,f\} \right\}$
  \item[(c)] \makebox[200pt][l]{$S_3=\left\{A \right\}$}
        (d)                     $S_4=\left\{ \{a\}, \emptyset, \{b,c,d\},\{e,f,g\} \right\}$
  \item[(e)]  $S_5=\left\{ \{a,c,d\},\{b,g\},\{e\},\{b,f\} \right\}$
    
  \end{enumerate}
%%%%%%%%%%%%%%%%%%%%%%%%%%%%%%%%%%%%%%%%%%%%%%%%%%%%%%%%%%%%%%%%%%%%%%%%%%%%%%%%%%%%%%%%%%%%%%%%%%%%

%%%%%%%%%%%%%%%%%%%%%%%%%%%%%%%%%%%%%%%%%%%%%%%%%%%%%%%%%%%%%%%%%%%%%%%%%%%%%%%%%%%%%%%%%%%%%%%%%%%%
\item Give an example of a partition of $\NN$ into three subsets.
%%%%%%%%%%%%%%%%%%%%%%%%%%%%%%%%%%%%%%%%%%%%%%%%%%%%%%%%%%%%%%%%%%%%%%%%%%%%%%%%%%%%%%%%%%%%%%%%%%%%


%%%%%%%%%%%%%%%%%%%%%%%%%%%%%%%%%%%%%%%%%%%%%%%%%%%%%%%%%%%%%%%%%%%%%%%%%%%%%%%%%%%%%%%%%%%%%%%%%%%%
\item   For $A=\{1,2\}$ and $B=\{\emptyset\}$ determine $\calP(A\times B)$.
%%%%%%%%%%%%%%%%%%%%%%%%%%%%%%%%%%%%%%%%%%%%%%%%%%%%%%%%%%%%%%%%%%%%%%%%%%%%%%%%%%%%%%%%%%%%%%%%%%%%

%%%%%%%%%%%%%%%%%%%%%%%%%%%%%%%%%%%%%%%%%%%%%%%%%%%%%%%%%%%%%%%%%%%%%%%%%%%%%%%%%%%%%%%%%%%%%%%%%%%%
\item
  (a) For a set $A$ with $|A|=2$, what is the largest possible value of $|A\cap\calP(A)|$?

  (b) What is the largest possible value of  $|A\cap\calP(A)|$ if $|A|=3$?

  (c) This should suggest another question to you.
      What is the answer to that question?  \newline (Do also give that question).
%%%%%%%%%%%%%%%%%%%%%%%%%%%%%%%%%%%%%%%%%%%%%%%%%%%%%%%%%%%%%%%%%%%%%%%%%%%%%%%%%%%%%%%%%%%%%%%%%%%%

%%%%%%%%%%%%%%%%%%%%%%%%%%%%%%%%%%%%%%%%%%%%%%%%%%%%%%%%%%%%%%%%%%%%%%%%%%%%%%%%%%%%%%%%%%%%%%%%%%%%
\item  Consider the sets $A$, $B$, $C$, and $D$ below.
       Which of the following statements are true?      
       Give an explanation for each false statement.

       $A = \{1,4,7,10,13,16,\dotsc\}$ \qquad $B=\{n\in\ZZ\sep n\mbox{ is prime and }n\neq 2\}$

       $C= \{ n\in\ZZ \sep n\mbox{ is odd}\}$ \qquad  $D=\{1,2,3,5,8,13,21,34,55,\dotsc\}$

       (a) $25\in A$ \qquad
       (b) $33\in D$ \qquad
       (c) $22\not\in A\cup D$ \qquad
       (d) $B\subseteq C$ \newline
       (e) $\emptyset \in C\cap D$ \qquad
       (f) $53\not\in B$ \qquad
       (g) $144\in D$ .

%%%%%%%%%%%%%%%%%%%%%%%%%%%%%%%%%%%%%%%%%%%%%%%%%%%%%%%%%%%%%%%%%%%%%%%%%%%%%%%%%%%%%%%%%%%%%%%%%%%%


%%%%%%%%%%%%%%%%%%%%%%%%%%%%%%%%%%%%%%%%%%%%%%%%%%%%%%%%%%%%%%%%%%%%%%%%%%%%%%%%%%%%%%%%%%%%%%%%%%%%
\item Give a negation of each of the following statements.
   \begin{enumerate}
     \item At least two of my library books are overdue.
     \item One of my two friends misplaced his homework assignment.
     \item No one expected that to happen.
     \item It is not often that my instructor teaches that course.
     \item It is surprising that two students received the same exam score.
   \end{enumerate}
%%%%%%%%%%%%%%%%%%%%%%%%%%%%%%%%%%%%%%%%%%%%%%%%%%%%%%%%%%%%%%%%%%%%%%%%%%%%%%%%%%%%%%%%%%%%%%%%%%%%

%%%%%%%%%%%%%%%%%%%%%%%%%%%%%%%%%%%%%%%%%%%%%%%%%%%%%%%%%%%%%%%%%%%%%%%%%%%%%%%%%%%%%%%%%%%%%%%%%%%%
\item Which of the following sentences are statements?
  \begin{enumerate}
   \item $3^2+4^2=5^2$.
   \item $a^2+b^2=c^2$.
   \item There exist integers $a$, $b$, and $c$ such that  $ a^2+b^2=c^2$.
   \item If $x^2= 4$, then $x=2$.
   \item For each real number $x$,  if $x^2= 4$, then $x=2$.
   \item For each real number $t$ , $\sin^2 t + \cos^2 t =1$.
   \item If $n$ is a prime number, then $n^2$ has three positive factors.
   \item $\sin x < \sin(\pi/4)$.
   \item Every parallelogram is a rectangle.
  \end{enumerate}
 Of those that are statements, which are true?
%%%%%%%%%%%%%%%%%%%%%%%%%%%%%%%%%%%%%%%%%%%%%%%%%%%%%%%%%%%%%%%%%%%%%%%%%%%%%%%%%%%%%%%%%%%%%%%%%%%%

%%%%%%%%%%%%%%%%%%%%%%%%%%%%%%%%%%%%%%%%%%%%%%%%%%%%%%%%%%%%%%%%%%%%%%%%%%%%%%%%%%%%%%%%%%%%%%%%%%%%
\item  Let $P(n)$ be the open sentence ``$n$ and $n+2$ are primes'' where $n$ ranges over the natural numbers $\NN$.
       Find six positive integers $n$ for which $P(n)$ is true.
%%%%%%%%%%%%%%%%%%%%%%%%%%%%%%%%%%%%%%%%%%%%%%%%%%%%%%%%%%%%%%%%%%%%%%%%%%%%%%%%%%%%%%%%%%%%%%%%%%%%

%%%%%%%%%%%%%%%%%%%%%%%%%%%%%%%%%%%%%%%%%%%%%%%%%%%%%%%%%%%%%%%%%%%%%%%%%%%%%%%%%%%%%%%%%%%%%%%%%%%%
\item Fill out the truth table for the expressions
  $P\land Q$, $\lsim P$, $\lsim Q$, $\lsim(P\land Q)$, $\lsim P\land\lsim Q$, $\lsim P\lor\lsim Q$, and 
   $(\lsim P\land\lsim Q)\lor Q$ :

  %  Tabular is an environment in LaTeX that formats tables.
  %  The {|c|c||c|c|c|c|c|c|c|} specifies that there are 9 columns, each with entries cenered (c) and | delimiters between them,
  %    *and* one at the beginning and end, and an extra one after the first two columns  
  %  \hline is 'horizontal line'
  %  The cells are typeset in ordinary text, and are delimited (separated) by the ampersands, &
  %  You wil need to make a similar table (that is not as wide) for problem 15.
  %
  \begin{tabular}{|c|c||c|c|c|c|c|c|c|}\hline
    $P$&$Q$&$P\land Q$&$\lsim P$& $\lsim Q$& $\lsim(P\land Q)$& $\lsim P\land\lsim Q$& $\lsim P\lor\lsim Q$
      &  $(\lsim P\land\lsim Q)\lor Q$\\ \hline\hline
     T &T&&&&&&&\\\hline
     T &F&&&&&&&\\\hline
     F &T&&&&&&&\\\hline
     F &F&&&&&&&\\\hline
    \end{tabular}\medskip
%%%%%%%%%%%%%%%%%%%%%%%%%%%%%%%%%%%%%%%%%%%%%%%%%%%%%%%%%%%%%%%%%%%%%%%%%%%%%%%%%%%%%%%%%%%%%%%%%%%%

%%%%%%%%%%%%%%%%%%%%%%%%%%%%%%%%%%%%%%%%%%%%%%%%%%%%%%%%%%%%%%%%%%%%%%%%%%%%%%%%%%%%%%%%%%%%%%%%%%%%
 \item  Identify the hypothesis and the conclusion for each of the following conditional statements.
  \begin{enumerate}
   \item If $a$ is an irrational number and $b$ is an irrational number, then $a\cdot b$ is an irrational number.
   \item If $p \neq 2$ and $p$ is an even number, then $p$ is not prime.
  \end{enumerate}
%%%%%%%%%%%%%%%%%%%%%%%%%%%%%%%%%%%%%%%%%%%%%%%%%%%%%%%%%%%%%%%%%%%%%%%%%%%%%%%%%%%%%%%%%%%%%%%%%%%%


%%%%%%%%%%%%%%%%%%%%%%%%%%%%%%%%%%%%%%%%%%%%%%%%%%%%%%%%%%%%%%%%%%%%%%%%%%%%%%%%%
\item  Rewrite the following statements in the form ``if $P$, then $Q$''.
  \begin{enumerate}

    \item ``One, if by land''.
    \item ``Candor implies equality''.
    \item ``Pepperoni only if pizza''.
    \item ``Inattentive when bored''.
    \item ``Slapstick is sufficient for comedy''.
    \item ``Quiet is necessary for sleep''.

  \end{enumerate}
%%%%%%%%%%%%%%%%%%%%%%%%%%%%%%%%%%%%%%%%%%%%%%%%%%%%%%%%%%%%%%%%%%%%%%%%%%%%%%%%%%%%%%%%%%%%%%%%%%%%

%%%%%%%%%%%%%%%%%%%%%%%%%%%%%%%%%%%%%%%%%%%%%%%%%%%%%%%%%%%%%%%%%%%%%%%%%%%%%%%%%%%%%%%%%%%%%%%%%%%%
\item Rewrite ``If the function $f$ is differentiable, then it is continuous'' in each of the six forms of the previous problem.
%%%%%%%%%%%%%%%%%%%%%%%%%%%%%%%%%%%%%%%%%%%%%%%%%%%%%%%%%%%%%%%%%%%%%%%%%%%%%%%%%%%%%%%%%%%%%%%%%%%%

%%%%%%%%%%%%%%%%%%%%%%%%%%%%%%%%%%%%%%%%%%%%%%%%%%%%%%%%%%%%%%%%%%%%%%%%%%%%%%%%%%%%%%%%%%%%%%%%%%%%
\item For statements $P$ and $Q$, construct the truth table for $(P\Rightarrow Q) \Rightarrow \lsim P$.
%%%%%%%%%%%%%%%%%%%%%%%%%%%%%%%%%%%%%%%%%%%%%%%%%%%%%%%%%%%%%%%%%%%%%%%%%%%%%%%%%%%%%%%%%%%%%%%%%%%%

  
\end{enumerate}  
%%%%%%%%%%%%%%%%%%%%%%%%%%%%%%%%%%%%%%%%%%%%%%%%%%%%%%%%%%%%%%%%%%%%%%%%%%%%%%%%%


\end{document}
%%%%%%%%%%%%%%%%%%%%%%%%%%%%%%%%%%%%%%%%%%%%%%%%%%%%%%%%%%%%%%%%%%%%%%%%%%%%%%%%%
