%Group05.tex
%Third group homework -- Math 629 
%
%  The percent sign is a comment character
%
%%%%%%%%%%%%%%%%%%%%%%%%%%%%%%%%%%%%%%%%%%%%%%%%%%%%%%%%%%%%%%%%%%%%%%%%%%%%%%%%%%
%
%   Look these up on line.  The first sets the type of document, and the next are for mathematics symbols, graphics and color
%
\documentclass[12pt]{article}
\usepackage{amssymb,amsmath}
\usepackage{graphicx}
\usepackage[usenames,dvipsnames,svgnames,table]{xcolor}
\usepackage{multirow}   % This is for more control over tables
%%%%%%%%%%%%%%%%%%%%%%%%%%%%%%%%  Layout     %%%%%%%%%%%%%%%%%%%%%%%%%%%%%%%%%%%%%%
\usepackage{vmargin}
\setpapersize{USletter}
\setmargrb{2cm}{1cm}{2cm}{1cm} % --- sets all four margins LTRB


%%%%%%%%%%%%%%%%%%%%%%%%%%%%%%%%%%%%%%%%%%%%%%%%%%%%%%%%%%%%%%%%%%%%%%%%%%%%%%%%%
\begin{document}
\LARGE 
\noindent
{\color{Maroon}History of Mathematics \hfill Math 629}\vspace{2pt}\\
\large
Third Group Homework: \hfill 14 February 2024\\
Not to be handed in, but should be discussed.
\normalsize\vspace{10pt}



\noindent{\bf Some additional work on cubics}

\begin{enumerate}

   
  \item  
    Here is another cubic to solve completely: $x^3+2x+4i=0$, where $i=\sqrt{-1}$ is the imaginary unit, a square root
    of -1.
    You will find the example on the bottom of page 3 of my notes, as well as the formula $(-i)^3=i$ useful for this.

    

  \item
    Watch the last part of Marcus du Satoy's documentary on the History of Mathematics.
    Can you find a discrepancy between what he presents and what the book presents about the history of solving the cubic?
    Please investigate this, seeking sources to corroborate one or the other.


  \item {\bf Challenge:}  Can you use methods, described either in the book, or in my notes, to solve the quartic,
    \[
    x^4 - 10 x^3 +4x + 8 = 0\,?
    \]
    
\end{enumerate}

\end{document}
%%%%%%%%%%%%%%%%%%%%%%%%%%%%%%%%%%%%%%%%%%%%%%%%%%%%%%%%%%%%%%%%%%%%%%%%%%%%%%%%%
