%HW1.tex
%First Homework -- Math 629 
%
%  The percent sign is a comment character
%
%%%%%%%%%%%%%%%%%%%%%%%%%%%%%%%%%%%%%%%%%%%%%%%%%%%%%%%%%%%%%%%%%%%%%%%%%%%%%%%%%%
%
%   Look these up on line.  The first sets the type of document, and the next are for mathematics symbols, graphics and color
%
\documentclass[12pt]{article}
\usepackage{amssymb,amsmath}
\usepackage{graphicx}
\usepackage[usenames,dvipsnames,svgnames,table]{xcolor}
\usepackage{multirow}   % This is for more control over tables
%%%%%%%%%%%%%%%%%%%%%%%%%%%%%%%%  Layout     %%%%%%%%%%%%%%%%%%%%%%%%%%%%%%%%%%%%%%
\usepackage{vmargin}
\setpapersize{USletter}
\setmargrb{2cm}{1cm}{2cm}{1cm} % --- sets all four margins LTRB


%%%%%%%%%%%%%%%%%%%%%%%%%%%%%%%%%%%%%%%%%%%%%%%%%%%%%%%%%%%%%%%%%%%%%%%%%%%%%%%%%
\begin{document}
\LARGE 
\noindent
{\color{Maroon}History of Mathematics \hfill Math 629}\vspace{2pt}\\
\large
First Homework: \hfill 18 January 2022\\
Due Monday 24 January 2022.
\normalsize\vspace{10pt}

\begin{enumerate}

\item {[20]}
    Write one or two paragraphs answering the following questions: What are the two kinds of numbers? How are they used in common life? What
    evidence is there that non-Human animals use/understand either type? How about babies or very young children? Which type of number lends
    itself to arithmetic? 

\item {[10]}
      Allen gives the formula $(n^3+n)/2$ for the (common) sum of the rows/columns of a magic square with side length $n$.
      Explain why this is the sum of any row/column of a magic square of side length $n$.

\item {[10]}
    In Allen's text, the Method of the Mean is discussed as a way to approximate the square root of a number $n$: Start with some initial
    guess, e.g.\  $a=1$.
    Then, replace $a$ by $(a + n/a)/2$, the average of $a$ and $n/a$ (which together multiply to $n$). On a calculator (or better) a
    computer carry out this procedure for a few steps (3-5), for some natural number $n$ that is not a perfect square. Record and hand in the
    (base ten, digital) steps that you compute. If you have access to a computer and can do this with many digits of precision, by all means
    use it, and report your answers to higher precision (such as 100 digits). If anyone gets a particularly interesting answer, share it in
    a post on Piazza. This problem is very relevant to modern computation; I'll explain this after we share our answers.

\item {[50]}    
    I'd like us all to practice some calculations using the sexagesimal system of the Mesopotamians. For this, let us use the notation that
    `;' represents the `sexagesimal point' and `,' is the delimiter between `places'. Thus `2,22' is 2*60+22=142, one-hundred and forty two,
    while `1;45' is 1+45/60=1.75. Do these using base 60 and show or explain your work. The purpose of this is to appreciate what it is like
    to compute in base 60. To that end, do not simply convert to base ten, do the computations, and convert back. Do them purely in base 60,
    employing the usual algorithms you know.

    \begin{enumerate}

    \item 
      Warm-up: Express the (decimal) numbers in sexagesimal: $45$,   $150$,    $3253$,    $17589$, and    $100,000$.

        
     \item        
        Simpler: $20 + 50 = W$,      $7*17 = X$,      $3,9 - 1,40 = Y$, and      $1,24*1,24 = Z$.

    \item 
        How about some division: 1/2 = V     1/3 = W     2/5 = X     7/4 = Y     2,16/3 = Z.

    \item 
      Repeating sexagesimals: Why is $1/59 = ;1,1,1,....$?
      My favorite decimal fraction is $1/7$. What is this in sexagesimal (multiply your answer by 7 to check) ?
      
      The decimal expansion of 1/11 has the form $0.090909090....$ It is a repeating decimal with period 2.
      
      Express the common fraction 1/7 as a repeating vigesimal. (E.g. $0;a,b,c,\dotsc = a/20 + b/20^2 + c/20^3 + \dotsb$).
      Compare the period of the repeats for this same number in decimal and in sexagesimal.
      Can you explain the relation between the different periods in the different bases?

    \item 
        Compare and contrast the different methods used to represent whole numbers used by Babylonians, Mayans, and by us in our decimal
        positional system. For each ancient system give an example of a computation or representation for which is was superior to the
        others, and one where it was inferior. You can include fractions for this second question. 

    \item {[Extra Credit]}
        Challenge (only if you have stamina and like this stuff; this is not required): Try to verify that $(1;24,51,10)^2$ is pretty close to
        2, as recorded on YBC 7289. What is the next term in sexagesimal? I found this interesting. 
  \end{enumerate}

    \item {[10]}
        Explain how the Babylonians used tables of squares ($n^2$) to facilitate multiplication of whole numbers. Compare this to the method we commonly use for calculations by hand. Which do you think makes more sense for sexagesimal calculations? Why?
        (You may want to look at my table of Babylonian squares, from the course webpage for week 1.)
        
\end{enumerate}

\end{document}
%%%%%%%%%%%%%%%%%%%%%%%%%%%%%%%%%%%%%%%%%%%%%%%%%%%%%%%%%%%%%%%%%%%%%%%%%%%%%%%%%
