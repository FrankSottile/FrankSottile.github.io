%HW6.tex
%Sixth Homework -- Math 300 
%
%  The percent sign is a comment character
%
%%%%%%%%%%%%%%%%%%%%%%%%%%%%%%%%%%%%%%%%%%%%%%%%%%%%%%%%%%%%%%%%%%%%%%%%%%%%%%%%%%
%
%   Look these up on line.  The first sets the type of document, and the next are for mathematics symbols, graphics and color
%
\documentclass[12pt]{article}
\usepackage{amssymb,amsmath}
\usepackage{graphicx}
\usepackage[usenames,dvipsnames,svgnames,table]{xcolor}
\usepackage{multirow}   % This is for more control over tables
%%%%%%%%%%%%%%%%%%%%%%%%%%%%%%%%  Layout     %%%%%%%%%%%%%%%%%%%%%%%%%%%%%%%%%%%%%%
\usepackage{vmargin}
\setpapersize{USletter}
\setmargrb{2cm}{1cm}{2cm}{1cm} % --- sets all four margins LTRB


\newcommand{\defn}[1]{{\color{blue}\sl #1}}
\newcommand{\deco}[1]{{\color{blue} #1}}

\newcommand{\NN}{{\mathbb N}}
\newcommand{\QQ}{{\mathbb Q}}
\newcommand{\RR}{{\mathbb R}}
\newcommand{\ZZ}{{\mathbb Z}}

%%%%%%%%%%%%%%%%%%%%%%%%%%%%%%%%%%%%%%%%%%%%%%%%%%%%%%%%%%%%%%%%%%%%%%%%%%%%%%%%%
\begin{document}
\LARGE 
\noindent
{\color{Maroon}Foundations of Mathematics \hfill Math 300 Sections 902, 905}\vspace{2pt}\\
\Large YOUR NAME\vspace{2pt}\\
\large
Sixth Homework: \hfill Due 12 October 2020
\normalsize    %%%% This returns the fontsize to 12 point
\vspace{10pt}  %%%% This makes a vertical space of 10 points (approx 70 points/inch) after the line break

\noindent{\color{blue}\rule{500pt}{2pt}}


\noindent {\color{Maroon}\bf Definition:}
The \defn{Fibonacci sequence} $\{f_n\mid n\geq 1\}$ is defined by $f_1=f_2=1$ and for $n\geq 2$, $f_{n+1}=f_{n}+f_{n-1}$.

  
%%%%%%%%%%%%%%%%%%%%%%%%%%%%%%%%%%%%%%%%%%%%%%%%%%%%%%%%%%%%%%%%%%%%%%%%%%%%%%%%%
\begin{enumerate}


%\newpage
%%%%%%%%%%%%%%%%%%%%%%%%%%%%%%%%%%%%%%%%%%%%%%%%%%%%%%%%%%%%%%%%%%%%%%%%%%%%%%%%%
\item{[20]} Please do parts (a) and (b) of Problem 17 for Section 4.3 on page 198 in the .pdf of the Sundstrom book.
%%%%%%%%%%%%%%%%%%%%%%%%%%%%%%%%%%%%%%%%%%%%%%%%%%%%%%%%%%%%%%%%%%%%%%%%%%%%%%%%%

%\newpage
%%%%%%%%%%%%%%%%%%%%%%%%%%%%%%%%%%%%%%%%%%%%%%%%%%%%%%%%%%%%%%%%%%%%%%%%%%%%%%%%%
\item{[10]} Write a proof in paragraph form of the inequality
$3^n > 1+2^n$  for $n\geq 2$ using mathematical induction.

%%%%%%%%%%%%%%%%%%%%%%%%%%%%%%%%%%%%%%%%%%%%%%%%%%%%%%%%%%%%%%%%%%%%%%%%%%%%%%%%%
 
%\newpage
%%%%%%%%%%%%%%%%%%%%%%%%%%%%%%%%%%%%%%%%%%%%%%%%%%%%%%%%%%%%%%%%%%%%%%%%%%%%%%%%%
\item{[10]}  Find a number $M$ and which of $\{<, =,>\}$ such that
  $\forall n\geq M, \ 2^n\ (<, =,>)\ n!$ is true, and then write a proof in paragraph form of this assertion
  using mathematical induction.
  
%%%%%%%%%%%%%%%%%%%%%%%%%%%%%%%%%%%%%%%%%%%%%%%%%%%%%%%%%%%%%%%%%%%%%%%%%%%%%%%%%

%\newpage
%%%%%%%%%%%%%%%%%%%%%%%%%%%%%%%%%%%%%%%%%%%%%%%%%%%%%%%%%%%%%%%%%%%%%%%%%%%%%%%%%
\item{[10]} For which natural numbers $n$ do there exist nonnegative integers $x$ and $y$ such that $n=4x+7y$?
   Justify your conclusion.
%%%%%%%%%%%%%%%%%%%%%%%%%%%%%%%%%%%%%%%%%%%%%%%%%%%%%%%%%%%%%%%%%%%%%%%%%%%%%%%%%

%\newpage
%%%%%%%%%%%%%%%%%%%%%%%%%%%%%%%%%%%%%%%%%%%%%%%%%%%%%%%%%%%%%%%%%%%%%%%%%%%%%%%%%
 \item{[10]} Consider the sequence $\{a_n\mid n\in\NN\}$  defined by $a_1=1$, $a_2=3$ and for each $n\in \NN$,
           $a_{n+2}=3 a_{n+1}-2a_n$.
   Calculate the first eight elements of this sequence. 

   Conjecture a formula for $a_n$ and prove it using induction.
%%%%%%%%%%%%%%%%%%%%%%%%%%%%%%%%%%%%%%%%%%%%%%%%%%%%%%%%%%%%%%%%%%%%%%%%%%%%%%%%%

%\newpage
%%%%%%%%%%%%%%%%%%%%%%%%%%%%%%%%%%%%%%%%%%%%%%%%%%%%%%%%%%%%%%%%%%%%%%%%%%%%%%%%%
 \item{[10]} Consider the sequence $\{a_n\mid n\in\NN\}$  defined by $a_1=a_2=1$ and for each $n\in \NN$,
           $a_{n+2}=\frac{1}{2}\left(a_{n+1}+\frac{2}{a_n}\right)$.
   Calculate the first six elements of this sequence.

   Prove, for all $n\in\NN$, that $1\leq a_n\leq 2$.
%%%%%%%%%%%%%%%%%%%%%%%%%%%%%%%%%%%%%%%%%%%%%%%%%%%%%%%%%%%%%%%%%%%%%%%%%%%%%%%%%

%\newpage
%%%%%%%%%%%%%%%%%%%%%%%%%%%%%%%%%%%%%%%%%%%%%%%%%%%%%%%%%%%%%%%%%%%%%%%%%%%%%%%%%
\item{[10]}  Compute the first 15 terms of the Fibonacci sequence.
           Note that the recursion $f_{n+1}=f_{n}+f_{n-1}$ may be rewritten $f_{n-1}=f_{n+1}-f_n$.
           Use this to extend the Fibonacci sequence to {\sl negative} integers and compute the values of $f_n$ for
           $-10\leq n \leq 0$.
           Conjecture a formula for $f_{-n}$ for $n\in\NN$ and prove it by induction.
%%%%%%%%%%%%%%%%%%%%%%%%%%%%%%%%%%%%%%%%%%%%%%%%%%%%%%%%%%%%%%%%%%%%%%%%%%%%%%%%%

%\newpage
%%%%%%%%%%%%%%%%%%%%%%%%%%%%%%%%%%%%%%%%%%%%%%%%%%%%%%%%%%%%%%%%%%%%%%%%%%%%%%%%%
\item{[10]} Prove that for every $n\in\NN$,   $f_{5n}$ is a multiple of 5.
%%%%%%%%%%%%%%%%%%%%%%%%%%%%%%%%%%%%%%%%%%%%%%%%%%%%%%%%%%%%%%%%%%%%%%%%%%%%%%%%%

%\newpage
%%%%%%%%%%%%%%%%%%%%%%%%%%%%%%%%%%%%%%%%%%%%%%%%%%%%%%%%%%%%%%%%%%%%%%%%%%%%%%%%%
\item{[10]} Look up the term \defn{Pythagrean triple} (it is in our book).
    Investigate the following\medskip

  {\bf Conjecture.}  {\sl For each natural number $n$, the numbers $f_nf_{n+3}$,
  $2f_{n+1}f_{n+2}$, and $(f_{n+1}^2+f_{n+2}^2)$ form a {\color{Blue}Pythagorean triple}.}\medskip

   If true, provide a proof, and if false, a counterexample.
%%%%%%%%%%%%%%%%%%%%%%%%%%%%%%%%%%%%%%%%%%%%%%%%%%%%%%%%%%%%%%%%%%%%%%%%%%%%%%%%%



%\newpage
%%%%%%%%%%%%%%%%%%%%%%%%%%%%%%%%%%%%%%%%%%%%%%%%%%%%%%%%%%%%%%%%%%%%%%%%%%%%%%%%%
\item{[20]} {\color{Magenta}Extra credit for the ambitious and bored.}
          For $k>2$, find (and prove) a formula for $f_{n+k}$ in terms of $f_n$ and $f_{n+1}$.
          Use it to give a proof by induction that for all $n,k\in\NN$ the
          Fibonacci number $f_{nk}$ is a multiple of $f_n$.

         {\color{Magenta}There will be a special assignment for this on Gradescope; hand it in separately.
         Your proof needs to be correct for full credit.}
%%%%%%%%%%%%%%%%%%%%%%%%%%%%%%%%%%%%%%%%%%%%%%%%%%%%%%%%%%%%%%%%%%%%%%%%%%%%%%%%%


\end{enumerate}
%%%%%%%%%%%%%%%%%%%%%%%%%%%%%%%%%%%%%%%%%%%%%%%%%%%%%%%%%%%%%%%%%%%%%%%%%%%%%%%%%


\end{document}
%%%%%%%%%%%%%%%%%%%%%%%%%%%%%%%%%%%%%%%%%%%%%%%%%%%%%%%%%%%%%%%%%%%%%%%%%%%%%%%%%


