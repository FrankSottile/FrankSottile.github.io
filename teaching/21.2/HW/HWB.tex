%HWB.tex
%Eleventh Homework -- Math 300H 
%
%  The percent sign is a comment character
%
%%%%%%%%%%%%%%%%%%%%%%%%%%%%%%%%%%%%%%%%%%%%%%%%%%%%%%%%%%%%%%%%%%%%%%%%%%%%%%%%%%
%
%   Look these up on line.  The first sets the type of document, and the next are for mathematics symbols, graphics and color
%
\documentclass[12pt]{article}
\usepackage{amssymb,amsmath,mathtools}
\usepackage{graphicx}
\usepackage[usenames,dvipsnames,svgnames,table]{xcolor}
\usepackage{multirow}   % This is for more control over tables
%%%%%%%%%%%%%%%%%%%%%%%%%%%%%%%%  Layout     %%%%%%%%%%%%%%%%%%%%%%%%%%%%%%%%%%%%%%
\usepackage{vmargin}
\setpapersize{USletter}
\setmargrb{1cm}{1cm}{2cm}{1cm} % --- sets all four margins LTRB


\newcommand{\RR}{{\mathbb R}}  % This is the backboard symbol for the real numbers.  Note how it is used below
\newcommand{\NN}{{\mathbb N}}  % 
\newcommand{\QQ}{{\mathbb Q}}  % 
\newcommand{\ZZ}{{\mathbb Z}}  %

\newcommand{\calP}{{\mathcal P}}  %Caligraphic P for power set

\newcommand{\sep}{{\ :\ }}     % This is for the : in our notation for building sets.
\newcommand{\lsim}{\mathord{\sim}}  %  This is to remove the extra spacing around \sim (it is a binary relation) for the logical not

\newcommand{\defcolor}[1]{{\color{blue}{#1}}}
\newcommand{\demph}[1]{{\color{blue}\sl{#1}}}
%%%%%%%%%%%%%%%%%%%%%%%%%%%%%%%%%%%%%%%%%%%%%%%%%%%%%%%%%%%%%%%%%%%%%%%%%%%%%%%%%
%
%   Please edit this as appropriate
%
\begin{document}
\LARGE 
\noindent
{\color{Maroon}Foundations of Mathematics \hfill Math 300H Section 970}\vspace{2pt}\\
\Large \vspace{2pt}\\
\large
Eleventh Homework: \hfill Due \  29 November 2021
\normalsize\medskip


\noindent{\color{blue}\rule{528.3675pt}{2pt}}


%%%%%%%%%%%%%%%%%%%%%%%%%%%%%%%%%%%%%%%%%%%%%%%%%%%%%%%%%%%%%%%%%%%%%%%%%%%%%%%%%
\begin{enumerate}  %  the \begin{..}  \end{..} stars and ends an environment.



%%%%%%%%%%%%%%%%%%%%%%%%%%%%%%%%%%%%%%%%%%%%%%%%%%%%%%%%%%%%%%%%%%%%%%%%%%%%%%%%%
\item Let $A$, $B$, and $C$ be nonempty sets, and suppose that  $f\colon A\to B$ and $g\colon B\to C$ are functions.
  Suppose that $g\circ f\colon A\to C$ is an injection.
  Prove that $f$ is an injection.

  Give an example of functions $f$ and $g$ with these properties illustrating that $g$ need not be an injection.
%%%%%%%%%%%%%%%%%%%%%%%%%%%%%%%%%%%%%%%%%%%%%%%%%%%%%%%%%%%%%%%%%%%%%%%%%%%%%%%%%


%%%%%%%%%%%%%%%%%%%%%%%%%%%%%%%%%%%%%%%%%%%%%%%%%%%%%%%%%%%%%%%%%%%%%%%%%%%%%%%%%
\item Let $A$, $B$, and $C$ be nonempty sets, and suppose that  $f\colon A\to B$ and $g\colon B\to C$ are functions.
  Suppose that $g\circ f\colon A\to C$ is a surjection.
  Prove that $g$ is a surjection.

  Give an example of functions $f$ and $g$ with these properties illustrating that $f$ need not be a surjection.
%%%%%%%%%%%%%%%%%%%%%%%%%%%%%%%%%%%%%%%%%%%%%%%%%%%%%%%%%%%%%%%%%%%%%%%%%%%%%%%%%


%%%%%%%%%%%%%%%%%%%%%%%%%%%%%%%%%%%%%%%%%%%%%%%%%%%%%%%%%%%%%%%%%%%%%%%%%%%%%%%%%
\item Suppose that $S$ and $T$ are nonempty sets.
  Let $f\colon S\to T$ be a function, $A,B$ be subsets of $S$ and $C,D$ be subsets of $T$.
  For $x\in S$ and $y\in T$, carefully explain what is means to say that 
  
 \begin{enumerate}
  \item $y\in f(A\cup B)$.
  \item $y\in f(A)\cap f(B)$.
  \item $x\in f^{-1}(C\cap D)$.
  \item $x\in f^{-1}(C) \cup f^{-1}(D)$.
 \end{enumerate}  
%%%%%%%%%%%%%%%%%%%%%%%%%%%%%%%%%%%%%%%%%%%%%%%%%%%%%%%%%%%%%%%%%%%%%%%%%%%%%%%%%


%%%%%%%%%%%%%%%%%%%%%%%%%%%%%%%%%%%%%%%%%%%%%%%%%%%%%%%%%%%%%%%%%%%%%%%%%%%%%%%%%
\item Let $A$, $B$, and $C$ be nonempty sets, and suppose that  $f\colon A\to B$, $g\colon B\to C$, and $h\colon B\to C$ are functions.
  For each of the following, prove or disprove:
  
 \begin{enumerate}
  \item  If $g\circ f= h\circ f$, then $g=h$.
  \item  If $f$ is one-to-one and $g\circ f= h\circ f$, then $g=h$.
 \end{enumerate}  
%%%%%%%%%%%%%%%%%%%%%%%%%%%%%%%%%%%%%%%%%%%%%%%%%%%%%%%%%%%%%%%%%%%%%%%%%%%%%%%%%


%%%%%%%%%%%%%%%%%%%%%%%%%%%%%%%%%%%%%%%%%%%%%%%%%%%%%%%%%%%%%%%%%%%%%%%%%%%%%%%%%
\item Let $\alpha=\begin{pmatrix} 1&2&3&4&5\\
                                  2&3&4&5&1 \end{pmatrix}$,
   $\beta=\begin{pmatrix} 1&2&3&4&5\\
                          2&5&1&4&3 \end{pmatrix}$, and
  $\gamma=\begin{pmatrix} 1&2&3&4&5\\
                          3&2&5&4&1 \end{pmatrix}$.
  Using the definitions given in the book, compute
  \[
  \alpha\circ \beta\,,\
  \beta\circ\alpha\,,\
  \alpha\circ\gamma\,,\
  \gamma\circ\beta\,,\
  \beta^{-1}\,.
  \]
%%%%%%%%%%%%%%%%%%%%%%%%%%%%%%%%%%%%%%%%%%%%%%%%%%%%%%%%%%%%%%%%%%%%%%%%%%%%%%%%%



%%%%%%%%%%%%%%%%%%%%%%%%%%%%%%%%%%%%%%%%%%%%%%%%%%%%%%%%%%%%%%%%%%%%%%%%%%%%%%%%%
\item Given a permutation $\sigma$, write \defcolor{$\sigma^2$} for $\sigma\circ\sigma$, and the obvious for higher powers.
   E.G.\ ($\sigma^3=\sigma\circ\sigma^2$, and etc.)
  For $\alpha,\beta,\gamma$ of Question 5, compute
  $\alpha^2,\alpha^3,\alpha^4,\alpha^5$, and do the same for $\beta$ and $\gamma$.
%%%%%%%%%%%%%%%%%%%%%%%%%%%%%%%%%%%%%%%%%%%%%%%%%%%%%%%%%%%%%%%%%%%%%%%%%%%%%%%%%



%%%%%%%%%%%%%%%%%%%%%%%%%%%%%%%%%%%%%%%%%%%%%%%%%%%%%%%%%%%%%%%%%%%%%%%%%%%%%%%%%
\item {\color{Maroon}{\bf Bonus.}}
      The \demph{order} of a permutation $\sigma$ is the least positive number $k$ such that $\sigma^k=e$, where $e$ is the identity
  permutation.
  (For example, the order of $\alpha$ is 5.)

  What are the orders of $\beta$ and $\gamma$?   How about the other permutations  computed in problem 6?
  ($\alpha\circ \beta$ and etc.)

  What is the maximal order of a permutation in $S_5$?  What is minimal order?
%%%%%%%%%%%%%%%%%%%%%%%%%%%%%%%%%%%%%%%%%%%%%%%%%%%%%%%%%%%%%%%%%%%%%%%%%%%%%%%%%

\end{enumerate}
%%%%%%%%%%%%%%%%%%%%%%%%%%%%%%%%%%%%%%%%%%%%%%%%%%%%%%%%%%%%%%%%%%%%%%%%%%%%%%%%%%%%%%%%%%%%%%%%%%%%

\end{document}

  

%%%%%%%%%%%%%%%%%%%%%%%%%%%%%%%%%%%%%%%%%%%%%%%%%%%%%%%%%%%%%%%%%%%%%%%%%%%%%%%%%
\item \qquad
%%%%%%%%%%%%%%%%%%%%%%%%%%%%%%%%%%%%%%%%%%%%%%%%%%%%%%%%%%%%%%%%%%%%%%%%%%%%%%%%%
 

%%%%%%%%%%%%%%%%%%%%%%%%%%%%%%%%%%%%%%%%%%%%%%%%%%%%%%%%%%%%%%%%%%%%%%%%%%%%%%%%%
\item \ 
%%%%%%%%%%%%%%%%%%%%%%%%%%%%%%%%%%%%%%%%%%%%%%%%%%%%%%%%%%%%%%%%%%%%%%%%%%%%%%%%%
