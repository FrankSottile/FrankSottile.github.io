%HWA.tex
%Tenth Homework -- Math 300H 
%
%  The percent sign is a comment character
%
%%%%%%%%%%%%%%%%%%%%%%%%%%%%%%%%%%%%%%%%%%%%%%%%%%%%%%%%%%%%%%%%%%%%%%%%%%%%%%%%%%
%
%   Look these up on line.  The first sets the type of document, and the next are for mathematics symbols, graphics and color
%
\documentclass[12pt]{article}
\usepackage{amssymb,amsmath,mathtools}
\usepackage{graphicx}
\usepackage[usenames,dvipsnames,svgnames,table]{xcolor}
\usepackage{multirow}   % This is for more control over tables
%%%%%%%%%%%%%%%%%%%%%%%%%%%%%%%%  Layout     %%%%%%%%%%%%%%%%%%%%%%%%%%%%%%%%%%%%%%
\usepackage{vmargin}
\setpapersize{USletter}
\setmargrb{1cm}{1cm}{2cm}{1cm} % --- sets all four margins LTRB


\newcommand{\RR}{{\mathbb R}}  % This is the backboard symbol for the real numbers.  Note how it is used below
\newcommand{\NN}{{\mathbb N}}  % 
\newcommand{\QQ}{{\mathbb Q}}  % 
\newcommand{\ZZ}{{\mathbb Z}}  %

\newcommand{\calP}{{\mathcal P}}  %Caligraphic P for power set

\newcommand{\sep}{{\ :\ }}     % This is for the : in our notation for building sets.
\newcommand{\lsim}{\mathord{\sim}}  %  This is to remove the extra spacing around \sim (it is a binary relation) for the logical not

\newcommand{\defcolor}[1]{{\color{blue}{#1}}}
\newcommand{\demph}[1]{{\color{blue}\sl{#1}}}
%%%%%%%%%%%%%%%%%%%%%%%%%%%%%%%%%%%%%%%%%%%%%%%%%%%%%%%%%%%%%%%%%%%%%%%%%%%%%%%%%
%
%   Please edit this as appropriate
%
\begin{document}
\LARGE 
\noindent
{\color{Maroon}Foundations of Mathematics \hfill Math 300H Section 970}\vspace{2pt}\\
\Large \vspace{2pt}\\
\large
Tenth Homework: \hfill Due \  15 November 2021
\normalsize\medskip


\noindent{\color{blue}\rule{528.3675pt}{2pt}}


%%%%%%%%%%%%%%%%%%%%%%%%%%%%%%%%%%%%%%%%%%%%%%%%%%%%%%%%%%%%%%%%%%%%%%%%%%%%%%%%%
\begin{enumerate}  %  the \begin{..}  \end{..} stars and ends an environment.

   
%%%%%%%%%%%%%%%%%%%%%%%%%%%%%%%%%%%%%%%%%%%%%%%%%%%%%%%%%%%%%%%%%%%%%%%%%%%%%%%%%
\item For $n\in \NN$, let $s(n)$ denote the sum of the digits of $n$, expressed in base 10.
  That is, if we write $n=a_k\dotsc a_1 a_0$ in base 10 so that
  \[
     n\ =\ \left(a_k\cdot 10^k\right) +   \left(a_{k-1}\cdot 10^{k-1}\right) +\dotsb + 
     \left(a_1\cdot 10\right) + a_0\,,
  \]
  then $s(n)=a_k+\dotsb+a_1+a_0$.
   \begin{enumerate}
     \item Use mathematical induction (or any other means, there are several valid proofs) to prove that for all $m\in\NN$, $10^m\equiv 1\mod 9$.
     Thus $[10^m]_9=[1]_9$.

     \item Use this to prove that $[n]_9 = [s(n)]_9$ and deduce that $9|n$ if and only if $9|s(n)$.

     \item Show that for $a,b\in\ZZ$, we have $[a+ b]_9=[s(a)+ s(b)]_9$ and $[a\cdot b]_9=[s(a)\cdot s(b)]_9$.
       This is the idea behind \demph{casting out nines}.
       Do not forget that you can use previously proven results in your proofs.
   \end{enumerate}
%%%%%%%%%%%%%%%%%%%%%%%%%%%%%%%%%%%%%%%%%%%%%%%%%%%%%%%%%%%%%%%%%%%%%%%%%%%%%%%%%


%%%%%%%%%%%%%%%%%%%%%%%%%%%%%%%%%%%%%%%%%%%%%%%%%%%%%%%%%%%%%%%%%%%%%%%%%%%%%%%%%
\item {\bf Proof Analysis.}   Which if the following is true for the proof below (is the statement true or false, and is
  the proof correct or incorrect.) ?

  {\bf Statement:} Every symmetric and transitive relation on a nonempty set $A$ is an equivalence relation.

  {\bf Proof:} Let $R$ be a symmetric and transitive relation on a nonempty set $A$.
  We need only to show that $R$ is reflexive.
  Let $x\in A$.
  Let $y\in A$ be such that $xRy$.
  As $R$ is symmetric, we have $yRx$.
  Now $xRy$ and $yRx$, so we conclude that $xRx$, by transitivity.
  Thus, $R$ is reflexive. \hfill $\Box$
%%%%%%%%%%%%%%%%%%%%%%%%%%%%%%%%%%%%%%%%%%%%%%%%%%%%%%%%%%%%%%%%%%%%%%%%%%%%%%%%%

  

%%%%%%%%%%%%%%%%%%%%%%%%%%%%%%%%%%%%%%%%%%%%%%%%%%%%%%%%%%%%%%%%%%%%%%%%%%%%%%%%%
\item  Let $A$ be a nonempty set.
  Suppose that $R$ is a relation from $A$ to $A$ that is both an equivalence relation and a function.
  What familiar function is $R$?
  Justify your answer.  
%%%%%%%%%%%%%%%%%%%%%%%%%%%%%%%%%%%%%%%%%%%%%%%%%%%%%%%%%%%%%%%%%%%%%%%%%%%%%%%%%

%%%%%%%%%%%%%%%%%%%%%%%%%%%%%%%%%%%%%%%%%%%%%%%%%%%%%%%%%%%%%%%%%%%%%%%%%%%%%%%%%
\item   Let $\nu$ be the function from $\NN$ to $\NN$ whose value at a positive integer $n$ is the number of digits in the American
  English spelling of the number $n$.
  For example $\nu(0)=4$, as `0' is written {\sf zero} with four letters.
  Similarly, $\nu(22)=9$, as {\sf twentytwo} has nine letters.

  If we restrict the domain of $\nu$ to $\{1,2,\dotsc,20\}$, what is its range?
%%%%%%%%%%%%%%%%%%%%%%%%%%%%%%%%%%%%%%%%%%%%%%%%%%%%%%%%%%%%%%%%%%%%%%%%%%%%%%%%%

%\newpage
%%%%%%%%%%%%%%%%%%%%%%%%%%%%%%%%%%%%%%%%%%%%%%%%%%%%%%%%%%%%%%%%%%%%%%%%%%%%%%%%%
\item A \demph{real function} is one whose domain and codomain are subsets of $\RR$.
  For each of the following real functions, determine their largest possible domain and their range.

  (a) The function $f$ defined by $f(x)=x/(x^2-3x-2)$.

  (b) The function $g$ defined by $g(x)=\ln(1-\cos(x))$.
%%%%%%%%%%%%%%%%%%%%%%%%%%%%%%%%%%%%%%%%%%%%%%%%%%%%%%%%%%%%%%%%%%%%%%%%%%%%%%%%%


%\newpage
%%%%%%%%%%%%%%%%%%%%%%%%%%%%%%%%%%%%%%%%%%%%%%%%%%%%%%%%%%%%%%%%%%%%%%%%%%%%%%%%%
\item  Let $s\colon\NN\to\NN$ be the function whose value $s(n)$ at a number $n$ is the sum of the distinct natural number divisors of
  $n$. 
  Compute the values of $s$ on the set $\{1,\dotsc,10\}$.
  Is the function $s$ injective?  Is it surjective?  Justify your conclusions.
%%%%%%%%%%%%%%%%%%%%%%%%%%%%%%%%%%%%%%%%%%%%%%%%%%%%%%%%%%%%%%%%%%%%%%%%%%%%%%%%%



%\newpage
%%%%%%%%%%%%%%%%%%%%%%%%%%%%%%%%%%%%%%%%%%%%%%%%%%%%%%%%%%%%%%%%%%%%%%%%%%%%%%%%%
\item  Let $d\colon\NN\to\NN$ be the function whose value $d(n)$ at a number $n$ is the number of distinct natural number divisors of
  $n$. 
  Compute the values of $d$ on the set $\{1,\dotsc,10\}$.
  Is the function $d$ injective?  Is it surjective?  Justify your conclusions.
%%%%%%%%%%%%%%%%%%%%%%%%%%%%%%%%%%%%%%%%%%%%%%%%%%%%%%%%%%%%%%%%%%%%%%%%%%%%%%%%%

  
%%%%%%%%%%%%%%%%%%%%%%%%%%%%%%%%%%%%%%%%%%%%%%%%%%%%%%%%%%%%%%%%%%%%%%%%%%%%%%%%%
\item For a function $f\colon A\to B$ and subsets $C$ anbd $D$ of $A$ and $E$ and $F$ of $B$, prove the following:

 \begin{enumerate}
  \item  $f(C\cap D)\subseteq f(C)\cap f(D)$.

  \item  $f^{-1}(E-F)= f^{-1}(E) - f^{-1}(F)$.

 \end{enumerate}  
%%%%%%%%%%%%%%%%%%%%%%%%%%%%%%%%%%%%%%%%%%%%%%%%%%%%%%%%%%%%%%%%%%%%%%%%%%%%%%%%%

  
%%%%%%%%%%%%%%%%%%%%%%%%%%%%%%%%%%%%%%%%%%%%%%%%%%%%%%%%%%%%%%%%%%%%%%%%%%%%%%%%%
  \item  Let $A$ and $B$ be sets.
    Recall the definitions of the identity functions $I_A\colon A\to A$ and  $I_B\colon B\to B$:
    For $a\in A$, $I_A(a)=a$ and for $b\in B$, $I_B(b)=b$.

    Let $f\colon A\to B$ be a function.
    Prove by a direct computation that $f=f\circ I_A$ and that $f=I_B\circ f$.
%%%%%%%%%%%%%%%%%%%%%%%%%%%%%%%%%%%%%%%%%%%%%%%%%%%%%%%%%%%%%%%%%%%%%%%%%%%%%%%%%

%%%%%%%%%%%%%%%%%%%%%%%%%%%%%%%%%%%%%%%%%%%%%%%%%%%%%%%%%%%%%%%%%%%%%%%%%%%%%%%%%
  \item  Let $A$ be a set.  Prove that the identity function $I_A$ is a bijection.
%%%%%%%%%%%%%%%%%%%%%%%%%%%%%%%%%%%%%%%%%%%%%%%%%%%%%%%%%%%%%%%%%%%%%%%%%%%%%%%%%

%%%%%%%%%%%%%%%%%%%%%%%%%%%%%%%%%%%%%%%%%%%%%%%%%%%%%%%%%%%%%%%%%%%%%%%%%%%%%%%%%
\item For each of the following, either give an example of functions $f\colon A\to B$ and $g\colon B\to C$ that satisfy the given
  properties, or explain why no such example exists.

 \begin{enumerate}
  \item The function $g$ is a surjection, but the function $g\circ f$ is not a surjection.

  \item The function $g$ is an injection, but the function $g\circ f$ is not an injection.

  \item The function $f$ is not a surjection, but the function $g\circ f$ is a surjection.

  \item The function $g$ is not an injection, but the function $g\circ f$ is an injection.

  \item The function $g$ is not an injection, but the function $g\circ f$ is a surjection.

 \end{enumerate}  

%%%%%%%%%%%%%%%%%%%%%%%%%%%%%%%%%%%%%%%%%%%%%%%%%%%%%%%%%%%%%%%%%%%%%%%%%%%%%%%%%


%%%%%%%%%%%%%%%%%%%%%%%%%%%%%%%%%%%%%%%%%%%%%%%%%%%%%%%%%%%%%%%%%%%%%%%%%%%%%%%%%
\item  For functions $f$, $g$, and $h$ with domain and codomain $\RR$, prove or disprove the following:
  
 \begin{enumerate}
    \item   $(g+h)\circ f = (g\circ f) + (h\circ f)$.

    \item   $f\circ(g+h) = (f\circ g) + (f\circ h)$.   

 \end{enumerate}  

  {\bf Definition:} The sum of two $g$ and $h$  with domain and codomain $\RR$ is defined to be the function $g+h$ whose value at a number
  $x\in \RR$ is $g(x)+h(x)$.
 
%%%%%%%%%%%%%%%%%%%%%%%%%%%%%%%%%%%%%%%%%%%%%%%%%%%%%%%%%%%%%%%%%%%%%%%%%%%%%%%%%

\end{enumerate}
%%%%%%%%%%%%%%%%%%%%%%%%%%%%%%%%%%%%%%%%%%%%%%%%%%%%%%%%%%%%%%%%%%%%%%%%%%%%%%%%%%%%%%%%%%%%%%%%%%%%

\end{document}

  

%%%%%%%%%%%%%%%%%%%%%%%%%%%%%%%%%%%%%%%%%%%%%%%%%%%%%%%%%%%%%%%%%%%%%%%%%%%%%%%%%
\item \ 
%%%%%%%%%%%%%%%%%%%%%%%%%%%%%%%%%%%%%%%%%%%%%%%%%%%%%%%%%%%%%%%%%%%%%%%%%%%%%%%%%
