%HWB.tex
%Eleventh Homework -- Math 300H 
%
%  The percent sign is a comment character
%
%%%%%%%%%%%%%%%%%%%%%%%%%%%%%%%%%%%%%%%%%%%%%%%%%%%%%%%%%%%%%%%%%%%%%%%%%%%%%%%%%%
%
%   Look these up on line.  The first sets the type of document, and the next are for mathematics symbols, graphics and color
%
\documentclass[12pt]{article}
\usepackage{amssymb,amsmath,mathtools}
\usepackage{graphicx}
\usepackage[usenames,dvipsnames,svgnames,table]{xcolor}
\usepackage{multirow}   % This is for more control over tables
%%%%%%%%%%%%%%%%%%%%%%%%%%%%%%%%  Layout     %%%%%%%%%%%%%%%%%%%%%%%%%%%%%%%%%%%%%%
\usepackage{vmargin}
\setpapersize{USletter}
\setmargrb{1cm}{1cm}{2cm}{1cm} % --- sets all four margins LTRB


\newcommand{\RR}{{\mathbb R}}  % This is the backboard symbol for the real numbers.  Note how it is used below
\newcommand{\NN}{{\mathbb N}}  % 
\newcommand{\QQ}{{\mathbb Q}}  % 
\newcommand{\ZZ}{{\mathbb Z}}  %

\newcommand{\calP}{{\mathcal P}}  %Caligraphic P for power set

\newcommand{\sep}{{\ :\ }}     % This is for the : in our notation for building sets.
\newcommand{\lsim}{\mathord{\sim}}  %  This is to remove the extra spacing around \sim (it is a binary relation) for the logical not

\newcommand{\defcolor}[1]{{\color{blue}{#1}}}
\newcommand{\demph}[1]{{\color{blue}\sl{#1}}}
%%%%%%%%%%%%%%%%%%%%%%%%%%%%%%%%%%%%%%%%%%%%%%%%%%%%%%%%%%%%%%%%%%%%%%%%%%%%%%%%%
%
%   Please edit this as appropriate
%
\begin{document}
\LARGE 
\noindent
{\color{Maroon}Foundations of Mathematics \hfill Math 300H Section 970}\vspace{2pt}\\
\Large \vspace{2pt}\\
\large
Eleventh Homework: \hfill Due \  22 November 2022
\normalsize\medskip


\noindent{\color{blue}\rule{528.3675pt}{2pt}}


%%%%%%%%%%%%%%%%%%%%%%%%%%%%%%%%%%%%%%%%%%%%%%%%%%%%%%%%%%%%%%%%%%%%%%%%%%%%%%%%%
\begin{enumerate}  %  the \begin{..}  \end{..} stars and ends an environment.


%%%%%%%%%%%%%%%%%%%%%%%%%%%%%%%%%%%%%%%%%%%%%%%%%%%%%%%%%%%%%%%%%%%%%%%%%%%%%%%%%
\item  Let $H$ be a nonempty subset of $\ZZ$.
  Suppose that the relation $R$ defined on $\ZZ$ by $a R b$ if $a - b\in  H$ is an equivalence relation.
  Verify the following
  \begin{enumerate}
    \item $0\in H$.
    \item If $a\in H$, then $-a\in H$.
    \item If $a,b\in H$, then $a+b\in H$.
  \end{enumerate}
%%%%%%%%%%%%%%%%%%%%%%%%%%%%%%%%%%%%%%%%%%%%%%%%%%%%%%%%%%%%%%%%%%%%%%%%%%%%%%%%%


%%%%%%%%%%%%%%%%%%%%%%%%%%%%%%%%%%%%%%%%%%%%%%%%%%%%%%%%%%%%%%%%%%%%%%%%%%%%%%%%%
\item Let $R$ be a relation defined on the set $\NN$ by $a R b$ if either $a \mid 2b$ or $b \mid 2a$.
       Prove or disprove: $R$ is an equivalence relation.
%%%%%%%%%%%%%%%%%%%%%%%%%%%%%%%%%%%%%%%%%%%%%%%%%%%%%%%%%%%%%%%%%%%%%%%%%%%%%%%%%


 
%\newpage
%%%%%%%%%%%%%%%%%%%%%%%%%%%%%%%%%%%%%%%%%%%%%%%%%%%%%%%%%%%%%%%%%%%%%%%%%%%%%%%%%%
%  \item  Determine all the congruence classes (equivalence classes) for the relation on the integers $\ZZ$ of congruence modulo 5.
%%%%%%%%%%%%%%%%%%%%%%%%%%%%%%%%%%%%%%%%%%%%%%%%%%%%%%%%%%%%%%%%%%%%%%%%%%%%%%%%%
  
%%%%%%%%%%%%%%%%%%%%%%%%%%%%%%%%%%%%%%%%%%%%%%%%%%%%%%%%%%%%%%%%%%%%%%%%%%%%%%%%%
\item The relation $R$ on $\ZZ$ defined by $a R b$ if $a^4 \equiv b^4 \mod 8$ is known to be an equivalence relation.
  Determine the distinct equivalence classes.
%%%%%%%%%%%%%%%%%%%%%%%%%%%%%%%%%%%%%%%%%%%%%%%%%%%%%%%%%%%%%%%%%%%%%%%%%%%%%%%%%

 %%%%%%%%%%%%%%%%%%%%%%%%%%%%%%%%%%%%%%%%%%%%%%%%%%%%%%%%%%%%%%%%%%%%%%%%%%%%%%%%%
\item In $\ZZ_{11}$, express the following sums and products as $[r]$, where $0\leq r < 11$.
  
  (a) $[7] + [5]$ \quad
  (b) $[7]\cdot [5]$ \quad
  (c) $[-82] + [207]$ \quad
  (d) $[-82] \cdot [207]$\,.
%%%%%%%%%%%%%%%%%%%%%%%%%%%%%%%%%%%%%%%%%%%%%%%%%%%%%%%%%%%%%%%%%%%%%%%%%%%%%%%%%
  
%\newpage
%%%%%%%%%%%%%%%%%%%%%%%%%%%%%%%%%%%%%%%%%%%%%%%%%%%%%%%%%%%%%%%%%%%%%%%%%%%%%%%%%
\item Compute the addition and multiplication tables for $\ZZ_5$.
%%%%%%%%%%%%%%%%%%%%%%%%%%%%%%%%%%%%%%%%%%%%%%%%%%%%%%%%%%%%%%%%%%%%%%%%%%%%%%%%%
  

%%%%%%%%%%%%%%%%%%%%%%%%%%%%%%%%%%%%%%%%%%%%%%%%%%%%%%%%%%%%%%%%%%%%%%%%%%%%%%%%%
\item Prove that the multiplication in $\ZZ_m$, for $m\geq 2$, defined by $[a]_m\cdot[b]_m=[a\cdot b]_m$ is well-defined.
  (See Result 4.11 in our book.)

  Hint: $ ab-cd = ab-cb + cb-cd = (a-c)b + c(b-d)$, the oldest trick in the book.
%%%%%%%%%%%%%%%%%%%%%%%%%%%%%%%%%%%%%%%%%%%%%%%%%%%%%%%%%%%%%%%%%%%%%%%%%%%%%%%%%

       
%%%%%%%%%%%%%%%%%%%%%%%%%%%%%%%%%%%%%%%%%%%%%%%%%%%%%%%%%%%%%%%%%%%%%%%%%%%%%%%%%
\item {\bf Proof Analysis.}   Which oxf the following is true for the proof below (is the statement true or false, and is
  the proof correct or incorrect.) ?

  {\bf Statement:} Every symmetric and transitive relation on a nonempty set $A$ is an equivalence relation.

  {\bf Proof:} Let $R$ be a symmetric and transitive relation on a nonempty set $A$.
  We need only to show that $R$ is reflexive.
  Let $x\in A$.
  Let $y\in A$ be such that $xRy$.
  As $R$ is symmetric, we have $yRx$.
  Now $xRy$ and $yRx$, so we conclude that $xRx$, by transitivity.
  Thus, $R$ is reflexive. \hfill $\Box$
%%%%%%%%%%%%%%%%%%%%%%%%%%%%%%%%%%%%%%%%%%%%%%%%%%%%%%%%%%%%%%%%%%%%%%%%%%%%%%%%%

  

%%%%%%%%%%%%%%%%%%%%%%%%%%%%%%%%%%%%%%%%%%%%%%%%%%%%%%%%%%%%%%%%%%%%%%%%%%%%%%%%%
\item  Let $A$ be a nonempty set.
  Suppose that $R$ is a relation from $A$ to $A$ that is both an equivalence relation and a function.
  What familiar function is $R$?
  Justify your answer.  
%%%%%%%%%%%%%%%%%%%%%%%%%%%%%%%%%%%%%%%%%%%%%%%%%%%%%%%%%%%%%%%%%%%%%%%%%%%%%%%%%

%%%%%%%%%%%%%%%%%%%%%%%%%%%%%%%%%%%%%%%%%%%%%%%%%%%%%%%%%%%%%%%%%%%%%%%%%%%%%%%%%
\item   Let $\nu$ be the function from $\NN$ to $\NN$ whose value at a positive integer $n$ is the number of digits in the American
  English spelling of the number $n$.
  For example $\nu(0)=4$, as `0' is written {\sf zero} with four letters.
  Similarly, $\nu(22)=9$, as {\sf twentytwo} has nine letters.

  If we restrict the domain of $\nu$ to $\{1,2,\dotsc,20\}$, what is its range?
%%%%%%%%%%%%%%%%%%%%%%%%%%%%%%%%%%%%%%%%%%%%%%%%%%%%%%%%%%%%%%%%%%%%%%%%%%%%%%%%%

%%%%%%%%%%%%%%%%%%%%%%%%%%%%%%%%%%%%%%%%%%%%%%%%%%%%%%%%%%%%%%%%%%%%%%%%%%%%%%%%%
\item A \demph{real function} is one whose domain and codomain are subsets of $\RR$.
  For each of the following real functions, determine their largest possible domain and their range.

  (a) The function $f$ defined by $f(x)=x/(x^2-3x-2)$.

  (b) The function $g$ defined by $g(x)=\ln(1-\cos(x))$.
%%%%%%%%%%%%%%%%%%%%%%%%%%%%%%%%%%%%%%%%%%%%%%%%%%%%%%%%%%%%%%%%%%%%%%%%%%%%%%%%%


  
\end{enumerate}
%%%%%%%%%%%%%%%%%%%%%%%%%%%%%%%%%%%%%%%%%%%%%%%%%%%%%%%%%%%%%%%%%%%%%%%%%%%%%%%%%%%%%%%%%%%%%%%%%%%%

\end{document}

  

%%%%%%%%%%%%%%%%%%%%%%%%%%%%%%%%%%%%%%%%%%%%%%%%%%%%%%%%%%%%%%%%%%%%%%%%%%%%%%%%%
\item \ 
%%%%%%%%%%%%%%%%%%%%%%%%%%%%%%%%%%%%%%%%%%%%%%%%%%%%%%%%%%%%%%%%%%%%%%%%%%%%%%%%%

%%%%%%%%%%%%%%%%%%%%%%%%%%%%%%%%%%%%%%%%%%%%%%%%%%%%%%%%%%%%%%%%%%%%%%%%%%%%%%%%%
\item For $n\in \NN$, let $s(n)$ denote the sum of the digits of $n$, expressed in base 10.
  That is, if we write $n=a_k\dotsc a_1 a_0$ in base 10 so that
  \[
     n\ =\ \left(a_k\cdot 10^k\right) +   \left(a_{k-1}\cdot 10^{k-1}\right) +\dotsb + 
     \left(a_1\cdot 10\right) + a_0\,,
  \]
  then $s(n)=a_k+\dotsb+a_1+a_0$.
   \begin{enumerate}
     \item Use mathematical induction (or any other means, there are several valid proofs) to prove that for all $m\in\NN$, $10^m\equiv 1\mod 9$.
     Thus $[10^m]_9=[1]_9$.

     \item Use this to prove that $[n]_9 = [s(n)]_9$ and deduce that $9|n$ if and only if $9|s(n)$.

     \item Show that for $a,b\in\NN$, we have $[a+ b]_9=[s(a)+ s(b)]_9$ and $[a\cdot b]_9=[s(a)\cdot s(b)]_9$.
       This is the idea behind \demph{casting out nines}.
       Do not forget that you can use previously proven results in your proofs.
   \end{enumerate}
%%%%%%%%%%%%%%%%%%%%%%%%%%%%%%%%%%%%%%%%%%%%%%%%%%%%%%%%%%%%%%%%%%%%%%%%%%%%%%%%%

