%HW11.tex
%
% Eleventh Homework for Graduate Algebra
% Frank Sottile
%%%%%%%%%%%%%%%%%%%%%%%%%%%%%%%%%%%%%%%%%%%%%%%%%%%%%%%%%%%%%%%%%%%%%%%
\documentclass[12pt]{article}
\usepackage{multicol,amsfonts, amssymb,  mathtools,amsmath}
\usepackage{colordvi,graphicx}
\headheight=8pt
%
\topmargin=-75pt
\textheight=720pt   \textwidth=560pt
\oddsidemargin=-60pt \evensidemargin=-60pt

\pagestyle{empty}

%%%%%%%%%%%%%%%%%%%%%%%%%%%%%%%%%%%%%%%%%%%%
\newcommand{\CC}{{\mathbb C}}
\newcommand{\KK}{{\mathbb K}}
\newcommand{\NN}{{\mathbb N}}
\newcommand{\QQ}{{\mathbb Q}}
\newcommand{\RR}{{\mathbb R}}
\newcommand{\TT}{{\mathbb T}}
\newcommand{\ZZ}{{\mathbb Z}}

\newcommand{\calA}{{\mathcal A}}
\newcommand{\be}{{\bf e}}
\newcommand{\bfi}{{\bf i}}
\newcommand{\bfj}{{\bf j}}

\newcommand{\Hom}{\mbox{Hom}}
\newcommand{\spec}{\mbox{spec}}
\newcommand{\cone}{\mbox{cone}}

\newcommand{\vect}[2]{(\begin{smallmatrix}#1\\#2\end{smallmatrix})}
\newcommand{\msp}{\hspace{8pt}}
%\newcommand{\Square}{\raisebox{-2pt}{\includegraphics{images/Square.eps}}}
\newcommand{\Square}{\raisebox{-2pt}{\Large$\square$}}

\def\Color#1#2{\special{color push cmyk #1}#2\special{color pop}}
%\def\Indigo#1{\Color{.42 1. 0. .49}{#1}}
\def\Indigo#1{\Color{1. .95 .05 .4}{#1}}
\def\MyViolet#1{\Color{.6 1. 0. .15}{#1}}
\def\TAMU#1{\Color{.15 1. .39 .69}{#1}}

\newcommand{\barsl}{\noindent\begin{minipage}[t]{590pt}
\Indigo{\rule{590pt}{1.2pt}}\vspace{-5.7mm}\\
\MyViolet{\rule{590pt}{1.2pt}}\vspace{-5.7mm}\\
\Blue{\rule{590pt}{1.2pt}}\vspace{-5.7mm}\\
\Green{\rule{590pt}{1.2pt}}\vspace{-5.7mm}\\
\Yellow{\rule{590pt}{1.2pt}}\vspace{-5.7mm}\\
\Orange{\rule{590pt}{1.2pt}}\vspace{-5.7mm}\\
\Red{\rule{590pt}{1.2pt}}\bigskip
\end{minipage}}


\newcommand{\barsn}{\noindent\begin{minipage}[t]{590pt}
\Indigo{\rule{590pt}{1.1pt}}\vspace{-4.5mm}\\
\MyViolet{\rule{590pt}{1.1pt}}\vspace{-4.5mm}\\
\Blue{\rule{590pt}{1.1pt}}\vspace{-4.5mm}\\
\Green{\rule{590pt}{1.1pt}}\vspace{-4.5mm}\\
\Yellow{\rule{590pt}{1.1pt}}\vspace{-4.5mm}\\
\Orange{\rule{590pt}{1.1pt}}\vspace{-4.5mm}\\
\Red{\rule{590pt}{1.1pt}}\bigskip
\end{minipage}}

\def\demph#1{\TAMU{{\sl #1}}}
\def\defcolor#1{\TAMU{#1}}

\begin{document}
\LARGE 
\noindent
Algebra \ \ Autumn 2025\vspace{1pt}\\
Frank Sottile\vspace{2pt}\\
\Large 13 November 2025 \hfill
\sf
 Eleventh Homework\makebox[20pt][l]{\ }
\normalsize\vspace{10pt}

\noindent
Write your answers neatly, in complete sentences.  
I highly recommend recopying your work before handing it in.
Correct and crisp proofs are greatly appreciated; oftentimes your work can be shortened and made clearer.

\barsn

\noindent\Maroon{{\sf Hand in at the start of class, Thursday 20 November:}} 

\begin{enumerate}
  \setcounter{enumi}{66}

%%%%%%%%%%%%%%%%%%%%%%%%%%%%%%%%%%%%%%%%%%%%%%%%%%%%%%%%%%%%%%%%%%%%%%%%%%%%%%%%%%%%%%%%%%%%%%%%%%%%
\item  
  Let  $SS=M_n(D)$ be the ring of $n\times n$ matrices over a division ring $D$.
  Show that $S$ has no proper ideals (Hint: use the elementary matrices $E_{a,b}$ whose only non-zero entry is in row $a$
  and column $b$---these form a `basis' of $S$.)
  
  Show that $S$ has zero divisorswhen $n>1$.

  Consequently, the zero-ideal $O$ is a prime ideal of $S$ that does not satisfy the conditions of prime ideal
  for commutative rings.
%%%%%%%%%%%%%%%%%%%%%%%%%%%%%%%%%%%%%%%%%%%%%%%%%%%%%%%%%%%%%%%%%%%%%%%%%%%%%%%%%%%%%%%%%%%%%%%%%%%%
 
%%%%%%%%%%%%%%%%%%%%%%%%%%%%%%%%%%%%%%%%%%%%%%%%%%%%%%%%%%%%%%%%%%%%%%%%%%%%%%%%%%%%%%%%%%%%%%%%%%%%


  
\end{enumerate}
%%%%%%%%%%%%%%%%%%%%%%%%%%%%%%%%%%%%%%%%%%%%%%%%%%%%%%%%%%%%%%%%%%%%%%%%%%%%%%%%%%%%%%%%%%%%%%%%%%%%


\barsn

\noindent\Maroon{{\sf Hand in at the start of class, Thursday 20 November:}} 

\begin{enumerate}
\setcounter{enumi}{67}

%%%%%%%%%%%%%%%%%%%%%%%%%%%%%%%%%%%%%%%%%%%%%%%%%%%%%%%%%%%%%%%%%%%%%%%%%%%%%%%%%%%%%%%%%%%%%%%%%%%%
%
\item 
      Define the binomial coefficient $\binom{n}{k}$ to be  $\frac{n!}{k!(n{-}k)!}$  for integers $0\leq k\leq n$.
       Let $R$ be a commutative ring.
       Prove the binomial theorem:\vspace{-4pt}
\[
   \forall a,b\in R\ \forall n\in\NN\qquad
    (a+b)^n\ =\ \sum_{k=0}^n \tbinom{n}{k}a^k b^{n-k}\ .
\] 
%%%%%%%%%%%%%%%%%%%%%%%%%%%%%%%%%%%%%%%%%%%%%%%%%%%%%%%%%%%%%%%%%%%%%%%%%%%%%%%%%%%%%%%%%%%%%%%%%%%%  



%%%%%%%%%%%%%%%%%%%%%%%%%%%%%%%%%%%%%%%%%%%%%%%%%%%%%%%%%%%%%%%%%%%%%%%%%%%%%%%%%%%%%%%%%%%%%%%%%%%%
%
\item 
      Suppose that $R$ is a commutative ring of characteristic $p$, a prime number.
      Prove that the map $a\mapsto a^p$ defined for $a\in R$ is a ring homomorphism.
      (This is called the \demph{Frobenius homomorphism}.)
%%%%%%%%%%%%%%%%%%%%%%%%%%%%%%%%%%%%%%%%%%%%%%%%%%%%%%%%%%%%%%%%%%%%%%%%%%%%%%%%%%%%%%%%%%%%%%%%%%%%  


%%%%%%%%%%%%%%%%%%%%%%%%%%%%%%%%%%%%%%%%%%%%%%%%%%%%%%%%%%%%%%%%%%%%%%%%%%%%%%%%%%%%%%%%%%%%%%%%%%%%
%
\item 
  Prove that the set consisting of zero and all zero divisors in a commutative ring contains at least one prime ideal.
\vspace{-2pt}
%%%%%%%%%%%%%%%%%%%%%%%%%%%%%%%%%%%%%%%%%%%%%%%%%%%%%%%%%%%%%%%%%%%%%%%%%%%%%%%%%%%%%%%%%%%%%%%%%%%%  
 

%%%%%%%%%%%%%%%%%%%%%%%%%%%%%%%%%%%%%%%%%%%%%%%%%%%%%%%%%%%%%%%%%%%%%%%%%%%%%%%%%%%%%%%%%%%%%%%%%%%%
%
\item
  Let $G$ be a finite group.
  Show that the center, \defcolor{$Z(\CC[G])$}, of the group algebra $\CC[G]$ (the elements that commute with every element of $\CC[G]$)
  has dimension as a $\CC$-vector space equal to the 
      number of conjugacy classes of $G$.
      (The expected answer is a huge hint for a possible basis for $Z(\CC[G])$.)

      Work out this hint.
      For $g\in G$, define
      \[
      \defcolor{e_{[g]}}\ \vcentcolon=\ \sum_{h\in G}  {^h}g\,,
      \]
      which only depends upon the conjugacy class $[g]$ of $g$.
      Show that these elements form a basis for $Z(\CC[G])$.
%%%%%%%%%%%%%%%%%%%%%%%%%%%%%%%%%%%%%%%%%%%%%%%%%%%%%%%%%%%%%%%%%%%%%%%%%%%%%%%%%%%%%%%%%%%%%%%%%%%%  


%%%%%%%%%%%%%%%%%%%%%%%%%%%%%%%%%%%%%%%%%%%%%%%%%%%%%%%%%%%%%%%%%%%%%%%%%%%%%%%%%%%%%%%%%%%%%%%%%%%%  
\item
  Let $R$ be a  ring and $G$ be an infinite multiplicative cyclic group with generator $\xi$.
       Prove or disprove:  The group ring $R[G]$ is isomorphic to the polynomial ring $R[x]$ in one indeterminate $x$.
%%%%%%%%%%%%%%%%%%%%%%%%%%%%%%%%%%%%%%%%%%%%%%%%%%%%%%%%%%%%%%%%%%%%%%%%%%%%%%%%%%%%%%%%%%%%%%%%%%%%

      
\end{enumerate}
%%%%%%%%%%%%%%%%%%%%%%%%%%%%%%%%%%%%%%%%%%%%%%%%%%%%%%%%%%%%%%%%%%%%%%%%%%%%%%%%%%%%%%%%%%%%%%%%%%%%

\end{document}

%%%%%%%%%%%%%%%%%%%%%%%%%%%%%%%%%%%%%%%%%%%%%%%%%%%%%%%%%%%%%%%%%%%%%%%%%%%%%%%%%%%%%%%%%%%%%%%%%%%%
\item  
%%%%%%%%%%%%%%%%%%%%%%%%%%%%%%%%%%%%%%%%%%%%%%%%%%%%%%%%%%%%%%%%%%%%%%%%%%%%%%%%%%%%%%%%%%%%%%%%%%%%

