%
%HW07.tex
%Seventh Homework -- Math 629 
%
%  The percent sign is a comment character
%
%%%%%%%%%%%%%%%%%%%%%%%%%%%%%%%%%%%%%%%%%%%%%%%%%%%%%%%%%%%%%%%%%%%%%%%%%%%%%%%%%%
%
%   Look these up on line.  The first sets the type of document, and the next are for mathematics symbols, graphics and color
%
\documentclass[12pt]{article}
\usepackage{amssymb,amsmath}
\usepackage{graphicx}
\usepackage[usenames,dvipsnames,svgnames,table]{xcolor}
\usepackage{multirow}   % This is for more control over tables
%%%%%%%%%%%%%%%%%%%%%%%%%%%%%%%%  Layout     %%%%%%%%%%%%%%%%%%%%%%%%%%%%%%%%%%%%%%
\usepackage{vmargin}
\setpapersize{USletter}
\setmargrb{2cm}{1cm}{2cm}{1cm} % --- sets all four margins LTRB


%%%%%%%%%%%%%%%%%%%%%%%%%%%%%%%%%%%%%%%%%%%%%%%%%%%%%%%%%%%%%%%%%%%%%%%%%%%%%%%%%
\begin{document}
\LARGE 
\noindent
{\color{Maroon}History of Mathematics \hfill Math 629}\vspace{2pt}\\
\large
Seventh Homework: \hfill 27 February 2024\\
Due Monday 4 March 2024.
\normalsize\vspace{10pt}

To hand in: We are using Gradescope for homework submission.


\begin{enumerate}

\item  {[10]}
     Exercise 10.5.1 from Stillwell.

\item  {[10]}
     Exercise 10.6.1 from Stillwell.


\item  {[10]}
     Exercise 10.6.2 from Stillwell. 

\item  {[10]}
     Exercise 10.6.3 from Stillwell. 

 
\item  {[10]}
  Exercise 10.7.3 from Stillwell.

  For this, use the formula $\zeta(1-s) = 2 (2\pi)^{-s} \cos\frac{s\pi}{2}\,\Gamma(s)\, \zeta(s)$.
  You need to recognise the sum as $\zeta(1-s)$ and then apply this formula.  Note that for $n$ a positive integer, $\Gamma(n)=(n-1)!$.

 
\item  {[10]}
  Write a coherent few sentences (maybe a paragraph) about at least one thing that is completely wrong with the Numberphile video.
  Be sure to be mathematically correct; you may need to recall your second semester of Calculus.

\end{enumerate}


\end{document}
%%%%%%%%%%%%%%%%%%%%%%%%%%%%%%%%%%%%%%%%%%%%%%%%%%%%%%%%%%%%%%%%%%%%%%%%%%%%%%%%%
