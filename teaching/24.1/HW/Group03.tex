%Group03.tex.tex
%Third group homework -- Math 629 
%
%  The percent sign is a comment character
%
%%%%%%%%%%%%%%%%%%%%%%%%%%%%%%%%%%%%%%%%%%%%%%%%%%%%%%%%%%%%%%%%%%%%%%%%%%%%%%%%%%
%
%   Look these up on line.  The first sets the type of document, and the next are for mathematics symbols, graphics and color
%
\documentclass[12pt]{article}
\usepackage{amssymb,amsmath}
\usepackage{graphicx}
\usepackage[usenames,dvipsnames,svgnames,table]{xcolor}
\usepackage{multirow}   % This is for more control over tables
%%%%%%%%%%%%%%%%%%%%%%%%%%%%%%%%  Layout     %%%%%%%%%%%%%%%%%%%%%%%%%%%%%%%%%%%%%%
\usepackage{vmargin}
\setpapersize{USletter}
\setmargrb{2cm}{1cm}{2cm}{1cm} % --- sets all four margins LTRB


%%%%%%%%%%%%%%%%%%%%%%%%%%%%%%%%%%%%%%%%%%%%%%%%%%%%%%%%%%%%%%%%%%%%%%%%%%%%%%%%%
\begin{document}
\LARGE 
\noindent
{\color{Maroon}History of Mathematics \hfill Math 629}\vspace{2pt}\\
\large
Second Group Homework: \hfill 30 January 2024\\
Not to be handed in, but should be discussed.
\normalsize\vspace{10pt}



\noindent{\bf Some numerology of the Greeks}

Do read Stillwell's exposition on how Euclid proved the Fundamental Theorem of Arithmetic, that every positive integer greater than 1 has an
essentially unique factorization into powers of prime numbers.



\begin{enumerate}

\item  Please do Stillwell's exercise 3.3.1, about the factors of a number of the form $2^{n-1}p$, where $p$ is a prime.

\item    Give the defining property of a perfect number. Prove that if $p=2^n-1$ is a prime number, then $2^{n-1}p$ is a perfect number.
  Write down four perfect numbers.
  How many perfect numbers are there currently known? 


\item Let us call a positive number {\color{blue}\it ample} if the sum of its divisors exceeds the number itself.
  For example, $1+2+3+4+6=16>12$, so twelve is an ample number.

  Show that any number of the form $2^{n-1}(2^n-1)$ is either perfect or ample.
  
  Give an ample number not of this form.


\end{enumerate}

\end{document}
%%%%%%%%%%%%%%%%%%%%%%%%%%%%%%%%%%%%%%%%%%%%%%%%%%%%%%%%%%%%%%%%%%%%%%%%%%%%%%%%%
