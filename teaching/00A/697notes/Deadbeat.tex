%
% wrapper.tex
%
%
%
%%%%%%%%%%%%%%%%%%%%%%%%%%%%%%%%%%%%%%%%%%%%%%%%%%%%%%%%%%%%
\documentclass[12pt]{report}
\usepackage{amssymb}
\usepackage{amsmath}
\usepackage{epsf}
%\textwidth 6.9in\oddsidemargin -0.2in
%\textwidth 6.5in\oddsidemargin 0in
\textwidth 6.in\oddsidemargin 0.2in
\textheight 9.3in\topmargin -0.65in   % For regular US paper
%\textheight 9.5in\topmargin -0.65in % For Paper a4

%%%%%%%%%%%%%%%%%%%%%%%%%%%%%%%%%%%%%%%%%%%%%%%%%
\newcommand{\Span}[1]{\langle #1 \rangle}

%\makeindex
%%%%%%%%%%%%%%%%%%%%%%%%%%%%%%%%%%%%%%%%%%%%%%%%%

\newtheorem{thm}{Theorem}[chapter]
\newtheorem{definition}[thm]{Definition}
\newtheorem{example}[thm]{Example}
\newtheorem{exercise}{Exercise}[chapter]
\newtheorem{lem}[thm]{Lemma}
\newtheorem{co}[thm]{Corollary}
\newtheorem{remark}[thm]{Remark}
\newtheorem{pr}[thm]{Proposition}
\newtheorem{pro}[thm]{Problem}
\newtheorem{alg}[thm]{Algorithm}
\newtheorem{no}[thm]{Notation}
\newtheorem{con}[thm]{Conjecture}
\newenvironment{re}{\begin{remark}\rm}{\end{remark}} 
\newenvironment{ex}{\begin{example}\rm}{\end{example}}
\newenvironment{exer}{\begin{exercise}\rm}{\end{exercise}} 
\newenvironment{de}{\begin{definition}\rm}{\end{definition}}

\numberwithin{equation}{chapter}
\numberwithin{figure}{chapter}
\renewcommand{\theequation}{\thechapter.\arabic{equation}}
 
%%% The following commands provide the AMS proof enviorment
\newcommand{\openbox}{\leavevmode
  \hbox to.77778em{%
  \hfil\vrule
  \vbox to.675em{\hrule width.6em\vfil\hrule}%
  \vrule\hfil}}
\newcommand{\proofname}{Proof}
\newenvironment{proof}[1][\proofname]{\par\normalfont
 \trivlist\item[\hskip\labelsep\itshape #1:]\ignorespaces
}{\hspace*{1cm}\hspace*{\fill}\openbox \medskip\endtrivlist}
%%% End of AMS proof enviorment

\newenvironment{sketch}{\begin{proof}[{\em Sketch of Proof}]}{\end{proof}}

\typeout{LaTeX 3 times in order to get the 
page-numbering of the table of contents correct!}

\def\2stack#1#2{\mathrel{\mathop{#1}\limits_{#2}}}
\def\3stack#1#2#3{\mathrel{\mathop{\mathop{#1}\limits_{#2}}\limits_{#3}}}

\newcommand{\R}{\mathbb{R}}
\newcommand{\K}{\mathbb{K}}
\newcommand{\C}{\mathbb{C}}
\newcommand{\Q}{\mathbb{Q}}
\newcommand{\Ps}{\mathbb{P}}
\newcommand{\Plu}{{\Ps^{\binom{n}{p}-1}}}
\newcommand{\Z}{\mathbb{Z}}
\newcommand{\Pn}{{\mathbb P}^n}
\newcommand{\Af}{\mathbb{A}}
\newcommand{\Grass}{\mbox{\rm Grass}}

\newcommand{\lex}{\succ_{\it lex}}
\newcommand{\dlex}{\succ_{\it dlx}}
\newcommand{\drl}{\succ_{\it drl}}

\newcommand{\ini}{{\mbox{\rm in}}}
\newcommand{\nf}{{\mbox{\rm nf}}}
\newcommand{\lcm}{{\mbox{\rm lcm}}}
\newcommand{\Spol}{{\mbox{\rm Spol}}}
\newcommand{\amod}{{\mbox{\rm \,mod\,}}}
\newcommand{\tii}{{\bf t}^{\bf i}}
\newcommand{\tj}{{\bf t}^{\bf j}}
\newcommand{\tk}{{\bf t}^{\bf k}}
\newcommand{\vd}{\vdots}


\newcommand{\rat}{{\relbar\rightarrow}}
\newcommand{\longrat}{{\relbar\relbar\rightarrow}}

\newcommand{\A}{\mathcal{A}}
\newcommand{\Ideal}{\mathcal{I}}
\newcommand{\Jideal}{\mathcal{J}}
\newcommand{\Var}{\mathcal{V}}

\newcommand{\Mat}{{\mbox{\rm Mat}}}
\newcommand{\Hom}{{\mbox{\rm Hom}}}

\newcommand{\h}{\hspace{1.5mm}}
\newcommand{\dis}{\displaystyle}
\newcommand{\ki}{\mbox{\raise.5ex\hbox{$\chi$}}\hspace{-.4ex}}
\newcommand{\m}{\hspace{1em}}
\newcommand{\mm}{\hspace{2em}}
\newcommand{\p}{\partial}
\newcommand{\eqr}[1]{~\mbox{$(${\rm \ref{#1}}$)$}}
\newcommand{\q}{\nolinebreak
      \mbox{\hspace{2em}\frame{\rule{0ex}{1.5ex}\mbox{\hspace{1ex}}}}}
\newcommand{\seteq}{\setcounter{equation}{0}}
\newcommand{\x}{\vspace*{1ex}}
\newcommand{\xx}{\vspace*{2ex}}
\newcommand{\for}{\mm\mbox{for }}
\newcommand{\g}{\mbox{{\rm Grass}$(p,n)$}}
\newcommand{\gp}{\mbox{{\rm Grass}$(p,m+p)$}}
\newcommand{\gm}{\mbox{{\rm Grass}$(m,m+p)$}}
\newcommand{\la}{\mbox{${\mathbf \wedge}^{p} {\mathbb K}^{n}$}}

\newcommand{\I}{\mbox{$\underline{i}$}}
\newcommand{\J}{\mbox{$\underline{j}$}}

\newcommand{\zwei}[2]{\left[
    \begin{array}{c} #1 \\ #2 \end{array} \right]} 
\newcommand{\vier}[4]{\left[ \begin{array}{ccc}
                   #1 &\;& #2 \\ #3 &\;& #4 \end{array} \right]}

\newcommand{\comment}[1]{\vspace{10mm}\hspace{15mm}
\begin{minipage}{5in}
\baselineskip 9mm {\LARGE #1} 
\end{minipage}\vspace{5mm}
\typeout{Here is a comment line}}
\newcommand{\note}[1]{\comment{#1}}


\begin{document} 
%%%%%%%%%%%%%%%%%%%%%%%%%%%%%%%%%%%%%%%%%%%%%%%%%%%%%%%%%%%%
%%%%%%%%%%%%%%%%%%%%%%%%%%%%%%%%%%%%%%%%%%%%%%%%%%%%%%%%%%%%
\begin{center}
\Large\bf The Problem of Deadbeat Control\\
Math 697R\\
19 September, 2000
\end{center}


%%%%%%%%%%%%%%%%%%%%%%%%%%%%%%%%%%%%%%%%%%%%%%%%%%%%%%%%%%%%%%
%%%%%%%%%%%%%%%%%%%%%%%%%%%%%%%%%%%%%%%%%%%%%%%%%%%%%%%%%%%%%%
% The Deadbeat Problem
%%%%%%%%%%%%%%%%%%%%%%%%%%%%%%%%%%%%%%%%%%%%%%%%%%%%%%%%%%%%%%

%\chapter{The Problem of Deadbeat Control}

Many problems in systems and control engineering are algebraic
geometric in nature. Our first example is the so called {\em
  deadbeat control problem}. \index{deadbeat control problem}

Let $A,B$ and $C$ be three real matrices of size $n\times n$,
$n\times m$ and $p\times n$ respectively. These three matrices
describe a discrete time linear system \index{linear system}
through:
\begin{equation}                          \label{LinSystem}
  \begin{array}{rcl}
    x_{t+1} & = & Ax_t+Bu_t,\\
     y_{t} & = & Cx_t.
  \end{array}
\end{equation}

The sequence of vectors
$$
\{x_t\mid t=0,1,2\ldots\}\subset \R^n
$$
describes the sequence of {\em states}
\index{state of a system}%
of the linear system. It does depend on the sequence of {\em
  inputs}
$$
\{u_t\mid t=0,1,2\ldots\}\subset \R^m.
$$
\index{input of a system} It is assumed that the control
engineer can choose the sequence of inputs freely. It is the
control task to drive a certain linear system from one particular
state into a desirable future state by choosing an appropriate
sequence of inputs.

As a first step we would like to understand what states can be
reached when the system starts in a particular location
$\bar{x}=:x_0$. If it is possible that every state
$\hat{x}\in\R^n$ can be reached from every state $\bar{x}\in\R^n$
after a finite number of inputs we will call the
system\eqr{LinSystem} a controllable system.
\index{controllable system}%
By iterating the equations one readily verifies
that the state $x_d$ after $d$ iterations is given by:
$$
x_d=A^d\hat{x}+\left[ A^{d-1}B\ A^{d-2}B\ \ldots \ AB\ 
  B\right] \left[
\begin{array}{c}
u_0\\ \vdots \\ v_{d-1}
\end{array}
\right].
$$

Denote by `colsp' the column space of a matrix. From our
calculation one sees that every state $x\in\R^n$ can be reached
after at most $d$ iterations if and only if:
$$
\mathrm{colsp}\ \left[ A^{d-1}B\ A^{d-2}B\ \ldots \ AB\ 
  B\right]= \R^n.
$$
The sequence of subspaces 
$$
\mathrm{colsp}\left[ B\right]\subset 
\mathrm{colsp}\left[AB\  B\right]\subset
\mathrm{colsp}\left[A^2B\ AB\ B\right]\subset \ldots
$$
is increasing in dimensions. If at a certain point 
$$
\mathrm{colsp}\ \left[ A^{d-1}B\ A^{d-2}B\ \ldots \ AB\ B\right]
=\mathrm{colsp}\ \left[ A^{d}B\ A^{d-1}B\ \ldots \ AB\ B\right]
$$
then it follows that the sequence becomes stationary. 
It therefore follows that system\eqr{LinSystem} is a controllable
system if and only if Kalman's controllability rank
condition (see e.g.~\cite{so90} for details) \index{Kalman}
$$
\mathrm{rank}\ \left[ A^{n-1}B\ A^{n-2}B\ \ldots \ AB\ B\right]=
n
$$
holds.


An important instance of above control problem is the zero
controllability problem: For this assume that the state at time
$t=0$ is at a certain location $\bar{x}=:x_0$. The zero
controllability problem asks for a {\em sequence of inputs}
$u_0,u_1,\ldots$ \index{input of a system} which will force the
resulting sequence of states to the origin. If this is always
possible one says that the system is zero-controllable.
\index{zero controllable}

Often it is not possible to observe the full state $x_t\in\R^n$
at a certain time and instead one is only able to observe the
{\em output vector} \index{output of a system} $y_t$. This vector
is related to the state vector $x_t$ through a linear map $C$.
The sequence of vectors
$$
\{y_t\mid t=0,1,2\ldots\}\subset \R^p
$$
is called the sequence of outputs. We say that the
system\eqr{LinSystem} has input number $m$, output number $p$ and
McMillan degree \index{McMillan degree} $n$. In the systems
theory literature the relation between inputs, states and outputs
is often depicted via the diagram:
$$
\begin{picture}(240,30)
\thicklines
  \put(  0,15){\line(1, 0){90}}   
 \put( 80,25){\line(1,-1){10}}   
 \put( 80, 5){\line(1, 1){10}}   
 \put( 25, 3){${\mathbf u_t}\in{\mathbb R}^m$}

 \put( 90, 0){\line(1,0){65}} 
 \put( 90,30){\line(1,0){65}}
 \put( 90, 0){\line(0,1){30}}  
 \put(155, 0){\line(0,1){30}}
 \put(105,12){${\mathbf x_t}\in{\mathbb R}^n$}

 \put(155,15){\line(1,0){90}}
 \put(235,25){\line(1,-1){10}}
 \put(235, 5){\line(1, 1){10}}
 \put(185, 3){$\mathbf{y_t}\in{\mathbb R}^p$}
\end{picture}
$$


If the control engineer can only observe the sequence of outputs
it is in general a difficult task to design a sequence of inputs
which will assure that the sequence of states and the sequence of
outputs both approach the origin.

The problem becomes even more difficult if this goal should be
achieved using {\em feedback}\index{feedback}.  In our context a
feedback law consists of a $m\times p$ matrix $F$ which relates
the inputs and the outputs through the control law
$$
u_t=Fy_t.
$$

Once a feedback has been chosen the dynamics of the system will
be governed by the equations:
\begin{equation}                \label{closed-loop}
  \begin{array}{rcl}
    x_{t+1} & = & (A+BFC)x_t,\\
     y_{t} & = & Cx_t.
  \end{array}
\end{equation}
System\eqr{closed-loop} is called the {\em closed loop system}
\index{closed loop system}%
as opposed to the open loop system\eqr{LinSystem}.
\index{open loop system}%
In terms of block diagrams the closed loop
system\eqr{closed-loop} is often depicted through:
$$
\begin{picture}(240,70)
\thicklines
 \put( 40,55){\line( 1, 0){50}}
 \put( 40,55){\line( 0,-1){40}}
 \put( 30,25){\line( 1,-1){10}}
 \put( 50,25){\line(-1,-1){10}}

 \put( 90,40){\line(1,0){65}} 
 \put( 90,70){\line(1,0){65}}
 \put( 90,40){\line(0,1){30}}  
 \put(155,40){\line(0,1){30}}
 \put(101,53){${\mathbf u_t}=F{\bf y_t}$}

 \put(200,55){\line(-1, 0){45}}
 \put(200,55){\line( 0,-1){40}}
 \put(155,55){\line( 1, 1){10}}
 \put(155,55){\line( 1,-1){10}}
 
 \put(  0,15){\line(1, 0){90}}   
 \put( 80,25){\line(1,-1){10}}   
 \put( 80, 5){\line(1, 1){10}}   
 \put( 25, 3){${\mathbf u_t}\in{\mathbb R}^m$}

 \put( 90, 0){\line(1,0){65}} 
 \put( 90,30){\line(1,0){65}}
 \put( 90, 0){\line(0,1){30}}  
 \put(155, 0){\line(0,1){30}}
 \put(105,12){${\mathbf x_t}\in{\mathbb R}^n$}

 \put(155,15){\line(1,0){90}}
 \put(235,25){\line(1,-1){10}}
 \put(235, 5){\line(1, 1){10}}
 \put(185, 3){${\mathbf y_t}\in{\mathbb R}^p$}
\end{picture}
$$
The deadbeat control problem 
\index{deadbeat control problem}%
for the design of a feedback law $u_t=Fy_t$ such that the closed
loop system will approach the origin independently of the initial
state $x_0$. This is goal is achieved if, and only if the
$n\times n$ matrix $(A+BFC)$ is
nilpotent\index{matrix!nilpotent}.

Algebraically, the deadbeat control problem
\index{deadbeat control problem}%
can therefore be formulated in the following way:
\begin{pro}
  Given matrices $A,B,C$ of size $n\times n$, $n\times m$ and
  $p\times n$ respectively. Is there a $m\times p$ matrix $F$
  such that $R:=A+BFC$ is nilpotent.
\end{pro}
Recall that a $n\times n$ matrix $R$ is
nilpotent\index{matrix!nilpotent} if and only if $R^n=0$.  Let us
view the $mp$ entries $f_{ij}$ of the matrix $F$ as variables and
let us view the entries of $A,B,C$ as constants.  The condition
$R^n=0$ then results in $n^2$ polynomial equations in $mp$
variables each of degree at most $n$. A general purpose symbolic
or numerical program will not be able to solve this system of
equations even for moderate sizes of $n$. One way of simplifying
the problem is by means of the following well known lemma. The
lemma is a direct consequence of the fact that the minimal
polynomial of a matrix divides the characteristic
polynomial\index{characteristic polynomial}.
\begin{lem}
  An $n\times n$ matrix $N$ is nilpotent if and only if the
  characteristic polynomial $\det(sI_n-N)=s^n$.
\end{lem}
Using this lemma we compute:
$$
\det(sI_n-R)=\det(sI_n-(A+BFC))
=s^n+a_{n-1}s^{n-1}+\cdots+a_1s+a_0.
$$
We view the coefficients $a_{n-1},\ldots,a_0$ as polynomials
in the $mp$ variables $f_{ij}$. Note that $a_t$ is a polynomial
of degree at most $n-t$. In this form we are confronted with the
problem of solving $n$ polynomial equations in $mp$ unknowns,
still a formidable task in general.

Surprisingly as we will see later in the book it is possible to
say quite a bit more about the general solution of this
polynomial system of equations. It will turn out that the system
of equations has in general no solution if $n>mp$. If $n\leq mp$
then ``generically'', \index{generic set} (this means for almost
all systems in a sense which will be made precise later) the
polynomial system will have solutions over the complex numbers.
Moreover if $n=mp$ the number of solutions is in almost all cases
equal to the finite number:
$$
d(m,p):=\frac{1!2!\cdots (p-1)! (mp)!}{m!(m+1)!\cdots (m+p-1)!}.
$$

The combinatorial formula $d(m,p)$ is interesting in many ways
and the formula has appeared in the Mathematics literature in
representation theory, in the theory of symmetric functions and
in the context of Schubert calculus. Actually it was Schubert in
the 19th century who showed:
\begin{quote}
  Given $mp$ linear subspaces of $\C^{m+p}$ in general position.
  Then there are exactly $d(m,p)$ different subspaces
  intersecting each of the $mp$ linear subspaces non-trivially.
\end{quote}
Why is the number of solutions in the deadbeat control problem
equal to $d(m,p)$ if $n=mp$? At this point this is rather a
mystery but we will develop the theory so that this becomes clear
later in the book. One thing we would like to mention already now
is the following: We already observed that the polynomial $a_t$
has degree at most $n-t$.  We can now think that we eliminate the
$n=mp$ variables $f_{it}$ one by one. If the number of solutions
is finite then this elimination process (see Part 2) will result
in an elimination polynomial of degree at most $(mp)!=n!\, $. The
total number of solutions should therefore be less than $(mp)!$
and one readily verifies that this is the case for the number
$d(m,p)$.  The following example illustrates the concepts which
we introduced so far:

\begin{ex}
Consider the matrices
  $$
  A := \left[ {\begin{array}{rrrr}
        0 & 0 & 1 & 0 \\
        0 & 0 & 1 & 0 \\
        0 & 1 & 0 & 0 \\
        1 & 0 & 1 & 2
\end{array}}
\right],\ \ B := \left[ {\begin{array}{rr}
      1 & 0 \\
      0 & 1 \\
      0 & 0 \\
      0 & 0
\end{array}}
\right], \ \ C := \left[ {\begin{array}{rrrr}
      1 & 0 & 0 & 0 \\
      0 & 1 & 0 & 2
\end{array}}
\right].
$$
The matrices $A,B,C$ describe a linear system of the
form\eqr{LinSystem} having $m=2$ inputs, $p=2$ outputs and
McMillan degree $n=4$.

We are interested in the construction of a feedback 
feedback matrix $F$ which solves the deadbeat control problem. 
Let
$F := \left[ {\begin{array}{cc}
      {f_{11}} & {f_{12}} \\
      {f_{21}} & {f_{22}}
\end{array}}
\right] $.  It is goal to find values for the parameters $f_{ij}$
such that the closed loop system matrix $R=A+BFC$ is
nilpotent. Using Maple or any other symbolic computation program
one readily computes $R$ as: 
$$
R = \left[ {\begin{array}{ccrc}
      {f_{11}} & {f_{12}} & 1 & 2\,{f_{12}} \\
      {f_{21}} & {f_{22}} & 1 & 2\,{f_{22}} \\
      0 & 1 & 0 & 0 \\
      1 & 0 & 1 & 2
\end{array}}
\right] .
$$
Using the program Maple\index{Maple} we compute the
characteristic polynomial in terms of the coefficients of $F$:
\begin{multline*}
  \chi(R,s) := s^{4}-2\,s^{3}-{f_{22}}\,s^{3}-{f_{11}}\,s^{3}\\
  +2\,{f_{22}} \,s^{2} - s^{2} + 2\,{f_{11}}\,s^{2} +
  {f_{11}}\,{f_{22}}\,
  s^{2}- {f _{21}}\,{f_{12}}\,s^{2}- 2\,{f_{12}}\,s^{2}\\
  + 2\,s - 2\,{f_{22}}\,s - 2\,{f_{11}}\,{f_{22}}\,s +
  {f_{11}}\,s+ 2\,{f_{21}}\,{f_{12}}\,s - {f_{21}}\,s\\
  - 2\,{f_{11}} + 2\,{f_{11}}\,{f_{22}} + 2\,{f_{21}} -
  2\,{f_{21}}\,{f_{12}} - 2\,{f_{22}} + 2\,{f_{12}}.
\end{multline*}
Equating the coefficients of $\chi(R,s)$ to zero results into the
system of 4 polynomial equations:
$$
\begin{array}{rcl}
  {f_{22}} + {f_{11}} + 2&=&0,  \\
{f_{11}}\,{f_{22}} - 1 - {f_{21}}\,{f_{12}} - 2\,{f_{12}} + 2\,
{f_{22}} + 2\,{f_{11}}&=&0,\\
2\,{f_{21}}\,{f_{12}} - 2\,{f_{22}} - {f_{21}} - 2\,{f_{11}}\,
{f_{22}} + {f_{11}} + 2&=&0,  \\
2\,{f_{11}}\,{f_{22}} - 2\,{f_{21}}\,{f_{12}} - 2\,{f_{22}} + 2\,
{f_{21}} + 2\,{f_{12}} - 2\,{f_{11}}&=&0.
\end{array}
$$
This is a system of 4 equations having degree 1 and 2
respectively. Elimination theory (see Part 2) tells us that the
system of polynomial equations has at most 8 solutions if the
number of solutions is finite.  Maple readily computes two
solutions:
$$
\begin{array}{rcl}
F_1&=&\vier{- {\frac{67}{52}} + {\frac{1}{52}} \,\sqrt{641}}{ -
  {\frac {45}{52}} + {\frac {3}{52}} \,\sqrt{641},}{ -
  {\frac{229}{52}} - {\frac{9}{52}} \,\sqrt{641}}{- 
  {\frac{37}{52}} - {\frac{1}{52}} \,\sqrt{641}},\medskip\\ 
F_2&=&\vier{- {\frac{67}{52}} -\frac{1}{52} \,\sqrt{641},}{-
   {\frac{45}{52}} - {\frac{3}{52}} \,\sqrt{641},}{- {\frac
    {229}{52}} + { \frac{9}{52}} \,\sqrt{641}}{- {\frac{37}{52}}
  + {\frac{1}{52}} \,\sqrt{641}}.
\end{array}
$$
Note that if $m=p=2$ the number $d(m,p)=2$ and not 8, as one
would naively expect.
\end{ex}

The deadbeat control problem is a special instance of the static
pole placement problem\index{pole placement!static}. In its
general form the static pole placement problem asks the following
question: 
\begin{pro}
  Given a monic polynomial
  $\varphi(s)=s^n+c_{n-1}s^{n-1}+\cdots+c_1s+c_0$ and given
  matrices $A,B$ and $C$ of size $n\times n$, $n\times m$ and
  $p\times n$ respectively. Is there a $m\times p$ feedback
  matrix $F$ such that the characteristic polynomial of $R=A+BFC$
  is $\varphi(s)$?
\end{pro}

If $\varphi(s)=s^n$ then the pole placement reduces to the
deadbeat control problem. In its general form the static pole
placement problem belongs to the most prominent problems in
systems and control theory. It is a major design problem in
control theory, where one often wants that the eigenvalues of the
closed loop characteristic polynomial
\index{characteristic polynomial}%
will be placed at a certain location.

In Part III of the book we will study the static pole placement
problem in detail. We will show how the problem is closely
related to Schubert calculus
\index{Schubert calculus}%
and how algebraic geometric methods do allow one to predict the
existence of solutions.


\begin{thebibliography}{1}

\bibitem{so90}
E.~D. Sontag.
\newblock {\em Mathematical Control Theory: Deterministic Finite Dimensional
  Systems}.
\newblock Springer Verlag, New York, 1990.

\end{thebibliography}


\end{document}

%%%%%%%%%%%%%%%%%%%%%%%%%%%%%%%%%%%%%%%%%%%%%%%%%%%%%%%%%%%%
%%%%%%%%%%%%%%%%%%%%%%%%%%%%%%%%%%%%%%%%%%%%%%%%%%%%%%%%%%%%
%%%%%%%%%%%%%%%  THE  END  %%%%%%%%%%%%%%%%%%%%%%%%%%%%%%%%%
%%%%%%%%%%%%%%%%%%%%%%%%%%%%%%%%%%%%%%%%%%%%%%%%%%%%%%%%%%%%
%%%%%%%%%%%%%%%%%%%%%%%%%%%%%%%%%%%%%%%%%%%%%%%%%%%%%%%%%%%%
  






