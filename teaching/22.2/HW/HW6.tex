%HW6.tex
%Sixth Homework -- Math 300H 
%
%  The percent sign is a comment character
%
%%%%%%%%%%%%%%%%%%%%%%%%%%%%%%%%%%%%%%%%%%%%%%%%%%%%%%%%%%%%%%%%%%%%%%%%%%%%%%%%%%
%
%   Look these up on line.  The first sets the type of document, and the next are for mathematics symbols, graphics and color
%
\documentclass[12pt]{article}
\usepackage{amssymb,amsmath,mathtools}
\usepackage{graphicx}
\usepackage[usenames,dvipsnames,svgnames,table]{xcolor}
\usepackage{multirow}   % This is for more control over tables
%%%%%%%%%%%%%%%%%%%%%%%%%%%%%%%%  Layout     %%%%%%%%%%%%%%%%%%%%%%%%%%%%%%%%%%%%%%
\usepackage{vmargin}
\setpapersize{USletter}
\setmargrb{2cm}{0.8cm}{2cm}{0.8cm} % --- sets all four margins LTRB


\newcommand{\RR}{{\mathbb R}}  % This is the backboard symbol for the real numbers.  Note how it is used below
\newcommand{\NN}{{\mathbb N}}  % 
\newcommand{\QQ}{{\mathbb Q}}  % 
\newcommand{\ZZ}{{\mathbb Z}}  %

\newcommand{\calP}{{\mathcal P}}  %Caligraphic P for power set

\newcommand{\sep}{{\ :\ }}     % This is for the : in our notation for building sets.
\newcommand{\lsim}{\mathord{\sim}}  %  This is to remove the extra spacing around \sim (it is a binary relation) for the logical not

\newcommand{\defcolor}[1]{{\color{blue}{#1}}}
\newcommand{\demph}[1]{{\color{blue}\sl{#1}}}
%%%%%%%%%%%%%%%%%%%%%%%%%%%%%%%%%%%%%%%%%%%%%%%%%%%%%%%%%%%%%%%%%%%%%%%%%%%%%%%%%
%
%   Please edit this as appropriate
%
\begin{document}
\LARGE 
\noindent
{\color{Maroon}Foundations of Mathematics \hfill Math 300H Section 970}\vspace{2pt}\\
\Large YOUR NAME\vspace{2pt}\\
\large
Sixth Homework: \hfill Due \  18 October 2022\\
Use English when possible.  Answers should not just be symbols.
\normalsize\smallskip




\noindent{\color{blue}\rule{500pt}{2pt}}\smallskip

%%%%%%%%%%%%%%%%%%%%%%%%%%%%%%%%%%%%%%%%%%%%%%%%%%%%%%%%%%%%%%%%%%%%%%%%%%%%%%%%%
\begin{enumerate}  %  the \begin{..}  \end{..} stars and ends an environment.


%%%%%%%%%%%%%%%%%%%%%%%%%%%%%%%%%%%%%%%%%%%%%%%%%%%%%%%%%%%%%%%%%%%%%%%%%%%%%%%%%
\item Please give formal definitions of the following mathematical terms that we have used in our course.
  These should be, in Frank's terms, proper mathematical definitions.
  
  \begin{enumerate}
    \item  A real number $x$ is \demph{rational}.

    \item A real number $x$ is \demph{irrational}.

    \item The \demph{Cartesian product $A\times B$} of two sets $A$ and $B$.

    \item For a real number $x$, the \demph{absolute value $|x|$} of $x$.

    \item The set \demph{theoretic difference}, \defcolor{$A-B$} of sets $A$ and $B$.
      
    \item That \defcolor{$a |b$}, for integers $a$ and $b$.

    \item That \defcolor{$a\equiv b \mod m$}, for integers $m$, $a$, and $b$.
      
\end{enumerate}

  
%%%%%%%%%%%%%%%%%%%%%%%%%%%%%%%%%%%%%%%%%%%%%%%%%%%%%%%%%%%%%%%%%%%%%%%%%%%%%%%%%

%%%%%%%%%%%%%%%%%%%%%%%%%%%%%%%%%%%%%%%%%%%%%%%%%%%%%%%%%%%%%%%%%%%%%%%%%%%%%%%%%
\item Please give a formal definition of the empty set, $\emptyset$, and of the subset relation, that is, for sets $A$ and $B$, give a
  definition of what it means to say that $A\subseteq B$.
  
      Use these in a (short) proof that ``For every sey $A$, $\emptyset\subseteq A$''
%%%%%%%%%%%%%%%%%%%%%%%%%%%%%%%%%%%%%%%%%%%%%%%%%%%%%%%%%%%%%%%%%%%%%%%%%%%%%%%%%

%%%%%%%%%%%%%%%%%%%%%%%%%%%%%%%%%%%%%%%%%%%%%%%%%%%%%%%%%%%%%%%%%%%%%%%%%%%%%%%%%
\item Let $P$, $Q$, and $R$ be statements.
      Give useful negations of the following statements.
     
\begin{enumerate}

  \item $P\Rightarrow Q$.

  \item  $P\Rightarrow(Q \vee R)$.

  \item  $P\Rightarrow(Q \wedge R)$.

  \item  $(P\vee Q)\Rightarrow R$.

  \item  $(P\wedge Q)\Rightarrow R$.

  \item  $(P\Rightarrow Q) \Rightarrow R$

  \item  $P\Rightarrow (Q \Rightarrow R)$

\end{enumerate}

%%%%%%%%%%%%%%%%%%%%%%%%%%%%%%%%%%%%%%%%%%%%%%%%%%%%%%%%%%%%%%%%%%%%%%%%%%%%%%%%%

\item  Recall that a real number $x$ is \demph{positive} if $x>0$.
     Consider the statement $P$: ``The sum of two real numbers is positive''.
  \begin{enumerate}
    \item Write $P$ as a statement of the form:
        ``some quantifier$\dotsc$, if $\dotsc$, then $\dotsc$.''
    \item Write the contrapositive of this statement in this form.
    \item Write the converse of this statement in this form.
    \item Write $\lsim P$ in this form.
    \item Prove whichever of $P$ or $\lsim P$ is true.
  \end{enumerate}
%%%%%%%%%%%%%%%%%%%%%%%%%%%%%%%%%%%%%%%%%%%%%%%%%%%%%%%%%%%%%%%%%%%%%%%%%%%%%%%%  


%%%%%%%%%%%%%%%%%%%%%%%%%%%%%%%%%%%%%%%%%%%%%%%%%%%%%%%%%%%%%%%%%%%%%%%%%%%%%%%%%
\item Let $a$, $b$, and $c$ be integers.

  (a) Write a crisp and correct proof of the statement that ``If $a|b$ and $b|c$, then $a|c$''.

    \mbox{\qquad} (Also, rewrite the statement using quantifiers.)

  (b) Do the same for the statement ``If $a|b$ and $a|c$, then $a|bc$''.
%%%%%%%%%%%%%%%%%%%%%%%%%%%%%%%%%%%%%%%%%%%%%%%%%%%%%%%%%%%%%%%%%%%%%%%%%%%%%%%%%



  
%%%%%%%%%%%%%%%%%%%%%%%%%%%%%%%%%%%%%%%%%%%%%%%%%%%%%%%%%%%%%%%%%%%%%%%%%%%%%%%%%%%%%%%%%%%%%%%%%%%%
\end{enumerate}
%%%%%%%%%%%%%%%%%%%%%%%%%%%%%%%%%%%%%%%%%%%%%%%%%%%%%%%%%%%%%%%%%%%%%%%%%%%%%%%%%%%%%%%%%%%%%%%%%%%%

\end{document}


%%%%%%%%%%%%%%%%%%%%%%%%%%%%%%%%%%%%%%%%%%%%%%%%%%%%%%%%%%%%%%%%%%%%%%%%%%%%%%%%%
\item 
%%%%%%%%%%%%%%%%%%%%%%%%%%%%%%%%%%%%%%%%%%%%%%%%%%%%%%%%%%%%%%%%%%%%%%%%%%%%%%%%%

%%%%%%%%%%%%%%%%%%%%%%%%%%%%%%%%%%%%%%%%%%%%%%%%%%%%%%%%%%%%%%%%%%%%%%%%%%%%%%%%%
\item \ 
%%%%%%%%%%%%%%%%%%%%%%%%%%%%%%%%%%%%%%%%%%%%%%%%%%%%%%%%%%%%%%%%%%%%%%%%%%%%%%%%%
