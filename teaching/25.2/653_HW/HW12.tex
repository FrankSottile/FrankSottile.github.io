%HW12.tex
%
% Ninth Homework for Graduate Algebra
% Frank Sottile
%%%%%%%%%%%%%%%%%%%%%%%%%%%%%%%%%%%%%%%%%%%%%%%%%%%%%%%%%%%%%%%%%%%%%%%
\documentclass[12pt]{article}
\usepackage{multicol,amsfonts, amssymb,  mathtools,amsmath}
\usepackage{colordvi,graphicx}
\headheight=8pt
%
\topmargin=-75pt
\textheight=720pt   \textwidth=560pt
\oddsidemargin=-60pt \evensidemargin=-60pt

\pagestyle{empty}

%%%%%%%%%%%%%%%%%%%%%%%%%%%%%%%%%%%%%%%%%%%%
\newcommand{\CC}{{\mathbb C}}
\newcommand{\KK}{{\mathbb K}}
\newcommand{\NN}{{\mathbb N}}
\newcommand{\QQ}{{\mathbb Q}}
\newcommand{\RR}{{\mathbb R}}
\newcommand{\TT}{{\mathbb T}}
\newcommand{\ZZ}{{\mathbb Z}}

\newcommand{\calA}{{\mathcal A}}
\newcommand{\be}{{\bf e}}
\newcommand{\bfi}{{\bf i}}
\newcommand{\bfj}{{\bf j}}

\newcommand{\Hom}{\mbox{Hom}}
\newcommand{\spec}{\mbox{spec}}
\newcommand{\cone}{\mbox{cone}}

\newcommand{\vect}[2]{(\begin{smallmatrix}#1\\#2\end{smallmatrix})}
\newcommand{\msp}{\hspace{8pt}}
%\newcommand{\Square}{\raisebox{-2pt}{\includegraphics{images/Square.eps}}}
\newcommand{\Square}{\raisebox{-2pt}{\Large$\square$}}

\def\Color#1#2{\special{color push cmyk #1}#2\special{color pop}}
%\def\Indigo#1{\Color{.42 1. 0. .49}{#1}}
\def\Indigo#1{\Color{1. .95 .05 .4}{#1}}
\def\MyViolet#1{\Color{.6 1. 0. .15}{#1}}
\def\TAMU#1{\Color{.15 1. .39 .69}{#1}}

\newcommand{\barsl}{\noindent\begin{minipage}[t]{590pt}
\Indigo{\rule{590pt}{1.2pt}}\vspace{-5.7mm}\\
\MyViolet{\rule{590pt}{1.2pt}}\vspace{-5.7mm}\\
\Blue{\rule{590pt}{1.2pt}}\vspace{-5.7mm}\\
\Green{\rule{590pt}{1.2pt}}\vspace{-5.7mm}\\
\Yellow{\rule{590pt}{1.2pt}}\vspace{-5.7mm}\\
\Orange{\rule{590pt}{1.2pt}}\vspace{-5.7mm}\\
\Red{\rule{590pt}{1.2pt}}\bigskip
\end{minipage}}


\newcommand{\barsn}{\noindent\begin{minipage}[t]{590pt}
\Indigo{\rule{590pt}{1.1pt}}\vspace{-4.5mm}\\
\MyViolet{\rule{590pt}{1.1pt}}\vspace{-4.5mm}\\
\Blue{\rule{590pt}{1.1pt}}\vspace{-4.5mm}\\
\Green{\rule{590pt}{1.1pt}}\vspace{-4.5mm}\\
\Yellow{\rule{590pt}{1.1pt}}\vspace{-4.5mm}\\
\Orange{\rule{590pt}{1.1pt}}\vspace{-4.5mm}\\
\Red{\rule{590pt}{1.1pt}}\bigskip
\end{minipage}}

\def\demph#1{\TAMU{{\sl #1}}}
\def\defcolor#1{\TAMU{#1}}

\begin{document}
\LARGE 
\noindent
Algebra \ \ Autumn 2025\vspace{1pt}\\
Frank Sottile\vspace{2pt}\\
\Large 20 November 2025 \hfill
\sf
 Twelfth Homework\makebox[20pt][l]{\ }
\normalsize\vspace{10pt}

\noindent
Write your answers neatly, in complete sentences.  
I highly recommend recopying your work before handing it in.
Correct and crisp proofs are greatly appreciated; oftentimes your work can be shortened and made clearer.

\barsn

\noindent\Maroon{{\sf Hand in at the start of class, Tuesday 2 December:}} 

\begin{enumerate}
  \setcounter{enumi}{72}

%%%%%%%%%%%%%%%%%%%%%%%%%%%%%%%%%%%%%%%%%%%%%%%%%%%%%%%%%%%%%%%%%%%%%%%%%%%%%%%%%%%%%%%%%%%%%%%%%%%%
%
\item  
Consider the ring $\ZZ[x]$ of univariate polynomials with integer coefficients. 
      Show that its ideal $I$ generated by $2$ and $x$ is not principal.\vspace{-2pt}
%%%%%%%%%%%%%%%%%%%%%%%%%%%%%%%%%%%%%%%%%%%%%%%%%%%%%%%%%%%%%%%%%%%%%%%%%%%%%%%%%%%%%%%%%%%%%%%%%%%%  

%%%%%%%%%%%%%%%%%%%%%%%%%%%%%%%%%%%%%%%%%%%%%%%%%%%%%%%%%%%%%%%%%%%%%%%%%%%%%%%%%%%%%%%%%%%%%%%%%%%%
\item  Let $p,r\in\NN$ with $p$ prime number and $r>0$.
  Determine the group of units in the ring $\ZZ/p^r\ZZ$.
  Show that it is cyclic unless $p=2$ and $r\geq 3$.
  (Hint, for the general case, show that it is the product of a cyclic group generated by $p{+}1$ and a cyclic group of order $p{-}1$.)
%%%%%%%%%%%%%%%%%%%%%%%%%%%%%%%%%%%%%%%%%%%%%%%%%%%%%%%%%%%%%%%%%%%%%%%%%%%%%%%%%%%%%%%%%%%%%%%%%%%%

%%%%%%%%%%%%%%%%%%%%%%%%%%%%%%%%%%%%%%%%%%%%%%%%%%%%%%%%%%%%%%%%%%%%%%%%%%%%%%%%%%%%%%%%%%%%%%%%%%%%
%
\item  
  For $\alpha=a+b\sqrt{-1}\in\ZZ[\sqrt{-1}]$, (the Gau{\ss}ian integers)
       set $N(\alpha):=a^2+b^2$.
       Determine the units in $\ZZ[\sqrt{-1}]$.
       Show that if $N(\alpha)$ is prime, then $\alpha$ is irreducible.
       Show that the same conclusion holds if $N(\alpha)=p^2$, where $p$ is a prime in
       $\ZZ$ that is congruent to 3 modulo 4.
%%%%%%%%%%%%%%%%%%%%%%%%%%%%%%%%%%%%%%%%%%%%%%%%%%%%%%%%%%%%%%%%%%%%%%%%%%%%%%%%%%%%%%%%%%%%%%%%%%%%  

%%%%%%%%%%%%%%%%%%%%%%%%%%%%%%%%%%%%%%%%%%%%%%%%%%%%%%%%%%%%%%%%%%%%%%%%%%%%%%%%%%%%%%%%%%%%%%%%%%%%
\item 
       Show that $\ZZ[\sqrt{-1}]$ is a unique factorization domain.
       (Hint: Show that it is a Euclidean domain.)
%%%%%%%%%%%%%%%%%%%%%%%%%%%%%%%%%%%%%%%%%%%%%%%%%%%%%%%%%%%%%%%%%%%%%%%%%%%%%%%%%%%%%%%%%%%%%%%%%%%%  

%%%%%%%%%%%%%%%%%%%%%%%%%%%%%%%%%%%%%%%%%%%%%%%%%%%%%%%%%%%%%%%%%%%%%%%%%%%%%%%%%%%%%%%%%%%%%%%%%%%%
\item       Using that $\ZZ[\sqrt{-1}]$ is a unique factorization domain, show that every
  prime $p$ that is congruent to 1 modulo 4 is the sum of two squares.

  Hint: use the cyclicity of the group of units of $\ZZ_p$
   to  show that there is a number $n\in\ZZ$ with $n^2\cong -1$ (mod $p$).
   Then use this to show that $p$ is reducible in $\ZZ[\sqrt{-1}]$
%%%%%%%%%%%%%%%%%%%%%%%%%%%%%%%%%%%%%%%%%%%%%%%%%%%%%%%%%%%%%%%%%%%%%%%%%%%%%%%%%%%%%%%%%%%%%%%%%%%%  

%%%%%%%%%%%%%%%%%%%%%%%%%%%%%%%%%%%%%%%%%%%%%%%%%%%%%%%%%%%%%%%%%%%%%%%%%%%%%%%%%%%%%%%%%%%%%%%%%%%%
\item  Let $S$ be a submonoid of the multiplicative monoid of a commutative ring $R$ 
   Identify the kernel of the canonical map $\iota \colon R \to R[S^{-1}]$.
%%%%%%%%%%%%%%%%%%%%%%%%%%%%%%%%%%%%%%%%%%%%%%%%%%%%%%%%%%%%%%%%%%%%%%%%%%%%%%%%%%%%%%%%%%%%%%%%%%%%

%%%%%%%%%%%%%%%%%%%%%%%%%%%%%%%%%%%%5%%%%%%%%%%%%%%%%%%%%%%%%%%%%%%%%%%%%%%%%%%%%%%%%%%%%%%%%%%%%%%%
 \item
   Let $S$ be a submonoid of an integral domain $R$ with $0\not\in S$.
        Show that if $R$ is a principal ideal domain, then so is $R[S^{-1}]$.
%%%%%%%%%%%%%%%%%%%%%%%%%%%%%%%%%%%%%%%%%%%%%%%%%%%%%%%%%%%%%%%%%%%%%%%%%%%%%%%%%%%%%%%%%%%%%%%%%%%%

%%%%%%%%%%%%%%%%%%%%%%%%%%%%%%%%%%%%%%%%%%%%%%%%%%%%%%%%%%%%%%%%%%%%%%%%%%%%%%%%%%%%%%%%%%%%%%%%%%%%
\item   Let $S$ be a submonoid of a commutative ring $R$ with $0\not\in S$.
  Let $P$ be a maximal element in the set of ideals of $R$ that do not meet $S$.
  Prove that $P$ is prime.
%%%%%%%%%%%%%%%%%%%%%%%%%%%%%%%%%%%%%%%%%%%%%%%%%%%%%%%%%%%%%%%%%%%%%%%%%%%%%%%%%%%%%%%%%%%%%%%%%%%%

%%%%%%%%%%%%%%%%%%%%%%%%%%%%%%%%%%%%%%%%%%%%%%%%%%%%%%%%%%%%%%%%%%%%%%%%%%%%%%%%%%%%%%%%%%%%%%%%%%%%
\item
        Let $p\in \ZZ$ be a prime number, so that $(p)$ is a prime ideal.
       What can be said about the relation between the quotient ring $\ZZ_p$ and the
       localization $\ZZ_{(p)}$ ?
%%%%%%%%%%%%%%%%%%%%%%%%%%%%%%%%%%%%%%%%%%%%%%%%%%%%%%%%%%%%%%%%%%%%%%%%%%%%%%%%%%%%%%%%%%%%%%%%%%%%  

%%%%%%%%%%%%%%%%%%%%%%%%%%%%%%%%%%%%%%%%%%%%%%%%%%%%%%%%%%%%%%%%%%%%%%%%%%%%%%%%%%%%%%%%%%%%%%%%%%%%  
\item 
  Show that a commutative ring $R$ is local if and only if for all $r,s\in R$, if $r+s=1$, then either $r$ or $s$ is a
  unit.
%%%%%%%%%%%%%%%%%%%%%%%%%%%%%%%%%%%%%%%%%%%%%%%%%%%%%%%%%%%%%%%%%%%%%%%%%%%%%%%%%%%%%%%%%%%%%%%%%%%%
 
      
\end{enumerate}
%%%%%%%%%%%%%%%%%%%%%%%%%%%%%%%%%%%%%%%%%%%%%%%%%%%%%%%%%%%%%%%%%%%%%%%%%%%%%%%%%%%%%%%%%%%%%%%%%%%%

\end{document}

%%%%%%%%%%%%%%%%%%%%%%%%%%%%%%%%%%%%%%%%%%%%%%%%%%%%%%%%%%%%%%%%%%%%%%%%%%%%%%%%%%%%%%%%%%%%%%%%%%%%
\item  
%%%%%%%%%%%%%%%%%%%%%%%%%%%%%%%%%%%%%%%%%%%%%%%%%%%%%%%%%%%%%%%%%%%%%%%%%%%%%%%%%%%%%%%%%%%%%%%%%%%%

