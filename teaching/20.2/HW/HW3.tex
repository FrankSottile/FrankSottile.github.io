%HW3.tex
%Third Homework -- Math 300 
%
%  The percent sign is a comment character
%
%%%%%%%%%%%%%%%%%%%%%%%%%%%%%%%%%%%%%%%%%%%%%%%%%%%%%%%%%%%%%%%%%%%%%%%%%%%%%%%%%%
%
%   Look these up on line.  The first sets the type of document, and the next are for mathematics symbols, graphics and color
%
\documentclass[12pt]{article}
\usepackage{amssymb,amsmath}
\usepackage{graphicx}
\usepackage[usenames,dvipsnames,svgnames,table]{xcolor}
\usepackage{multirow}   % This is for more control over tables
%%%%%%%%%%%%%%%%%%%%%%%%%%%%%%%%  Layout     %%%%%%%%%%%%%%%%%%%%%%%%%%%%%%%%%%%%%%
\usepackage{vmargin}
\setpapersize{USletter}
\setmargrb{2cm}{1cm}{2cm}{1cm} % --- sets all four margins LTRB


\newcommand{\defn}[1]{{\color{blue}\sl #1}}
\newcommand{\deco}[1]{{\color{blue} #1}}

\newcommand{\NN}{{\mathbb N}}
\newcommand{\QQ}{{\mathbb Q}}
\newcommand{\RR}{{\mathbb R}}
\newcommand{\ZZ}{{\mathbb Z}}

%%%%%%%%%%%%%%%%%%%%%%%%%%%%%%%%%%%%%%%%%%%%%%%%%%%%%%%%%%%%%%%%%%%%%%%%%%%%%%%%%
\begin{document}
\LARGE 
\noindent
{\color{Maroon}Foundations of Mathematics \hfill Math 300 Sections 902, 905}\vspace{2pt}\\
\Large YOUR NAME\vspace{2pt}\\
\large
Third Homework: \hfill Due 14 September 2020
\normalsize    %%%% This returns the fontsize to 12 point
\vspace{10pt}  %%%% This makes a vertical space of 10 points (approx 70 points/inch) after the line break


%%%%%%%%%%%%%%%%%%%%%%%%%%%%%%%%%%%%%%%%%%%%%%%%%%%%%%%%%%%%%%%%%%%%%%%%%%%%%%%%%
\begin{enumerate}

 
%%%%%%%%%%%%%%%%%%%%%%%%%%%%%%%%%%%%%%%%%%%%%%%%%%%%%%%%%%%%%%%%%%%%%%%%%%%%%%%%%
\item Rewrite the following English sentences (which are  mathematical statements) as sentences involving quantifiers.
 \begin{enumerate}
   \item A trangle has three sides.
   \item The square of a real number is nonnegative.
   \item Some Aggies are not Human.
   \item An integer is necessarily prime or composite.
   \item Some even numbers are divisible by two and are divisible by seven.
   \item The sum of two even integers is an odd integer.
   \item Irrational numbers are real.
 \end{enumerate}
%%%%%%%%%%%%%%%%%%%%%%%%%%%%%%%%%%%%%%%%%%%%%%%%%%%%%%%%%%%%%%%%%%%%%%%%%%%%%%%%%

%\newpage
%%%%%%%%%%%%%%%%%%%%%%%%%%%%%%%%%%%%%%%%%%%%%%%%%%%%%%%%%%%%%%%%%%%%%%%%%%%%%%%%%
\item Negate each of the quantified statements from Question 1, again as English sentences.
%%%%%%%%%%%%%%%%%%%%%%%%%%%%%%%%%%%%%%%%%%%%%%%%%%%%%%%%%%%%%%%%%%%%%%%%%%%%%%%%%

%\newpage
%%%%%%%%%%%%%%%%%%%%%%%%%%%%%%%%%%%%%%%%%%%%%%%%%%%%%%%%%%%%%%%%%%%%%%%%%%%%%%%%%
\item Recall the following property of the integers:\newline
      ``If $n$ is an integer, then there is an integer $m$ with the property that $n+m=0$.''
 \begin{enumerate}
   \item Write this as a statement involving quantifiers.
   \item Give a useful negation of this statement.
   \item What is this property called?
 \end{enumerate}
%%%%%%%%%%%%%%%%%%%%%%%%%%%%%%%%%%%%%%%%%%%%%%%%%%%%%%%%%%%%%%%%%%%%%%%%%%%%%%%%%

%\newpage
%%%%%%%%%%%%%%%%%%%%%%%%%%%%%%%%%%%%%%%%%%%%%%%%%%%%%%%%%%%%%%%%%%%%%%%%%%%%%%%%%
\item  Negate each of the following statements (which are important definitions in mathematics).
  Assume that the symbols $f$, $K$, $a$, and $l$ are defined.
  \begin{enumerate}

    \item  For every $x\in K$, if $x\neq 0$, then there is a $y\in K$ such that $xy=1$.
    \item  For every real number $\epsilon >0$, there is a $\delta>0$ such that if $x\in\RR$ with $x\neq a$ and $|x-a|<\delta$, then
      $|f(x)-l|<\epsilon$. 
    \item   For every real number $\epsilon >0$, there is a $\delta>0$ such that if $x,y\in\RR$ with $|x-y|<\delta$, then
      $|f(x)-f(y)|<\epsilon$. 

  \end{enumerate}


%\newpage
%%%%%%%%%%%%%%%%%%%%%%%%%%%%%%%%%%%%%%%%%%%%%%%%%%%%%%%%%%%%%%%%%%%%%%%%%%%%%%%%%  
\item Is the following statement a tautology?
  \[
     (\forall x\in U) (P(x))\ \longrightarrow\ (\exists x\in U)(P(x))\,.
  \]
  Why or why not?
  Justify your assertions.
%%%%%%%%%%%%%%%%%%%%%%%%%%%%%%%%%%%%%%%%%%%%%%%%%%%%%%%%%%%%%%%%%%%%%%%%%%%%%%%%%

\newpage
%%%%%%%%%%%%%%%%%%%%%%%%%%%%%%%%%%%%%%%%%%%%%%%%%%%%%%%%%%%%%%%%%%%%%%%%%%%%%%%%%
\item Prove the following statement:

     For integers $a$, $b$, and $c$, if $a|b$ and $a|c$, then $a|(b+c)$.
  \begin{enumerate}
  \item  Construct a ``know-show'' table for a proof of this statement.
         You may find it useful to recycle LaTeX code from HW1.
   \item  Write your proof in paragraph form.  
  \end{enumerate}

%\newpage
%%%%%%%%%%%%%%%%%%%%%%%%%%%%%%%%%%%%%%%%%%%%%%%%%%%%%%%%%%%%%%%%%%%%%%%%%%%%%%%%%
\item Prove or find counterexamples to following statements.
      Write negations of the false statements in English.
  \begin{enumerate}
  \item  For all integers $a$, we have $\sqrt{a^2}=a$.
  \item  For all integers $a,b,c$ with $a\neq 0$, if $a|(bc)$ then $a|b$ or $a|c$.
  \item  For all integers $a,b$ with $a\neq 0$, if $a|b$, then $a^2|b^2$.
  \item  For all real numbers $x,y$ we have $\sqrt{x^2+y^2}>2xy$.
  \item  For all integers $a$, $b$, and $c$ with $a\neq 0$, if $a$ divides $(b-1)$ and $a$ divides $(c-1)$, then
    $a$ divides $(bc-1)$.
  \item For all integers $a$, $b$, and $c$ with $a\neq 0$, if $a$ divides  both $b-c$ and $b+c$, then  $a$ divides $b$.
  \end{enumerate}

%\newpage
%%%%%%%%%%%%%%%%%%%%%%%%%%%%%%%%%%%%%%%%%%%%%%%%%%%%%%%%%%%%%%%%%%%%%%%%%%%%%%%%%
\item Let $n$ be a positive integer and consider the statement we explored about congruence modulo $n$:

  For any integers $a,b,c,d$ if $a\equiv b \mod n$ and $c\equiv d\mod n$, then
    $(a+c)\equiv (b+d) \mod n$.
  
  \begin{enumerate}
  \item  Construct a ``know-show'' table for a proof of this statement.
         You may find it useful to recycle LaTeX code from HW1.
   \item  Write your proof in paragraph form.  
  \end{enumerate}
%%%%%%%%%%%%%%%%%%%%%%%%%%%%%%%%%%%%%%%%%%%%%%%%%%%%%%%%%%%%%%%%%%%%%%%%%%%%%%%%%


%\newpage
%%%%%%%%%%%%%%%%%%%%%%%%%%%%%%%%%%%%%%%%%%%%%%%%%%%%%%%%%%%%%%%%%%%%%%%%%%%%%%%%%
\item Repeat the previous question, but replace addition by multiplication.
%  {\bf\color{red}This may be hard}
%%%%%%%%%%%%%%%%%%%%%%%%%%%%%%%%%%%%%%%%%%%%%%%%%%%%%%%%%%%%%%%%%%%%%%%%%%%%%%%%%

%\newpage
%%%%%%%%%%%%%%%%%%%%%%%%%%%%%%%%%%%%%%%%%%%%%%%%%%%%%%%%%%%%%%%%%%%%%%%%%%%%%%%%%
\item Prove or find counterexamples to following statements.
  \begin{enumerate}
  \item  If $a$ is an integer with $a\equiv 2 \mod 6$, then $a^2 \equiv 4 \mod 6$.
  \item  If $a$ is an integer with $a^2 \equiv 4 \mod 6$, then  $a\equiv 2 \mod 6$.
  \end{enumerate}
%%%%%%%%%%%%%%%%%%%%%%%%%%%%%%%%%%%%%%%%%%%%%%%%%%%%%%%%%%%%%%%%%%%%%%%%%%%%%%%%%


%\newpage
%%%%%%%%%%%%%%%%%%%%%%%%%%%%%%%%%%%%%%%%%%%%%%%%%%%%%%%%%%%%%%%%%%%%%%%%%%%%%%%%%
\item Consider Statement (e) in Problem 7.
  \begin{enumerate}
  \item  Rewrite this as a statment involving congruences.
  \item  Formulate a useful generalization of you statement in part (a) of this problem.
  \end{enumerate}
%%%%%%%%%%%%%%%%%%%%%%%%%%%%%%%%%%%%%%%%%%%%%%%%%%%%%%%%%%%%%%%%%%%%%%%%%%%%%%%%%


\end{enumerate}
%%%%%%%%%%%%%%%%%%%%%%%%%%%%%%%%%%%%%%%%%%%%%%%%%%%%%%%%%%%%%%%%%%%%%%%%%%%%%%%%%


\end{document}
%%%%%%%%%%%%%%%%%%%%%%%%%%%%%%%%%%%%%%%%%%%%%%%%%%%%%%%%%%%%%%%%%%%%%%%%%%%%%%%%%

