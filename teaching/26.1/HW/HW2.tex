%HW2.tex
%
% Second Homework for Computational Algebraic Geometry
% Frank Sottile
%%%%%%%%%%%%%%%%%%%%%%%%%%%%%%%%%%%%%%%%%%%%%%%%%%%%%%%%%%%%%%%%%%%%%%%
\documentclass[12pt]{article}
\usepackage{multicol,amssymb,amsmath}
\usepackage{colordvi,graphicx}
\headheight=8pt
%
\topmargin=-75pt
\textheight=720pt   \textwidth=595pt
\oddsidemargin=-60pt \evensidemargin=-60pt

\pagestyle{empty}

%%%%%%%%%%%%%%%%%%%%%%%%%%%%%%%%%%%%%%%%%%%%
\newcommand{\CC}{{\mathbb C}}
\newcommand{\KK}{{\mathbb K}}
\newcommand{\NN}{{\mathbb N}}
\newcommand{\RR}{{\mathbb R}}
\newcommand{\calI}{{\mathcal I}}
\newcommand{\calV}{{\mathcal V}}
\newcommand{\be}{{\bf e}}

\newcommand{\Hom}{\mbox{Hom}}
\newcommand{\spec}{\mbox{spec}}
\newcommand{\cone}{\mbox{cone}}

\newcommand{\vect}[2]{(\begin{smallmatrix}#1\\#2\end{smallmatrix})}

\def\Color#1#2{\special{color push cmyk #1}#2\special{color pop}}
%\def\Indigo#1{\Color{.42 1. 0. .49}{#1}}
\def\Indigo#1{\Color{1. .95 .05 .4}{#1}}
\def\MyViolet#1{\Color{.6 1. 0. .15}{#1}}

\newcommand{\barsl}{\noindent\begin{minipage}[t]{590pt}
\Indigo{\rule{590pt}{1.2pt}}\vspace{-5.7mm}\\
\MyViolet{\rule{590pt}{1.2pt}}\vspace{-5.7mm}\\
\Blue{\rule{590pt}{1.2pt}}\vspace{-5.7mm}\\
\Green{\rule{590pt}{1.2pt}}\vspace{-5.7mm}\\
\Yellow{\rule{590pt}{1.2pt}}\vspace{-5.7mm}\\
\Orange{\rule{590pt}{1.2pt}}\vspace{-5.7mm}\\
\Red{\rule{590pt}{1.2pt}}\smallskip
\end{minipage}}


\newcommand{\barsn}{\noindent\begin{minipage}[t]{590pt}
\Indigo{\rule{590pt}{1.1pt}}\vspace{-4.5mm}\\
\MyViolet{\rule{590pt}{1.1pt}}\vspace{-4.5mm}\\
\Blue{\rule{590pt}{1.1pt}}\vspace{-4.5mm}\\
\Green{\rule{590pt}{1.1pt}}\vspace{-4.5mm}\\
\Yellow{\rule{590pt}{1.1pt}}\vspace{-4.5mm}\\
\Orange{\rule{590pt}{1.1pt}}\vspace{-4.5mm}\\
\Red{\rule{590pt}{1.1pt}}\bigskip
\end{minipage}}


\def\demph#1{\Maroon{{\sl #1}}}
\def\defcolor#1{\Maroon{#1}}

\begin{document}
\LARGE
\noindent
Math 648: 
Computational Algebraic Geometry\vspace{2pt}\\
Frank Sottile\vspace{2pt}\\
10 February 2026\hfill
\Large\sf
 Second Homework\makebox[40pt][l]{\ }
\large\vspace{10pt}

\barsl

%\normalsize

\begin{center}
\begin{minipage}[t]{560pt}
%%%%%%%%%%%%%%%%%%%%%%%%%%%%%%%%%%%%%%%%%%%%%%%%%%%%%%%%%%%%%%%%%%%%%%%%%%%%%%%%%%%%%%%%%%%%%%%%%%%%
\noindent\Maroon{{\large\sf   Please hand these into Frank on Tuesday 17 February 2026.}}

\begin{enumerate}

%%%%%%%%%%%%%%%%%%%%%%%%%%%%%%%%%%%%%%%%%%%%%%%%%%%%%%%%%%%%%%%%%%%%%%%%%%%%%%%%%%%%%%%%%%%%%%%%%%%%
\item  
  A quartic surface in 4-space is defined by two linearly independent quadratic polynomials $F=G=0$ in four variables.
  A line in 4-space may be represented by a point $p=(a,b,c,0)$ and a direction vector $v=(e,f,g,1)$ (six coordinates), and
  parameterized by $\lambda\mapsto p+\lambda v$.
  Evaluating $F$ and $G$ at $p+\lambda v$ gives two quadratic polynomials in $\lambda$.
  The line $p+\lambda v$ lies on the quartic surface $\calV(F,G)$ exactly when the six coefficients of $\lambda$ in these
  two quadratic polynomials vanish.

  Modify (including the comments) the file for the 27 lines to compute the number of lines on such a quartic surface in
  4-space.
  How many lines did you find?
  Explain this computation and what you learned in a paragraph (hand in with the other problems)
  and email the resulting Macaulay2 or Singular script to   Sottile.
%%%%%%%%%%%%%%%%%%%%%%%%%%%%%%%%%%%%%%%%%%%%%%%%%%%%%%%%%%%%%%%%%%%%%%%%%%%%%%%%%%%%%%%%%%%%%%%%%%%%

%%%%%%%%%%%%%%%%%%%%%%%%%%%%%%%%%%%%%%%%%%%%%%%%%%%%%%%%%%%%%%%%%%%%%%%%%%%%%%%%%%%%%%%%%%%%%%%%%%%%
\item   Show that the radical of a monomial ideal is a monomial ideal, and that a monomial ideal is radical if and only if it
     has square-free generators. 
     (Square-free means that no variable occurs to a power greater than 1.)\newline
     \Violet{A monomial ideal is {\sl square-free} when it has square-free generators.}

    Show that if an ideal $I$ has a square-free initial ideal, then $I$ is radical.
       Give an example to show that the converse of this statement is false.

%%%%%%%%%%%%%%%%%%%%%%%%%%%%%%%%%%%%%%%%%%%%%%%%%%%%%%%%%%%%%%%%%%%%%%%%%%%%%%%%%%%%%%%%%%%%%%%%%%%%

%%%%%%%%%%%%%%%%%%%%%%%%%%%%%%%%%%%%%%%%%%%%%%%%%%%%%%%%%%%%%%%%%%%%%%%%%%%%%%%%%%%%%%%%%%%%%%%%%%%%       
\item Suppose that $x_1\prec x_2\prec \dotsb$.  \Red{{\sf This is the reverse from other problems.}}\vspace{-3pt}
    \begin{enumerate}
    \item Use Buchberger's algorithm to compute by hand the reduced Gr\"obner basis for $\langle x_1+x_2, x_1x_2\rangle$,
           and for $\langle x_1+x_2+x_3, x_1x_2+x_1x_3+x_2x_3, x_1x_2x_3\rangle$ with respect to the degree
           reverse lexicographic order.
    \item Do the same for the lexicographic order.
    \item  These polynomials are the \Blue{{\sl elementary symmetric polynomials} $e_k(x_1,\dotsc,x_n)$} for $k\leq n$ when $n=2,3$.
      Generalize this to arbitrary $n$.
 \end{enumerate}
%%%%%%%%%%%%%%%%%%%%%%%%%%%%%%%%%%%%%%%%%%%%%%%%%%%%%%%%%%%%%%%%%%%%%%%%%%%%%%%%%%%%%%%%%%%%%%%%%%%%    

\end{enumerate}
%%%%%%%%%%%%%%%%%%%%%%%%%%%%%%%%%%%%%%%%%%%%%%%%%%%%%%%%%%%%%%%%%%%%%%%%%%%%%%%%%%%%%%%%%%%%%%%%%%%%

\end{minipage}\qquad
\end{center}

\end{document}
