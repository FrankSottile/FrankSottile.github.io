\documentclass[12pt]{article}

\usepackage{aurical}
\usepackage[T1]{fontenc}
\usepackage{amssymb,amsmath,amsthm} 
\usepackage[margin=1.2in]{geometry}
\usepackage{graphicx,ctable,booktabs}
\usepackage{amsfonts}
\usepackage[section]{placeins}
\usepackage[mathscr]{euscript}
\usepackage[margin=0.5cm]{caption}
\usepackage{subcaption}
\usepackage{listings}
\lstloadlanguages{[5.2]Mathematica}
\newcommand{\genby}[1]{\langle #1 \rangle}
\usepackage{enumerate}
\usepackage{listings}

\usepackage{blindtext}
\newenvironment{texto}[1][\textwidth]{
   \begin{center} % so the minipage is centered
   \begin{minipage}[t]{#1}
   \raggedright % so the minipage's text is left justified
   \hspace{-.5 cm}
}{
  \end{minipage}
   \end{center}
}

\newenvironment{example}[1][Example]{\begin{trivlist}
\item[\hskip \labelsep {\bfseries #1}]}{\end{trivlist}}
\newenvironment{nonexample}[1][Nonexample]{\begin{trivlist}
\item[\hskip \labelsep {\bfseries #1}]}{\end{trivlist}}
\newenvironment{remark}[1][Remark]{\begin{trivlist}
\item[\hskip \labelsep {\bfseries #1}]}{\end{trivlist}}




\newcommand{\ds}{\displaystyle}
\newcommand{\ZZ}{\mathbb{Z}}
\newcommand{\QQ}{\mathbb{Q}}
\newcommand{\RR}{\mathbb{R}}
\newcommand{\CC}{\mathbb{C}}
\newcommand{\EE}{\mathcal{E}}
\newcommand{\FF}{\mathcal{F}}
\newcommand{\BB}{\mathcal{B}}
\newcommand{\VV}{\mathcal{V}}
\newcommand{\HH}{\mathcal{H}}
\newcommand{\lcm}{\text{lcm}}
\newcommand{\ini}{\text{in}}
\renewcommand{\NN}{\mathbb{N}}
\newcommand{\ceil}[1]{\lceil#1 \rceil}
\newcommand{\floor}[1]{\lfloor#1 \rfloor}

\newtheorem{theorem}{Theorem}
\newtheorem{prop}{Proposition}
\newtheorem{lemma}{Lemma}
\newtheorem{corollary}{Corollary}
\newtheorem{conjecture}{Conjecture}
\newtheorem{claim}{Claim}


\theoremstyle{definition}
\newtheorem{definition}{Definition}
\newtheorem{exercise}{Exercise}

\usepackage{fancyhdr}
\pagestyle{fancy}
\fancyfoot[L]{\small\scshape TACO Seminar} 
\rfoot{\footnotesize Madeline Brandt} 
\rhead{ } 
\renewcommand{\headrulewidth}{0pt} 

\usepackage{aurical}
\usepackage[T1]{fontenc}

\begin{document}

\title{\textsc{Ideals and Varieties Exercises}}
\author{Praise Adeyemo}

\maketitle

\thispagestyle{empty}

\begin{exercise} Let $\mathbb{A}_K^{n^2}$ be identified with the set of $M_{n \times n}$ matrices.
\begin{enumerate}
\item Show that the general linear group $GL_n(K) \subset \mathbb{A}_K^{n^2}$ of invertible matrices is not algebraic.
\item How can it be made algebraic?
\item Repeat both parts for the algebraic torus $(\mathbb{C}^*)^n \subset \mathbb{C}^n$.
\end{enumerate}
\end{exercise}

\begin{exercise}
Let $f(x)= x^4 + x^3 - x^2 + x - 2$, and $g(x) = x^3 + x^2 + x+1$. Use the Sylvester matrix of $f$ and $g$ to investigate whether or not the polynomials share a common root in $\mathbb{Q}[x]$. Then, do this with the Euclidean algorithm.
\end{exercise}

\begin{exercise} How would you describe lines in $\mathbb{A}_K^2$? What is the algebraic interpretation of your description?
\end{exercise}

\begin{exercise} A map $\phi : \mathbb{A}_k^2 \rightarrow \mathbb{A}_K^2$ is an \emph{affine transformation} if and only if there exists a vector $v \in \mathbb{K}^2$ and a matrix $A \in GL_2(K)$ such that $\phi(x) = Ax+v$ for all $x \in \mathbb{A}_K^2$. Show that the set of all affine transformations, denoted by $\text{Aff}(\mathbb{A}_K^2)$, forms a group.
\end{exercise}

\begin{exercise} Show that the Zariski topology on $\mathbb{A}_k^n$ is not Hausdorff when $K$ is an infinite field. What happens when $K$ is finite?
\end{exercise}

\begin{exercise} The equation $x^2+y^2 = z^2$ has many solutions over $\mathbb{Z}$. If $(a,b,c)$ is a solution and $n \in \mathbb{N}$, then $(n \cdot a, n \cdot b, n\cdot c)$ is also a solution. Find infinitely many solutions where $(a,b,c)$ have no common factor greater than 1.
\end{exercise}

\begin{exercise} Show that $V(x+y,x^2) = V(x+y,y^2)$. 
\end{exercise}

\begin{exercise} Show that the set of matrices in $\mathbb{A}_k^{n^2}$ with a repeated eigenvalue is an algebraic set.
\end{exercise}

\maketitle


\end{document}