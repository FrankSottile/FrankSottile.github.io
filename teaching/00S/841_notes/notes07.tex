%notes07.tex
%
% Frank Sottile
%  2000
% Madison, WI
%
\documentclass[12pt]{amsart}
\usepackage{amssymb}
\usepackage{epsf}

\headheight=8pt     \topmargin=-10pt
\textheight=644pt   \textwidth=432pt
\oddsidemargin=18pt \evensidemargin=18pt

\def\silentfootnote#1{{\let\thefootnote\relax\footnotetext{#1}}}

\newcommand{\DOT}{{\mbox{\Large\bf .}}}

\newcommand{\G}{{\bf G}}

%\def\baselinestretch{1.2}

\begin{document}
\begin{center}
\Large
The Geometry of Algebraic Groups\\
\large
Math 841\\
Frank Sottile\\
{\tt notes07}
\end{center}\bigskip

\silentfootnote{\sl Version of 2 April 2000} 


\section{Solvable Subgroups}

We begin with a simple topological lemma.
\medskip

\noindent{\bf Lemma. }
{\it
Let $S$ be a connected subset of an algebraic group $G$ containing
the identity $e$.
Then the subgroup $\langle S \rangle$ generated by $S$ is connected.
If $S$ is irreducible and constructible, then $\langle S \rangle$ is closed.
}\medskip

\begin{proof}
Since $S$ is connected, $S^{-1}$, the collection of inverses of elements of
$S$ is too, and so we may replace $S$ by $S\cup S^{-1}$.
Since $S\times S$ is connected in $G\times G$, we see that its image
$S^2$ under the multiplication map is also connected.
Similarly for $S^3, S^4$ and so on.
But the subgroup generated by $S$ is their union, 
$S\cup S^2\cup S^3\cup\cdots$, and hence is connected.

For the last statement, observe that each $S^i$ is irreducible and
constructible, and so $\overline{S^i}$ is irreducible.
Since irreducible closed sets have the increasing chain peoperty, we see
that there is some $i$ with $\overline{S^i}=\overline{S^j}$ for $i<j$, and
so $\overline{\langle S \rangle}=\overline{S^i}$.
Let $V\subset S^i$ be an open dense subset of $\overline{S^i}$.
For $x\in \overline{\langle S \rangle}$, we have $x V^{-1}$ is also open and
dense, and so $\emptyset \neq x V^{-1}\cap V$, which implies that 
$\overline{\langle S \rangle}=V\cdot V\subset S^i\cdot S^i\subset 
\langle S \rangle$.
\end{proof}


A special consequence is that the commutator subgroup 
$[G,G]$ of a connected algebraic group is a closed connected subgroup, as it
is the group generated by all commutators  
$[g,h]:=ghg^{-1}h^{-1}$ for $g,h\in G$, and these commutators form an
irreducible constructible set containing the identity $e$.
\bigskip


A group $G$ is {\sl solvable} if $G$ has a finite chain of subgroups 
$G \vartriangleright G_1 \vartriangleright \cdots
 \vartriangleright G_a = \{e\}$ with each $G_{i+1}$ normal in $G_i$ with the
succcessive quotients $G_{i}/G_{i+1}$ abelian.
This is equivalent to the requirement that the derived series of $G$
terminates with the identity.
(The derived series of $G$ is the sequence of normal subgroups 
$G\vartriangleright [G,G]=:G_1 \vartriangleright [G_1,G_1]\cdots$.)
\medskip


\noindent{\bf Lemma. }
{\it
Let $G$ be an algebraic group and $A,B$ subgroups of $G$.
\begin{enumerate}
 \item[1)] $\overline{[B,B]}=[\overline{B},\overline{B}]$.

 \item[2)] If $B$ is solvable, then $\overline{B}$ is also solvable.

 \item[3)] If $A,B$ are normal and solvable, then $AB$ is also normal and
           solvable. 

 \item[4)] There is a unique maximal cinnected normal solvable subgroup of
    $G$, which is necessarily closed.

\end{enumerate}
}\medskip

\begin{proof}
Consider the continuous map $G\times G\to G$ given by $(g,h)\mapsto [g,h]$.
Let $C_B$ be the image of $B\times B$ under this map.
Then we have $C_B\subset C_{\overline{B}}\subset \overline{C_B}$,
which implies that  $\overline{[B,B]}\supset[\overline{B},\overline{B}]$,
which is closed and contains $[B,B]$, forcing equality.

For (2), define subgroups $B_i$ of $B$ by $B_0:=B$ and $B_{i+1}=[B_i,B_i]$.
Then by (1), we have $\overline{B_i}= (\overline{B})_i$.
Since $B$ is solvable, wehave $B_j=\{e\}$, for some j, forcing
$(\overline{B})_j=\{e\}$.


Since $A$ and $B$ are both normal, $AB$ is, too.
To see that $AB$ is solveble, define $B_i$ by $B_0:=B$ and
$B_{i+1}:=[B_i,B_i]$, and the same for $A_i$.
Then $AB\supset AB_1\supset B_2 \subset \cdots A \supset A_1 \supset \cdots
\supset {e}$. is a decreasing chain of normal subgroups of $AB$ with each
quotient abelian.


The last statement is an immediate consequence of the first three.
\end{proof}


A consequence of this Lemma is that an algebraic group $G$ has a unique maximal
normal solvable subgroup $R(G)$, called its radical.
A connected algebraic group $G$ is called {\sl semisimple} if $R(G)=\{e\}$.
When $k$ is algebraically closed, semisimple algebraic groups are classified
by their root systems.



\section{Borel Fixed Point Theorem}


Let $V\simeq k^n$.
The flag variety ${\mathcal F}(V)$ is
the set of 
(complete) flags in $V$, which are sequence of $k$-subspaces
$$
   F_{\DOT}\ :\ F_1\subset F_2\subset\cdots\subset F_n\ =\ V\,.
$$
We may regard ${\mathcal F}(V)$ as a projective variety as follows:
It is naturally a subset of the product of Grassmannians
$$
  {\bf G}_1V\times {\bf G}_2 V\times \cdots\times {\bf G}_nV\,.
$$
To see that is it closed, and hence a projective variety,
consider the condition for $F_{i-1}\subset F_i$ in
the product of affine pieces $G_{\alpha}\times G_\beta$ of 
$\G_{i-1}\times \G_i$.
It is an easy exercise to write down $n-i$ linear forms whose entries depend
linearly on the coordinates in $G_\beta$ such that a plane $F_i\in G_\beta$
is the kernel of the forms evaluated at $F_i$.
Then the equations for $F_{i-1}\subset F_i$ in $G_{\alpha}\times G_\beta$ 
are simply the multilinear equations that $F_{i-1}\in G_\alpha$
be annihilated by these linear forms.
\medskip


\noindent{\bf Theorem. }
{\it 
Let $G$ be a connected, solvable algebraic group and $\emptyset\neq X$ a
projective variety on which $G$ acts.
Then $G$ has a fixed point on $X$.
}\medskip

\begin{proof}
If $\dim G=0$, then $G=\{e\}$, and there is nothing to prove.
We induct on the dimension of $G$.

Let $H:=[G,G]$, which is a solvable, connected, and proper subgroup of $G$.
(Since $G$ is connected, $C_G:=\{ghg^{-1}h^{-1}\mid g,h\in G\}$ is
connected, which 
implies that $[G,G]=\langle C_G \rangle$ is connected.)
Let $X^H$ be the points of $X$ fixed by $H$, which is nonempty by
induction.
This fixed point set $X^H$ is closed, as 
$X^H=\Delta_X \cap \bigcap_{g\in G} \Gamma_g$, where 
$\Gamma_g\subset X\times X$
is the graph of multiplication by $g\in G$, and $\delta_X\simeq X$
is the diagonal in $X\times X$.
Thus $X^Hl$ is projective and we have $G.H\subset X^H$ as $H$ is normal in
$G$. 
Without loss of generality, we may assume that $X=X^H$.


By our Theorem on orbits, there exists an $x\in X$ with $G.x$ closed.
Then the stabilizer subgroup $G_x:=\{g\in G\mid g.x=x\}$ of $x$ is closed
(it is the inverse image of $\{x\}$ under the map $G\to X$ given by 
$g\mapsto g.x$)
and it contains $H=[G,G]$, so it is also normal.
Since $G/G_x\simeq G.x$, with $G/G_x$ an algebraic group (hence affine) and 
$G.x$ projective, we must have that each is a single point.
Thus $G_x=G$ and $x\in X^G$.
\end{proof}



This has an important consequence in the theory of algebraic groups, the 
Lie-Kolchin Theorem:
\medskip


\noindent{\bf Lie-Kolchin Theorem. }
{\it
Let $G\subset GL(V)$ be a connected solvable subgroup.
Then there is an ordered  basis for $V$ for which $G$ is a subset of the
upper triangular matrices with respect to that basis.
}\medskip

\begin{proof}
Observe that a subgroup $G$ of $GL(V)$ consists of upper triangular matrices
with respect to 
some (ordered) basis $e_1,\ldots,e_n$ if and only if it stabilizes the flag
defined by that basis:
$$
  \langle e_1 \rangle\subset
  \langle e_1,e_2 \rangle\subset\cdots\subset
  \langle e_1,e_2,\ldots,e_n \rangle\,.
$$
Thus $G$ is upper triangular with respect to some basis if and only if 
its natural action on the flag variety 
${\mathcal F}(V)$ has a fixed point.
Replace $G$ by its closure in $GL(V)$, which is still solvable.
Then the action of $G$ on ${\mathcal F}(V)$ has a fixed point, $F_{\DOT}$.
If we choose an ordered basis $e_1,\ldots,e_n$ for $V$ such that 
$F_i=\langle e_1,\ldots,e_i\rangle$, then $G$ will be upper triangular with
respect to that basis.
(Here, $G$ acts on the right of $V$.)
\end{proof}

\section{Borel Subgroups and Flag Varieties}

A {\sl Borel subgroup} of an algebraic group $G$ is a maximal connected
solvable subgroup.  (Hence closed).
Since any Borel subgroup $B$ has $B\subset G^{\circ}$, the identity
component of $G$, we assume that $G$ is connected.
\medskip

\noindent{\bf Theorem. }
{\it
Let $B$ be any Borel subgroup of $G$.
Then $G/B$ is a projective variety and all other Borel subgroups are
conjugate to $B$ in $G$.}\medskip

\begin{proof}
Let $S\subset G$ be a Borel subgroup of maximal dimension.
Then there exists a representation $\rho:G\to GL(V)$ and a 1-dimensional
subspace $F_1$ of $V$ for which $S$ is its stabilizer, by Chevalley's
Theorem. 
By the Lie-Kolchin Theorem, $S$ stabilizes a flag in $V/F_1$.
Let $F_{\DOT}$ be the lift of that flag to $V$.
Since $S$ is the stabilizer of $F_1$,
it is also the stabilizer of $F\!_{\DOT}$.

Thus $G/B\simeq G.F_{\DOT}\subset {\mathcal F}(V)$.
Let $x\in \overline{G.F_{\DOT}}$.
Then the stabilizer of $x$ is a solvable subgroup of
$G$ (as it is a subgroup of the solvable subgroup of upper triangular
matrices in $GL(V)$ with respect to the flag represented by $x$).
This subgroup is closed, and thus its dimension is at most that of $S$, as
$S$ has maximal dimension among all solvable subgroups of $G$.
Thus the dimension of the orbit $G.x$ is at least the dimension of 
$G.F\!_{\DOT}$.
(This is by the Theorem on the dimension of fibres:
\medskip

\noindent{\bf Theorem. }
{\it 
Suppose $f:X\to Y$ is a map of irreducible algebraic varieties with
$f(X)=Y$. 
Then there is an open subset $U$ of $Y$ such that for every $y\in U$, the
fibre $f^{-1}(y)$ of the map over $y$ has  dimension equal to
$\dim X - \dim Y$.
}\medskip

We apply this Theorem to the map $G\to G.x$, whose fibre is the isotropy
subgroup of $x$.)
By our Theorem on orbits, either the dimension of $g.x$ is less than
that of $G.F_{\DOT}$ or else $G.x=G.F_{\DOT}$.

We conclude that $G.F_{\DOT}$ is closed, which implies that $G/S$ is a
projective variety.
Any Borel subgroup $B$ of $G$ acts on $G/S$ on the left, and thus has a
fixed point $xS$.
Thus we have $BxS=xS$, which implies that $x^{-1}BxS=S$ or that 
$x^{-1}Bx\subset S$.
Since each is a Borel subgroup, we have $x^{-1}Bx=S$, which proves both
parts of the Theorem.
\end{proof}


Suppose that $H$ is a closed subgroup of $G$ and that the quotient $G/H$ is
a projective variety.
Let $B$ be a Borel subgroup of $G$, which acts on the left of $G/H$.
Then by the Borel Fixed Point Theorem, $B$ has a fixed point $x.H$ on $G/H$. 
As in the proof of the proceeding Theorem, this implies that 
$B\subset x^{-1}Hx$.
Similarly, if $H$ is a closed subgroup of $G$ containing a Borel subgroup
$B$ of $G$, then we have a surjection $G/B\to G/H$ of a projective variety
onto a quasi-projective variety.
Since the image of a projective variety under any algebraic map is closed,
we conclude that $G/H$ is a projective variety.
A subgroup $P$ of an algebraic group $G$ is {\sl parabolic} if $G/P$ is
projective.  
We conclude:
\medskip

\noindent{\bf Corollary. }
{\it
A closed subgroup $P$ of $G$ is parabolic if and only if it contains a Borel
subgroup of $G$.
A connected subgroup $H$ of $G$ is a Borel subgroup if and only if $H$ is
solvable and $G/H$ is projective.}\medskip


\noindent{\bf Definition. }
A {\sl flag variety} (or generalized flag manifold) is a projective
homogeneous space $G/P$ of an algebraic group $G$.
We wish to classify these flag varieties and study their geometry.
\medskip

Consider the group $G=GL(n,k)$.
The group $B$ of upper triangular matrices is a Borel subgroup of $G$, and
the flag variety $G/B$ in this case is the space of complete flags
${\mathcal F}_n:={\mathcal F}(k^n)$ defined previously.
Given any subset $S\subset \{1,2,\ldots,n-1\}$, consider the variety of
partial flags ${\mathcal F}_S={\mathcal F}_S(k^n)$ given by flags
$$
   F_{s_1}\subset F_{s_2}\subset \cdots \subset F_{s_m}\subset k^n\,,
$$
where $S=\{s_1<s_2<\cdots<s_m\}$ and $\dim F_{s_i}=s_i$.
We call $S$ the {\sl type} of the flag.

There is an obvious surjective `forgetful' map from 
${\mathcal F}_n\to {\mathcal F}_S$ given by sending a complete flag
$F_{\DOT}$ to the partial flag formed by the subspaces in $F_{\DOT}$ whose
dimension is in $S$.
Thus we see that these partial flag varieties are projective varieties;
either as they are the image of a projective variety, or else directly in
the same way we argued that ${\mathcal F}_n$ was projective.

If $e_1,\ldots, e_n$ is a basis for $k^n$, then we let the standard flag
$E_{\DOT}$ be the flag whose $i$th subspace $E_i$ is spanned by
$e_1,\ldots, e_i$.
Then $B$ is the isotropy subgroup of $E_{\DOT}$, and we define $P_S$ to be
the isotropy subgroup of the subflag of $E_{\DOT}$ of type $S$.
If $T\supset S$, then $P_T\subset P_S$ and so the projection from 
${\mathcal F}_n$ to ${\mathcal F}_S$ factors through ${\mathcal F}_T$. 

In this way we obtain a lattice of parabolic subgroups of $G$ containing $B$ 
which is anti-isomorphic to the lattice of subsets of $\{1,\ldots,n-1\}$.

There is even more geometric information in this structure.
Let $A_{n-1}$ be the graph consisting of a path of length $n-1$:
$$
  \epsfbox{figures/An.eps}
$$
This graph corresponds to the flag manifold ${\mathcal F}_n$.
If we have a subset $S\subset \{1,\ldots,n-1\}$, then we may consider the 
induced subgraph that remains when we remove the vertices from $A_{n-1}$
labeled by $S$. 
Below, we show $A_5$ and the induced subgraph when $S=\{2,3\}$:
$$
  \epsfbox{figures/A6.eps}
$$
This subgraph encodes the geometry of the fibres 
${\mathcal F}_n\to{\mathcal F}_S$ in the following way.
Each connected component of the subgraph is isomorphic to some $A_{m-1}$.
Let $\mbox{Fib}_S$ be the product of flag varieties ${\mathcal F}_m$, one
for each component of the subgraph.
Then $\mbox{Fib}_S$ is isomorphic to the fibre of the the projection
${\mathcal F}_n\to{\mathcal F}_S$.

Consider this in the example of 
$\pi\colon{\mathcal F}_6\to {\mathcal F}_{\{2,3\}}$.
This map is given by 
$$
  \pi\ \colon\ 
  F_1\subset F_2\subset F_3\subset F_4\subset F_5\ \longmapsto\ 
  F_2\subset F_3\,.
$$
The inverse image $\pi^{-1}(F_2\subset F_3)$ is given by choosing
subspaces $F_1,F_4$, and $F_5$ which `fill out' the flag 
$F_2\subset F_3$.
We may choose $F_1$ to be any 1-dimensional subspace of the 2-dimensional
vector space $F_2$, and $F_4\subset F_5$ corresponds to any complete flag in 
$k^6/F_3\simeq k^3$.
In this way we see that 
$$
\pi^{-1}(F_2\subset F_3)\ \simeq\  
   {\mathcal F}(F_2)\times {\mathcal F}(k^6/F_3)\ \simeq\ 
   {\mathcal F}_2\times {\mathcal F}_3\,,
$$
in agreement with the description given above.





\end{document}
