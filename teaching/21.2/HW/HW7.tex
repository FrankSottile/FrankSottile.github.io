%HW7.tex
%Seventh Homework -- Math 300H 
%
%  The percent sign is a comment character
%
%%%%%%%%%%%%%%%%%%%%%%%%%%%%%%%%%%%%%%%%%%%%%%%%%%%%%%%%%%%%%%%%%%%%%%%%%%%%%%%%%%
%
%   Look these up on line.  The first sets the type of document, and the next are for mathematics symbols, graphics and color
%
\documentclass[12pt]{article}
\usepackage{amssymb,amsmath,mathtools}
\usepackage{graphicx}
\usepackage[usenames,dvipsnames,svgnames,table]{xcolor}
\usepackage{multirow}   % This is for more control over tables
%%%%%%%%%%%%%%%%%%%%%%%%%%%%%%%%  Layout     %%%%%%%%%%%%%%%%%%%%%%%%%%%%%%%%%%%%%%
\usepackage{vmargin}
\setpapersize{USletter}
\setmargrb{1cm}{1cm}{2cm}{1cm} % --- sets all four margins LTRB


\newcommand{\RR}{{\mathbb R}}  % This is the backboard symbol for the real numbers.  Note how it is used below
\newcommand{\NN}{{\mathbb N}}  % 
\newcommand{\QQ}{{\mathbb Q}}  % 
\newcommand{\ZZ}{{\mathbb Z}}  %

\newcommand{\calP}{{\mathcal P}}  %Caligraphic P for power set

\newcommand{\sep}{{\ :\ }}     % This is for the : in our notation for building sets.
\newcommand{\lsim}{\mathord{\sim}}  %  This is to remove the extra spacing around \sim (it is a binary relation) for the logical not

\newcommand{\defcolor}[1]{{\color{blue}{#1}}}
\newcommand{\demph}[1]{{\color{blue}\sl{#1}}}
%%%%%%%%%%%%%%%%%%%%%%%%%%%%%%%%%%%%%%%%%%%%%%%%%%%%%%%%%%%%%%%%%%%%%%%%%%%%%%%%%
%
%   Please edit this as appropriate
%
\begin{document}
\LARGE 
\noindent
{\color{Maroon}Foundations of Mathematics \hfill Math 300H Section 970}\vspace{2pt}\\
\Large \vspace{2pt}\\
\large
Seventh Homework: \hfill Due \  25 October 2021
\normalsize\medskip


\noindent{\color{blue}\rule{528.3675pt}{2pt}}


\noindent {\color{Maroon}\bf Definition:}
The \demph{Fibonacci sequence} $\{f_n\mid n\geq 1\}$ is defined by $f_1=f_2=1$ and for $n\geq 2$, $f_{n+1}=f_{n}+f_{n-1}$.

\noindent {\color{Maroon}\bf Read:}   Chapters 7 and 8 in Fourth edition.
(Chapter 8 in fourth edition is Chapter 7 in third.  The new Chapter 7 in Fourth edition is a review of proof techniques.)


\noindent{\color{blue}\rule{528.3675pt}{2pt}}


%%%%%%%%%%%%%%%%%%%%%%%%%%%%%%%%%%%%%%%%%%%%%%%%%%%%%%%%%%%%%%%%%%%%%%%%%%%%%%%%%
\begin{enumerate}  %  the \begin{..}  \end{..} stars and ends an environment.

%%%%%%%%%%%%%%%%%%%%%%%%%%%%%%%%%%%%%%%%%%%%%%%%%%%%%%%%%%%%%%%%%%%%%%%%%%%%%%%%%
\item  Compute the first 15 terms of the Fibonacci sequence (this will help for later problems).
           Note that the recursion $f_{n+1}=f_{n}+f_{n-1}$ may be rewritten $f_{n-1}=f_{n+1}-f_n$.
           Use this to extend the Fibonacci sequence to {\sl negative} integers and compute the values of $f_n$ for
           $-10\leq n \leq 0$.
           Conjecture a formula for $f_{-n}$ for $n\in\NN$ and prove it by induction.
%%%%%%%%%%%%%%%%%%%%%%%%%%%%%%%%%%%%%%%%%%%%%%%%%%%%%%%%%%%%%%%%%%%%%%%%%%%%%%%%%


%%%%%%%%%%%%%%%%%%%%%%%%%%%%%%%%%%%%%%%%%%%%%%%%%%%%%%%%%%%%%%%%%%%%%%%%%%%%%%%%%
\item Explore sums of squares of the Fibonacci numbers and conjecture a formula for 
  \[
       f_1^2 + f_2^2 + f_3^2 + \dotsb + f_n^2\,.
  \]
  Prove your formula.
%%%%%%%%%%%%%%%%%%%%%%%%%%%%%%%%%%%%%%%%%%%%%%%%%%%%%%%%%%%%%%%%%%%%%%%%%%%%%%%%%


%%%%%%%%%%%%%%%%%%%%%%%%%%%%%%%%%%%%%%%%%%%%%%%%%%%%%%%%%%%%%%%%%%%%%%%%%%%%%%%%%
\item Look up the term \demph{Pythagrean triple} (it is in our book).
    Investigate the following\medskip

  {\bf Conjecture.}  {\sl For each natural number $n$, the numbers $f_nf_{n+3}$,
  $2f_{n+1}f_{n+2}$, and $(f_{n+1}^2+f_{n+2}^2)$ form a {\color{Blue}Pythagorean triple}.}\medskip

   If true, provide a proof, and if false, a counterexample.
%%%%%%%%%%%%%%%%%%%%%%%%%%%%%%%%%%%%%%%%%%%%%%%%%%%%%%%%%%%%%%%%%%%%%%%%%%%%%%%%%


%%%%%%%%%%%%%%%%%%%%%%%%%%%%%%%%%%%%%%%%%%%%%%%%%%%%%%%%%%%%%%%%%%%%%%%%%%%%%%%%%
 \item  {\color{blue}This is a little more substantial than the other problems.}
   For each integer $\ell\geq 1$, find (and prove) a formula for $f_{n+\ell}$ in terms of $f_n$ and $f_{n+1}$.
          Use this formula to give a proof by induction that for all $n,k\in\NN$ the
          Fibonacci number $f_{nk}$ is a multiple of $f_n$.
%%%%%%%%%%%%%%%%%%%%%%%%%%%%%%%%%%%%%%%%%%%%%%%%%%%%%%%%%%%%%%%%%%%%%%%%%%%%%%%%%

   
%%%%%%%%%%%%%%%%%%%%%%%%%%%%%%%%%%%%%%%%%%%%%%%%%%%%%%%%%%%%%%%%%%%%%%%%%%%%%%%%%
\item  Suppose that $a_0, a_1,a_2,\dotsc$ is a sequence of rational numbers such that
  $a_0=\frac{1}{2}$,
  $a_1=\frac{2}{3}$, and for every $n\geq 2$, we have
  $a_n=\frac{a_{n-2}}{a_{n-1}}$, then for every positive integer $n$,
  \[
  a_n\ =\ \left\{ \begin{array}{rcl}
    {\displaystyle\frac{3^{f_n}}{2^{f_{n+1}}}} &\ & \mbox{if $n$ is even\,,\ and}\vspace{10pt}\\
    {\displaystyle\frac{2^{f_{n+1}}}{3^{f_n}}} & & \mbox{if $n$ is odd\,.}
  \end{array}\right.
  \]
  Prove this statement by induction.

  Computing the first few instances gives:
  $\frac{1}{2},\frac{2}{3},\frac{3}{4},\frac{8}{9},\frac{27}{32},\frac{256}{243},\frac{6561}{8192}...$.

  After factoring we obtain:
  $\frac{3^0}{2^1},\frac{2^1}{3^1}, \frac{3^1}{2^2},\frac{2^3}{3^2}, \frac{3^3}{2^5},\frac{2^8}{3^5},
  \frac{3^8}{2^{13}},\frac{2^{21}}{3^{13}},\ldots$. 
%%%%%%%%%%%%%%%%%%%%%%%%%%%%%%%%%%%%%%%%%%%%%%%%%%%%%%%%%%%%%%%%%%%%%%%%%%%%%%%%%

  \newpage

%%%%%%%%%%%%%%%%%%%%%%%%%%%%%%%%%%%%%%%%%%%%%%%%%%%%%%%%%%%%%%%%%%%%%%%%%%%%%%%%%
\item Evaluate the proposed proof of the following statement.\newline
  {\bf Theorem.}  {\sl For every positive integer $n$, we have \ 
    $1+3+5+\dotsb+(2n-1)=n^2$.}\newline
  {\bf Proof.}  We proceed by induction.  Note that the formula holds for  $n=1$.
  Assume that  $1+3+5+\dotsb+(2k-1)=k^2$ for a positive integer $k$.
  We prove that $1+3+5+\dotsb+(2k+1)=(k+1)^2$.
  Observe that\vspace{-5pt}
  \begin{eqnarray*}
   1+3+5+\dotsb+(2k+1)&=&(k+1)^2\\
   1+3+5+\dotsb+(2k-1)+(2k+1)&=&(k+1)^2\\
   k^2+(2k+1)&=&(k+1)^2\\
   (k+1)^2&=&(k+1)^2\,.
   \begin{picture}(1,0)\put(110,0){$\Box$}\end{picture}
  \end{eqnarray*}
%%%%%%%%%%%%%%%%%%%%%%%%%%%%%%%%%%%%%%%%%%%%%%%%%%%%%%%%%%%%%%%%%%%%%%%%%%%%%%%%%


%%%%%%%%%%%%%%%%%%%%%%%%%%%%%%%%%%%%%%%%%%%%%%%%%%%%%%%%%%%%%%%%%%%%%%%%%%%%%%%%%
\item For which natural numbers $n$ do there exist nonnegative integers $x$ and $y$ such that $n=4x+7y$?
  Justify your conclusion.  (You might consider induction.)

  Here are the first few integers $n$ of this form
  \[
  0, 4, 7, 8, 11, 12, 14, 15, 16, 18, 19, 20, 21, 22, 23, \dotsc
  \]
%%%%%%%%%%%%%%%%%%%%%%%%%%%%%%%%%%%%%%%%%%%%%%%%%%%%%%%%%%%%%%%%%%%%%%%%%%%%%%%%%


%%%%%%%%%%%%%%%%%%%%%%%%%%%%%%%%%%%%%%%%%%%%%%%%%%%%%%%%%%%%%%%%%%%%%%%%%%%%%%%%%
\item  Suppose that $a$ and $b$ are integers such that $a+b$ is even.
  Prove that there exist integers $x$ and $y$ such that $x^2-y^2=ab$.
%%%%%%%%%%%%%%%%%%%%%%%%%%%%%%%%%%%%%%%%%%%%%%%%%%%%%%%%%%%%%%%%%%%%%%%%%%%%%%%%%

%%%%%%%%%%%%%%%%%%%%%%%%%%%%%%%%%%%%%%%%%%%%%%%%%%%%%%%%%%%%%%%%%%%%%%%%%%%%%%%%%
\item  Prove or disprove.
  \begin{enumerate}
    \item   Let $A$, $B$, $C$, and $D$ be sets with $A\subseteq C$ and $B\subseteq D$.
      If $A$ and $B$ are disjoint, then $C$ and $D$ are disjoint.

    \item Every even integer can be expressed as the sum of two odd integers.
  \end{enumerate}
%%%%%%%%%%%%%%%%%%%%%%%%%%%%%%%%%%%%%%%%%%%%%%%%%%%%%%%%%%%%%%%%%%%%%%%%%%%%%%%%%


%%%%%%%%%%%%%%%%%%%%%%%%%%%%%%%%%%%%%%%%%%%%%%%%%%%%%%%%%%%%%%%%%%%%%%%%%%%%%%%%%
\item  Prove or disprove.
  \begin{enumerate}
    \item  
  There is a real number solution of the equation $x^4+x^2+1=0$.
\item  
  There exist positive integers $a$ and $b$ such that $a^2-b^2=101$.
  \end{enumerate}
%%%%%%%%%%%%%%%%%%%%%%%%%%%%%%%%%%%%%%%%%%%%%%%%%%%%%%%%%%%%%%%%%%%%%%%%%%%%%%%%%


  
\end{enumerate}
%%%%%%%%%%%%%%%%%%%%%%%%%%%%%%%%%%%%%%%%%%%%%%%%%%%%%%%%%%%%%%%%%%%%%%%%%%%%%%%%%%%%%%%%%%%%%%%%%%%%

\end{document}


%%%%%%%%%%%%%%%%%%%%%%%%%%%%%%%%%%%%%%%%%%%%%%%%%%%%%%%%%%%%%%%%%%%%%%%%%%%%%%%%%
\item \ 
%%%%%%%%%%%%%%%%%%%%%%%%%%%%%%%%%%%%%%%%%%%%%%%%%%%%%%%%%%%%%%%%%%%%%%%%%%%%%%%%%
