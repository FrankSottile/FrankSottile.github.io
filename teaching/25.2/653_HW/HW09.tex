%HW09.tex
%
% Ninth Homework for Graduate Algebra
% Frank Sottile
%%%%%%%%%%%%%%%%%%%%%%%%%%%%%%%%%%%%%%%%%%%%%%%%%%%%%%%%%%%%%%%%%%%%%%%
\documentclass[12pt]{article}
\usepackage{multicol,amsfonts, amssymb,  mathtools,amsmath}
\usepackage{colordvi,graphicx}
\headheight=8pt
%
\topmargin=-75pt
\textheight=720pt   \textwidth=560pt
\oddsidemargin=-60pt \evensidemargin=-60pt

\pagestyle{empty}

%%%%%%%%%%%%%%%%%%%%%%%%%%%%%%%%%%%%%%%%%%%%
\newcommand{\CC}{{\mathbb C}}
\newcommand{\KK}{{\mathbb K}}
\newcommand{\NN}{{\mathbb N}}
\newcommand{\PP}{{\mathbb P}}
\newcommand{\QQ}{{\mathbb Q}}
\newcommand{\RR}{{\mathbb R}}
\newcommand{\TT}{{\mathbb T}}
\newcommand{\ZZ}{{\mathbb Z}}

\newcommand{\calA}{{\mathcal A}}
\newcommand{\calC}{{\mathcal C}}
\newcommand{\calG}{{\mathcal G}}
\newcommand{\calAb}{{\mathcal Ab}}
\newcommand{\be}{{\bf e}}
\newcommand{\bfi}{{\bf i}}
\newcommand{\bfj}{{\bf j}}

\newcommand{\Aut}{\mbox{Aut}}
\newcommand{\Hom}{\mbox{Hom}}
\newcommand{\Mor}{\mbox{Mor}}
\newcommand{\ab}{\mbox{ab}}
\newcommand{\spec}{\mbox{spec}}
\newcommand{\cone}{\mbox{cone}}

\newcommand{\vect}[2]{(\begin{smallmatrix}#1\\#2\end{smallmatrix})}
\newcommand{\msp}{\hspace{8pt}}
%\newcommand{\Square}{\raisebox{-2pt}{\includegraphics{images/Square.eps}}}
\newcommand{\Square}{\raisebox{-2pt}{\Large$\square$}}

\def\Color#1#2{\special{color push cmyk #1}#2\special{color pop}}
%\def\Indigo#1{\Color{.42 1. 0. .49}{#1}}
\def\Indigo#1{\Color{1. .95 .05 .4}{#1}}
\def\MyViolet#1{\Color{.6 1. 0. .15}{#1}}
\def\TAMU#1{\Color{.15 1. .39 .69}{#1}}

\newcommand{\barsl}{\noindent\begin{minipage}[t]{590pt}
\Indigo{\rule{590pt}{1.2pt}}\vspace{-5.7mm}\\
\MyViolet{\rule{590pt}{1.2pt}}\vspace{-5.7mm}\\
\Blue{\rule{590pt}{1.2pt}}\vspace{-5.7mm}\\
\Green{\rule{590pt}{1.2pt}}\vspace{-5.7mm}\\
\Yellow{\rule{590pt}{1.2pt}}\vspace{-5.7mm}\\
\Orange{\rule{590pt}{1.2pt}}\vspace{-5.7mm}\\
\Red{\rule{590pt}{1.2pt}}\bigskip
\end{minipage}}


\newcommand{\barsn}{\noindent\begin{minipage}[t]{590pt}
\Indigo{\rule{590pt}{1.1pt}}\vspace{-4.5mm}\\
\MyViolet{\rule{590pt}{1.1pt}}\vspace{-4.5mm}\\
\Blue{\rule{590pt}{1.1pt}}\vspace{-4.5mm}\\
\Green{\rule{590pt}{1.1pt}}\vspace{-4.5mm}\\
\Yellow{\rule{590pt}{1.1pt}}\vspace{-4.5mm}\\
\Orange{\rule{590pt}{1.1pt}}\vspace{-4.5mm}\\
\Red{\rule{590pt}{1.1pt}}\bigskip
\end{minipage}}

\def\demph#1{\TAMU{{\sl #1}}}
\def\defcolor#1{\TAMU{#1}}

\begin{document}
\LARGE 
\noindent
Algebra \ \ Autumn 2025\vspace{1pt}\\
Frank Sottile\vspace{2pt}\\
\Large 30 October 2025 \hfill
\sf
 Ninth Homework\makebox[20pt][l]{\ }
\normalsize\vspace{10pt}

\noindent
Write your answers neatly, in complete sentences.  
I highly recommend recopying your work before handing it in.
Correct and crisp proofs are greatly appreciated; oftentimes your work can be shortened and made clearer.

\barsn

\noindent\Maroon{{\sf Hand in at the start of class, Thursday 6 November:}}

\noindent\Cyan{While the statements are long, they are mostly exercises in the definitions, and should have short answers.}

\begin{enumerate}

%%%%%%%%%%%%%%%%%%%%%%%%%%%%%%%%%%%%%%%%%%%%%%%%%%%%%%%%%%%%%%%%%%%%%%%%%%%%%%%%%%%%%%%%%%%%%%%%%%%%
\item  Let $p\in\NN$ be a prime.
  For positive integers $n<m$ we have the natural surjection $\ZZ/p^n\ZZ \twoheadleftarrow \ZZ/p^m\ZZ$ induced by
  the inclusions $p^n\ZZ\supset p^m\ZZ$.  These form an inverse system.
  Define the \demph{$p$-adic integers} \defcolor{$\widehat{\ZZ}_p$} to be the  limit of this inverse system,
  \[
  \widehat{\ZZ}_p\ \vcentcolon=\  \lim_{\longleftarrow} \ZZ/p^n\ZZ\,.
  \]
  Using the universal property of inverse limits (or a direct construction), show that the natural map
  $\ZZ\to {\ZZ}_p$ induced by the surjections $\ZZ\twoheadrightarrow \ZZ/p^n\ZZ$ is in injection.

  Lang and Dr.\ Witherspoon define more general inverse limits over partially ordered sets.
  Let \defcolor{$\PP$} be the set of positive integers, which forms a poset under divisibility.
  Under $m\mapsto m\ZZ$, the set of free abelian subgroups of $\ZZ$  is a poset (opposite to $\PP$), and
  the collection of quotients $\ZZ/m\ZZ \twoheadleftarrow \ZZ/mn\ZZ$ forms an inverse  directed system indexed by $\PP$.
  Define
  \[
  \defcolor{\widehat{\ZZ}}\ \vcentcolon=\ \lim_{\longleftarrow} \ZZ/m\ZZ\ , \mbox{\quad the inverse limit along $\PP$.}
  \]
  Show that as before there is a natural injection  $\ZZ\hookrightarrow \widehat{\ZZ}$ and for every prime number $p$ there is similarly
  an injection  $\widehat{\ZZ}_p \hookrightarrow \widehat{\ZZ}$ compatible with the map of the first part.  
%%%%%%%%%%%%%%%%%%%%%%%%%%%%%%%%%%%%%%%%%%%%%%%%%%%%%%%%%%%%%%%%%%%%%%%%%%%%%%%%%%%%%%%%%%%%%%%%%%%%



%%%%%%%%%%%%%%%%%%%%%%%%%%%%%%%%%%%%%%%%%%%%%%%%%%%%%%%%%%%%%%%%%%%%%%%%%%%%%%%%%%%%%%%%%%%%%%%%%%%%
\item  \Magenta{Double question}  Let $A$, $B$, $C$, and $D$ be abelian groups with group homomorphisms
  \[
     f\colon A\to B\quad\mbox{ and }\quad
     g\colon A\to C\quad\mbox{ and }\quad
     h\colon B\to D\quad\mbox{ and }\quad
     k\colon C\to D\ .
   \]     
\begin{enumerate}
  \item Show that  the construction of the fiber product $B\times_D C$ from Lang is in fact a fiber product of abelian groups
    over $D$.  (E.g.\ show that it satisfies the universal property.)
  
  \item Show that the pullback of a surjective homomorphism is surjective.
  
  \item Show that the construction of the fiber coproduct (pushout) $B\otimes_A C$ from Lang is in fact a fiber coproduct of
    abelian groups. (E.g.\ show that it satisfies the universal property.)
  
  \item Show that the pushout of an  injective homomorphism is injective.

  \item Is there any relation between the fiber product and fiber coproduct (of abelian groups)?  
\end{enumerate}  
%%%%%%%%%%%%%%%%%%%%%%%%%%%%%%%%%%%%%%%%%%%%%%%%%%%%%%%%%%%%%%%%%%%%%%%%%%%%%%%%%%%%%%%%%%%%%%%%%%%%


%%%%%%%%%%%%%%%%%%%%%%%%%%%%%%%%%%%%%%%%%%%%%%%%%%%%%%%%%%%%%%%%%%%%%%%%%%%%%%%%%%%%%%%%%%%%%%%%%%%%
\item Let $\calC$ be a locally small category (for objects $A,B$ of $\calC$, $\Mor(A,B)$ is a set).
  \begin{enumerate}
    \item Show that for every object $A$ of $\calC$, $\Mor(A,A)$ is a monoid under composition.
    \item For every object $A$ of $\calC$, let $\defcolor{\Aut(A)}\subset \Mor(A,A)$ be its subset of automorphisms.
          Show that $\Aut(A)$ forms a group.
   \end{enumerate}
%%%%%%%%%%%%%%%%%%%%%%%%%%%%%%%%%%%%%%%%%%%%%%%%%%%%%%%%%%%%%%%%%%%%%%%%%%%%%%%%%%%%%%%%%%%%%%%%%%%%

  \Magenta{Continued on the other side}


%%%%%%%%%%%%%%%%%%%%%%%%%%%%%%%%%%%%%%%%%%%%%%%%%%%%%%%%%%%%%%%%%%%%%%%%%%%%%%%%%%%%%%%%%%%%%%%%%%%%
\item
  Let $G$ be a fixed group.
  Show that the following forms a category:
  The objects are all $G$-sets (sets $S$ equipped with an action of $G$).
  For two $G$ sets $S,T$, $\Mor(S,T)$ consists of all functions $\phi\colon S\to T$ such that
  \[
     \phi(g.s)\ =\ g.\phi(s) \qquad \mbox{for all }g\in G\mbox{ and } s\in S\,.
  \]
  Composition is composition of functions.   
%%%%%%%%%%%%%%%%%%%%%%%%%%%%%%%%%%%%%%%%%%%%%%%%%%%%%%%%%%%%%%%%%%%%%%%%%%%%%%%%%%%%%%%%%%%%%%%%%%%%


%%%%%%%%%%%%%%%%%%%%%%%%%%%%%%%%%%%%%%%%%%%%%%%%%%%%%%%%%%%%%%%%%%%%%%%%%%%%%%%%%%%%%%%%%%%%%%%%%%%%
\item  Let $\calG$ be the category of all groups and $\calAb$ the category of abelian groups.
  For each group $G$, define $\ab(G)\vcentcolon= G/[G,G]$, where $[G,G]$ is the commutator group of $G$.
  \begin{enumerate}
    \item
      For a group homomorphism $\phi\colon G\to H$, show that $\phi$ induces a homomorphism
      $\ab(\phi)$ from $\ab(G)$ to $\ab(H)$.
      
    \item
      Prove that $\ab$ is a functor from $\calG$ to $\calAb$.  (This is the \demph{abelianization functor}.)

    \item
      As every abelian group is a group, show that we have a forgetful function $\iota$ from  $\calAb$ to $\calG$.
      Show that $\ab$ is adjoint to $\iota$.    
  \end{enumerate}
%%%%%%%%%%%%%%%%%%%%%%%%%%%%%%%%%%%%%%%%%%%%%%%%%%%%%%%%%%%%%%%%%%%%%%%%%%%%%%%%%%%%%%%%%%%%%%%%%%%%

%%%%%%%%%%%%%%%%%%%%%%%%%%%%%%%%%%%%%%%%%%%%%%%%%%%%%%%%%%%%%%%%%%%%%%%%%%%%%%%%%%%%%%%%%%%%%%%%%%%%
\item   Show that every nonidentity element of a free group has infinite order.
%%%%%%%%%%%%%%%%%%%%%%%%%%%%%%%%%%%%%%%%%%%%%%%%%%%%%%%%%%%%%%%%%%%%%%%%%%%%%%%%%%%%%%%%%%%%%%%%%%%%

%%%%%%%%%%%%%%%%%%%%%%%%%%%%%%%%%%%%%%%%%%%%%%%%%%%%%%%%%%%%%%%%%%%%%%%%%%%%%%%%%%%%%%%%%%%%%%%%%%%%
\item  Let $F$ be a free group and let $n\in\PP$.
  Define $N$ to be the subgroup generated by all $n$-th powers of elements of
  $F$, $N=\langle x^n \mid x\in F\rangle$.
  Show that $N$ is a normal subgroup of $F$.
%%%%%%%%%%%%%%%%%%%%%%%%%%%%%%%%%%%%%%%%%%%%%%%%%%%%%%%%%%%%%%%%%%%%%%%%%%%%%%%%%%%%%%%%%%%%%%%%%%%%

%%%%%%%%%%%%%%%%%%%%%%%%%%%%%%%%%%%%%%%%%%%%%%%%%%%%%%%%%%%%%%%%%%%%%%%%%%%%%%%%%%%%%%%%%%%%%%%%%%%%
\item  Let $G$ and $H$ be groups.
  Show that their free product $G*H$ is a coproduct in the category of groups.
%%%%%%%%%%%%%%%%%%%%%%%%%%%%%%%%%%%%%%%%%%%%%%%%%%%%%%%%%%%%%%%%%%%%%%%%%%%%%%%%%%%%%%%%%%%%%%%%%%%%

      
\end{enumerate}
%%%%%%%%%%%%%%%%%%%%%%%%%%%%%%%%%%%%%%%%%%%%%%%%%%%%%%%%%%%%%%%%%%%%%%%%%%%%%%%%%%%%%%%%%%%%%%%%%%%%

\end{document}

%%%%%%%%%%%%%%%%%%%%%%%%%%%%%%%%%%%%%%%%%%%%%%%%%%%%%%%%%%%%%%%%%%%%%%%%%%%%%%%%%%%%%%%%%%%%%%%%%%%%
\item  
%%%%%%%%%%%%%%%%%%%%%%%%%%%%%%%%%%%%%%%%%%%%%%%%%%%%%%%%%%%%%%%%%%%%%%%%%%%%%%%%%%%%%%%%%%%%%%%%%%%%

