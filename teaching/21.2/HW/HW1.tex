%Homework1.tex
%First Homework -- Math 300H 
%
%  The percent sign is a comment character
%
%%%%%%%%%%%%%%%%%%%%%%%%%%%%%%%%%%%%%%%%%%%%%%%%%%%%%%%%%%%%%%%%%%%%%%%%%%%%%%%%%%
%
%   Look these up on line.  The first sets the type of document, and the next are for mathematics symbols, graphics and color
%
\documentclass[12pt]{article}
\usepackage{amssymb,amsmath}
\usepackage{graphicx}
\usepackage[usenames,dvipsnames,svgnames,table]{xcolor}
\usepackage{multirow}   % This is for more control over tables
%%%%%%%%%%%%%%%%%%%%%%%%%%%%%%%%  Layout     %%%%%%%%%%%%%%%%%%%%%%%%%%%%%%%%%%%%%%
\usepackage{vmargin}
\setpapersize{USletter}
\setmargrb{2cm}{1cm}{2cm}{1cm} % --- sets all four margins LTRB


\newcommand{\RR}{{\mathbb R}}  % This is the backboard symbol for the real numbers.  Note how it is used below
\newcommand{\NN}{{\mathbb N}}  % 
\newcommand{\ZZ}{{\mathbb Z}}  %

\newcommand{\calP}{{\mathcal P}}  %Caligraphic P for power set

\newcommand{\sep}{{\ :\ }}     % This is for the : in our notation for building sets.

%%%%%%%%%%%%%%%%%%%%%%%%%%%%%%%%%%%%%%%%%%%%%%%%%%%%%%%%%%%%%%%%%%%%%%%%%%%%%%%%%
\begin{document}
\LARGE 
\noindent
{\color{Maroon}Foundations of Mathematics \hfill Math 300H Section 970}\vspace{2pt}\\
\Large YOUR NAME\vspace{2pt}\\
\large
First Homework: \hfill 6 September 2021
\normalsize\vspace{10pt}


%%%%%%%%%%%%%%%%%%%%%%%%%%%%%%%%%%%%%%%%%%%%%%%%%%%%%%%%%%%%%%%%%%%%%%%%%%%%%%%%%
\begin{enumerate}  %  the \begin{..}  \end{..} stars and ends an environment.

%%%%%%%%%%%%%%%%%%%%%%%%%%%%%%%%%%%%%%%%%%%%%%%%%%%%%%%%%%%%%%%%%%%%%%%%%%%%%%%%%%%%%%%%%%%%%%%%%%%%
\item Write each of the following sets by listing its elements within braces.
  \begin{enumerate}  %Notice how this is a nested listing environment, and note how it is numbered.

  \item $A=\{n\in\mathbb{Z} \sep  -4\leq n\leq 4\}$

  \item $B=\{n\in{\mathbb Z} \sep  n^2\leq 7\}$

  \item $C=\{n\in{\mathbb N} \sep  n^3< 100\}$

  \item $D=\{x\in\RR \sep  x^3-x=0\}$
 
  \item $E=\{x\in\RR \sep  x^2+1=0\}$
     
  \end{enumerate}
%%%%%%%%%%%%%%%%%%%%%%%%%%%%%%%%%%%%%%%%%%%%%%%%%%%%%%%%%%%%%%%%%%%%%%%%%%%%%%%%


%%%%%%%%%%%%%%%%%%%%%%%%%%%%%%%%%%%%%%%%%%%%%%%%%%%%%%%%%%%%%%%%%%%%%%%%%%%%%%%%%%%%%%%%%%%%%%%%%%%%
\item Let $S=\{-10,-9,-8,\dotsc,8,9,10\}$.
  Describe each of the following sets as $\{x\in S \sep  p(x)\}$, where $p(x)$ is some condition on $x$.
  \begin{enumerate}  

  \item $A=\{-10,-9,\dotsc,-1,1,\ldots,9,10\}$  %Notice how \dotsc and \ldots are about the same.  There is a reason to prefer \dotsc.

  \item $B=\{-10,-9,\dotsc,-1,0\}$

  \item $C=\{-5,-4,\dotsc,0,1,\dotsc,7\}$

  \item $D=\{-10,-9,-8,\dotsc,4,6,7,\dotsc,10\}$
 
     
  \end{enumerate}
%%%%%%%%%%%%%%%%%%%%%%%%%%%%%%%%%%%%%%%%%%%%%%%%%%%%%%%%%%%%%%%%%%%%%%%%%%%%%%%%

%%%%%%%%%%%%%%%%%%%%%%%%%%%%%%%%%%%%%%%%%%%%%%%%%%%%%%%%%%%%%%%%%%%%%%%%%%%%%%%%
\item  Give examples of three sets $A$, $B$, and $C$ such that 

  \begin{enumerate} 
  \item $A\subseteq  B \subsetneq C$

  \item $A \in B$, $B\in C$, and $A\not\in C$

  \item $A\in B$ and $A\subsetneq C$
    
  \end{enumerate}
%%%%%%%%%%%%%%%%%%%%%%%%%%%%%%%%%%%%%%%%

  
%%%%%%%%%%%%%%%%%%%%%%%%%%%%%%%%%%%%%%%%%%%%%%%%%%%%%%%%%%%%%%%%%%%%%%%%%%%%%%%%
\item  Which of the following sets are equal?

  \makebox[150pt][l]{$A=\{ n\in\ZZ \sep  |n|<2\}$} \makebox[150pt][l]{$B=\{ n\in\ZZ \sep  n^2\leq 1\}$}

  \makebox[150pt][l]{$C=\{ n\in\ZZ \sep  n^3=n\}$} \makebox[150pt][l]{$D=\{-1, 0, 1\}$}

  \makebox[150pt][l]{$E=\{ n\in\ZZ \sep  n^2\leq n\}$} 
%%%%%%%%%%%%%%%%%%%%%%%%%%%%%%%%%%%%%%%%%%%%%%%%%%%%%%%%%%%%%%%%%%%%%%%%%%%%%%%%%

  
%%%%%%%%%%%%%%%%%%%%%%%%%%%%%%%%%%%%%%%%%%%%%%%%%%%%%%%%%%%%%%%%%%%%%%%%%%%%%%%%%
\item Let $A=\{\emptyset, \spadesuit, \Psi\}$.  Determine which of the following are true or false.
  
  \begin{itemize}
  \item[(a)] \makebox[100pt][l]{$\spadesuit\subseteq \calP(A)$}
        (e)  \makebox[100pt][l]{$\emptyset\subseteq \calP(A)$}
        (i)  \makebox[100pt][l]{$\{\emptyset,\{\spadesuit\}\}\subseteq \calP(A)$}
  \item[(b)] \makebox[100pt][l]{$\Psi\in \calP(A)$}
        (f)  \makebox[100pt][l]{$\emptyset\in \calP(A)$}
        (j)  \makebox[100pt][l]{$\{\emptyset,\{\spadesuit\}\}\in \calP(A)$}
  \item[(c)] \makebox[100pt][l]{$\{\Psi\}\subseteq \calP(A)$}
        (g)  \makebox[100pt][l]{$\{\emptyset\}\subseteq \calP(A)$}
        (k)  \makebox[100pt][l]{$A\subseteq \calP(A)$}
  \item[(d)] \makebox[100pt][l]{$\{\spadesuit\}\in \calP(A)$}
        (h)  \makebox[100pt][l]{$\{\emptyset\} \in \calP(A)$}
        (l)  \makebox[100pt][l]{$A\in \calP(A)$}
 \end{itemize}

%%%%%%%%%%%%%%%%%%%%%%%%%%%%%%%%%%%%%%%%%%%%%%%%%%%%%%%%%%%%%%%%%%%%%%%%%%%%%%%%%
\item  Let  $U$ be some universal set.
  Investigate the two sets $A-(B-C)$ and $(A-B)-C$. \newline
  Are they the same? different? Is one a subset of the other?

  Make a conjecture about their relation, and explain why it holds.
%%%%%%%%%%%%%%%%%%%%%%%%%%%%%%%%%%%%%%%%%%%%%%%%%%%%%%%%%%%%%%%%%%%%%%%%%%%%%%%%%

%%%%%%%%%%%%%%%%%%%%%%%%%%%%%%%%%%%%%%%%%%%%%%%%%%%%%%%%%%%%%%%%%%%%%%%%%%%%%%%%%%%%%%%%%%%%%%%%%%%%
\item Determine whether the following statements are true or false.

  \begin{enumerate}
  \item  If $\{1\}\in\calP(A)$, then $1\in A$ but $\{1\}\not\in A$.

  \item If $A$, $B$, and $C$ are sets such that $A\subsetneq\calP(B)\subsetneq C$ and $|A|=2$, then $|C|$ can be 5, but it cannot be $4$.

  \item If a set $B$ has one more element than a set $A$, then $\calP(B)$ has at least two more elements than $\calP(A)$.

  \item If four sets $A$, $B$, $C$, and $D$ are subsets of $\{1,2,3\}$ such that $|A|=|B|=|C|=|D|=2$, then at least two of these sets are
    equal. 

  \end{enumerate}
%%%%%%%%%%%%%%%%%%%%%%%%%%%%%%%%%%%%%%%%%%%%%%%%%%%%%%%%%%%%%%%%%%%%%%%%%%%%%%%%%%%%%%%%%%%%%%%%%%%%

%%%%%%%%%%%%%%%%%%%%%%%%%%%%%%%%%%%%%%%%%%%%%%%%%%%%%%%%%%%%%%%%%%%%%%%%%%%%%%%%%%%%%%%%%%%%%%%%%%%%
\item Determine the cardinality of each of the following sets.

  \begin{enumerate}
  \item  $A=\{1,2,3,\{1,2,3\},4,\{4\}\}$

  \item  $B=\{x\in\RR \sep  |x|=-\pi\}$

  \item $C=\{m\in\NN \sep  2<m\leq 5\}$

  \item $D=\{n\in\NN \sep  n<0\}$

  \item $E=\{k\in\NN \sep  1\leq k^2\leq 100\}$

  \item $F=\{k\in\ZZ \sep  1\leq k^2\leq 100\}$

  \end{enumerate}
%%%%%%%%%%%%%%%%%%%%%%%%%%%%%%%%%%%%%%%%%%%%%%%%%%%%%%%%%%%%%%%%%%%%%%%%%%%%%%%%%%%%%%%%%%%%%%%%%%%%

%%%%%%%%%%%%%%%%%%%%%%%%%%%%%%%%%%%%%%%%%%%%%%%%%%%%%%%%%%%%%%%%%%%%%%%%%%%%%%%%%%%%%%%%%%%%%%%%%%%%
\item Given examples of three sets $A$, $B$, and $C$ such that $B\neq C$ but $B-A=C-A$.
%%%%%%%%%%%%%%%%%%%%%%%%%%%%%%%%%%%%%%%%%%%%%%%%%%%%%%%%%%%%%%%%%%%%%%%%%%%%%%%%%%%%%%%%%%%%%%%%%%%%

%%%%%%%%%%%%%%%%%%%%%%%%%%%%%%%%%%%%%%%%%%%%%%%%%%%%%%%%%%%%%%%%%%%%%%%%%%%%%%%%%%%%%%%%%%%%%%%%%%%%
\item Given an example of two subsets $A$ and $B$ of $\{1,2,3\}$ such that all of the following sets are different:
  $A\cup B$, $A\cup\overline{B}$, $\overline{A}\cup B$,     $\overline{A}\cup\overline{B}$,
  $A\cap B$, $A\cap\overline{B}$, $\overline{A}\cap B$, and $\overline{A}\cap\overline{B}$.
%%%%%%%%%%%%%%%%%%%%%%%%%%%%%%%%%%%%%%%%%%%%%%%%%%%%%%%%%%%%%%%%%%%%%%%%%%%%%%%%%%%%%%%%%%%%%%%%%%%%

\end{enumerate}  
%%%%%%%%%%%%%%%%%%%%%%%%%%%%%%%%%%%%%%%%%%%%%%%%%%%%%%%%%%%%%%%%%%%%%%%%%%%%%%%%%


\end{document}
%%%%%%%%%%%%%%%%%%%%%%%%%%%%%%%%%%%%%%%%%%%%%%%%%%%%%%%%%%%%%%%%%%%%%%%%%%%%%%%%%
