%HWA.tex
%Tenth Homework -- Math 300 
%
%  The percent sign is a comment character
%
%%%%%%%%%%%%%%%%%%%%%%%%%%%%%%%%%%%%%%%%%%%%%%%%%%%%%%%%%%%%%%%%%%%%%%%%%%%%%%%%%%
%
%   Look these up on line.  The first sets the type of document, and the next are for mathematics symbols, graphics and color
%
\documentclass[12pt]{article}
\usepackage{amssymb,amsmath}
\usepackage{graphicx}
\usepackage[usenames,dvipsnames,svgnames,table]{xcolor}
\usepackage{multirow}   % This is for more control over tables
%%%%%%%%%%%%%%%%%%%%%%%%%%%%%%%%  Layout     %%%%%%%%%%%%%%%%%%%%%%%%%%%%%%%%%%%%%%
\usepackage{vmargin}
\setpapersize{USletter}
\setmargrb{1cm}{.1cm}{2cm}{.5cm} % --- sets all four margins LTRB


\newcommand{\defn}[1]{{\color{blue}\sl #1}}
\newcommand{\deco}[1]{{\color{blue} #1}}

\newcommand{\calP}{{\mathcal P}}

\newcommand{\NN}{{\mathbb N}}
\newcommand{\QQ}{{\mathbb Q}}
\newcommand{\RR}{{\mathbb R}}
\newcommand{\ZZ}{{\mathbb Z}}

%%%%%%%%%%%%%%%%%%%%%%%%%%%%%%%%%%%%%%%%%%%%%%%%%%%%%%%%%%%%%%%%%%%%%%%%%%%%%%%%%
\begin{document}
\LARGE 
\noindent
{\color{Maroon}Foundations of Mathematics \hfill Math 300 Sections 902, 905}\vspace{2pt}\\
\Large YOUR NAME\vspace{2pt}\\
\large
Tenth Homework: \hfill Due 19 November 2020
\normalsize    %%%% This returns the fontsize to 12 point
%\vspace{10pt}  %%%% This makes a vertical space of 10 points (approx 70 points/inch) after the line break

\noindent{\color{blue}\rule{529pt}{2pt}}

  
%%%%%%%%%%%%%%%%%%%%%%%%%%%%%%%%%%%%%%%%%%%%%%%%%%%%%%%%%%%%%%%%%%%%%%%%%%%%%%%%%
\begin{enumerate}


%\newpage
%%%%%%%%%%%%%%%%%%%%%%%%%%%%%%%%%%%%%%%%%%%%%%%%%%%%%%%%%%%%%%%%%%%%%%%%%%%%%%%%%
\item{[14]} Do all parts of Problem 16  in the Exercises for Section 7.2 in the Sundstrom book. 
%%%%%%%%%%%%%%%%%%%%%%%%%%%%%%%%%%%%%%%%%%%%%%%%%%%%%%%%%%%%%%%%%%%%%%%%%%%%%%%%%



%\newpage
%%%%%%%%%%%%%%%%%%%%%%%%%%%%%%%%%%%%%%%%%%%%%%%%%%%%%%%%%%%%%%%%%%%%%%%%%%%%%%%%%
\item{[10]}  Let $A=\{0,1\}$.
  Determine all the relations $R$ on $A$.

  Which of these are equivalence relations?
%%%%%%%%%%%%%%%%%%%%%%%%%%%%%%%%%%%%%%%%%%%%%%%%%%%%%%%%%%%%%%%%%%%%%%%%%%%%%%%%%

%\newpage
%%%%%%%%%%%%%%%%%%%%%%%%%%%%%%%%%%%%%%%%%%%%%%%%%%%%%%%%%%%%%%%%%%%%%%%%%%%%%%%%%
\item{[8]}  Let $A$ be a set.
  As a function is also a relation, what can be said if a function $f\colon A\to A$ is
  also an equivalence relation?
  (E.g.\ is this possible, and if so, what does this say about $f$?)
%%%%%%%%%%%%%%%%%%%%%%%%%%%%%%%%%%%%%%%%%%%%%%%%%%%%%%%%%%%%%%%%%%%%%%%%%%%%%%%%%


%\newpage
%%%%%%%%%%%%%%%%%%%%%%%%%%%%%%%%%%%%%%%%%%%%%%%%%%%%%%%%%%%%%%%%%%%%%%%%%%%%%%%%%
\item{[10]}  Consider the relation $\sim$ on the power set $\mathcal{P}(U)$ of some set $U$, where
  for $A,B\subseteq U$, we have $A\sim B$ if and only if there is a bijection $f\colon A\to B$.
  Prove that this is a equivalence relation.
  Determine the equivalence class $[\emptyset]$.
%%%%%%%%%%%%%%%%%%%%%%%%%%%%%%%%%%%%%%%%%%%%%%%%%%%%%%%%%%%%%%%%%%%%%%%%%%%%%%%%%

%\newpage
%%%%%%%%%%%%%%%%%%%%%%%%%%%%%%%%%%%%%%%%%%%%%%%%%%%%%%%%%%%%%%%%%%%%%%%%%%%%%%%%%
  \item{[8]}  Consider the relation $\sim$ on the power set $\mathcal{P}(U)$ of some set $U$, where
    for $A,B\subseteq U$, we have $A\sim B$ if and only if  $A\cap B=\emptyset$.
    Is this an equivalence relation?
    If not, is it reflexive, symmetric, or transitive?
    Justify your conclusions.
%%%%%%%%%%%%%%%%%%%%%%%%%%%%%%%%%%%%%%%%%%%%%%%%%%%%%%%%%%%%%%%%%%%%%%%%%%%%%%%%%


%\newpage
%%%%%%%%%%%%%%%%%%%%%%%%%%%%%%%%%%%%%%%%%%%%%%%%%%%%%%%%%%%%%%%%%%%%%%%%%%%%%%%%%
  \item{[12]}  Consider the relation $q$ on $\ZZ\times \ZZ$ where, for integers $a,b,c,d$, we have
    $(a,b)\sim (c,d)$ if $ad=bc$.
    Show that this is not an equivalence relation.
    What if we restrict to $\ZZ\times\NN$?
%%%%%%%%%%%%%%%%%%%%%%%%%%%%%%%%%%%%%%%%%%%%%%%%%%%%%%%%%%%%%%%%%%%%%%%%%%%%%%%%%

%\newpage
%%%%%%%%%%%%%%%%%%%%%%%%%%%%%%%%%%%%%%%%%%%%%%%%%%%%%%%%%%%%%%%%%%%%%%%%%%%%%%%%%
  \item{[8]}  Determine all the congruence classes (equivalence classes) for the relation on the integers $\ZZ$ of congruence modulo 5.
%%%%%%%%%%%%%%%%%%%%%%%%%%%%%%%%%%%%%%%%%%%%%%%%%%%%%%%%%%%%%%%%%%%%%%%%%%%%%%%%%

%\newpage
%%%%%%%%%%%%%%%%%%%%%%%%%%%%%%%%%%%%%%%%%%%%%%%%%%%%%%%%%%%%%%%%%%%%%%%%%%%%%%%%%
  \item{[8]}  The relation $\sim$ on $\ZZ$ defined by $a\sim b$ if $3a+4b\equiv 0 \mod 7$ is an equivalence relation (you can check this, but
    it is not necessary).
    Determine all distinct equivalence classes for this equivalence relation.
%%%%%%%%%%%%%%%%%%%%%%%%%%%%%%%%%%%%%%%%%%%%%%%%%%%%%%%%%%%%%%%%%%%%%%%%%%%%%%%%%

%\newpage
%%%%%%%%%%%%%%%%%%%%%%%%%%%%%%%%%%%%%%%%%%%%%%%%%%%%%%%%%%%%%%%%%%%%%%%%%%%%%%%%%
\item{[8]} Compute the addition and multiplication tables for $\ZZ_5$.
%%%%%%%%%%%%%%%%%%%%%%%%%%%%%%%%%%%%%%%%%%%%%%%%%%%%%%%%%%%%%%%%%%%%%%%%%%%%%%%%%



%\newpage
%%%%%%%%%%%%%%%%%%%%%%%%%%%%%%%%%%%%%%%%%%%%%%%%%%%%%%%%%%%%%%%%%%%%%%%%%%%%%%%%%
\item{[14]} For $n\in \NN$, let $s(n)$ denote the sum of the digits of $n$, expressed in base 10.
  That is, if we write $n=a_k\dotsc a_1 a_0$ in base 10 so that
  \[
     n\ =\ \left(a_k\cdot 10^k\right) +   \left(a_{k-1}\cdot 10^{k-1}\right) +\dotsb + 
     \left(a_1\cdot 10\right) + a_0\,.
     \]

     (a) Use mathematical induction to prove that for all $n\in\NN$, $10^n\equiv 1\mod 9$.
     Thus $[10^n]_9=[1]_9$.

     (b) Use this to prove that $[n]_9 = [s(n)]_9$ and deduce that $9|n$ if and only if $9|s(n)$.

     (c) Show that for $a,b\in\ZZ$, we have $[a\cdot b]_9=[s(a)\cdot s(b)]_9$.
         This is the idea behind \defn{casting out nines}.
%%%%%%%%%%%%%%%%%%%%%%%%%%%%%%%%%%%%%%%%%%%%%%%%%%%%%%%%%%%%%%%%%%%%%%%%%%%%%%%%%



%%%%%%%%%%%%%%%%%%%%%%%%%%%%%%%%%%%%%%%%%%%%%%%%%%%%%%%%%%%%%%%%%%%%%%%%%%%%%%%%%
\end{enumerate}
%%%%%%%%%%%%%%%%%%%%%%%%%%%%%%%%%%%%%%%%%%%%%%%%%%%%%%%%%%%%%%%%%%%%%%%%%%%%%%%%%


\end{document}
%%%%%%%%%%%%%%%%%%%%%%%%%%%%%%%%%%%%%%%%%%%%%%%%%%%%%%%%%%%%%%%%%%%%%%%%%%%%%%%%%


