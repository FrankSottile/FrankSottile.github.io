%HW9.tex
%Ninnth Homework -- Math 300H 
%
%  The percent sign is a comment character
%
%%%%%%%%%%%%%%%%%%%%%%%%%%%%%%%%%%%%%%%%%%%%%%%%%%%%%%%%%%%%%%%%%%%%%%%%%%%%%%%%%%
%
%   Look these up on line.  The first sets the type of document, and the next are for mathematics symbols, graphics and color
%
\documentclass[12pt]{article}
\usepackage{amssymb,amsmath,mathtools}
\usepackage{graphicx}
\usepackage[usenames,dvipsnames,svgnames,table]{xcolor}
\usepackage{multirow}   % This is for more control over tables
%%%%%%%%%%%%%%%%%%%%%%%%%%%%%%%%  Layout     %%%%%%%%%%%%%%%%%%%%%%%%%%%%%%%%%%%%%%
\usepackage{vmargin}
\setpapersize{USletter}
\setmargrb{1cm}{0.5cm}{2cm}{1cm} % --- sets all four margins LTRB


\newcommand{\RR}{{\mathbb R}}  % This is the backboard symbol for the real numbers.  Note how it is used below
\newcommand{\NN}{{\mathbb N}}  % 
\newcommand{\QQ}{{\mathbb Q}}  % 
\newcommand{\ZZ}{{\mathbb Z}}  %

\newcommand{\calP}{{\mathcal P}}  %Caligraphic P for power set

\newcommand{\sep}{{\ :\ }}     % This is for the : in our notation for building sets.
\newcommand{\lsim}{\mathord{\sim}}  %  This is to remove the extra spacing around \sim (it is a binary relation) for the logical not

\newcommand{\defcolor}[1]{{\color{blue}{#1}}}
\newcommand{\demph}[1]{{\color{blue}\sl{#1}}}
%%%%%%%%%%%%%%%%%%%%%%%%%%%%%%%%%%%%%%%%%%%%%%%%%%%%%%%%%%%%%%%%%%%%%%%%%%%%%%%%%
%
%   Please edit this as appropriate
%
\begin{document}
\LARGE 
\noindent
{\color{Maroon}Foundations of Mathematics \hfill Math 300H Section 970}\vspace{2pt}\\
\Large \vspace{2pt}\\
\large
Ninth Homework: \hfill Due \  8 November 2022
\normalsize\medskip


\noindent{\color{blue}\rule{528.3675pt}{2pt}}


\noindent {\color{Maroon}\bf Definition:}
The \demph{Fibonacci sequence} $\{f_n\mid n\geq 1\}$ is defined by $f_1=f_2=1$ and for $n\geq 2$, $f_{n+1}=f_{n}+f_{n-1}$.

\noindent {\color{Maroon}\bf Read:}   Chapters 7 and 8 in Fourth edition.
(Chapter 8 in fourth edition is Chapter 7 in third.  The new Chapter 7 in Fourth edition is a review of proof techniques.)


\noindent{\color{blue}\rule{528.3675pt}{2pt}}


%%%%%%%%%%%%%%%%%%%%%%%%%%%%%%%%%%%%%%%%%%%%%%%%%%%%%%%%%%%%%%%%%%%%%%%%%%%%%%%%%
\begin{enumerate}  %  the \begin{..}  \end{..} stars and ends an environment.



%%%%%%%%%%%%%%%%%%%%%%%%%%%%%%%%%%%%%%%%%%%%%%%%%%%%%%%%%%%%%%%%%%%%%%%%%%%%%%%%%
\item Look up the term \demph{Pythagrean triple} (it is in our book).
    Investigate the following

  {\bf Conjecture.}  {\sl For each natural number $n$, the numbers $f_nf_{n+3}$,
  $2f_{n+1}f_{n+2}$, and $(f_{n+1}^2+f_{n+2}^2)$ form a {\color{Blue}Pythagorean triple}.}

   If true, provide a proof, and if false, a counterexample.
%%%%%%%%%%%%%%%%%%%%%%%%%%%%%%%%%%%%%%%%%%%%%%%%%%%%%%%%%%%%%%%%%%%%%%%%%%%%%%%%%


%%%%%%%%%%%%%%%%%%%%%%%%%%%%%%%%%%%%%%%%%%%%%%%%%%%%%%%%%%%%%%%%%%%%%%%%%%%%%%%%%
 \item 
   Find (and prove) a formula for $f_{n+5}$ in terms of $f_n$ and $f_{n+1}$.
          Use this formula to give a proof by induction that for all $n\in\NN$ the
          Fibonacci number $f_{5n}$ is a multiple of $f_n$.
%%%%%%%%%%%%%%%%%%%%%%%%%%%%%%%%%%%%%%%%%%%%%%%%%%%%%%%%%%%%%%%%%%%%%%%%%%%%%%%%%


%%%%%%%%%%%%%%%%%%%%%%%%%%%%%%%%%%%%%%%%%%%%%%%%%%%%%%%%%%%%%%%%%%%%%%%%%%%%%%%%%
 \item   Explore the relation between $f_k$ and $f_{kn}$ for small values of $k$, and make a conjecture.
%%%%%%%%%%%%%%%%%%%%%%%%%%%%%%%%%%%%%%%%%%%%%%%%%%%%%%%%%%%%%%%%%%%%%%%%%%%%%%%%%


%%%%%%%%%%%%%%%%%%%%%%%%%%%%%%%%%%%%%%%%%%%%%%%%%%%%%%%%%%%%%%%%%%%%%%%%%%%%%%%%%
 \item   Find natural numbers $a$ and $b$ such that $a^2+b^2=10$, and then natural numbers $c$ and $d$ such that
   $c^2+d^2=10^2$.

   Prove the following statement by mathematical induction:
   For every natural number $n$, there are natural numbers $x$ and $y$ such that $x^2+y^2=10^n$.
%%%%%%%%%%%%%%%%%%%%%%%%%%%%%%%%%%%%%%%%%%%%%%%%%%%%%%%%%%%%%%%%%%%%%%%%%%%%%%%%%
   


%%%%%%%%%%%%%%%%%%%%%%%%%%%%%%%%%%%%%%%%%%%%%%%%%%%%%%%%%%%%%%%%%%%%%%%%%%%%%%%%%
\item  Suppose that $a$ and $b$ are integers such that $a+b$ is even.
  Prove that there exist integers $x$ and $y$ such that $x^2-y^2=ab$.
%%%%%%%%%%%%%%%%%%%%%%%%%%%%%%%%%%%%%%%%%%%%%%%%%%%%%%%%%%%%%%%%%%%%%%%%%%%%%%%%%

%%%%%%%%%%%%%%%%%%%%%%%%%%%%%%%%%%%%%%%%%%%%%%%%%%%%%%%%%%%%%%%%%%%%%%%%%%%%%%%%%
\item  Prove or disprove.
  \begin{enumerate}
    \item   Let $A$, $B$, $C$, and $D$ be sets with $A\subseteq C$ and $B\subseteq D$.
      If $A$ and $B$ are disjoint, then $C$ and $D$ are disjoint.

    \item Every even integer can be expressed as the sum of two odd integers.
  \end{enumerate}
%%%%%%%%%%%%%%%%%%%%%%%%%%%%%%%%%%%%%%%%%%%%%%%%%%%%%%%%%%%%%%%%%%%%%%%%%%%%%%%%%


%%%%%%%%%%%%%%%%%%%%%%%%%%%%%%%%%%%%%%%%%%%%%%%%%%%%%%%%%%%%%%%%%%%%%%%%%%%%%%%%%
\item  Prove or disprove.
  \begin{enumerate}
    \item  
  There is a real number solution of the equation $x^4+x^2+1=0$.
\item  
  There exist positive integers $a$ and $b$ such that $a^2-b^2=101$.
  \end{enumerate}
%%%%%%%%%%%%%%%%%%%%%%%%%%%%%%%%%%%%%%%%%%%%%%%%%%%%%%%%%%%%%%%%%%%%%%%%%%%%%%%%%
   


%%%%%%%%%%%%%%%%%%%%%%%%%%%%%%%%%%%%%%%%%%%%%%%%%%%%%%%%%%%%%%%%%%%%%%%%%%%%%%%%%
\item  Evaluate the proof of the following statement.

  {\bf Statement.} Let $x,y,z\in\ZZ$ be such that $3x+5y=7z$.
  If at least one of $x$, $y$, or $z$ is odd, then at least one of  $x$, $y$, or $z$ is even.

  {\bf Proof.}  Let $x,y,z\in\ZZ$ be such that $3x+5y=7z$.
  Assume, to the contrary, that none of  $x$, $y$, or $z$ is odd andthat none of  $x$, $y$, or $z$ is even.
  This is impossible.  \hfill$\Box$
  
%%%%%%%%%%%%%%%%%%%%%%%%%%%%%%%%%%%%%%%%%%%%%%%%%%%%%%%%%%%%%%%%%%%%%%%%%%%%%%%%%
   


  
\end{enumerate}
%%%%%%%%%%%%%%%%%%%%%%%%%%%%%%%%%%%%%%%%%%%%%%%%%%%%%%%%%%%%%%%%%%%%%%%%%%%%%%%%%%%%%%%%%%%%%%%%%%%%

\end{document}


%%%%%%%%%%%%%%%%%%%%%%%%%%%%%%%%%%%%%%%%%%%%%%%%%%%%%%%%%%%%%%%%%%%%%%%%%%%%%%%%%
\item \ 
%%%%%%%%%%%%%%%%%%%%%%%%%%%%%%%%%%%%%%%%%%%%%%%%%%%%%%%%%%%%%%%%%%%%%%%%%%%%%%%%%
