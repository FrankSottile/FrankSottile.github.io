%HW4.tex
%Fourth Homework -- Math 300 
%
%  The percent sign is a comment character
%
%%%%%%%%%%%%%%%%%%%%%%%%%%%%%%%%%%%%%%%%%%%%%%%%%%%%%%%%%%%%%%%%%%%%%%%%%%%%%%%%%%
%
%   Look these up on line.  The first sets the type of document, and the next are for mathematics symbols, graphics and color
%
\documentclass[12pt]{article}
\usepackage{amssymb,amsmath}
\usepackage{graphicx}
\usepackage[usenames,dvipsnames,svgnames,table]{xcolor}
\usepackage{multirow}   % This is for more control over tables
%%%%%%%%%%%%%%%%%%%%%%%%%%%%%%%%  Layout     %%%%%%%%%%%%%%%%%%%%%%%%%%%%%%%%%%%%%%
\usepackage{vmargin}
\setpapersize{USletter}
\setmargrb{2cm}{1cm}{2cm}{1cm} % --- sets all four margins LTRB


\newcommand{\defn}[1]{{\color{blue}\sl #1}}
\newcommand{\deco}[1]{{\color{blue} #1}}

\newcommand{\NN}{{\mathbb N}}
\newcommand{\QQ}{{\mathbb Q}}
\newcommand{\RR}{{\mathbb R}}
\newcommand{\ZZ}{{\mathbb Z}}

%%%%%%%%%%%%%%%%%%%%%%%%%%%%%%%%%%%%%%%%%%%%%%%%%%%%%%%%%%%%%%%%%%%%%%%%%%%%%%%%%
\begin{document}
\LARGE 
\noindent
{\color{Maroon}Foundations of Mathematics \hfill Math 300 Sections 902, 905}\vspace{2pt}\\
\Large YOUR NAME\vspace{2pt}\\
\large
Third Homework: \hfill Due 28 September 2020
\normalsize    %%%% This returns the fontsize to 12 point
\vspace{10pt}  %%%% This makes a vertical space of 10 points (approx 70 points/inch) after the line break

\noindent{\color{blue}\rule{500pt}{2pt}}

\noindent {\color{Maroon}\bf Definition:} An integer $a$ is \defn{even} if there is an integer $k$ such that $n=2k$.
An integer $a$ is \defn{odd} if there is an integer $k$ such that $n=2k{+}1$.

\noindent {\color{Maroon}\bf Definition:} Let $a$ and $b$ be integers.
We say that $a$ \defn{divides} $b$ and write $a|b$ if there is an integer $c$ such that $ac=b$.

\noindent {\color{Maroon}\bf Definition:}
A real number $x$ is a \defn{rational number} when there exists integers $n,d$ with $d\neq 0$ such that  $x=n/d$.
  A real number that is not rational is an \defn{irrational number}.


\noindent {\color{Maroon}\bf Definition:}
Suppose that $a$ is a nonnegative real number.
The \defn{square root of $a$} is the unique nonnegative real number $r$ such that $r^2=a$.
We write \deco{$\sqrt{a}$} for the square root of $a$.

\noindent {\color{Maroon}\bf Definition:}
Suppose that $a,b$ are positive real numbers.
The \defn{logarithm of $a$ in base $b$}, written \defn{$\log_b(a)$}, is the unique real number $r$ such that $b^r=a$.

  
  
%%%%%%%%%%%%%%%%%%%%%%%%%%%%%%%%%%%%%%%%%%%%%%%%%%%%%%%%%%%%%%%%%%%%%%%%%%%%%%%%%
\begin{enumerate}

 
%%%%%%%%%%%%%%%%%%%%%%%%%%%%%%%%%%%%%%%%%%%%%%%%%%%%%%%%%%%%%%%%%%%%%%%%%%%%%%%%%
 \item Consider the following statement:\newline
  ``Let $n\in\ZZ$.   If $5{\not|}(n^2+4)$, then $5{\not|}(n-1)$ and $5{\not|}(n+1)$.''
   \begin{enumerate}
   \item  Write its contrapositive
   \item  Construct a ``know-show'' table for a proof of this statement, in the form of a direct proof of the
     contrapositive.
       (You may find it useful to recycle code from previous homeworks)
   \item  Write your proof in paragraph form.  
  \end{enumerate}
%%%%%%%%%%%%%%%%%%%%%%%%%%%%%%%%%%%%%%%%%%%%%%%%%%%%%%%%%%%%%%%%%%%%%%%%%%%%%%%%%


%\newpage
%%%%%%%%%%%%%%%%%%%%%%%%%%%%%%%%%%%%%%%%%%%%%%%%%%%%%%%%%%%%%%%%%%%%%%%%%%%%%%%%%
\item Write a proof in paragraph form of the following statement:
 ``If $n^2$ is even, then $n$ is even.''
%%%%%%%%%%%%%%%%%%%%%%%%%%%%%%%%%%%%%%%%%%%%%%%%%%%%%%%%%%%%%%%%%%%%%%%%%%%%%%%%%


%\newpage
%%%%%%%%%%%%%%%%%%%%%%%%%%%%%%%%%%%%%%%%%%%%%%%%%%%%%%%%%%%%%%%%%%%%%%%%%%%%%%%%%
\item Write a proof in paragraph form of the following statement:
 ``If $nm$ is even, then $m$ is even or $n$ is even.''
%%%%%%%%%%%%%%%%%%%%%%%%%%%%%%%%%%%%%%%%%%%%%%%%%%%%%%%%%%%%%%%%%%%%%%%%%%%%%%%%%


  
%\newpage
%%%%%%%%%%%%%%%%%%%%%%%%%%%%%%%%%%%%%%%%%%%%%%%%%%%%%%%%%%%%%%%%%%%%%%%%%%%%%%%%%
\item 
  Is the following statement true or false? (If true, give a proof, if false, give a counterexample.)\newline
  ``For each positive real number $x$, if $x$ is irrational, then $\sqrt{x}$ is irrational.''
%%%%%%%%%%%%%%%%%%%%%%%%%%%%%%%%%%%%%%%%%%%%%%%%%%%%%%%%%%%%%%%%%%%%%%%%%%%%%%%%%

%\newpage
%%%%%%%%%%%%%%%%%%%%%%%%%%%%%%%%%%%%%%%%%%%%%%%%%%%%%%%%%%%%%%%%%%%%%%%%%%%%%%%%%
 \item Consider the definitions given on page 55 in the Sundstrom text on set equality and subsets.
   \begin{enumerate}
   \item  Write each definition more mathematically in terms of elements of the sets, quantifiers and implications.
   \item  Write a proof in paragraph form of the statement: Two sets $A$ and $B$ are equal if and only if $A\subset B$ and
     $B\subset A$.
     {\color{Magenta}Both $\subset$ and $\subseteq$ denote 'subset'.}
  \end{enumerate}
%%%%%%%%%%%%%%%%%%%%%%%%%%%%%%%%%%%%%%%%%%%%%%%%%%%%%%%%%%%%%%%%%%%%%%%%%%%%%%%%%

   
%\newpage
%%%%%%%%%%%%%%%%%%%%%%%%%%%%%%%%%%%%%%%%%%%%%%%%%%%%%%%%%%%%%%%%%%%%%%%%%%%%%%%%  
\item  Write a proof in paragraph form of the following statement.\newline
 ``For all real numbers $x$ and $y$, $x^2=y^2$ if and only if $x=y$ or $x=-y$.''\newline
{\color{Magenta}(You may use that $\forall a,b\in\RR$, \ $ab=0\ \to\ a=0$ or $b=0$, but do not use anything about square roots,
  which could be a recipe for a misstep.)}
  

%\newpage
%%%%%%%%%%%%%%%%%%%%%%%%%%%%%%%%%%%%%%%%%%%%%%%%%%%%%%%%%%%%%%%%%%%%%%%%%%%%%%%%%
\item Suppose that $a,b,c$ are real numbers and that $ax^2+bx+c=0$ has two different solutions.
      Prove that the sum of the two solutions equals $-b/a$.
%%%%%%%%%%%%%%%%%%%%%%%%%%%%%%%%%%%%%%%%%%%%%%%%%%%%%%%%%%%%%%%%%%%%%%%%%%%%%%%%%

%\newpage
%%%%%%%%%%%%%%%%%%%%%%%%%%%%%%%%%%%%%%%%%%%%%%%%%%%%%%%%%%%%%%%%%%%%%%%%%%%%%%%%%
\item Using the definitions, prove by cases that for every integer $n$, $n^2-n+41$ is odd.
%%%%%%%%%%%%%%%%%%%%%%%%%%%%%%%%%%%%%%%%%%%%%%%%%%%%%%%%%%%%%%%%%%%%%%%%%%%%%%%%%

%\newpage
%%%%%%%%%%%%%%%%%%%%%%%%%%%%%%%%%%%%%%%%%%%%%%%%%%%%%%%%%%%%%%%%%%%%%%%%%%%%%%%%%
\item Prove that if $m$ is odd, then $m^2\equiv 1\mod 8$.
%%%%%%%%%%%%%%%%%%%%%%%%%%%%%%%%%%%%%%%%%%%%%%%%%%%%%%%%%%%%%%%%%%%%%%%%%%%%%%%%%

%\newpage
%%%%%%%%%%%%%%%%%%%%%%%%%%%%%%%%%%%%%%%%%%%%%%%%%%%%%%%%%%%%%%%%%%%%%%%%%%%%%%%%%
\item For all integers $a,b,c$ with $a\neq 0$, if $a{\not|}(bc)$ then $a{\not|}b$ and $a{\not|}c$.
%%%%%%%%%%%%%%%%%%%%%%%%%%%%%%%%%%%%%%%%%%%%%%%%%%%%%%%%%%%%%%%%%%%%%%%%%%%%%%%%%


%\newpage
%%%%%%%%%%%%%%%%%%%%%%%%%%%%%%%%%%%%%%%%%%%%%%%%%%%%%%%%%%%%%%%%%%%%%%%%%%%%%%%%%
\item
  Write a proof in paragraph form of the following statement:
  ``For all positive real numbers $x,y$  we have $\sqrt{xy}\leq \frac{x+y}{2}$, and we have equality if and only if $x=y$.''
%%%%%%%%%%%%%%%%%%%%%%%%%%%%%%%%%%%%%%%%%%%%%%%%%%%%%%%%%%%%%%%%%%%%%%%%%%%%%%%%%
 

%\newpage
%%%%%%%%%%%%%%%%%%%%%%%%%%%%%%%%%%%%%%%%%%%%%%%%%%%%%%%%%%%%%%%%%%%%%%%%%%%%%%%%%
\item
Prove by {\it reductio ad absurdum} that an integer cannot be both even and odd.
%%%%%%%%%%%%%%%%%%%%%%%%%%%%%%%%%%%%%%%%%%%%%%%%%%%%%%%%%%%%%%%%%%%%%%%%%%%%%%%%%

%\newpage
%%%%%%%%%%%%%%%%%%%%%%%%%%%%%%%%%%%%%%%%%%%%%%%%%%%%%%%%%%%%%%%%%%%%%%%%%%%%%%%%%
\item Prove the following by contradiction ({\it reductio ad absurdum}):\newline
     For all integers $n$, if $n^2$ is odd, then $n$ is odd. 
%%%%%%%%%%%%%%%%%%%%%%%%%%%%%%%%%%%%%%%%%%%%%%%%%%%%%%%%%%%%%%%%%%%%%%%%%%%%%%%%%

%\newpage
%%%%%%%%%%%%%%%%%%%%%%%%%%%%%%%%%%%%%%%%%%%%%%%%%%%%%%%%%%%%%%%%%%%%%%%%%%%%%%%%%
 \item Prove that $\log_2 5$ is an irrational number.\newline
     Can you find a (true) generalization of this statement, replacing 2 and/or 5 by other, nearly arbitrary positive integers?
%%%%%%%%%%%%%%%%%%%%%%%%%%%%%%%%%%%%%%%%%%%%%%%%%%%%%%%%%%%%%%%%%%%%%%%%%%%%%%%%%


%%%%%%%%%%%%%%%%%%%%%%%%%%%%%%%%%%%%%%%%%%%%%%%%%%%%%%%%%%%%%%%%%%%%%%%%%%%%%%%%%
\item Prove the following by contradiction ({\it reductio ad absurdum}):\newline
   For all real numbers $a$ and $b$ with $b\geq 0$, if $a^2\geq b$, then either $a\geq \sqrt{b}$ or $a\leq -\sqrt{b}$.
%%%%%%%%%%%%%%%%%%%%%%%%%%%%%%%%%%%%%%%%%%%%%%%%%%%%%%%%%%%%%%%%%%%%%%%%%%%%%%%%%


%\newpage
%%%%%%%%%%%%%%%%%%%%%%%%%%%%%%%%%%%%%%%%%%%%%%%%%%%%%%%%%%%%%%%%%%%%%%%%%%%%%%%%%
 \item Is the following proposition true or false?  (Justify your conclusion with a proof or counterexample).\newline
   ``For all nonnegative real numbers $x$ and $y$, $\sqrt{x+y}\leq\sqrt{x}+\sqrt{y}$.
%%%%%%%%%%%%%%%%%%%%%%%%%%%%%%%%%%%%%%%%%%%%%%%%%%%%%%%%%%%%%%%%%%%%%%%%%%%%%%%%%


\end{enumerate}
%%%%%%%%%%%%%%%%%%%%%%%%%%%%%%%%%%%%%%%%%%%%%%%%%%%%%%%%%%%%%%%%%%%%%%%%%%%%%%%%%


\end{document}
%%%%%%%%%%%%%%%%%%%%%%%%%%%%%%%%%%%%%%%%%%%%%%%%%%%%%%%%%%%%%%%%%%%%%%%%%%%%%%%%%

