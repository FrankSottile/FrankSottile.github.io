%Group01.tex
%First Group Homework -- Math 629 
%
%  The percent sign is a comment character
%
%%%%%%%%%%%%%%%%%%%%%%%%%%%%%%%%%%%%%%%%%%%%%%%%%%%%%%%%%%%%%%%%%%%%%%%%%%%%%%%%%%
%
%   Look these up on line.  The first sets the type of document, and the next are for mathematics symbols, graphics and color
%
\documentclass[12pt]{article}
\usepackage{amssymb,amsmath}
\usepackage{graphicx}
\usepackage{cancel}
\usepackage[usenames,dvipsnames,svgnames,table]{xcolor}
\usepackage{multirow}   % This is for more control over tables
%%%%%%%%%%%%%%%%%%%%%%%%%%%%%%%%  Layout     %%%%%%%%%%%%%%%%%%%%%%%%%%%%%%%%%%%%%%
\usepackage{vmargin}
\setpapersize{USletter}
\setmargrb{2cm}{1cm}{2cm}{1cm} % --- sets all four margins LTRB


%%%%%%%%%%%%%%%%%%%%%%%%%%%%%%%%%%%%%%%%%%%%%%%%%%%%%%%%%%%%%%%%%%%%%%%%%%%%%%%%%
\begin{document}
\LARGE 
\noindent
{\color{Maroon}History of Mathematics \hfill Math 629}\vspace{2pt}\\
\large
First Group Homework: \hfill 18 January 2022\\
Not to be handed in, but should be disscused.
\normalsize\vspace{10pt}

\noindent{\bf Standard algorithms for computation in other bases}

 I'd like us all to practice some calculations using the sexagesimal system of the Mesopotamians. For this, let us use the notation that
    `;' represents the `sexagesimal point' and `,' is the delimiter between `places'. Thus `2,22' is 2*60+22=142, one-hundred and forty two,
    while `1;45' is 1+45/60=1.75. Do these using base 60 and show or explain your work. The purpose of this is to appreciate what it is like
    to compute in base 60. To that end, do not simply convert to base ten, do the computations, and convert back. Do them purely in base 60,
    employing the usual algorithms you know.

    \begin{enumerate}

    \item 
      Warm-up: Express the (decimal) numbers in sexagesimal: $45$,   $150$,    $3253$,    $17589$, and    \newline $10^5=100,000$.

        
     \item        
        Simpler: $20 + 50 = W$,      $7*17 = X$,      $3,9 - 1,40 = Y$, and      $1,24*1,24 = Z$.

    \item 
        How about some division: $1/2 = V$, \      $1/3 = W$, \      $2/5 = X$, \      $7/4 = Y$,  and      $2,16/3 = Z$.

    \item 
      Repeating sexagesimals: Why is $1/59 =\ ;1,1,1,....$?
      My favorite decimal fraction is $1/7$. What is this in sexagesimal (multiply your answer by 7 to check) ?
      
      The decimal expansion of 1/11 has the form $0.090909090....$ It is a repeating decimal with period 2.
      
      Express the common fraction 1/7 as a repeating vigesimal. (E.g. $0;a,b,c,\dotsc = a/20 + b/20^2 + c/20^3 + \dotsb$).
      Compare the period of the repeats for this same number in decimal and in sexagesimal.
      Can you explain the relation between the different periods in the different bases?

    \item 
        Compare and contrast the different methods used to represent whole numbers used by Babylonians, Mayans, and by us in our decimal
        positional system. For each ancient system give an example of a computation or representation for which is was superior to the
        others, and one where it was inferior. You can include fractions for this second question. 

    \item {[Challenge]}
      Challenge (only if you, like me, have stamina and like this stuff; this is not required):
       Try to verify that $(1;24,51,10)^2$ is pretty close to
        2, as recorded on YBC 7289. What is the next term in sexagesimal? I found this interesting. 
    \end{enumerate}


    {\color{Maroon}
      Recall that many of use a method to add and multiply numbers in decimal that involves lining up in columns (for place value) and then
      using single-digit additions and multiplications before combining them to get the answer, as below:
\begin{center}
    \begin{tabular}{cccl}
      &\scriptsize{1}&&\\
      &1&3&\quad $8+3=11$\\
      +&2&8&\quad $1+1+2=4$\\ \cline{1-3}
      &4&1\end{tabular}
\qquad
     \begin{tabular}{ccccl}
      &&\scriptsize{\cancel{2}}&&\\
      &&1&3&\quad $8\cdot 3=24$\qquad $8\cdot 1 + 2 = 10$\\
      $\times$&&2&8&\quad $2\cdot 3=6$\\ \cline{2-4}
      &1&0&4\\
      +&2&6&\\ \cline{1-4}
      &3&6&4    \end{tabular}
\end{center}

I will not try to typeset long division, but you should remember that has an algorithm involving place value and smaller calculations, too.
(As does extracting square roots!)\smallskip


Let me now demonstrate some computations in sexagesimal.   First, a conversion from decimal:
\[
288\ =\ 2\cdot 60 + 48\ =\ 4, 48\,.
\]
Let's do a couple of additions
 \begin{center}
    \begin{tabular}{crrl}
      &1,&15&\quad\\
      +&2,&6&\quad\\ \cline{1-3}
      &3,&21\end{tabular}
\qquad
     \begin{tabular}{crrl}
      &\scriptsize{1}&&\\
      &12,&35&\quad $35+41=1\cdot 60 +16=1,16$\\
      $+$&2,&41&\quad $1+12+2=15$\\ \cline{1-3}
      &15,&16\end{tabular}
 \end{center}
 How about a subtraction:
 \begin{center}
     \begin{tabular}{crrl}
      &5&1,8&\\
      &\cancel{6},&\cancel{8}&\quad (regroup: 6,8 = 5 + 1,8)\\
      $-$&2,&11&\quad $11=8+3$ so $1,8-11=1,0 - 3 = 57$.\\ \cline{1-3}
      &3,&57 &\quad $5-2=3$\end{tabular}
\end{center}
Now it is time for a multiplication:
 \begin{center}
     \begin{tabular}{ccrrl}
      &&\scriptsize{\cancel{2}}&&\\
      &&1,&12&\quad $12\cdot 13=156=2\cdot 60 + 36= 2,36$\\
      $\times$&&2,&13&\quad $13+2=15$\\ \cline{2-4}
      &&15,&36\\
      +&2,&24,&\\ \cline{1-4}
      &2,&39,&36    \end{tabular}
 \end{center}

 Before I leave this, here is a historical question:  When or where did these methods of doing arithmetical calculations arise, and what
 effect did they have?  (Our book does not discuss this.)
}
\end{document}
%%%%%%%%%%%%%%%%%%%%%%%%%%%%%%%%%%%%%%%%%%%%%%%%%%%%%%%%%%%%%%%%%%%%%%%%%%%%%%%%%
