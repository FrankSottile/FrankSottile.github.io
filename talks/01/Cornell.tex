\documentclass[12pt]{article}
\usepackage{epsf}
\begin{document}

Common transversals and tangents in P^3

Frank Sottile

Abstract: The following geometric problem has its origins in
computational vision: Determine the (degenerate) configurations of two
lines l1 and l2 and two spheres in R3 for which there are infinitely
many lines simultaneously transversal to l1 and l2 and tangent to both
spheres. We generalize this, replacing the spheres by quadric surfaces
in P3. In this setting, the question has an amazing answer. Fixing the
two lines to be skew and one quadric, the set of degenerate second
quadrics is a curve of degree 24 in the P9 of quadrics which is in
fact the union of 12 plane conics! Moreover, there are examples where
all 12 degenerate families are real. We describe the symbolic
computation behind this result and give some vivid pictures.
