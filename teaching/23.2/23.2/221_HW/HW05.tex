%HW05.tex
%Fifth Homework -- Math 221H
% 
%
% Frank Sottile
% 14 September 2023 
%
%%%%%%%%%%%%%%%%%%%%%%%%%%%%%%%%%%%%%%%%%%%%%%%%%%%%%%%%
\documentclass[12pt]{article}
\usepackage{amssymb,amsmath}
\usepackage{graphicx}
\usepackage[usenames,dvipsnames,svgnames,table]{xcolor}
\usepackage{multirow}   % This is for more control over tables
%%%%%%%%%%%%%%%%%%%%%%%%%%%%%%%%  Layout     %%%%%%%%%%%%%%%%%%%%%%%%%%%%%%%%%%%%%%
\usepackage{vmargin}
\setpapersize{USletter}
\setmargrb{1cm}{0.5cm}{2cm}{0.5cm} % --- sets all four margins LTRB

%%%%%%%%%%%%%%%%%%%%%%%%%%%%%%%%%%%%%%%%  Macros  %%%%%%%%%%%%%%%%%%%%%%%%%%%%%%%%%%%%%%%%
\newcommand{\RR}{{\mathbb R}}  % This is the backboard bold symbol for the real numbers.  Note how it is used below
\newcommand{\NN}{{\mathbb N}}  % 
\newcommand{\ZZ}{{\mathbb Z}}  %

\newcommand{\calP}{{\mathcal P}}  %Caligraphic P for power set

\newcommand{\bfa}{{\bf a}}    %Vectors
\newcommand{\bfb}{{\bf b}}    %Vectors
\newcommand{\bfi}{{\bf i}}    %Unit Vectors
\newcommand{\bfj}{{\bf j}}    %Unit Vectors
\newcommand{\bfk}{{\bf k}}    %Unit Vectors

\newcommand{\sep}{{\ :\ }}     % This is for the : in our notation for building sets.
                               % an acceptable (and common) alternative is \mid  (try it!)
%%%%%%%%%%%%%%%%%%%%%%%%%%%%%%%%%%%%%%%%%%%%%%%%%%%%%%%%%%%%%%%%%%%%%%%%
\begin{document}
\LARGE 
\noindent
{\color{Maroon}Honors Multivariate Calculus \hfill Math 221H Section 201}\vspace{2pt}\\
\large
Fifth Homework:\hfill 
Due in recitation: Thursday 21 September 2023\\

\normalsize
    {\bf {\color{Maroon}Homework about partial derivatives and tangent planes}}

%%%%%%%%%%%%%%%%%%%%%%%%%%%%%%%%%%%%%%%%%%%%%%%%%%%%%%%%%%%%%%%%%%%%%%%%%%%%%%%%%%%%%%%%%%%%%%%%%%%%
\begin{enumerate}


%%%%%%%%%%%%%%%%%%%%%%%%%%%%%%%%%%%%%%%%%%%%%%%%%%%%%%%%%%%%%%%%%%%%%%%%%%%%%%%%%%%%%%%%%%%%%%%%%%%%
  
\item  Find the indicated partial derivatives.

  (a) $xyz=\sin(xy+yz+xz)$ \quad
      ${\displaystyle\frac{\partial z}{\partial x},\ \frac{\partial z}{\partial y}}$.

  (b) $f(x,y,z)=\sqrt{x^2+y^2+z^2}$ \quad $f_x(3,0,4)$, $f_y(0,5,12)$, $f_z(3,4,12)$.

  (c) $u=t^2x^3y^4$ \quad $u_t$, $u_x$, $u_y$.

%%%%%%%%%%%%%%%%%%%%%%%%%%%%%%%%%%%%%%%%%%%%%%%%%%%%%%%%%%%%%%%%%%%%%%%%%%%%%%%%%%%%%%%%%%%%%%%%%%%%

%%%%%%%%%%%%%%%%%%%%%%%%%%%%%%%%%%%%%%%%%%%%%%%%%%%%%%%%%%%%%%%%%%%%%%%%%%%%%%%%%%%%%%%%%%%%%%%%%%%%
  
\item  Find all second partial derivatives.

  (a) $f(x,y)=x^{\ln y}$ \qquad (b) $f(x,y)=\tan^2(3x + 2y)$ \qquad (c) $f(x,y,z)=x y^2 z^3$

%%%%%%%%%%%%%%%%%%%%%%%%%%%%%%%%%%%%%%%%%%%%%%%%%%%%%%%%%%%%%%%%%%%%%%%%%%%%%%%%%%%%%%%%%%%%%%%%%%%%

%%%%%%%%%%%%%%%%%%%%%%%%%%%%%%%%%%%%%%%%%%%%%%%%%%%%%%%%%%%%%%%%%%%%%%%%%%%%%%%%%%%%%%%%%%%%%%%%%%%%
  
\item    Suppose that $f$ is a function of three variables with all higher order partial derivaties continuous.
  How many $n$th order (mixed) partial derivaties does $f$ have, where $n=1,2,3$?
  What is the general formula ?  

%%%%%%%%%%%%%%%%%%%%%%%%%%%%%%%%%%%%%%%%%%%%%%%%%%%%%%%%%%%%%%%%%%%%%%%%%%%%%%%%%%%%%%%%%%%%%%%%%%%%

%%%%%%%%%%%%%%%%%%%%%%%%%%%%%%%%%%%%%%%%%%%%%%%%%%%%%%%%%%%%%%%%%%%%%%%%%%%%%%%%%%%%%%%%%%%%%%%%%%%%
  
\item  Determine which of the following functions satisfies Laplace's equation, $u_{xx}+u_{yy}=0$.

  (a) $u=x^3+3xy$. \qquad
  (b) $u=\sin x \cosh y + \cos x \sinh y$. \qquad
  (c) $u=e^{-x} \cos y - e^{-y} \cos x$.

%%%%%%%%%%%%%%%%%%%%%%%%%%%%%%%%%%%%%%%%%%%%%%%%%%%%%%%%%%%%%%%%%%%%%%%%%%%%%%%%%%%%%%%%%%%%%%%%%%%%

\item Find an equation for the tangent plane to the given surface at the given point.

  (a) $z=x^2 - 2y^2 + 5y$ \quad $(1,2,3)$  \qquad

  (b) $z=\ln(3x+2y)$ \quad $(1,-1,0)$

%%%%%%%%%%%%%%%%%%%%%%%%%%%%%%%%%%%%%%%%%%%%%%%%%%%%%%%%%%%%%%%%%%%%%%%%%%%%%%%%%%%%%%%%%%%%%%%%%%%%

%%%%%%%%%%%%%%%%%%%%%%%%%%%%%%%%%%%%%%%%%%%%%%%%%%%%%%%%%%%%%%%%%%%%%%%%%%%%%%%%%%%%%%%%%%%%%%%%%%%%

\item   Use differentials to approximate the value of $f$ at the given point.

  (a) $f(x,y,z)= x^2 y^3 z^4$ \quad $(3.01, 2.02, 0.99 )$ \qquad

  (b)  $f(x,y,z)= x y^2 \sin(\pi z)$ \quad $(3.99, 2.98, 2.06 )$ 

%%%%%%%%%%%%%%%%%%%%%%%%%%%%%%%%%%%%%%%%%%%%%%%%%%%%%%%%%%%%%%%%%%%%%%%%%%%%%%%%%%%%%%%%%%%%%%%%%%%%

%%%%%%%%%%%%%%%%%%%%%%%%%%%%%%%%%%%%%%%%%%%%%%%%%%%%%%%%%%%%%%%%%%%%%%%%%%%%%%%%%%%%%%%%%%%%%%%%%%%%

\item   Use differentials to approximate the number  \
     $(\sqrt{63} + \sqrt[3]{9})^4$.

%%%%%%%%%%%%%%%%%%%%%%%%%%%%%%%%%%%%%%%%%%%%%%%%%%%%%%%%%%%%%%%%%%%%%%%%%%%%%%%%%%%%%%%%%%%%%%%%%%%%

%%%%%%%%%%%%%%%%%%%%%%%%%%%%%%%%%%%%%%%%%%%%%%%%%%%%%%%%%%%%%%%%%%%%%%%%%%%%%%%%%%%%%%%%%%%%%%%%%%%%

\item   You observe your competitor's pile of sand, and want to know how much sand it holds.  Viewing it remotely, you note that it is in
  the shape of a right circular cone 6 metres high and measure the angle of repose to be 60 degrees. 

(a) Given these measurements, how much sand does the pile hold?

(b) If you reckon that your height measurement might be off by 10 centimetrers, and trust that you measured the angle of repose to within 2
  degrees, what is the error in the volume?  (use differentials and do not neglect that you are using degrees and not radians.


  
\end{enumerate}



\end{document}
%%%%%%%%%%%%%%%%%%%%%%%%%%%%%%%%%%%%%%%%%%%%%%%%%%%%%%%%%%%%%%%%%%%
