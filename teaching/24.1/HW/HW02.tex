%HW2.tex
%Second Homework -- Math 629 
%
%  The percent sign is a comment character
%
%%%%%%%%%%%%%%%%%%%%%%%%%%%%%%%%%%%%%%%%%%%%%%%%%%%%%%%%%%%%%%%%%%%%%%%%%%%%%%%%%%
%
%   Look these up on line.  The first sets the type of document, and the next are for mathematics symbols, graphics and color
%
\documentclass[12pt]{article}
\usepackage{amssymb,amsmath}
\usepackage{graphicx}
\usepackage[usenames,dvipsnames,svgnames,table]{xcolor}
\usepackage{multirow}   % This is for more control over tables
%%%%%%%%%%%%%%%%%%%%%%%%%%%%%%%%  Layout     %%%%%%%%%%%%%%%%%%%%%%%%%%%%%%%%%%%%%%
\usepackage{vmargin}
\setpapersize{USletter}
\setmargrb{2cm}{1cm}{2cm}{1cm} % --- sets all four margins LTRB


%%%%%%%%%%%%%%%%%%%%%%%%%%%%%%%%%%%%%%%%%%%%%%%%%%%%%%%%%%%%%%%%%%%%%%%%%%%%%%%%%
\begin{document}
\LARGE 
\noindent
{\color{Maroon}History of Mathematics \hfill Math 629}\vspace{2pt}\\
\large
Second Homework: \hfill 25 January 2022\\
Due Monday 31 January 2022.
\normalsize\vspace{10pt}

      Some of these problems will require you to find other material than just what is in the readings.
To hand in: We are using Gradescope for homework submission.


\begin{enumerate}

\item    Who was the British mathematician who made the remark at the top of the page?
  (The Greeks are `fellows of another college'.)
  What do you think he meant by this?


\item     Greek mathematics deeply affected at least two US presidents. Research and write about the influence of Euclid and Pythagoras on
  US Presidents. (There is one US president for each of these Greek mathematicians.)
  For each, there is at least one very interesting
  detail/story. Find it, and describe it. 


\item     What were the three geometric problems of antiquity? (Sometimes called the Three Classical Problems.)
  For one, describe it origins (at least the best that you can find out). Were any solved by the Greeks (in any fashion)?
  This may require some additional reading or research.


\item     Do Exercise 1.3.4 in Stillwell. Note that $t$ is the vertical coordinate of the point where the longish secant meets the vertical
  axis. How is this related to the `world's sneakiest substitution' from Calculus? 


\item     Do Exercise 1.4.2 of Stillwell.
  This is one of the easiest proofs of the Pythagorean Theorem, and is often presented in elementary geometry classes.


\item     Do Exercises 2.4.1 and 2.4.2 from Stillwell.


%
%  Notice how this was commented out.
%
%\item     Do Exercises 2.5.1 and 2.5.2 from Stillwell.
       
\end{enumerate}

\end{document}
%%%%%%%%%%%%%%%%%%%%%%%%%%%%%%%%%%%%%%%%%%%%%%%%%%%%%%%%%%%%%%%%%%%%%%%%%%%%%%%%%
