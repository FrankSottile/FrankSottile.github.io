%Homework2.tex
%Second Homework -- Math 300 
%
%  The percent sign is a comment character
%
%%%%%%%%%%%%%%%%%%%%%%%%%%%%%%%%%%%%%%%%%%%%%%%%%%%%%%%%%%%%%%%%%%%%%%%%%%%%%%%%%%
%
%   Look these up on line.  The first sets the type of document, and the next are for mathematics symbols, graphics and color
%
\documentclass[12pt]{article}
\usepackage{amssymb,amsmath}
\usepackage{graphicx}
\usepackage[usenames,dvipsnames,svgnames,table]{xcolor}
\usepackage{multirow}   % This is for more control over tables
%%%%%%%%%%%%%%%%%%%%%%%%%%%%%%%%  Layout     %%%%%%%%%%%%%%%%%%%%%%%%%%%%%%%%%%%%%%
\usepackage{vmargin}
\setpapersize{USletter}
\setmargrb{2cm}{1cm}{2cm}{1cm} % --- sets all four margins LTRB


\newcommand{\defn}[1]{{\color{blue}\sl #1}}
\newcommand{\deco}[1]{{\color{blue} #1}}

\newcommand{\NN}{{\mathbb N}}
\newcommand{\QQ}{{\mathbb Q}}
\newcommand{\RR}{{\mathbb R}}
\newcommand{\ZZ}{{\mathbb Z}}

%%%%%%%%%%%%%%%%%%%%%%%%%%%%%%%%%%%%%%%%%%%%%%%%%%%%%%%%%%%%%%%%%%%%%%%%%%%%%%%%%
\begin{document}
\LARGE 
\noindent
{\color{Maroon}Foundations of Mathematics \hfill Math 300 Sections 902, 905}\vspace{2pt}\\
\Large YOUR NAME\vspace{2pt}\\
\large
Second Homework: \hfill Due 7 September 2020
\normalsize\vspace{10pt}


Note:  \deco{$\RR$} is the real numbers, \deco{$\QQ$} is the rational numbers, \deco{$\ZZ$} is the integers, and
\deco{$\NN$} is the natural numbers (which begin with 0).

%%%%%%%%%%%%%%%%%%%%%%%%%%%%%%%%%%%%%%%%%%%%%%%%%%%%%%%%%%%%%%%%%%%%%%%%%%%%%%%%%
\begin{enumerate}

%%%%%%%%%%%%%%%%%%%%%%%%%%%%%%%%%%%%%%%%%%%%%%%%%%%%%%%%%%%%%%%%%%%%%%%%%%%%%%%%%
\item  Rewrite the following statements in the form ``if $P$, then $Q$''.
  \begin{enumerate}

    \item ``One, if by land''.
    \item ``Candor implies equality''.
    \item ``Pepperoni only if pizza''.
    \item ``Inattentive when bored''.
    \item ``Slapstick is sufficient for comedy''.
    \item ``Quiet is necessary for sleep''.

  \end{enumerate}


%\newpage
%%%%%%%%%%%%%%%%%%%%%%%%%%%%%%%%%%%%%%%%%%%%%%%%%%%%%%%%%%%%%%%%%%%%%%%%%%%%%%%%%  
\item Rewrite ``If the function $f$ is differentiable, then it is continuous'' in each of the six forms of the previous problem.


%\newpage
%%%%%%%%%%%%%%%%%%%%%%%%%%%%%%%%%%%%%%%%%%%%%%%%%%%%%%%%%%%%%%%%%%%%%%%%%%%%%%%%%
\item Express the following statements in the form ``If P then Q''.
  For example, ``A Hexagon has six sides'' becomes ``If $H$ is a hexagon, then $H$ has six sides''.
  \begin{enumerate}
   
   \item An integer is odd or even.
   \item All positive real numbers have a square root.
   \item All angles of of an equilateral triangle are equal.
   \item 1 is the smallest positive integer.
   \item When the product of two integers is even, then both integers are even.
     
  \end{enumerate}
  

%\newpage
%%%%%%%%%%%%%%%%%%%%%%%%%%%%%%%%%%%%%%%%%%%%%%%%%%%%%%%%%%%%%%%%%%%%%%%%%%%%%%%%%
\item Fill out the truth table for the expressions
  $P\wedge Q$, $\lnot P$, $\lnot Q$, $\lnot(P\wedge Q)$, $\lnot P\wedge\lnot Q$, $\lnot P\vee\lnot Q$, and 
   $(\lnot P\wedge\lnot Q)\vee Q$:

  \begin{tabular}{|c|c||c|c|c|c|c|c|c|}\hline
    $P$&$Q$&$P\wedge Q$&$\lnot P$& $\lnot Q$& $\lnot(P\wedge Q)$& $\lnot P\wedge\lnot Q$& $\lnot P\vee\lnot Q$
      &  $(\lnot P\wedge\lnot Q)\vee Q$\\ \hline\hline
     T &T&&&&&&&\\
     T &F&&&&&&&\\
     F &T&&&&&&&\\
     F &F&&&&&&&
    \end{tabular}\medskip
  
%\newpage
%%%%%%%%%%%%%%%%%%%%%%%%%%%%%%%%%%%%%%%%%%%%%%%%%%%%%%%%%%%%%%%%%%%%%%%%%%%%%%%%%
\item  Which of the following expressions are tautologies?  Which are contradictions?
  \begin{enumerate}
    
   \item $(P \to Q)\vee(Q\to P)$
   \item $(P\wedge Q)\vee(\lnot P\wedge \lnot Q)$
   \item $P\to (Q\to P)$
   \item $(P\wedge Q)\wedge(Q\to\lnot P)$
    
  \end{enumerate}

%\newpage
%%%%%%%%%%%%%%%%%%%%%%%%%%%%%%%%%%%%%%%%%%%%%%%%%%%%%%%%%%%%%%%%%%%%%%%%%%%%%%%%%
\item  Rewrite ``Friendship is necessary and sufficient for happiness'' in nine additional, equivalent ways.


%\newpage
%%%%%%%%%%%%%%%%%%%%%%%%%%%%%%%%%%%%%%%%%%%%%%%%%%%%%%%%%%%%%%%%%%%%%%%%%%%%%%%%%
\item Suppose that each of the following statements is true.
  \begin{itemize}
  \item Ibukun is in middle school.
  \item Ibukun got 90 on her German test or  Adeola got 90 on her German test.
  \item If Adeola got 90 on her German test, then Ibukun is not in middle school.
  \end{itemize}
  If possible, determine the truth values of each of the following statements.  Explain your reasoning.
  \begin{enumerate}
   \item  Ibukun  got 90 on her German test.
   \item  Adeola  got 90 on her German test.
   \item  Either Adeola or Ibukun did not get 90 on the German test.
  \end{enumerate}
 
 
%\newpage
%%%%%%%%%%%%%%%%%%%%%%%%%%%%%%%%%%%%%%%%%%%%%%%%%%%%%%%%%%%%%%%%%%%%%%%%%%%%%%%%%
\item
  For statements $P$, $Q$, and $R$:
  \begin{enumerate}
    
  \item Show that $[(P\to Q)\wedge P]\to Q$ is a tautology.
      {\bf Note:} In logic, this is an important rule of logic called \defn{modus ponens}.
  \item Show that $[(P\to Q)\wedge (Q\to R)]\to (P\to R)$ is a tautology.
      {\bf Note:} In logic, this is an important rule of logic called \defn{syllogism}.

    \item Give example of a valid syllogism involving Socrates.
          Give example of a false syllogism involving Socrates.

  \end{enumerate}


%\newpage
%%%%%%%%%%%%%%%%%%%%%%%%%%%%%%%%%%%%%%%%%%%%%%%%%%%%%%%%%%%%%%%%%%%%%%%%%%%%%%%%%
\item Fill out a truth table (with $8=2^3$ rows) for the two  expressions $(P\vee Q)\wedge(P\vee R)$
       and  $P\vee(Q\wedge R)$.   What do you observe?
  
%%%%%%%%%%%%%%%%%%%%%%%%%%%%%%%%%%%%%%%%%%%%%%%%%%%%%%%%%%%%%%%%%%%%%%%%%%%%%%%%%


%%%%%%%%%%%%%%%%%%%%%%%%%%%%%%%%%%%%%%%%%%%%%%%%%%%%%%%%%%%%%%%%%%%%%%%%%%%%%%%%%
%
%  Above is from the worksheet from Day 3
%  Next comes the worksheet from Day 4
%%%%%%%%%%%%%%%%%%%%%%%%%%%%%%%%%%%%%%%%%%%%%%%%%%%%%%%%%%%%%%%%%%%%%%%%%%%%%%%%%

%\newpage
%%%%%%%%%%%%%%%%%%%%%%%%%%%%%%%%%%%%%%%%%%%%%%%%%%%%%%%%%%%%%%%%%%%%%%%%%%%%%%%%%
\item Write the converse and contrapositive of the following conditional statements:
  \begin{enumerate}
    \item If it rains, then the grass is wet.
    \item $\alpha^2=25$ if $\alpha=5$.
    \item The integer $a$ is odd only if $3a$ is odd.
    \item ``Inattentive when bored''.
    \item ``Quiet is necessary for sleep''.
    \item ``Pepperoni is necessary for Pizza''.
  \end{enumerate}

%\newpage
%%%%%%%%%%%%%%%%%%%%%%%%%%%%%%%%%%%%%%%%%%%%%%%%%%%%%%%%%%%%%%%%%%%%%%%%%%%%%%%%%  
%
\item Give the contrapositive and converse of each of the implications in Problem 1.
     Write them in the form ``If $P$, then $Q$''.

\newpage
%%%%%%%%%%%%%%%%%%%%%%%%%%%%%%%%%%%%%%%%%%%%%%%%%%%%%%%%%%%%%%%%%%%%%%%%%%%%%%%%%
\item Write a useful negation of each of the following statements.
  Do not leave a negation as the prefix of a statement.
  For example, the negation of ``I will water my garden and pick basil'' is
  ``I will not water my garden or I will not pick basil''.
  \begin{enumerate}
  \item   You will walk or take the bus.
  \item   Knowledge is necessary for truth.
  \item   That was Country and Western.
  \item   That was Country or Western.
  \item   If you wash the dishes or put away the laundry, you can have some chocolate.
  \item   Hard work is necessary for success.
  \end{enumerate}
 
%\newpage
%%%%%%%%%%%%%%%%%%%%%%%%%%%%%%%%%%%%%%%%%%%%%%%%%%%%%%%%%%%%%%%%%%%%%%%%%%%%%%%%%
\item  Let $a$, $b$, and $c$ be integers.
  Consider the following conditional statement:
  \[\mbox{If $a$ divides $bc$, then $a$ divides $b$ or $a$ divides $c$.}\]
  Which of the following statements have the same meaning as this conditional statement, and which are negations of this
  conditional statement:
  \begin{enumerate}
   \item If $a$ divides $b$ or $a$ divides $c$, then $a$ divides $bc$.
   \item If $a$ does not divide $b$ or $a$ does not divide $c$, then $a$ does not divide $bc$.
   \item $a$ divides $bc$, $a$ does not divide $b$, and $a$ does not divide $c$.
   \item If $a$ does not divide $b$ and $a$ does not divide $c$, then $a$ does not divide $bc$.
   \item $a$ does not divide $bc$ or $a$  divides $b$ or $a$  divides $c$.
   \item If $a$ divides $bc$ and $a$ does not divide $c$, then $a$ divides $b$.
   \item If $a$ divides $bc$ or $a$ does not divide $b$, then $a$ divides $c$.
  \end{enumerate}

%\newpage
%%%%%%%%%%%%%%%%%%%%%%%%%%%%%%%%%%%%%%%%%%%%%%%%%%%%%%%%%%%%%%%%%%%%%%%%%%%%%%%%%
\item Use the roster method to specify the elements in each of the following sets and then write a sentence in English
  describing the set.
  \begin{enumerate}

   \item $\{ x\in{\RR} \mid x^2-2x-4=0\}$.
   \item $\{n\in \ZZ\mid n^2<27\}$.
   \item $\{n\in \NN\mid n^2<27\}$.
   \item $\{ x\in{\QQ} \mid x^2-2x-4=0\}$.
    
  \end{enumerate}


%\newpage
%%%%%%%%%%%%%%%%%%%%%%%%%%%%%%%%%%%%%%%%%%%%%%%%%%%%%%%%%%%%%%%%%%%%%%%%%%%%%%%%%
\item Use set builder notation to specify the following sets.
   \begin{enumerate}

   \item The set of all natural numbers with square at least 15.
   \item The set of all odd integers.
   \item The set of all real numbers at most 10 whose square exceeds $3$.
   \item The set of positive rational numbers.
    
  \end{enumerate}
 
\end{enumerate}
%%%%%%%%%%%%%%%%%%%%%%%%%%%%%%%%%%%%%%%%%%%%%%%%%%%%%%%%%%%%%%%%%%%%%%%%%%%%%%%%%



\end{document}
  

\end{enumerate}
%%%%%%%%%%%%%%%%%%%%%%%%%%%%%%%%%%%%%%%%%%%%%%%%%%%%%%%%%%%%%%%%%%%%%%%%%%%%%%%%%


\end{document}
%%%%%%%%%%%%%%%%%%%%%%%%%%%%%%%%%%%%%%%%%%%%%%%%%%%%%%%%%%%%%%%%%%%%%%%%%%%%%%%%%
