%HW9.tex
%Ninth Homework -- Math 300 
%
%  The percent sign is a comment character
%
%%%%%%%%%%%%%%%%%%%%%%%%%%%%%%%%%%%%%%%%%%%%%%%%%%%%%%%%%%%%%%%%%%%%%%%%%%%%%%%%%%
%
%   Look these up on line.  The first sets the type of document, and the next are for mathematics symbols, graphics and color
%
\documentclass[12pt]{article}
\usepackage{amssymb,amsmath}
\usepackage{graphicx}
\usepackage[usenames,dvipsnames,svgnames,table]{xcolor}
\usepackage{multirow}   % This is for more control over tables
%%%%%%%%%%%%%%%%%%%%%%%%%%%%%%%%  Layout     %%%%%%%%%%%%%%%%%%%%%%%%%%%%%%%%%%%%%%
\usepackage{vmargin}
\setpapersize{USletter}
\setmargrb{1cm}{.1cm}{2cm}{.5cm} % --- sets all four margins LTRB


\newcommand{\defn}[1]{{\color{blue}\sl #1}}
\newcommand{\deco}[1]{{\color{blue} #1}}

\newcommand{\calP}{{\mathcal P}}

\newcommand{\NN}{{\mathbb N}}
\newcommand{\QQ}{{\mathbb Q}}
\newcommand{\RR}{{\mathbb R}}
\newcommand{\ZZ}{{\mathbb Z}}

%%%%%%%%%%%%%%%%%%%%%%%%%%%%%%%%%%%%%%%%%%%%%%%%%%%%%%%%%%%%%%%%%%%%%%%%%%%%%%%%%
\begin{document}
\LARGE 
\noindent
{\color{Maroon}Foundations of Mathematics \hfill Math 300 Sections 902, 905}\vspace{2pt}\\
\Large YOUR NAME\vspace{2pt}\\
\large
Ninth Homework: \hfill Due {\color{red}11} November 2020
\normalsize    %%%% This returns the fontsize to 12 point
%\vspace{10pt}  %%%% This makes a vertical space of 10 points (approx 70 points/inch) after the line break

\noindent{\color{blue}\rule{529pt}{2pt}}

  
%%%%%%%%%%%%%%%%%%%%%%%%%%%%%%%%%%%%%%%%%%%%%%%%%%%%%%%%%%%%%%%%%%%%%%%%%%%%%%%%%
\begin{enumerate}


%\newpage
%%%%%%%%%%%%%%%%%%%%%%%%%%%%%%%%%%%%%%%%%%%%%%%%%%%%%%%%%%%%%%%%%%%%%%%%%%%%%%%%%
\item{[16]} Do all parts of Problem 17 in the Exercises for Section 6.3 in the Sundstrom book.
%%%%%%%%%%%%%%%%%%%%%%%%%%%%%%%%%%%%%%%%%%%%%%%%%%%%%%%%%%%%%%%%%%%%%%%%%%%%%%%%%



%\newpage
%%%%%%%%%%%%%%%%%%%%%%%%%%%%%%%%%%%%%%%%%%%%%%%%%%%%%%%%%%%%%%%%%%%%%%%%%%%%%%%%%
  \item{[10]}  Let $A$ and $B$ be sets.
    Recall the definitions of the identity functions $I_A\colon A\to A$ and  $I_B\colon B\to B$:
    For $a\in A$, $I_A(a)=a$ and for $b\in B$, $I_B(b)=b$.

    Let $f\colon A\to B$ be a function.
    Prove by a direct computation that $f=f\circ I_A$ and that $f=I_B\circ f$.
%%%%%%%%%%%%%%%%%%%%%%%%%%%%%%%%%%%%%%%%%%%%%%%%%%%%%%%%%%%%%%%%%%%%%%%%%%%%%%%%%

%\newpage
%%%%%%%%%%%%%%%%%%%%%%%%%%%%%%%%%%%%%%%%%%%%%%%%%%%%%%%%%%%%%%%%%%%%%%%%%%%%%%%%%
  \item{[10]}  Let $A$ be a set.  Prove that the identity function $I_A$ is a bijection.
%%%%%%%%%%%%%%%%%%%%%%%%%%%%%%%%%%%%%%%%%%%%%%%%%%%%%%%%%%%%%%%%%%%%%%%%%%%%%%%%%

%\newpage
%%%%%%%%%%%%%%%%%%%%%%%%%%%%%%%%%%%%%%%%%%%%%%%%%%%%%%%%%%%%%%%%%%%%%%%%%%%%%%%%%
\item{[15]} For each of the following, either give an example of functions $f\colon A\to B$ and $g\colon B\to C$ that satisfy the given
  properties, or explain why no such example exists.

 \begin{enumerate}
  \item The function $g$ is a surjection, but the function $g\circ f$ is not a surjection.

  \item The function $g$ is an injection, but the function $g\circ f$ is not an injection.

  \item The function $f$ is not a surjection, but the function $g\circ f$ is a surjection.

  \item The function $g$ is not a surjection, but the function $g\circ f$ is a surjection.

  \item The function $g$ is not an injection, but the function $g\circ f$ is a surjection.

 \end{enumerate}  

%%%%%%%%%%%%%%%%%%%%%%%%%%%%%%%%%%%%%%%%%%%%%%%%%%%%%%%%%%%%%%%%%%%%%%%%%%%%%%%%%


%\newpage
%%%%%%%%%%%%%%%%%%%%%%%%%%%%%%%%%%%%%%%%%%%%%%%%%%%%%%%%%%%%%%%%%%%%%%%%%%%%%%%%%
\item{[17]}   Let $f\colon A\to B$ and $g\colon B\to A$ be functions.
            Recall the identity functions  $I_A\colon A\to A$ and  $I_B\colon B\to B$.
        Preferably using theorems previously proven in the class (state those that you use), show the following.

 \begin{enumerate}
  \item If $g\circ f=I_A$, then $f$ is an injection.
  \item If $f\circ g=I_B$, then $f$ is a surjection.
  \item If $g\circ f=I_A$ and $f\circ b=I_B$, then $f$ and $g$ are bijections and $g=f^{-1}$.
 \end{enumerate}  

%%%%%%%%%%%%%%%%%%%%%%%%%%%%%%%%%%%%%%%%%%%%%%%%%%%%%%%%%%%%%%%%%%%%%%%%%%%%%%%%%


%\newpage
%%%%%%%%%%%%%%%%%%%%%%%%%%%%%%%%%%%%%%%%%%%%%%%%%%%%%%%%%%%%%%%%%%%%%%%%%%%%%%%%%
\item{[12]} Let $f\colon S\to T$ be a function, $A,B$ be subsets of $S$ and $C,D$ be subsets of $T$.
  For $x\in S$ and $y\in T$, carefully explain what is means to say that 
  
 \begin{enumerate}
  \item $y\in f(A\cup B)$.
  \item $y\in f(A)\cap f(B)$.
  \item $x\in f^{-1}(C\cap D)$.
  \item $x\in f^{-1}(C) \cup f^{-1}(D)$.
 \end{enumerate}  
%%%%%%%%%%%%%%%%%%%%%%%%%%%%%%%%%%%%%%%%%%%%%%%%%%%%%%%%%%%%%%%%%%%%%%%%%%%%%%%%%

%\newpage
%%%%%%%%%%%%%%%%%%%%%%%%%%%%%%%%%%%%%%%%%%%%%%%%%%%%%%%%%%%%%%%%%%%%%%%%%%%%%%%%%
\item{[10]} Let $f\colon \RR\to \RR$ be defined by $f(x)=-2x+1$ and let
  \[
  A \ :=\ [2,5] \qquad
  B \ :=\ [-1,3] \qquad
  C \ :=\ [-2,3] \qquad
  D \ :=\ [1,4] 
  \]  
  Find each of the following sets:
 \begin{enumerate}
  \item $f(A)$
  \item $f^{-1}(C)$
  \item $f^{-1}(C\cap D)$
  \item $f^{-1}(f(B))$
  \item $f^{-1}(C)\cup f^{-1}(D)$
 \end{enumerate}  
%%%%%%%%%%%%%%%%%%%%%%%%%%%%%%%%%%%%%%%%%%%%%%%%%%%%%%%%%%%%%%%%%%%%%%%%%%%%%%%%%

%\newpage
%%%%%%%%%%%%%%%%%%%%%%%%%%%%%%%%%%%%%%%%%%%%%%%%%%%%%%%%%%%%%%%%%%%%%%%%%%%%%%%%%
\item{[10]}  Let $f\colon A\to B$ be a function and $T\subset B$.
           Prove that $T\supseteq f(f^{-1}(T))$.  
%%%%%%%%%%%%%%%%%%%%%%%%%%%%%%%%%%%%%%%%%%%%%%%%%%%%%%%%%%%%%%%%%%%%%%%%%%%%%%%%%

%%%%%%%%%%%%%%%%%%%%%%%%%%%%%%%%%%%%%%%%%%%%%%%%%%%%%%%%%%%%%%%%%%%%%%%%%%%%%%%%%
\end{enumerate}
%%%%%%%%%%%%%%%%%%%%%%%%%%%%%%%%%%%%%%%%%%%%%%%%%%%%%%%%%%%%%%%%%%%%%%%%%%%%%%%%%


\end{document}
%%%%%%%%%%%%%%%%%%%%%%%%%%%%%%%%%%%%%%%%%%%%%%%%%%%%%%%%%%%%%%%%%%%%%%%%%%%%%%%%%


