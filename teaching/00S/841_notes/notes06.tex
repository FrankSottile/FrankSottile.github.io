%notes06.tex
%
% Frank Sottile
% 6 March 2000
% Madison, WI
%
\documentclass[12pt]{amsart}
\usepackage{amssymb}
\usepackage{epsf}

\headheight=8pt     \topmargin=-10pt
\textheight=644pt   \textwidth=432pt
\oddsidemargin=18pt \evensidemargin=18pt

\def\silentfootnote#1{{\let\thefootnote\relax\footnotetext{#1}}}

\newcommand{\G}{{\bf G}}
\newcommand{\Var}{{\mathcal V}}

%\def\baselinestretch{1.2}

\begin{document}
\begin{center}
\Large
The Geometry of Algebraic Groups\\
\large
Math 841\\
Frank Sottile\\
{\tt notes06}
\end{center}\bigskip

\silentfootnote{\sl Version of 2 April 2000} 

\section{Maps of Projective Varieties}

A quotient $F(s)/G(s)$ of polynomials $F,G\in k[s_0,s_1,\ldots,s_n]$ has a
well-defined value at $\xi\in{\mathbb P}^n$ only if $G(\xi)\neq 0$, $F$ and
$G$ are homogeneous, and $\deg F=\deg G$.
It follows that there are few regular functions on projective varieties.
We first expand our notion of varieties, and then define maps on these
varieties. 
\medskip

\noindent{\bf Definition. }
A {\it quasi-projective variety} $Y$ is an open subset of a projective
variety $X$.
That is,  a locally closed subset of projective space.
\medskip

\noindent{\bf Definition. }
Let $X\subset{\mathbb P}^n$ be a quasi-projective variety.
A {\it regular map} $\varphi\colon X\to{\mathbb P}^m$ is an $(m+1)$-tuple of
homogeneous forms $(F_0,F_1,\ldots,F_m)$ all of the same degree such that 
$$
  \Var(F_0,\ldots,F_m)\cap X\ =\ \emptyset\,.
$$
Such an $(m+1)$-tuple defines a function on the points of $X$ by
$\varphi(x)=[F_0(x),F_1(x),\ldots,F_m(x)]$.
Since the forms $F_i$ all have the same degree, this gives a well-defined
point in ${\mathbb P}^m$ for each $x\in X$.
Another $(m+1)$-tuple of homogeneous forms $G_0,\ldots,G_m$ all of the same 
degree with $\Var(G_0,\ldots,G_m)\cap X=\emptyset$ defines the same map 
if we have
$$
  \mbox{rank }\left[\begin{array}{cccc}
      F_0&F_1&\cdots&F_m\\G_0&G_1&\cdots&G_m\end{array}\right]\ =\ 1
  \eqno{(*) \ }
$$
on $X$.
That is, if all $2\times 2$ minors vanish on $X$ (lie in the ideal of $X$).
\smallskip

More generally, a regular map $\varphi$ on a quasi-projective variety $X$ is
a collection of $(m+1)$-tuples of homogeneous polynomials with each
$(m+1)$-tuple consisting of polynomials of the same degree,
and this collection has the following property:
For every point $x$ of $X$, there is some polynomial in some $(m+1)$-tuple 
that does not vanish at $x$ and, given any two $(m+1)$-tuples 
$F_0,\ldots,F_m$ and $G_0,\ldots,G_m$ we have $(*)$, at points of $X$ where
some polynomial in each $(m+1)$-tuple does not vanish.
\smallskip

An important (and simple) example of a regular map is a linear projection.
Let $\Lambda_0,\Lambda_1,\ldots,\Lambda_m$ be (independent) linear forms.
These give a map $\varphi$ which is defined on 
${\mathbb P}^n-E$, where $E$ is the common zero locus of the linear forms
$\Lambda_0,\ldots,\Lambda_m$, that is 
$E={\mathbb P}(\mbox{kernel}(M))$, where $M$ is the matrix whose columns are
the $\Lambda_i$.


A map $\varphi\colon X\to Y$ of quasi-projective varieties is an 
{\it isomorphism} if it has a regular inverse.
\medskip

\noindent{\bf Definition. }
A quasi-projective variety $X$ is {\it projective} (respectively 
{\it affine}) if if it is isomorphic to a closed subset of 
${\mathbb P}^n$ (respectively ${\mathbb A}^n$ or an affine piece of 
${\mathbb P}^n$).
\medskip

A quasi-projective variety $X\in {\mathbb P}^n$ has the property that each
point $x\in X$ has 
an open neighbourhood $U$ which is isomorphic to an affine variety.
First, $x$ lies in some affine piece of ${\mathbb P}^n$, and so we may
replace $X$ by its intersection with that affine piece and assume that $X$
is a locally closed subset of ${\mathbb A}^n$.
Let $Y:=\overline{X}$, and affine variety.
If we set $Z=Y-X$, then there is some $f\in k[Y]$ which vanishes on the
closed subvariety $Z$, but not at $x$.
Then the principal open subset $Y_f$ is a neighbourhood of $X$ in $X$, and
it is an affine variety, as we saw previously.

This is very similar to the property that manifolds have---a smooth manifold
is a space where each point has a neighbourhood isomorphic to a ball, with
the gluing maps all $C^\infty$.
One may (but we do not!) more generally define an algebraic variety to be a
something where each point has a neighbourhood isomorphic to an affine
variety with the gluing maps regular (polynomial) maps.
We instead stay concrete with our notion of quasi-projective varieties.


An important example of a regular map is the {\it Segre embedding}, which
enables us to define products of quasi-projective varieties.
\begin{eqnarray*}
  {\mathbb P}^n\times{\mathbb P^m}&\longrightarrow&
       {\mathbb P}^{nm+n+m}\\
   {[x_0,\ldots,x_n]} 
   \times [y_0,\ldots,y_m]
       &\longmapsto&
    [\ldots,x_iy_j,\ldots]
\end{eqnarray*}
If we let $z_{ij}$ be the coordinates of ${\mathbb P}^{nm+n+m}$,
then $\varphi$ is defined by $z_{ij}=x_iy_j$.
The image of $\varphi$ is defined by the quadratic polynomials
$z_{ij}z_{kl}=z_{il}z_{jk}$.
The map $\varphi$ is an isomorphism onto its image.
The inverse map to $\varphi$ is given (for each $k,l$) by
$$
  [\ldots,z_{ij},\ldots]\ \longmapsto\ 
  [z_{0l},z_{1l},\ldots,z_{nl}]\times
  [z_{k0},z_{k1},\ldots,z_{km}]\,.
$$

The Segre embedding allows us to define products $X\times Y$ of
quasi-projective varieties $X$ and $Y$.


We study some further properties of maps of projective varieties.
Using the local description of a quasi-projective variety given above, we may
show that the image of a quasi-projective variety under a regular map is
constructible. 
One may do much better if the variety is projective.
To see this, we begin with a definition.

Given a regular map $\varphi\colon X\to Y$, let 
$\Gamma_\varphi\subset X\times Y$ be its graph, the set of points
$\{(x,\varphi(x))\mid x\in X\}$. 
\medskip

\noindent{\bf Lemma. }
{\it
The graph of a regular map is closed.}\medskip

\begin{proof}
It suffices to prove this locally in $X\times Y$, and so we may assume that
$X$ and $Y$ are affine varieties with $Y\in {\mathbb A}^m$ and $\varphi$ is
given by an $m$-tuple $f_1,\ldots,f_m$ of regular functions on $X$.
But then the graph of $\varphi$ is defined by 
$y_1-f_1,\ldots,y_m-f_m$, and hence us closed in $X\times Y$.
\end{proof}

\noindent{\bf Theorem. }
{\it 
The image of a projective variety under a regular map is closed.
}\medskip

By the previous Lemma, this follows from:
\medskip

\noindent{\bf Theorem. }
{\it
If $X$ is a projective variety and $Y$ is a quasi-projective variety, then
the projection map $X\times Y \to Y$ is a closed mapping.
(It takes closed sets to closed sets.)}\medskip


\begin{proof}
Since $X$ is a closed subset of projective space, we can assume that
$X={\mathbb P}^n$.
Since closedness is local, we may assume that $Y$ is affine, and then that
$Y={\mathbb A}^m$.
With these reductions, we need to show the following:
Suppose $Z\subset{\mathbb P}^n\times{\mathbb A}^m$ is a subvariety and
$\wp\colon{\mathbb P}^n\times{\mathbb A}^m\to{\mathbb A}^m$ is the
projection map.
Then $\wp(Z)$ is a closed subset of ${\mathbb A}^m$.

Let $g_1,\ldots,g_l$ define $Z$, where $g_i=g_i(s;t)$.
For $y\in {\mathbb A}^m$, the inverse image $\wp^{-1}(y)\cap Z$ consists of
all solutions to the equations
$$
  g_1(s;y)\ =\ g_1(s;y)\ =\ \cdots\ =\ g_l(s;y)\ =\ 0\,.
$$
This is non-empty only if 
$$
  {\mathfrak m}_0^a\ \not\subset\ 
   \langle g_1(s;y),\ldots,g_l(s;y)\rangle\quad
   \forall a > 0\,.
$$
If we set $T_a:=\{y\in{\mathbb A}^m\mid {\mathfrak m}_0^a\ \not\subset\ 
   \langle g_1(s;y),\ldots,g_l(s;y)\rangle\}$, then 
$$
  \wp(Z)\ =\ \bigcap_{a>0} T_a\,.
$$
And so $\wp(Z)$ will be closed if each $T_a$ is closed.

Each function $g_i\in k[t_1,\ldots,t_m][s_0,\ldots,s_n]$ is a homogeneous
polynomial of degree $b_i$ in the variables $s_0,\ldots,s_n$ whose
coefficients are themselves polynomials in $k[t_1,\ldots,t_m]$.
We have ${\mathfrak m}_0^a\subset\langle g_1(s;y),\ldots,g_l(s;y)\rangle$
if and only if for every degree $a$ form $G$, there are homogeneous
polynomials $h_{i,G}(s)$ of degree $a-k_i$ such that 
$$
  G\ =\ \sum_i h_{i,G}g(s,y)\,.
$$

We express this in terms of linear algebra.
Let $A_j$ be the $k$-vector space of degree $j$ homogeneous forms in
$s_0,\ldots,s_n$ and consider the linear map:
\begin{eqnarray*}
\varphi(y)\colon A_{a-k_1}\oplus A_{a-k_2}\oplus\cdots\oplus A_{a-k_l}
       &\longrightarrow& A_a\\
(h_1(s),\ldots,h_l(s))&\longmapsto& \sum h_i(s) g_i(s,y)
\end{eqnarray*}
Then $T_a=\{y\mid \varphi(y)\mbox{ is not onto}\}$.
But this is defined by the vanishing of all $a\times a$ minors of a matrix
representing $\varphi(y)$, which is a collection of polynomials in $y$.
\end{proof}

\noindent{\bf Corollary. }
{\it
If $X$ is a connected projective variety, and 
$\varphi\colon X\to {\mathbb A}^1$ a regular map, then $\varphi$ is
constant map. 
}\medskip

\begin{proof}
If we consider ${\mathbb A}^1\subset{\mathbb P}^1$, then we see that the
image of $\varphi$ is a closed subset of ${\mathbb P}^1$ which is contained
in ${\mathbb A}^1$.
But the only such closed subsets are finite sets of points.
Since $X$ is connected, its image under $\varphi$ is too, and so it consists
of a single point.
\end{proof}

This shows that projective varieties have few regular functions, as we
claimed earlier.
Similarly, a map from a projective variety into an affine variety must be
constant. 
This fact has an interesting consequence for algebraic groups.
\medskip

\noindent{\bf Corollary. }
{\it
Connected projective algebraic groups have no non-trivial algebraic
representations. 
}\medskip

Thus if we want to study groups via their representations, we must restrict
ourselves to linear algebraic groups, which are all affine.
Projective algebraic groups also have the interesting property that they are 
necessarily abelian!
%%%%%%%%%%%%%%%%%%%%%%%%%%%%%%%%


\end{document}


