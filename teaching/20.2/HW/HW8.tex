%HW8.tex
%Eighth Homework -- Math 300 
%
%  The percent sign is a comment character
%
%%%%%%%%%%%%%%%%%%%%%%%%%%%%%%%%%%%%%%%%%%%%%%%%%%%%%%%%%%%%%%%%%%%%%%%%%%%%%%%%%%
%
%   Look these up on line.  The first sets the type of document, and the next are for mathematics symbols, graphics and color
%
\documentclass[12pt]{article}
\usepackage{amssymb,amsmath}
\usepackage{graphicx}
\usepackage[usenames,dvipsnames,svgnames,table]{xcolor}
\usepackage{multirow}   % This is for more control over tables
%%%%%%%%%%%%%%%%%%%%%%%%%%%%%%%%  Layout     %%%%%%%%%%%%%%%%%%%%%%%%%%%%%%%%%%%%%%
\usepackage{vmargin}
\setpapersize{USletter}
\setmargrb{1cm}{.1cm}{2cm}{1cm} % --- sets all four margins LTRB


\newcommand{\defn}[1]{{\color{blue}\sl #1}}
\newcommand{\deco}[1]{{\color{blue} #1}}

\newcommand{\calP}{{\mathcal P}}

\newcommand{\NN}{{\mathbb N}}
\newcommand{\QQ}{{\mathbb Q}}
\newcommand{\RR}{{\mathbb R}}
\newcommand{\ZZ}{{\mathbb Z}}

%%%%%%%%%%%%%%%%%%%%%%%%%%%%%%%%%%%%%%%%%%%%%%%%%%%%%%%%%%%%%%%%%%%%%%%%%%%%%%%%%
\begin{document}
\LARGE 
\noindent
{\color{Maroon}Foundations of Mathematics \hfill Math 300 Sections 902, 905}\vspace{2pt}\\
\Large YOUR NAME\vspace{2pt}\\
\large
Seventh Homework: \hfill Due 2 November 2020
\normalsize    %%%% This returns the fontsize to 12 point
%\vspace{10pt}  %%%% This makes a vertical space of 10 points (approx 70 points/inch) after the line break

\noindent{\color{blue}\rule{529pt}{2pt}}

  
%%%%%%%%%%%%%%%%%%%%%%%%%%%%%%%%%%%%%%%%%%%%%%%%%%%%%%%%%%%%%%%%%%%%%%%%%%%%%%%%%
\begin{enumerate}


%\newpage
%%%%%%%%%%%%%%%%%%%%%%%%%%%%%%%%%%%%%%%%%%%%%%%%%%%%%%%%%%%%%%%%%%%%%%%%%%%%%%%%%
\item{[14]} Do all parts of Problem 12 in the Exercises for Section 5.3 in the Sundstrom book.
%%%%%%%%%%%%%%%%%%%%%%%%%%%%%%%%%%%%%%%%%%%%%%%%%%%%%%%%%%%%%%%%%%%%%%%%%%%%%%%%%

%\newpage
%%%%%%%%%%%%%%%%%%%%%%%%%%%%%%%%%%%%%%%%%%%%%%%%%%%%%%%%%%%%%%%%%%%%%%%%%%%%%%%%%
\item{[16]} Sketch a picture (with labels) of each of the following in the Cartesian plane $\RR^2$ with the usual conventions
    (the first coordinate is horizontal, while the second is vertical):

  \makebox[250pt][l]{(a)  $[0,2]\times [1,3]$}\ (b) $(0,2)\times (1,3]$

  \makebox[250pt][l]{(c) $[2,3]\times \{1\}$}\  (d) $\{2\}\times [3,4]$ 

  \makebox[250pt][l]{(e) $\RR\times (2,4)$}\    (f) $(1,3]\times \RR$

  \makebox[250pt][l]{(g) $\RR\times \{-1\}$}\   (h) $\{-1\}\times [1,\infty)$


%%%%%%%%%%%%%%%%%%%%%%%%%%%%%%%%%%%%%%%%%%%%%%%%%%%%%%%%%%%%%%%%%%%%%%%%%%%%%%%%%


%\newpage
%%%%%%%%%%%%%%%%%%%%%%%%%%%%%%%%%%%%%%%%%%%%%%%%%%%%%%%%%%%%%%%%%%%%%%%%%%%%%%%%%
\item{[10]}  Is the following proposition true or false?  Justify your conclusion.
  \[
    \mbox{ Let $A$, $B$, and $C$ be sets with $A\neq\emptyset$. \quad
    If $A\times B= A\times C$, then $B=C$.}
  \]
  Explain where the assumption $A\neq\emptyset$ is needed.
  What happens when $A=\emptyset$?
%%%%%%%%%%%%%%%%%%%%%%%%%%%%%%%%%%%%%%%%%%%%%%%%%%%%%%%%%%%%%%%%%%%%%%%%%%%%%%%%%


%\newpage
%%%%%%%%%%%%%%%%%%%%%%%%%%%%%%%%%%%%%%%%%%%%%%%%%%%%%%%%%%%%%%%%%%%%%%%%%%%%%%%%%
\item{[10]}   For each positive integer $n\in\NN$, let $\deco{A_n}:=(-\frac{1}{n}, 1{-}\frac{1}{n})\subset\RR$.
  
  Determine each of \qquad ${\displaystyle \bigcap_{k\in\NN} A_k}$ \qquad and \qquad 
  $\bigcup\{ A_\ell\mid \ell\in\NN\}$.
%%%%%%%%%%%%%%%%%%%%%%%%%%%%%%%%%%%%%%%%%%%%%%%%%%%%%%%%%%%%%%%%%%%%%%%%%%%%%%%%%



%\newpage
%%%%%%%%%%%%%%%%%%%%%%%%%%%%%%%%%%%%%%%%%%%%%%%%%%%%%%%%%%%%%%%%%%%%%%%%%%%%%%%%%
\item{[8]}   Let $\nu$ be the function from $\NN$ to $\NN$ whose value at a positive integer $n$ is the number of digits in the American
  English spelling of the number $n$.
  For example $\nu(0)=4$, as '0' is written {\sf zero} with four letters.
  Similarly, $\nu(22)=9$, as {\sf twentytwo} has nine letters.

  If we restrict the domain of $\nu$ to $\{1,2,\dotsc,20\}$, what is its range?
%%%%%%%%%%%%%%%%%%%%%%%%%%%%%%%%%%%%%%%%%%%%%%%%%%%%%%%%%%%%%%%%%%%%%%%%%%%%%%%%%

%\newpage
%%%%%%%%%%%%%%%%%%%%%%%%%%%%%%%%%%%%%%%%%%%%%%%%%%%%%%%%%%%%%%%%%%%%%%%%%%%%%%%%%
\item{[10]} A \defn{real function} is one whose domain and codomain are subsets of $\RR$.
  For each of the following real functions, determine their largest possible domain and their range.

  (a) The function $f$ defined by $f(x)=x/(x^2-3x-2)$.

  (b) The function $g$ defined by $g(x)=\ln(1-\cos(x))$.
%%%%%%%%%%%%%%%%%%%%%%%%%%%%%%%%%%%%%%%%%%%%%%%%%%%%%%%%%%%%%%%%%%%%%%%%%%%%%%%%%


%\newpage
%%%%%%%%%%%%%%%%%%%%%%%%%%%%%%%%%%%%%%%%%%%%%%%%%%%%%%%%%%%%%%%%%%%%%%%%%%%%%%%%%
\item{[8]}  Let $s\colon\NN\to\NN$ be the function whose value $s(n)$ at a number $n$ is the sum of the distinct natural number divisors of
  $n$. 
  Compute the values of $s$ on the set $\{1,\dotsc,10\}$.
  Is the function $s$ injective?  Is it surjective?  Justify your conclusions.
%%%%%%%%%%%%%%%%%%%%%%%%%%%%%%%%%%%%%%%%%%%%%%%%%%%%%%%%%%%%%%%%%%%%%%%%%%%%%%%%%



%\newpage
%%%%%%%%%%%%%%%%%%%%%%%%%%%%%%%%%%%%%%%%%%%%%%%%%%%%%%%%%%%%%%%%%%%%%%%%%%%%%%%%%
\item{[8]}  Let $d\colon\NN\to\NN$ be the function whose value $d(n)$ at a number $n$ is the number of distinct natural number divisors of
  $n$. 
  Compute the values of $d$ on the set $\{1,\dotsc,10\}$.
  Is the function $d$ injective?  Is it surjective?  Justify your conclusions.
%%%%%%%%%%%%%%%%%%%%%%%%%%%%%%%%%%%%%%%%%%%%%%%%%%%%%%%%%%%%%%%%%%%%%%%%%%%%%%%%%

%\newpage
%%%%%%%%%%%%%%%%%%%%%%%%%%%%%%%%%%%%%%%%%%%%%%%%%%%%%%%%%%%%%%%%%%%%%%%%%%%%%%%%%
\item{[8]}  Let $h\colon\RR\times\RR$ be the function defined for $(x,y)\in\RR\times\RR$  by
        $h(x,y):=x^2y+3xy-2y+3$.
       Is the function $h$ injective?   Is it surjective?   Justify your conclusions.
%%%%%%%%%%%%%%%%%%%%%%%%%%%%%%%%%%%%%%%%%%%%%%%%%%%%%%%%%%%%%%%%%%%%%%%%%%%%%%%%%


%\newpage
%%%%%%%%%%%%%%%%%%%%%%%%%%%%%%%%%%%%%%%%%%%%%%%%%%%%%%%%%%%%%%%%%%%%%%%%%%%%%%%%%
\item{[8]} Define $\varphi\colon \NN\to\ZZ$ be the function defined for $n\in\NN$ by $f(n)=\dfrac{1+(-1)^n(2n-1)}{4}$.
       Is the function $\varphi$ injective?   Is it surjective?   Justify your conclusions.
%%%%%%%%%%%%%%%%%%%%%%%%%%%%%%%%%%%%%%%%%%%%%%%%%%%%%%%%%%%%%%%%%%%%%%%%%%%%%%%%%





%%%%%%%%%%%%%%%%%%%%%%%%%%%%%%%%%%%%%%%%%%%%%%%%%%%%%%%%%%%%%%%%%%%%%%%%%%%%%%%%%
\end{enumerate}
%%%%%%%%%%%%%%%%%%%%%%%%%%%%%%%%%%%%%%%%%%%%%%%%%%%%%%%%%%%%%%%%%%%%%%%%%%%%%%%%%


\end{document}
%%%%%%%%%%%%%%%%%%%%%%%%%%%%%%%%%%%%%%%%%%%%%%%%%%%%%%%%%%%%%%%%%%%%%%%%%%%%%%%%%


