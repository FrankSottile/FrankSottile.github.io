%HW09.tex
%Ninth Homework -- Math 629 
%
%  The percent sign is a comment character
%
%%%%%%%%%%%%%%%%%%%%%%%%%%%%%%%%%%%%%%%%%%%%%%%%%%%%%%%%%%%%%%%%%%%%%%%%%%%%%%%%%%
%
%   Look these up on line.  The first sets the type of document, and the next are for mathematics symbols, graphics and color
%
\documentclass[12pt]{article}
\usepackage{amssymb,amsmath}
\usepackage{graphicx}
\usepackage[usenames,dvipsnames,svgnames,table]{xcolor}
\usepackage{multirow}   % This is for more control over tables
%%%%%%%%%%%%%%%%%%%%%%%%%%%%%%%%  Layout     %%%%%%%%%%%%%%%%%%%%%%%%%%%%%%%%%%%%%%
\usepackage{vmargin}
\setpapersize{USletter}
\setmargrb{2cm}{1cm}{2cm}{1cm} % --- sets all four margins LTRB


%%%%%%%%%%%%%%%%%%%%%%%%%%%%%%%%%%%%%%%%%%%%%%%%%%%%%%%%%%%%%%%%%%%%%%%%%%%%%%%%%
\begin{document}
\LARGE 
\noindent
{\color{Maroon}History of Mathematics \hfill Math 629}\vspace{2pt}\\
\large
Ninth Homework: \hfill 19 March 2022\\
Due Monday 28 March 2022.
\normalsize\vspace{10pt}

To hand in: We are using Gradescope for homework submission.


\begin{enumerate}

\item  {[15]}
     Exercise 13.5.1 from Stillwell.

\item  {[15]}
     Exercise 13.5.2 from Stillwell.

\item  {[10]}
  Problem 1 from d'Alembert's solution of wave equation.
  
\item  {[15]}
     Problem 2 from d'Alembert's solution of wave equation.

\item  {[10]}
     Problem 3 from d'Alembert's solution of wave equation.

\item  {[10]}
       Exercise 1 on Mixed Partial Derivatives

\item  {[10]}
       Exercise 2 on Mixed Partial Derivatives

\item  {[15]}
       Exercise 3 on Mixed Partial Derivatives


 \end{enumerate}
\end{document}
%%%%%%%%%%%%%%%%%%%%%%%%%%%%%%%%%%%%%%%%%%%%%%%%%%%%%%%%%%%%%%%%%%%%%%%%%%%%%%%%%
